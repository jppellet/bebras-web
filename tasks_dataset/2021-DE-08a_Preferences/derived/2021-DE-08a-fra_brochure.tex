% Definition of the meta information: task difficulties, task ID, task title, task country; definition of the variables as well as their scope is in commands.tex
\setcounter{taskAgeDifficulty3to4}{2}
\setcounter{taskAgeDifficulty5to6}{1}
\setcounter{taskAgeDifficulty7to8}{0}
\setcounter{taskAgeDifficulty9to10}{0}
\setcounter{taskAgeDifficulty11to13}{0}
\renewcommand{\taskTitle}{Cadeau favori}
\renewcommand{\taskCountry}{DE}

% include this task only if for the age groups being processed this task is relevant
\ifthenelse{
  \(\boolean{age3to4} \AND \(\value{taskAgeDifficulty3to4} > 0\)\) \OR
  \(\boolean{age5to6} \AND \(\value{taskAgeDifficulty5to6} > 0\)\) \OR
  \(\boolean{age7to8} \AND \(\value{taskAgeDifficulty7to8} > 0\)\) \OR
  \(\boolean{age9to10} \AND \(\value{taskAgeDifficulty9to10} > 0\)\) \OR
  \(\boolean{age11to13} \AND \(\value{taskAgeDifficulty11to13} > 0\)\)}{

\newchapter{\taskTitle}

% task body
La famille castor a trois cadeaux pour ses trois enfants. Chaque enfant indique d’abord son cadeau favori, puis son second choix. Les cadeaux doivent être bien distribués:

\begin{enumerate}
  \item Le plus d’enfants possible doivent recevoir leur cadeau favori.
  \item Les autres enfants doivent recevoir leur second choix.
\end{enumerate}



% question (as \emph{})
{\em
Donne les bons cadeaux aux enfants.

{\centering%
\includesvg[scale=0.75]{\taskGraphicsFolder/graphics/2021-DE-08a-question-compatible.svg}\par}


}

% answer alternatives (as \begin{enumerate}[A)]) or interactivity


% from here on this is only included if solutions are processed
\ifthenelse{\boolean{solutions}}{
\newpage

% answer explanation
\section*{\BrochureSolution}
Voici la seule manière de distribuer les cadeaux en respectant les deux conditions.

{\centering%
\includesvg[scale=0.75]{\taskGraphicsFolder/graphics/2021-DE-08a-solution-compatible.svg}\par}

Seul le deuxième castor désire le troisième cadeau, c’est donc lui qui doit le recevoir. Sinon, un autre castor recevrait un cadeau qui n’est ni son cadeau favori, ni son deuxième choix. La distribution des deux autres cadeaux est claire: chaque castor reçoit son cadeau favori.



% it's informatics
\section*{\BrochureItsInformatics}
Dans cet exercice, nous avons affaire à un \emph{problème d’affectation} univoque: nous voulons affecter les cadeaux de manière à ce que tous les enfants recoivent un cadeau. Les enfants n’ont ici pas qu’un seul souhait, mais une liste de préférence. De tels problèmes d’affectation avec listes de préférence peuvent devenir très compliqués. L’informatique nous aide à résoudre de tels problèmes rapidement.

Une possibilité est de donner une valeur aux affectations: le cadeau favori a la valeur $1$ et le deuxième choix la valeur $2$. Un \emph{couplage} (\emph{matching} en anglais) est optimal s’il n’existe pas d’autre couplage avec plus de premiers choix distribués. Un tel couplage est appelé \emph{couplage parfait de poids minimum}. Il existe beaucoup de problèmes d’affectation. L’un deux est appelé \emph{problème des marriages stables} (\emph{Stable Marriage Problem} en anglais). Intéressant? L’informatique est une branche très variée!



% keywords and websites (as \begin{itemize})
\section*{\BrochureWebsitesAndKeywords}
{\raggedright
\begin{itemize}
  \item Problème d’affectation: \href{https://fr.wikipedia.org/wiki/Probl\%C3\%A8me_d\%27affectation}{\BrochureUrlText{https://fr.wikipedia.org/wiki/Problème\_d’affectation}}
  \item Couplage: \href{https://fr.wikipedia.org/wiki/Couplage_(th\%C3\%A9orie_des_graphes)}{\BrochureUrlText{https://fr.wikipedia.org/wiki/Couplage\_(théorie\_des\_graphes)}}
\end{itemize}


}

% end of ifthen for excluding the solutions
}{}

% all authors
% ATTENTION: you HAVE to make sure an according entry is in ../main/authors.tex.
% Syntax: \def\AuthorLastnameF{} (Lastname is last name, F is first letter of first name, this serves as a marker for ../main/authors.tex)
\def\AuthorPohlW{} % \ifdefined\AuthorPohlW \BrochureFlag{de}{} Wolfgang Pohl\fi
\def\AuthorVoborilF{} % \ifdefined\AuthorVoborilF \BrochureFlag{at}{} Florentina Voboril\fi
\def\AuthorKinciusV{} % \ifdefined\AuthorKinciusV \BrochureFlag{lt}{} Vaidotas Kinčius\fi
\def\AuthorFreiF{} % \ifdefined\AuthorFreiF \BrochureFlag{ch}{} Fabian Frei\fi
\def\AuthorPelletE{} % \ifdefined\AuthorPelletE \BrochureFlag{ch}{} Elsa Pellet\fi

\newpage}{}
