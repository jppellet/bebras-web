\documentclass[a4paper,11pt]{report}
\usepackage[T1]{fontenc}
\usepackage[utf8]{inputenc}

\usepackage[italian]{babel}
\AtBeginDocument{\def\labelitemi{$\bullet$}}

\usepackage{etoolbox}

\usepackage[margin=2cm]{geometry}
\usepackage{changepage}
\makeatletter
\renewenvironment{adjustwidth}[2]{%
    \begin{list}{}{%
    \partopsep\z@%
    \topsep\z@%
    \listparindent\parindent%
    \parsep\parskip%
    \@ifmtarg{#1}{\setlength{\leftmargin}{\z@}}%
                 {\setlength{\leftmargin}{#1}}%
    \@ifmtarg{#2}{\setlength{\rightmargin}{\z@}}%
                 {\setlength{\rightmargin}{#2}}%
    }
    \item[]}{\end{list}}
\makeatother

\newcommand{\BrochureUrlText}[1]{\texttt{#1}}
\usepackage{setspace}
\setstretch{1.15}

\usepackage{tabularx}
\usepackage{booktabs}
\usepackage{makecell}
\usepackage{multirow}
\renewcommand\theadfont{\bfseries}
\renewcommand{\tabularxcolumn}[1]{>{}m{#1}}
\newcolumntype{R}{>{\raggedleft\arraybackslash}X}
\newcolumntype{C}{>{\centering\arraybackslash}X}
\newcolumntype{L}{>{\raggedright\arraybackslash}X}
\newcolumntype{J}{>{\arraybackslash}X}

\newcommand{\BrochureInlineCode}[1]{{\ttfamily #1}}

\usepackage{amssymb}
\usepackage{amsmath}

\usepackage[babel=true,maxlevel=3]{csquotes}
\DeclareQuoteStyle{bebras-ch-eng}{“}[” ]{”}{‘}[”’ ]{’}\DeclareQuoteStyle{bebras-ch-deu}{«}[» ]{»}{“}[»› ]{”}
\DeclareQuoteStyle{bebras-ch-fra}{«\thinspace{}}[» ]{\thinspace{}»}{“}[»\thinspace{}› ]{”}
\DeclareQuoteStyle{bebras-ch-ita}{«}[» ]{»}{“}[»› ]{”}
\setquotestyle{bebras-ch-ita}

\usepackage{hyperref}
\usepackage{graphicx}
\usepackage{svg}
\svgsetup{inkscapeversion=1,inkscapearea=page}
\usepackage{wrapfig}

\usepackage{enumitem}
\setlist{nosep,itemsep=.5ex}

\setlength{\parindent}{0pt}
\setlength{\parskip}{2ex}
\raggedbottom

\usepackage{fancyhdr}
\usepackage{lastpage}
\pagestyle{fancy}

\fancyhf{}
\renewcommand{\headrulewidth}{0pt}
\renewcommand{\footrulewidth}{0.4pt}
\lfoot{\scriptsize © 2023 Bebras (CC BY-SA 4.0)}
\cfoot{\scriptsize\itshape 2023-BE-01 Le commissioni di Emma}
\rfoot{\scriptsize Page~\thepage{}/\pageref*{LastPage}}

\newcommand{\taskGraphicsFolder}{..}

\begin{document}

\section*{\centering{} 2023-BE-01 Le commissioni di Emma}


\subsection*{Body}

Emma è a casa \raisebox{\dimexpr -0.5ex -0.2ex \relax}[0pt][0pt]{\includesvg[width=11.5px]{\taskGraphicsFolder/graphics/2023-BE-01-inline-meinstandort.svg}}.  Deve svolgere tre compiti
e tornare:

\begin{itemize}
  \item ritirare un pacco al chiosco\raisebox{\dimexpr -0.5ex -0.2ex \relax}[0pt][0pt]{\includesvg[width=14.4px]{\taskGraphicsFolder/graphics/2023-BE-01-inline-kiosk.svg}}
  \item comprare frutta al mercato \raisebox{\dimexpr -0.5ex -0.2ex \relax}[0pt][0pt]{\includesvg[width=10.8px]{\taskGraphicsFolder/graphics/2023-BE-01-inline-markt.svg}} e
  \item Andare in farmacia \raisebox{\dimexpr -0.5ex -0.2ex \relax}[0pt][0pt]{\includesvg[width=14.4px]{\taskGraphicsFolder/graphics/2023-BE-01-inline-apotheke.svg}} per prendere una medicina.
\end{itemize}

Emma non sa quanto tempo impiegherà in ogni negozio.
Ma il viaggio dovrebbe essere il più breve possibile.

Emma ha scritto su una mappa quanti minuti dedicherà all’attività di
spostamento tra le singole località della città.

Ha anche segnato sulla planimetria il percorso che sta facendo.

Emma ha bisogno in questo caso di un totale di ${6 + 3 + 7 + 9 + 3 + 6 + 4 = 38}$ minuti per completare il percorso.

{\centering%
\includesvg[scale=0.35]{\taskGraphicsFolder/graphics/2023-BE-01-deu-challenge.svg}\par}

Emma si chiede se si può essere ancora più veloci.
Forse è utile percorrere alcune strade più di una volta?

{\em


\subsection*{Question/Challenge - for the brochures}

Determina il percorso più breve che Emma può intraprendere per completare i suoi tre compiti.

{\centering%
\includesvg[scale=0.35]{\taskGraphicsFolder/graphics/2023-BE-01-deu-challengebrochure.svg}\par}

}


\subsection*{Interactivity instruction - for the online challenge}

Fa clic su una freccia per selezionare o deselezionare il percorso nella direzione della freccia del percorso più breve.
In basso a sinistra è possibile vedere quanti minuti impiega Emma per percorrere i percorsi selezionati.
Al termine, fa clic su \enquote{Salva risposta}.

\begingroup
\renewcommand{\arraystretch}{1.5}
\subsection*{Answer Options/Interactivity Description}

Die Pfeile und Linien haben zwei Zustände: ausgewählt und abgewählt.
Ein Klick auf einen Pfeil bewirkt einen Zustandswechsel:

Abgewählt -> Ausgewählt:
Der Pfeil ist danach hervorgehoben, die zugehörige Linie auch (sie war möglicherweise auch vorher schon hervorgehoben).

Ausgewählt -> Abgewählt:
Der Pfeil ist danach nicht mehr hervorgehoben.  Die zugehörige Linie ist danach (a) immer noch hervorgehoben, wenn der Pfeil in Gegenrichtung ausgewählt ist, oder (b) sonst nicht hervorgehoben.

Zu Beginn sind die zum Beispielweg gehörenden Linien mit jeweils einem der beiden Pfeile so ausgewählt, dass sich ein Rundweg ergibt.

\endgroup

\subsection*{Answer Explanation}

Questa è la soluzione:

{\centering%
\includesvg[scale=0.35]{\taskGraphicsFolder/graphics/2023-BE-01-deu-solution.svg}\par}

Emma può camminare lungo i percorsi selezionati (o nella direzione opposta):

{\centering%
\includesvg[width=324.7px]{\taskGraphicsFolder/graphics/2023-BE-01-weg.svg}\par}

Per percorrere questa distanza ha bisogno di ${6 + 3 + 3 + 4 + 4 + 3 + 3 + 6 + 4 = 36}$ minuti.

\begin{tabularx}{\columnwidth}{ @{} J l @{} }
  Vogliamo giustificare perché non può esistere un percorso ancora più breve. Per farlo, utilizziamo una rappresentazione semplificata del piano. & \makecell[l]{\includesvg[scale=0.35]{\taskGraphicsFolder/graphics/graph1.svg}} \\ 
  Possiamo ignorare i percorsi disegnati in grigio. Esistono percorsi più brevi tra i luoghi collegati dai percorsi, ossia attraverso altri luoghi. & \makecell[l]{\includesvg[scale=0.35]{\taskGraphicsFolder/graphics/graph2.svg}} \\ 
  Possiamo anche ignorare il parco, Emma infatti non deve andare al parco. Inoltre, per ogni percorso che passa per il parco, esiste un’alternativa più breve. & \makecell[l]{\includesvg[scale=0.35]{\taskGraphicsFolder/graphics/graph3.svg}} \\ 
  Emma deve andare in farmacia \raisebox{\dimexpr -0.5ex -0.2ex \relax}[0pt][0pt]{\includesvg[width=14.4px]{\taskGraphicsFolder/graphics/2023-BE-01-inline-apotheke.svg}} e al chiosco \raisebox{\dimexpr -0.5ex -0.2ex \relax}[0pt][0pt]{\includesvg[width=14.4px]{\taskGraphicsFolder/graphics/2023-BE-01-inline-kiosk.svg}}.  Può arrivarci solo dal panificio \raisebox{\dimexpr -0.5ex -0.2ex \relax}[0pt][0pt]{\includesvg[width=14.4px]{\taskGraphicsFolder/graphics/2023-BE-01-inline-baeckerei.svg}} o dalla scuola \raisebox{\dimexpr -0.5ex -0.2ex \relax}[0pt][0pt]{\includesvg[width=15.2px]{\taskGraphicsFolder/graphics/2023-BE-01-inline-schule.svg}}. Deve percorrere a piedi la distanza tra questi luoghi. Ci vogliono ${3 + 3 = 6}$, ovvero $12$ minuti in totale. Ricordiamolo e combiniamo i due luoghi di cui sopra con quelli di cui sotto in uno solo. & \makecell[l]{\includesvg[scale=0.35]{\taskGraphicsFolder/graphics/graph4.svg}} \\ 
  Ora rimane solo il grafico a destra. L’inizio e la fine del percorso sono qui \raisebox{\dimexpr -0.5ex -0.2ex \relax}[0pt][0pt]{\includesvg[width=11.5px]{\taskGraphicsFolder/graphics/2023-BE-01-inline-meinstandort.svg}}. Questi tre luoghi (\raisebox{\dimexpr -0.5ex -0.2ex \relax}[0pt][0pt]{\includesvg[width=14.4px]{\taskGraphicsFolder/graphics/2023-BE-01-inline-baeckerei.svg}} \raisebox{\dimexpr -0.5ex -0.2ex \relax}[0pt][0pt]{\includesvg[width=15.2px]{\taskGraphicsFolder/graphics/2023-BE-01-inline-schule.svg}} \raisebox{\dimexpr -0.5ex -0.2ex \relax}[0pt][0pt]{\includesvg[width=10.8px]{\taskGraphicsFolder/graphics/2023-BE-01-inline-markt.svg}}) devono essere visitati. Il percorso più breve che soddisfa questo requisito passa attraverso tutti e cinque i luoghi e lungo tutti i percorsi tranne quello grigio e richiede ${4 + 6 + 4 + 6 = 24}$ minuti. Con i $12$ minuti di cui sopra, fanno $36$ minuti. Le considerazioni precedenti dimostrano che non può esistere un percorso più breve. & \makecell[l]{\includesvg[scale=0.35]{\taskGraphicsFolder/graphics/graph5.svg}}
\end{tabularx}


\subsection*{This is Informatics}

Per giustificare la risposta corretta è stata utilizzata una rappresentazione semplificata della mappa.
Sarebbe stato possibile presentare la mappa in modo molto più astratto:

{\centering%
\includesvg[scale=0.35]{\taskGraphicsFolder/graphics/graph1-abstract-compatible.svg}\par}

Questa rappresentazione contiene tutte le informazioni importanti per il percorso di Emma, ovvero

\begin{itemize}
  \item Oggetti: i luoghi, con segnati i luoghi importanti per il percorso;
  \item e relazioni tra gli oggetti: le distanze tra i luoghi per ognuno dei quali è indicata una lunghezza.
\end{itemize}

Uno strumento importante per modellare le relazioni tra gli oggetti sono i \emph{grafi}.
I grafi sono costituiti da nodi (per gli oggetti) e da bordi (coppie di oggetti, per le relazioni).
La mappa di Emma può essere modellata come un \emph{grafo pesato},
dove alle singole relazioni vengono assegnati dei valori numerici (i \emph{pesi}).

L’informatica è interessata a domande che possono essere poste in relazione ai grafi,
e per gli algoritmi che possono essere utilizzati per rispondere alle domande.
Una domanda importante per i grafi pesati è:
Qual è il percorso più breve (o più veloce) tra due nodi?
La \enquote{domanda del grafico} in questo compito è simile:
Qual è il viaggio di andata e ritorno più breve da un nodo che visita molti altri nodi?
L’informatica conosce molti algoritmi in grado di determinare in modo efficiente i percorsi più brevi nei grafi.
Tali algoritmi sono implementati in software per, ad esempio, la pianificazione di itinerari.


\subsection*{This is Computational Thinking}

Optional - not to be filled 2023


\subsection*{Informatics Keywords and Websites}

\begin{itemize}
  \item Grafi: \href{https://it.wikipedia.org/wiki/Grafo}{\BrochureUrlText{https://it.wikipedia.org/wiki/Grafo}}
\end{itemize}


\subsection*{Computational Thinking Keywords and Websites}

Optional - not to be filled 2023


\end{document}
