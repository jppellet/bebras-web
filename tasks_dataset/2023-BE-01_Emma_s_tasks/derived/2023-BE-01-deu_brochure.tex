% Definition of the meta information: task difficulties, task ID, task title, task country; definition of the variables as well as their scope is in commands.tex
\setcounter{taskAgeDifficulty3to4}{0}
\setcounter{taskAgeDifficulty5to6}{4}
\setcounter{taskAgeDifficulty7to8}{4}
\setcounter{taskAgeDifficulty9to10}{3}
\setcounter{taskAgeDifficulty11to13}{2}
\renewcommand{\taskTitle}{Emma erledigt}
\renewcommand{\taskCountry}{BE}

% include this task only if for the age groups being processed this task is relevant
\ifthenelse{
  \(\boolean{age3to4} \AND \(\value{taskAgeDifficulty3to4} > 0\)\) \OR
  \(\boolean{age5to6} \AND \(\value{taskAgeDifficulty5to6} > 0\)\) \OR
  \(\boolean{age7to8} \AND \(\value{taskAgeDifficulty7to8} > 0\)\) \OR
  \(\boolean{age9to10} \AND \(\value{taskAgeDifficulty9to10} > 0\)\) \OR
  \(\boolean{age11to13} \AND \(\value{taskAgeDifficulty11to13} > 0\)\)}{

\newchapter{\taskTitle}

% task body
Emma ist zu Hause \raisebox{\dimexpr -0.5ex -0.2ex \relax}[0pt][0pt]{\includesvg[width=11.5px]{\taskGraphicsFolder/graphics/2023-BE-01-inline-meinstandort.svg}}.  Sie soll drei Aufgaben erledigen
und zurückkommen:

\begin{itemize}
  \item beim Kiosk \raisebox{\dimexpr -0.5ex -0.2ex \relax}[0pt][0pt]{\includesvg[width=14.4px]{\taskGraphicsFolder/graphics/2023-BE-01-inline-kiosk.svg}} ein Päckchen abholen,
  \item auf dem Markt \raisebox{\dimexpr -0.5ex -0.2ex \relax}[0pt][0pt]{\includesvg[width=10.8px]{\taskGraphicsFolder/graphics/2023-BE-01-inline-markt.svg}} Obst kaufen und
  \item in der Apotheke \raisebox{\dimexpr -0.5ex -0.2ex \relax}[0pt][0pt]{\includesvg[width=14.4px]{\taskGraphicsFolder/graphics/2023-BE-01-inline-apotheke.svg}} ein Medikament besorgen.
\end{itemize}

Emma weiss nicht, wie lange sie in jedem Geschäft brauchen wird.
Aber zumindest ihr Weg soll so kurz wie möglich sein.

Auf einem Plan hat Emma eingetragen, wie viele Minuten sie für die
Strecken zwischen einzelnen Orten ihrer Stadt benötigt.
Ausserdem hat sie im Plan markiert, welche Strecke sie auf ihrem Weg geht.

Für diesen Weg benötigt Emma insgesamt ${6 + 3 + 7 + 9 + 3 + 6 + 4 = 38}$ Minuten.

{\centering%
\includesvg[scale=0.35]{\taskGraphicsFolder/graphics/2023-BE-01-deu-challenge.svg}\par}

Emma überlegt, ob es noch schneller geht.
Vielleicht hilft es, manche Strecken hin und zurück zu gehen?



% question (as \emph{})
{\em
Bestimme den kürzesten Weg, den Emma gehen kann, um ihre drei Aufgaben zu erledigen.

{\centering%
\includesvg[scale=0.35]{\taskGraphicsFolder/graphics/2023-BE-01-deu-challengebrochure.svg}\par}


}

% answer alternatives (as \begin{enumerate}[A)]) or interactivity


% from here on this is only included if solutions are processed
\ifthenelse{\boolean{solutions}}{
\newpage

% answer explanation
\section*{\BrochureSolution}
So ist es richtig:

{\centering%
\includesvg[scale=0.35]{\taskGraphicsFolder/graphics/2023-BE-01-deu-solution.svg}\par}

Emma kann so entlang der ausgewählten Strecken gehen (oder in die Gegenrichtung):

{\centering%
\includesvg[scale=0.35]{\taskGraphicsFolder/graphics/2023-BE-01-weg.svg}\par}

Für diesen Weg braucht sie ${6 + 3 + 3 + 4 + 4 + 3 + 3 + 6 + 4 = 36}$ Minuten.

\begin{tabularx}{\columnwidth}{ @{} J l @{} }
  Nun wollen wir begründen, warum es keinen noch kürzeren Weg geben kann. Dazu benutzen wir eine vereinfachte Darstellung des Plans. & \makecell[l]{\includesvg[scale=0.35]{\taskGraphicsFolder/graphics/graph1.svg}} \\ 
  Die grau gezeichneten Strecken können wir ignorieren. Es gibt kürzere Wege zwischen den durch die Strecken verbundenen Orte, nämlich über andere Orte. & \makecell[l]{\includesvg[scale=0.35]{\taskGraphicsFolder/graphics/graph2.svg}} \\ 
  Auch den Park können wir ignorieren.  Emma muss nicht zum Park. Zudem gibt es für jeden Weg, der über den Park geht, eine kürzere Alternative. & \makecell[l]{\includesvg[scale=0.35]{\taskGraphicsFolder/graphics/graph3.svg}} \\ 
  Emma muss zur Apotheke \raisebox{\dimexpr -0.5ex -0.2ex \relax}[0pt][0pt]{\includesvg[width=14.4px]{\taskGraphicsFolder/graphics/2023-BE-01-inline-apotheke.svg}} und zum Kiosk \raisebox{\dimexpr -0.5ex -0.2ex \relax}[0pt][0pt]{\includesvg[width=14.4px]{\taskGraphicsFolder/graphics/2023-BE-01-inline-kiosk.svg}} gehen.  Dorthin kommt sie jeweils nur von der Bäckerei \raisebox{\dimexpr -0.5ex -0.2ex \relax}[0pt][0pt]{\includesvg[width=14.4px]{\taskGraphicsFolder/graphics/2023-BE-01-inline-baeckerei.svg}} bzw. der Schule \raisebox{\dimexpr -0.5ex -0.2ex \relax}[0pt][0pt]{\includesvg[width=15.2px]{\taskGraphicsFolder/graphics/2023-BE-01-inline-schule.svg}}. Sie muss jeweils die Strecke zwischen diesen Orten hin- und her gehen. Das dauert jeweils ${3 + 3 = 6}$, ingesamt also $12$ Minuten. Das merken wir uns und fassen nun die beiden Orte oben mit denen darunter zu einem zusammen. & \makecell[l]{\includesvg[scale=0.35]{\taskGraphicsFolder/graphics/graph4.svg}} \\ 
  Nun bleibt nur noch der Plan rechts übrig.  Start und Ende des Weges ist hier \raisebox{\dimexpr -0.5ex -0.2ex \relax}[0pt][0pt]{\includesvg[width=11.5px]{\taskGraphicsFolder/graphics/2023-BE-01-inline-meinstandort.svg}}. Diese drei Orte (\raisebox{\dimexpr -0.5ex -0.2ex \relax}[0pt][0pt]{\includesvg[width=14.4px]{\taskGraphicsFolder/graphics/2023-BE-01-inline-baeckerei.svg}} \raisebox{\dimexpr -0.5ex -0.2ex \relax}[0pt][0pt]{\includesvg[width=15.2px]{\taskGraphicsFolder/graphics/2023-BE-01-inline-schule.svg}} \raisebox{\dimexpr -0.5ex -0.2ex \relax}[0pt][0pt]{\includesvg[width=10.8px]{\taskGraphicsFolder/graphics/2023-BE-01-inline-markt.svg}}) müssen besucht werden. Der kürzeste Weg, der das erfüllt, geht über alle fünf Orte und entlang aller Strecken ausser der grauen und dauert ${4 + 6 + 4 + 4 + 6 = 24}$ Minuten. Mit den $12$ Minuten von oben macht das $36$ Minuten. Die vorherigen Überlegungen zeigen, dass es keinen kürzeren Weg geben kann. & \makecell[l]{\includesvg[scale=0.35]{\taskGraphicsFolder/graphics/graph5.svg}}
\end{tabularx}



% it's informatics
\section*{\BrochureItsInformatics}
Für die Begründung der richtigen Antwort wurde eine vereinfachte Darstellung des Plans benutzt.
Es wäre möglich gewesen, den Plan noch deutlich abstrakter darzustellen:

{\centering%
\includesvg[scale=0.35]{\taskGraphicsFolder/graphics/graph1-abstract-compatible.svg}\par}

Diese Darstellung enthält alle für Emmas Weg wichtigen Informationen, nämlich

\begin{itemize}
  \item Objekte: die Orte, wobei die für den Weg wichtigen Orte markiert sind;
  \item und Beziehungen zwischen den Objekten: die Strecken zwischen den Orten, für die jeweils eine Länge angegeben ist.
\end{itemize}

Ein wichtiges Werkzeug zur Modellierung von Beziehungen zwischen Objekten sind \emph{Graphen}.
Graphen bestehen aus Knoten (für die Objekte) und Kanten (Paare von Objekten, für die Beziehungen).
Emmas Plan lässt sich als \emph{gewichteter Graph} modellieren,
bei denen die einzelnen Beziehungen mit Zahlenwerten (den \emph{Gewichten}) versehen werden.

Die Informatik interessiert sich für Fragen, die in Bezug auf Graphen gestellt werden können,
und für Algorithmen, mit denen man die Fragen beantworten kann.
Eine für gewichtete Graphen bedeutsame Frage lautet:
Was ist der kürzeste (oder schnellste) Weg zwischen zwei Knoten?
Die \enquote{Graphen-Frage} in dieser Biberaufgabe ist ähnlich:
Was ist der kürzeste Rundweg von einem Knoten aus, bei dem eine Menge anderer Knoten besucht werden?
Die Informatik kennt viele Algorithmen, die kürzeste Wege in Graphen effizient bestimmen können.
Solche Algorithmen werden zum Beispiel in Software zur Routenplanung implementiert.



% keywords and websites (as \begin{itemize})
\section*{\BrochureWebsitesAndKeywords}
{\raggedright
\begin{itemize}
  \item Weg (in Graphen): \href{https://de.wikipedia.org/wiki/Weg_(Graphentheorie)}{\BrochureUrlText{https://de.wikipedia.org/wiki/Weg\_(Graphentheorie)}}
  \item Kürzeste Wege:  Abenteuer Informatik, Kapitel $1$  (\href{http://abenteuer-informatik.de/dasbuch.html}{\BrochureUrlText{http://abenteuer-informatik.de/dasbuch.html}})
\end{itemize}


}

% end of ifthen for excluding the solutions
}{}

% all authors
% ATTENTION: you HAVE to make sure an according entry is in ../main/authors.tex.
% Syntax: \def\AuthorLastnameF{} (Lastname is last name, F is first letter of first name, this serves as a marker for ../main/authors.tex)
\def\AuthorPohlW{} % \ifdefined\AuthorPohlW \BrochureFlag{de}{} Wolfgang Pohl\fi
\def\AuthorDatzkoThutS{} % \ifdefined\AuthorDatzkoThutS \BrochureFlag{de}{} Susanne Datzko-Thut\fi
\def\AuthorCoolsaetK{} % \ifdefined\AuthorCoolsaetK \BrochureFlag{be}{} Kris Coolsaet\fi

\newpage}{}
