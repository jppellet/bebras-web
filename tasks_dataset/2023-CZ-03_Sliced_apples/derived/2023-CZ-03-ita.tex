\documentclass[a4paper,11pt]{report}
\usepackage[T1]{fontenc}
\usepackage[utf8]{inputenc}

\usepackage[italian]{babel}
\AtBeginDocument{\def\labelitemi{$\bullet$}}

\usepackage{etoolbox}

\usepackage[margin=2cm]{geometry}
\usepackage{changepage}
\makeatletter
\renewenvironment{adjustwidth}[2]{%
    \begin{list}{}{%
    \partopsep\z@%
    \topsep\z@%
    \listparindent\parindent%
    \parsep\parskip%
    \@ifmtarg{#1}{\setlength{\leftmargin}{\z@}}%
                 {\setlength{\leftmargin}{#1}}%
    \@ifmtarg{#2}{\setlength{\rightmargin}{\z@}}%
                 {\setlength{\rightmargin}{#2}}%
    }
    \item[]}{\end{list}}
\makeatother

\newcommand{\BrochureUrlText}[1]{\texttt{#1}}
\usepackage{setspace}
\setstretch{1.15}

\usepackage{tabularx}
\usepackage{booktabs}
\usepackage{makecell}
\usepackage{multirow}
\renewcommand\theadfont{\bfseries}
\renewcommand{\tabularxcolumn}[1]{>{}m{#1}}
\newcolumntype{R}{>{\raggedleft\arraybackslash}X}
\newcolumntype{C}{>{\centering\arraybackslash}X}
\newcolumntype{L}{>{\raggedright\arraybackslash}X}
\newcolumntype{J}{>{\arraybackslash}X}

\newcommand{\BrochureInlineCode}[1]{{\ttfamily #1}}

\usepackage{amssymb}
\usepackage{amsmath}

\usepackage[babel=true,maxlevel=3]{csquotes}
\DeclareQuoteStyle{bebras-ch-eng}{“}[” ]{”}{‘}[”’ ]{’}\DeclareQuoteStyle{bebras-ch-deu}{«}[» ]{»}{“}[»› ]{”}
\DeclareQuoteStyle{bebras-ch-fra}{«\thinspace{}}[» ]{\thinspace{}»}{“}[»\thinspace{}› ]{”}
\DeclareQuoteStyle{bebras-ch-ita}{«}[» ]{»}{“}[»› ]{”}
\setquotestyle{bebras-ch-ita}

\usepackage{hyperref}
\usepackage{graphicx}
\usepackage{svg}
\svgsetup{inkscapeversion=1,inkscapearea=page}
\usepackage{wrapfig}

\usepackage{enumitem}
\setlist{nosep,itemsep=.5ex}

\setlength{\parindent}{0pt}
\setlength{\parskip}{2ex}
\raggedbottom

\usepackage{fancyhdr}
\usepackage{lastpage}
\pagestyle{fancy}

\fancyhf{}
\renewcommand{\headrulewidth}{0pt}
\renewcommand{\footrulewidth}{0.4pt}
\lfoot{\scriptsize © 2023 Bebras (CC BY-SA 4.0)}
\cfoot{\scriptsize\itshape 2023-CZ-03 Dimezzare le mele}
\rfoot{\scriptsize Page~\thepage{}/\pageref*{LastPage}}

\newcommand{\taskGraphicsFolder}{..}

\begin{document}

\section*{\centering{} 2023-CZ-03 Dimezzare le mele}


\subsection*{Body}

Le mele possono essere divise in metà superiore e inferiore.
Alcuni torsoli di mela rimangono nella metà superiore, gli altri in quella inferiore.
Dai fori e dai torsoli della mela si vede che le metà si incastrano:

{\centering%
\includesvg[scale=1]{\taskGraphicsFolder/graphics/2023-CZ-03-taskbody.svg}\par}

Questo è quello che fanno nella Repubblica Ceca a Natale. Gala taglia a metà quattro mele.
Mette le metà superiori e quelle inferiori in due file.

{\em


\subsection*{Question/Challenge - for the brochures}

Quali metà di mele stanno bene insieme? Abbina le metà della mela l’una all’altra.

{\centering%
\includesvg[scale=1]{\taskGraphicsFolder/graphics/2023-CZ-03-question-brochure.svg}\par}

}


\subsection*{Interactivity instruction - for the online challenge}

Traccia delle linee tra le metà della mela che si incastrino tra loro. Al termine, fa clic su \enquote{Salva risposta}.

\begingroup
\renewcommand{\arraystretch}{1.5}
\subsection*{Answer Options/Interactivity Description}

$4$ apple slices are draggables in $4$ containers. The draggables can be dropped in the fields with question mark.

\endgroup

\subsection*{Answer Explanation}

La risposta corretta:

{\centering%
\includesvg[scale=1]{\taskGraphicsFolder/graphics/2023-CZ-03-solution_draglines.svg}\par}

Ogni mela ha $5$ torsoli. Due metà di mele uguali devono avere un totale di $5$ torsoli. Queste metà possono quindi essere facilmente abbinate:

{\centering%
\includesvg[scale=1]{\taskGraphicsFolder/graphics/2023-CZ-03-explanation1.svg}\par}

Le altre quattro metà non sono così facili da assegnare. Le due metà superiori hanno $3$ torsoli ciascuna, mentre le due metà inferiori hanno $2$ torsoli ciascuna. Pertanto, osserviamo più da vicino i modelli dei torsoli delle mele.  Perché quando due metà si uniscono, anche i modelli si uniscono. Per vedere questo, tuttavia, potrebbe essere necessario girare le metà.  A questo punto le metà possono essere disposte come mostrato nell’immagine seguente: a sinistra, la metà superiore ha tre semi direttamente uno dietro l’altro e poi due fori (S-S-S-F-F), la metà inferiore ha tre fori direttamente uno dietro l’altro e poi due semi (F-F-F-S-S): le metà si incastrano. Anche le metà destre si incastrano tra loro: lo schema della metà superiore è (se si parte dall’alto al centro) S-F-S-F-S, quello della metà inferiore è F-S-F-S-F.

{\centering%
\includesvg[scale=1]{\taskGraphicsFolder/graphics/2023-CZ-03-explanation-ita-compatible.svg}\par}


\subsection*{This is Informatics}

Nella spiegazione della risposta corretta, abbiamo visto che se due metà combaciano, non solo i numeri dei semi combaciano, ma anche le sequenze dei semi e dei fori (cioè i compartimenti vuoti del torsolo). Per un corretto abbinamento delle metà, occorre quindi considerare anche queste sequenze. Non è sufficiente sapere quanti semi ci sono in ogni metà.

Domande simili sorgono per i problemi che devono essere risolti con l’aiuto di programmi informatici. Gli informatici devono pensare a come descrivere le informazioni che il programma deve prendere in considerazione come dati. Spesso gli informatici cercando di semplificare questo \enquote{modello} il più possibile.  Dopotutto, i programmi semplici sono meno soggetti a errori. Nel caso del problema delle metà della mela in questo compito, inizialmente sembrava sufficiente descrivere le metà con il solo numero di semi. In seguito si è capito che questo non è sufficiente in tutti i casi. Per descrivere le metà della mela in un programma per computer, deve essere possibile descrivere un ordine. Questo può essere fatto, ad esempio, con l’aiuto della struttura dati \emph{lista}, disponibile nella maggior parte dei linguaggi di programmazione.


\subsection*{This is Computational Thinking}

–


\subsection*{Informatics Keywords and Websites}

\begin{itemize}
  \item Ordine
  \item Lista: \href{https://it.wikipedia.org/wiki/Lista_concatenata}{\BrochureUrlText{https://it.wikipedia.org/wiki/Lista\_concatenata}}
\end{itemize}


\subsection*{Computational Thinking Keywords and Websites}

–


\end{document}
