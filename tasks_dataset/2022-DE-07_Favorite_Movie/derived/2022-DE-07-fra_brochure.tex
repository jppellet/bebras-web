% Definition of the meta information: task difficulties, task ID, task title, task country; definition of the variables as well as their scope is in commands.tex
\setcounter{taskAgeDifficulty3to4}{0}
\setcounter{taskAgeDifficulty5to6}{0}
\setcounter{taskAgeDifficulty7to8}{3}
\setcounter{taskAgeDifficulty9to10}{2}
\setcounter{taskAgeDifficulty11to13}{1}
\renewcommand{\taskTitle}{Soirée ciné}
\renewcommand{\taskCountry}{DE}

% include this task only if for the age groups being processed this task is relevant
\ifthenelse{
  \(\boolean{age3to4} \AND \(\value{taskAgeDifficulty3to4} > 0\)\) \OR
  \(\boolean{age5to6} \AND \(\value{taskAgeDifficulty5to6} > 0\)\) \OR
  \(\boolean{age7to8} \AND \(\value{taskAgeDifficulty7to8} > 0\)\) \OR
  \(\boolean{age9to10} \AND \(\value{taskAgeDifficulty9to10} > 0\)\) \OR
  \(\boolean{age11to13} \AND \(\value{taskAgeDifficulty11to13} > 0\)\)}{

\newchapter{\taskTitle}

% task body
Quelques amis veulent regarder un film ensemble. ils ont le choix entre sept films. Pour prendre une décision, chaque personne évalue chaque film: bon \raisebox{-0.5ex}[0pt][0pt]{\includesvg[width=10.8px]{\taskGraphicsFolder/graphics/2022-DE-07-good.svg}}, moyen \raisebox{-0.5ex}[0pt][0pt]{\includesvg[width=10.8px]{\taskGraphicsFolder/graphics/2022-DE-07-ok.svg}} ou mauvais \raisebox{-0.5ex}[0pt][0pt]{\includesvg[width=10.8px]{\taskGraphicsFolder/graphics/2022-DE-07-bad.svg}}.

Tu vois le résultat ci-dessous. Malheureusement, il n’y a pas de favori pour la soirée.

Un film est un “favori” quand il reçoit de chaque personne la meilleure évaluation parmi les films. Par exemple, le film $1$ n’est pas un favori parce que Niklaus a donné sa meilleure évaluation à un autre film, le film $4$.

Ada aimerait devoir convaincre le moins de personnes possible de changer leurs évaluations pour avoir un favori malgré tout.



% question (as \emph{})
{\em
Aide Ada en changeant le moins d’évaluations possible pour qu’il y ait un favori.

{\centering%
\includesvg[scale=0.2]{\taskGraphicsFolder/graphics/2022-DE-07-taskbody-compatible.svg}\par}


}

% answer alternatives (as \begin{enumerate}[A)]) or interactivity


% from here on this is only included if solutions are processed
\ifthenelse{\boolean{solutions}}{
\newpage

% answer explanation
\section*{\BrochureSolution}
Il n’y a pas de favori au départ. Pour chaque film, nous trouvons des amis qui ont mieux évalué d’autres films.

\begin{tabular}{ @{} l l @{} }
  {\setstretch{1.0}\thead[lb]{Film}} & {\setstretch{1.0}\thead[lb]{Amis qui ont mieux évalué d’autres films}} \\ 
\midrule
  \makecell[l]{\includesvg[scale=0.2]{\taskGraphicsFolder/graphics/2022-DE-07-explanation1.svg}} & $4$: Nancy, Niklaus, Grace et Rozsa \\ 
  \makecell[l]{\includesvg[scale=0.2]{\taskGraphicsFolder/graphics/2022-DE-07-explanation2.svg}} & $3$: Niklaus, Edsger et Rozsa \\ 
  \makecell[l]{\includesvg[scale=0.2]{\taskGraphicsFolder/graphics/2022-DE-07-explanation3.svg}} & $3$: Niklaus, Edsger et Rozsa \\ 
  \makecell[l]{\includesvg[scale=0.2]{\taskGraphicsFolder/graphics/2022-DE-07-explanation4.svg}} & $3$: Nancy, Edsger et Rozsa \\ 
  \makecell[l]{\includesvg[scale=0.2]{\taskGraphicsFolder/graphics/2022-DE-07-explanation5.svg}} & $3$: Nancy, Grace et Edsger \\ 
  \makecell[l]{\includesvg[scale=0.2]{\taskGraphicsFolder/graphics/2022-DE-07-explanation6.svg}} & $2$: Niklaus et Rozsa \\ 
  \makecell[l]{\includesvg[scale=0.2]{\taskGraphicsFolder/graphics/2022-DE-07-explanation7.svg}} & $3$: Niklaus, Grace et Rozsa
\end{tabular}

Il n’y a que deux amis qui ont mieux évalué d’autres films que le film $6$. Tous les autres films ont été moins bien évalués que d’autres par plus de deux des amis. Il ne suffit donc pas qu’un seul ami change son évaluation pour qu’il y ait un favori. Ada doit convaincre Niklaus et Rozsa de faire du film $6$ leur film préféré, et elle obtient alors un favori après deux changements.

Quelles évaluations Niklaus et Rosza peuvent-ils changer afin que le film $6$ ait leur meilleure évaluation? Chacun a trois possibilités:

\begin{itemize}
  \item Niklaus peut changer son évaluation du film $6$ en \raisebox{-0.5ex}[0pt][0pt]{\includesvg[width=10.8px]{\taskGraphicsFolder/graphics/2022-DE-07-ok.svg}} ou \raisebox{-0.5ex}[0pt][0pt]{\includesvg[width=10.8px]{\taskGraphicsFolder/graphics/2022-DE-07-good.svg}}, ou alors changer son évaluation du film $4$ en \raisebox{-0.5ex}[0pt][0pt]{\includesvg[width=10.8px]{\taskGraphicsFolder/graphics/2022-DE-07-bad.svg}}. Dans les trois cas, le film $6$ obtient sa meilleure évaluation.
  \item Rozsa peut changer son évaluation du film $6$ en \raisebox{-0.5ex}[0pt][0pt]{\includesvg[width=10.8px]{\taskGraphicsFolder/graphics/2022-DE-07-good.svg}}, ou alors changer son évaluation du film $5$ en \raisebox{-0.5ex}[0pt][0pt]{\includesvg[width=10.8px]{\taskGraphicsFolder/graphics/2022-DE-07-ok.svg}} ou \raisebox{-0.5ex}[0pt][0pt]{\includesvg[width=10.8px]{\taskGraphicsFolder/graphics/2022-DE-07-bad.svg}}. Dans les trois cas, le film $6$ obtient sa meilleure évaluation.
\end{itemize}

Chacune des trois possibilités de Niklaus et Rosza peuvent être combinées de n’importe quelle manière. Il existe en tout ${3 \times 3 = 9}$ possibilités de ne changer que deux évaluations afin d’obtenir un favori.



% it's informatics
\section*{\BrochureItsInformatics}
Comment procédons-nous pour résoudre cet exercice? Une idée est de vérifier pour chaque film et chaque personne s’il y a un autre film que cette personne a mieux évalué. Dans notre cas, on obtient ainsi la table plus haut. Cette table nous aide à trouver quelles personnes doivent changer leurs évaluations pour que nous obtenions un favori avec le moins de changements possible.

Ada peut utiliser cet algorithme pour résoudre son problème. Mais cet algorithme est-il \emph{efficace}? Ada pourrait-elle résoudre le problème encore plus vite?

Dans ce qui suit, nous dénotons le nombre de film par ${M}$ et le nombre d’amis par ${F}$. Ada doit considérer les ${M \times F}$ entrées individuellement et doit prendre en compte les ${M-1}$ autres évaluations de la même personne pour chaque entrée. Ada doit donc considérer ${M \times (M-1) \times F}$ évaluations en tout.

Pour déterminer si une évaluation est problématique, Ada ne doit la comparer qu’à la meilleure évaluation de la même personne. Si cette personne a mieux évalué un autre film, le film considéré par Ada ne peut pas être un favori.

Autrement dit, Si Ada commence par déterminer la meilleure évaluation de chaque personne (en considérant toutes les ${M \times F}$ évaluations), elle peut déterminer pour chacune des ${M \times F}$ évaluations si elle est moins bonne que la meilleure évaluation de la même personne.

Cet algorithme alternatif qui commence par rechercher la meilleure évaluation permet à Ada de considérer ${2 \times M \times F}$ évaluations en tout. Pour ${M=7}$ et ${F=6}$, il s’agit de $84$ comparaisons, alors que le premier algorithme avait besoin de $252$ comparaisons. Le deuxième algorithme résoud aussi correctement le problème d’Ada, mais beaucoup plus efficacement que le premier.

Une des tâches les plus importantes en informatique est de résoudre des problèmes non seulement de manière correcte, mais aussi le plus efficacement possible. On arrive plus rapidement à des solutions avec des ordinateurs plus rapides, mais si aucun algorithme efficace n’est connu, même les ordinateurs les plus rapides peuvent atteindre leurs limites.



% keywords and websites (as \begin{itemize})
\section*{\BrochureWebsitesAndKeywords}
{\raggedright
\begin{itemize}
  \item Algorithme: \href{https://fr.wikipedia.org/wiki/Algorithme}{\BrochureUrlText{https://fr.wikipedia.org/wiki/Algorithme}}
  \item Complexité (efficacité): \href{https://fr.wikipedia.org/wiki/Analyse_de_la_complexit\%C3\%A9_des_algorithmes}{\BrochureUrlText{https://fr.wikipedia.org/wiki/Analyse\_de\_la\_complexité\_des\_algorithmes}}
\end{itemize}


}

% end of ifthen for excluding the solutions
}{}

% all authors
% ATTENTION: you HAVE to make sure an according entry is in ../main/authors.tex.
% Syntax: \def\AuthorLastnameF{} (Lastname is last name, F is first letter of first name, this serves as a marker for ../main/authors.tex)
\def\AuthorPohlW{} % \ifdefined\AuthorPohlW \BrochureFlag{de}{} Wolfgang Pohl\fi
\def\AuthorSukovicG{} % \ifdefined\AuthorSukovicG \BrochureFlag{me}{} Goran Sukovic\fi
\def\AuthorPluharZ{} % \ifdefined\AuthorPluharZ \BrochureFlag{hu}{} Zsuzsa Pluhár\fi
\def\AuthorSerafiniG{} % \ifdefined\AuthorSerafiniG \BrochureFlag{ch}{} Giovanni Serafini\fi
\def\AuthorWeigendM{} % \ifdefined\AuthorWeigendM \BrochureFlag{de}{} Michael Weigend\fi
\def\AuthorDatzkoS{} % \ifdefined\AuthorDatzkoS \BrochureFlag{ch}{} Susanne Datzko\fi
\def\AuthorPelletE{} % \ifdefined\AuthorPelletE \BrochureFlag{ch}{} Elsa Pellet\fi

\newpage}{}
