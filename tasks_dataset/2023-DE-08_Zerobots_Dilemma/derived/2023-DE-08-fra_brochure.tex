% Definition of the meta information: task difficulties, task ID, task title, task country; definition of the variables as well as their scope is in commands.tex
\setcounter{taskAgeDifficulty3to4}{0}
\setcounter{taskAgeDifficulty5to6}{0}
\setcounter{taskAgeDifficulty7to8}{4}
\setcounter{taskAgeDifficulty9to10}{3}
\setcounter{taskAgeDifficulty11to13}{2}
\renewcommand{\taskTitle}{La mission de Zérobot}
\renewcommand{\taskCountry}{DE}

% include this task only if for the age groups being processed this task is relevant
\ifthenelse{
  \(\boolean{age3to4} \AND \(\value{taskAgeDifficulty3to4} > 0\)\) \OR
  \(\boolean{age5to6} \AND \(\value{taskAgeDifficulty5to6} > 0\)\) \OR
  \(\boolean{age7to8} \AND \(\value{taskAgeDifficulty7to8} > 0\)\) \OR
  \(\boolean{age9to10} \AND \(\value{taskAgeDifficulty9to10} > 0\)\) \OR
  \(\boolean{age11to13} \AND \(\value{taskAgeDifficulty11to13} > 0\)\)}{

\newchapter{\taskTitle}

% task body
Zérobot a un réservoir de carburant échangeable. Il se déplace dans une grille: vers le haut, le bas, la gauche et la droite. Le niveau de son réservoir baisse de $1$ à chaque déplacement d’une case.

Il y a des réservoirs de rechange sur certaines cases; le chiffre écrit dessus indique leur niveau de carburant. Lorsque Zérobot arrive sur une de ces cases, il change son réservoir indépendemment du niveau de carburant de celui-ci. Il prend le réservoir de rechange, dépose son réservoir précédent sur la case et continue sa route.

La position de Zérobot et le niveau de son réservoir sont représentés comme cela sur l’image: \raisebox{-0.5ex}[0pt][0pt]{\includesvg[scale=0.6]{\taskGraphicsFolder/graphics/2023-DE-08a-zerobot_9_compatible.svg}}

{\centering%
\includesvg[scale=0.6]{\taskGraphicsFolder/graphics/2023-DE-08a-challenge_compatible.svg}\par}

Alarme: les réservoirs sont défectueux et pourraient exploser!

Voici la mission de Zérobot: il doit aller à la station de base \raisebox{-0.5ex}[0pt][0pt]{\includesvg[scale=0.6]{\taskGraphicsFolder/graphics/2023-DE-08-basisstation.svg}} en vidant tous les réservoirs (niveau $0$).



% question (as \emph{})
{\em
Comment Zérobot doit-il se déplacer pour accomplir sa mission?


}

% answer alternatives (as \begin{enumerate}[A)]) or interactivity


% from here on this is only included if solutions are processed
\ifthenelse{\boolean{solutions}}{
\newpage

% answer explanation
\section*{\BrochureSolution}
Voici la bonne réponse:

{\centering%
\includesvg[scale=0.6]{\taskGraphicsFolder/graphics/2023-DE-08-solution.svg}\par}

Zérobot peut rouler jusqu’à la station de base en $15$ déplacements de manière à ce que tous les réservoirs aient un niveau de carburant~$0$:

{\centering%
\includesvg[width=1\linewidth]{\taskGraphicsFolder/graphics/2023-DE-08-explanation_steps_compatible.svg}\par}

Pour pouvoir expliquer la bonne réponse plus facilement, nous nommons les cases contenant les réservoirs et la station de base A, B et C, respectivement.

Zérobot se déplace de trois cases jusqu’à la case~A et échange \raisebox{-0.5ex}[0pt][0pt]{\includesvg[width=13px]{\taskGraphicsFolder/graphics/2023-DE-08-tank0-withoutdiget.svg}} (niveau~$6$) contre \raisebox{-0.5ex}[0pt][0pt]{\includesvg[width=13px]{\taskGraphicsFolder/graphics/2023-DE-08-tank1-withoutdiget.svg}}~(niveau~$3$). Il se déplace ensuite de trois cases jusqu’à la case~B et échange \raisebox{-0.5ex}[0pt][0pt]{\includesvg[width=13px]{\taskGraphicsFolder/graphics/2023-DE-08-tank1-withoutdiget.svg}}~(niveau~$0$) contre \raisebox{-0.5ex}[0pt][0pt]{\includesvg[width=13px]{\taskGraphicsFolder/graphics/2023-DE-08-tank2-withoutdiget.svg}}~(niveau~$3$). Il se déplace ensuite à nouveau jusqu’à la case~A et échange \raisebox{-0.5ex}[0pt][0pt]{\includesvg[width=13px]{\taskGraphicsFolder/graphics/2023-DE-08-tank2-withoutdiget.svg}}~(niveau~$0$) contre \raisebox{-0.5ex}[0pt][0pt]{\includesvg[width=13px]{\taskGraphicsFolder/graphics/2023-DE-08-tank0-withoutdiget.svg}}~(niveau~$6$). Il se déplace ensuite de $6$~cases jusqu’à la station de base~C. \raisebox{-0.5ex}[0pt][0pt]{\includesvg[width=13px]{\taskGraphicsFolder/graphics/2023-DE-08-tank0-withoutdiget.svg}}~a ensuite le niveau de carburant~$0$. Mission accomplie!

Est-ce la seule bonne réponse? Zérobot doit effectuer exactement $15$~déplacements: $15$~déplacements sont nécessaires pour utiliser tout le carburant disponible, soit ${9 + 3 + 3 = 15}$~unités. Il n’y a pas assez de carburant pour plus de déplacements. Pour vider tous les réservoirs, Zérobot doit passer par les deux cases contenant les réservoirs de rechange, et même deux fois par la case~A. Si Zérobot commençait par passer par la case~B, il aurait besoin de $17$~déplacements pour atteindre la station de base, ce qui n’est pas possible. La réponse ci-dessus est donc la seule solution possible.



% it's informatics
\section*{\BrochureItsInformatics}
Cet exercice du Castor aborde des problèmes fondamentaux de la mobilité autonome: chaque robot mobile autonome (comme une voiture autonome) doit considérer quelle quantité d’énergie, sous forme de carburant ou de charge des batteries, il a à disposition lorsqu’il planifie ses activités. D’un côté, il doit s’assurer d’atteindre une station service ou une station de charge avant d’avoir épuisé ses réserves d’énergie; d’un autres côté, il y a certaines conditions à respecter. Dans cet exercice, la condition est que tout le carburant doit être utilisé à la fin de la mission. En réalité, les conditions sont plutôt liées à la position et disponibilité de stations de charge. Les logiciels qui dirigent les robots mobiles contiennent des éléments qui assurent un niveau d’énergie suffisant par la recharge (systèmes de contrôle de batterie intelligents).

En plus de cela, des programmes informatiques sont utilisés pour planifier et gérer des réseaux de recharge efficaces. Les informaticiens et informaticiennes étudient le problème du placement des stations de charge: les stations de charge pour les robots mobiles doivent être placées de manière à ce qu’un robot ayant une certaine charge minimale puisse atteindre une des stations de charge disponible. Des protocoles de communication entre les stations de charge et les voitures autonomes comme l’OCPP (\emph{Open Charge Point Protocol}) ont été dévelopés.



% keywords and websites (as \begin{itemize})
\section*{\BrochureWebsitesAndKeywords}
{\raggedright
\begin{itemize}
  \item Véhicule autonome: \href{https://fr.wikipedia.org/wiki/V\%C3\%A9hicule_autonome}{\BrochureUrlText{https://fr.wikipedia.org/wiki/Véhicule\_autonome}}
  \item Station de recharge: \href{https://fr.wikipedia.org/wiki/Station_de_recharge\#Infrastructures_de_recharge}{\BrochureUrlText{https://fr.wikipedia.org/wiki/Station\_de\_recharge\#Infrastructures\_de\_recharge}}
\end{itemize}


}

% end of ifthen for excluding the solutions
}{}

% all authors
% ATTENTION: you HAVE to make sure an according entry is in ../main/authors.tex.
% Syntax: \def\AuthorLastnameF{} (Lastname is last name, F is first letter of first name, this serves as a marker for ../main/authors.tex)
\def\AuthorPohlW{} % \ifdefined\AuthorPohlW \BrochureFlag{de}{} Wolfgang Pohl\fi
\def\AuthorCavalcanteL{} % \ifdefined\AuthorCavalcanteL \BrochureFlag{br}{} Leonardo Cavalcante\fi
\def\AuthorShahV{} % \ifdefined\AuthorShahV \BrochureFlag{in}{} Vipul Shah\fi
\def\AuthorEscherleN{} % \ifdefined\AuthorEscherleN \BrochureFlag{ch}{} Nora A.~Escherle\fi
\def\AuthorM{} % \ifdefined\AuthorM \BrochureFlag{de}{} Michael Weigend \fi
\def\AuthorDatzkoThutS{} % \ifdefined\AuthorDatzkoThutS \BrochureFlag{de}{} Susanne Datzko-Thut\fi
\def\AuthorPelletE{} % \ifdefined\AuthorPelletE \BrochureFlag{ch}{} Elsa Pellet\fi

\newpage}{}
