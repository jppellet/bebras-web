% Definition of the meta information: task difficulties, task ID, task title, task country; definition of the variables as well as their scope is in commands.tex
\setcounter{taskAgeDifficulty3to4}{0}
\setcounter{taskAgeDifficulty5to6}{0}
\setcounter{taskAgeDifficulty7to8}{1}
\setcounter{taskAgeDifficulty9to10}{0}
\setcounter{taskAgeDifficulty11to13}{0}
\renewcommand{\taskTitle}{Pirat Biberbart}
\renewcommand{\taskCountry}{UY}

% include this task only if for the age groups being processed this task is relevant
\ifthenelse{
  \(\boolean{age3to4} \AND \(\value{taskAgeDifficulty3to4} > 0\)\) \OR
  \(\boolean{age5to6} \AND \(\value{taskAgeDifficulty5to6} > 0\)\) \OR
  \(\boolean{age7to8} \AND \(\value{taskAgeDifficulty7to8} > 0\)\) \OR
  \(\boolean{age9to10} \AND \(\value{taskAgeDifficulty9to10} > 0\)\) \OR
  \(\boolean{age11to13} \AND \(\value{taskAgeDifficulty11to13} > 0\)\)}{

\newchapter{\taskTitle}

% task body
Auf einer Insel gibt es drei Schatzkisten:
Eine Kiste ist am Fuss des Vulkans, die zweite ist unter einer Palme, und die dritte ist am Strand.
Alle Kisten sind leer.

{\centering%
\includesvg[width=360.8px]{\taskGraphicsFolder/graphics/2023-UY-01-question.svg}\par}

An einem Tag kreuzt der Pirat Biberbart auf, füllt eine der Kisten mit Gold und verschliesst sie.
Am gleichen Tag sind drei Touristinnen auf der Insel: Anita, Britta und Carla.
Jede macht ein Foto: eine, bevor Biberbart Gold in eine Kiste gefüllt hat, die anderen beiden danach.

\begin{tabularx}{\columnwidth}{ @{} J J J @{} }
  {\setstretch{1.0}\thead[lb]{Anitas Foto}} & {\setstretch{1.0}\thead[lb]{Brittas Foto}} & {\setstretch{1.0}\thead[lb]{Carlas Foto}} \\ 
\midrule
  \makecell[l]{… zeigt die Kiste am Strand. \\ ${~}$ \\ ${~}$} & \makecell[l]{… zeigt die zwei Kisten unter \\ der Palme und am Strand. \\ ${~}$} & \makecell[l]{… zeigt die zwei Kisten unter \\ der Palme und am Fuss des \\ Vulkans.} \\ 
  \makecell[l]{\includesvg[scale=0.35]{\taskGraphicsFolder/graphics/2023-UY-01_AnitaPhoto.svg}} & \makecell[l]{\includesvg[scale=0.35]{\taskGraphicsFolder/graphics/2023-UY-01_BrittaPhoto.svg}} & \makecell[l]{\includesvg[scale=0.35]{\taskGraphicsFolder/graphics/2023-UY-01_CarlaPhoto.svg}}
\end{tabularx}

Auf den Fotos sind alle Kisten leer. Biberbart hatte also Glück, dass keine Touristin sein Gold gefunden hat.



% question (as \emph{})
{\em
In welcher Schatzkiste ist das Gold?


}

% answer alternatives (as \begin{enumerate}[A)]) or interactivity


% from here on this is only included if solutions are processed
\ifthenelse{\boolean{solutions}}{
\newpage

% answer explanation
\section*{\BrochureSolution}
So ist es richtig:

{\centering%
\includesvg[width=360.8px]{\taskGraphicsFolder/graphics/2023-UY-01-solution.svg}\par}

Das Gold ist in der Schatzkiste am Fuss des Vulkans.

Wir prüfen für jede Kiste, ob darin das Gold sein kann.  Dazu untersuchen wir jeweils, ob in diesem Fall die Fotos mit der Geschichte übereinstimmen.

\begin{enumerate}
  \item \textbf{Die Kiste unter der Palme.}
Brittas und Carlas Fotos zeigen die leere Schatzkiste unter der Palme. Wäre dies die Kiste mit dem Gold, müssten beide Fotos gemacht worden sein, bevor die Kiste gefüllt wurde. Wir wissen aber, dass nur eine Touristin ihr Foto gemacht hat, bevor Biberbart Gold in eine Kiste gefüllt hat. Die Annahme, dass das Gold in der Kiste unter der Palme ist, ergibt also einen Widerspruch. Daraus schliessen wir, dass in der Kiste unter der Palme kein Gold ist.
  \item \textbf{Die Kiste am Strand.}
Auf Anitas und Brittas Fotos ist die Schatzkiste am Strand leer. Wäre dies die Kiste mit dem Gold, müssten beide Fotos gemacht worden sein, bevor die Kiste gefüllt wurde. Wir wissen aber, dass nur eine Touristin ihr Foto gemacht hat, bevor Biberbart Gold in eine Kiste gefüllt hat. Die Annahme, dass das Gold in der Kiste am Strand ist, ergibt also einen Widerspruch. Daraus schliessen wir, dass in der Kiste am Strand kein Gold ist.
  \item \textbf{Die Kiste am Fuß des Vulkans}
… ist nur auf Carlas Foto und ist dort leer. Wäre dies die Kiste mit dem Gold, könnte Carla die Touristin sein, die ihr Foto gemacht hat, bevor Biberbart Gold in eine Kiste gefüllt hat. Auf Anitas und Brittas Fotos ist die Kiste am Fuß des Vulkans nicht zu sehen.  Anita und Britta können also die Touristinnen sein, die ihre Fotos danach gemacht haben.  Die Annahme, dass das Gold in der Kiste am Fuß des Vulkans ist, ergibt \emph{keinen} Widerspruch.
\end{enumerate}

Da sich das Gold in einer der drei Kisten befinden muss, können wir insgesamt schlussfolgern, dass das Gold tatsächlich in der Kiste am Fuß des Vulkans ist.
$3$. \textbf{Die Kiste am Fuss des Vulkans}
… ist nur auf Carlas Foto und ist dort leer. Wäre dies die Kiste mit dem Gold, könnte Carla die Touristin sein, die ihr Foto gemacht hat, bevor Biberbart Gold in eine Kiste gefüllt hat. Auf Anitas und Brittas Fotos ist die Kiste am Fuss des Vulkans nicht zu sehen.  Anita und Britta können also die Touristinnen sein, die ihre Fotos danach gemacht haben.  Die Annahme, dass das Gold in der Kiste am Fuss des Vulkans ist, ergibt \emph{keinen} Widerspruch.

Da sich das Gold in einer der drei Kisten befinden muss, können wir insgesamt schlussfolgern, dass das Gold tatsächlich in der Kiste am Fuss des Vulkans ist.



% it's informatics
\section*{\BrochureItsInformatics}
Beim Beantworten dieser Biberaufgabe hat \emph{logisches Schlussfolgern} geholfen.
Wir haben die drei Fotos und unser Wissen über die Situation auf der Insel verwendet, um Gründe zu finden, wieso bestimmte Annahmen zutreffen könnten oder eben nicht.
Widersprüche zu konstruieren spielt beim logischen Schlussfolgern eine besonders wichtige Rolle.
Wenn aus einer Annahme rein logisch eine Aussage folgt, aber Annahme und Aussage nicht gleichzeitig wahr sein können, dann können wir mit Sicherheit sagen, dass die Annahme falsch ist.

Logik spielt eine überaus wichtige Rolle in vielen Bereichen der Informatik: Die Schaltungen in der Computer-Hardware, ob in den Recheneinheiten oder in Speichermedien, sind Realisierungen von logischen Operationen.  Mit logischen Verknüpfungen können komplexe Bedingungen in Programmen oder komplexe Abfragen an Datenbanken formuliert werden.  Das Verhalten von Programmen kann mit Hilfe logischer Kalküle beschrieben und verifiziert werden.  Und \emph{logische Programmiersprachen} arbeiten direkt mit logischen Aussagen und logischem Schlussfolgern, um Berechnungen durchzuführen.



% keywords and websites (as \begin{itemize})
\section*{\BrochureWebsitesAndKeywords}
{\raggedright
\begin{itemize}
  \item Logisches Schlussfolgern: \href{https://de.wikipedia.org/wiki/Schlussfolgerung}{\BrochureUrlText{https://de.wikipedia.org/wiki/Schlussfolgerung}}
  \item Logische Programmierung: \href{https://de.wikipedia.org/wiki/Logische_Programmierung}{\BrochureUrlText{https://de.wikipedia.org/wiki/Logische\_Programmierung}}
  \item Prolog: \href{https://de.wikipedia.org/wiki/Prolog_(Programmiersprache)}{\BrochureUrlText{https://de.wikipedia.org/wiki/Prolog\_(Programmiersprache)}}
\end{itemize}


}

% end of ifthen for excluding the solutions
}{}

% all authors
% ATTENTION: you HAVE to make sure an according entry is in ../main/authors.tex.
% Syntax: \def\AuthorLastnameF{} (Lastname is last name, F is first letter of first name, this serves as a marker for ../main/authors.tex)
\def\AuthorEscherleN{} % \ifdefined\AuthorEscherleN \BrochureFlag{ch}{} Nora A.~Escherle\fi
\def\AuthorPluharZ{} % \ifdefined\AuthorPluharZ \BrochureFlag{hu}{} Zsuzsa Pluhár\fi
\def\AuthorDatzkoC{} % \ifdefined\AuthorDatzkoC \BrochureFlag{hu}{} Christian Datzko\fi
\def\AuthorWettsteinM{} % \ifdefined\AuthorWettsteinM \BrochureFlag{ch}{} Manuel Wettstein\fi
\def\AuthorSerafiniG{} % \ifdefined\AuthorSerafiniG \BrochureFlag{ch}{} Giovanni Serafini\fi
\def\AuthorSeulkiK{} % \ifdefined\AuthorSeulkiK \BrochureFlag{kr}{} Kim Seulki\fi
\def\AuthorHironM{} % \ifdefined\AuthorHironM \BrochureFlag{fr}{} Mathias Hiron\fi
\def\AuthorWillekesK{} % \ifdefined\AuthorWillekesK \BrochureFlag{nl}{} Kyra Willekes\fi
\def\AuthorChanS{} % \ifdefined\AuthorChanS \BrochureFlag{ca}{} Sarah Chan\fi
\def\AuthorCrivelliL{} % \ifdefined\AuthorCrivelliL \BrochureFlag{uy}{} Lucia Crivelli\fi
\def\AuthorCuriM{} % \ifdefined\AuthorCuriM \BrochureFlag{uy}{} María Eugenia Curi\fi
\def\AuthorGerosaA{} % \ifdefined\AuthorGerosaA \BrochureFlag{uy}{} Anaclara Gerosa\fi
\def\AuthorKoleszarV{} % \ifdefined\AuthorKoleszarV \BrochureFlag{uy}{} Víctor Koleszar\fi
\def\AuthorSchunkR{} % \ifdefined\AuthorSchunkR \BrochureFlag{uy}{} Rosario Schunk\fi
\def\AuthorOyhenardG{} % \ifdefined\AuthorOyhenardG \BrochureFlag{uy}{} Graciela Oyhenard\fi
\def\AuthorPohlW{} % \ifdefined\AuthorPohlW \BrochureFlag{de}{} Wolfgang Pohl\fi
\def\AuthorDatzkoThutS{} % \ifdefined\AuthorDatzkoThutS \BrochureFlag{de}{} Susanne Datzko-Thut\fi

\newpage}{}
