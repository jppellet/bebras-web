% Definition of the meta information: task difficulties, task ID, task title, task country; definition of the variables as well as their scope is in commands.tex
\setcounter{taskAgeDifficulty3to4}{0}
\setcounter{taskAgeDifficulty5to6}{1}
\setcounter{taskAgeDifficulty7to8}{0}
\setcounter{taskAgeDifficulty9to10}{0}
\setcounter{taskAgeDifficulty11to13}{0}
\renewcommand{\taskTitle}{Foto}
\renewcommand{\taskCountry}{LT}

% include this task only if for the age groups being processed this task is relevant
\ifthenelse{
  \(\boolean{age3to4} \AND \(\value{taskAgeDifficulty3to4} > 0\)\) \OR
  \(\boolean{age5to6} \AND \(\value{taskAgeDifficulty5to6} > 0\)\) \OR
  \(\boolean{age7to8} \AND \(\value{taskAgeDifficulty7to8} > 0\)\) \OR
  \(\boolean{age9to10} \AND \(\value{taskAgeDifficulty9to10} > 0\)\) \OR
  \(\boolean{age11to13} \AND \(\value{taskAgeDifficulty11to13} > 0\)\)}{

\newchapter{\taskTitle}

% task body
{\centering%
\includesvg[width=180.4px]{\taskGraphicsFolder/graphics/2023-LT-02-task.svg}\par}

Il castoro ha scattato una foto.



% question (as \emph{})
{\em
Di quale delle quattro foto si tratta?


}

% answer alternatives (as \begin{enumerate}[A)]) or interactivity
\begin{tabular}{ @{} c c c c @{} }
  \makecell[c]{\includesvg[width=108.2px]{\taskGraphicsFolder/graphics/2023-LT-02asw-A.svg}} & \makecell[c]{\includesvg[width=108.2px]{\taskGraphicsFolder/graphics/2023-LT-02asw-B.svg}} & \makecell[c]{\includesvg[width=108.2px]{\taskGraphicsFolder/graphics/2023-LT-02asw-C.svg}} & \makecell[c]{\includesvg[width=108.2px]{\taskGraphicsFolder/graphics/2023-LT-02asw-D.svg}} \\ 
  A) & B) & C) & D)
\end{tabular}



% from here on this is only included if solutions are processed
\ifthenelse{\boolean{solutions}}{
\newpage

% answer explanation
\section*{\BrochureSolution}
La risposta corretta è D). \raisebox{-0.5ex}{\includesvg[width=108.2px]{\taskGraphicsFolder/graphics/2023-LT-02asw-D.svg}}

I tronchi d’albero fotografati dal castoro sono disposti in cerchio. Per scoprire quale foto è quella giusta, osserviamo l’ordine dei tronchi in questa disposizione. Scegliamo un tronco (ad esempio quello appuntito) e gli diamo il numero $1$. Poi determiniamo quale tronco si trova a sinistra e gli diamo il numero $2$. Procediamo così finché tutti i tronchi hanno un numero. Nella situazione fotografata dal castoro, i tronchi sono in quest’ordine: $1$ (tronco appuntito) - $2$ (tronco marrone con foglie) - $3$ (tronco di betulla) - $4$ (tronco marrone spesso).

{\centering%
\includesvg[scale=0.16]{\taskGraphicsFolder/graphics/2023-LT-02-explanation.svg}\par}

Ora guardiamo la sequenza dei tronchi nelle foto da A a D. Iniziamo come sopra con il tronco affilato $1$ e andiamo sempre verso sinistra:

\begin{itemize}
  \item Foto A: $1$ – $3$ – $2$ – 4
  \item Foto B: $1$ – $4$ – $3$ – 2
  \item Foto C: $1$ – $3$ – $4$ – 2
  \item Foto D: $1$ – $2$ – $3$ – 4
\end{itemize}

Solo la foto D mostra l’ordine corretto.

{\centering%
\includesvg[scale=0.16]{\taskGraphicsFolder/graphics/2023-LT-02-explanationD.svg}\par}



% it's informatics
\section*{\BrochureItsInformatics}
In questo compito, si considera l’ordine dei tronchi d’albero. Ciò che è possibile con pochi \emph{elementi} (qui quattro tronchi d’albero) semplicemente \enquote{guardando} e confrontando le coppie vicine, richiede una procedura automatizzata per problemi con molti più elementi. In un programma informatico che deve elaborare elementi vicini, gli elementi possono essere memorizzati in una struttura di dati adatta, come una lista concatenata:

{\centering%
\includesvg[width=360.8px]{\taskGraphicsFolder/graphics/2023-LT-02-linkedlist.svg}\par}

In una \emph{lista concatenata}, ogni dato è memorizzato in un singolo nodo. Inoltre, in ogni nodo è memorizzato un \emph{riferimento} al nodo successivo della lista. Se l’ultimo nodo contiene un riferimento al primo nodo, si tratta di una struttura di dati anulare. Questo è importante nell’esempio, in modo da poter iniziare da qualsiasi tronco d’albero e scorrere la lista.



% keywords and websites (as \begin{itemize})
\section*{\BrochureWebsitesAndKeywords}
{\raggedright
\begin{itemize}
  \item Lista concatenata: \href{https://it.wikipedia.org/wiki/Lista_concatenata}{\BrochureUrlText{https://it.wikipedia.org/wiki/Lista\_concatenata}}
\end{itemize}


}

% end of ifthen for excluding the solutions
}{}

% all authors
% ATTENTION: you HAVE to make sure an according entry is in ../main/authors.tex.
% Syntax: \def\AuthorLastnameF{} (Lastname is last name, F is first letter of first name, this serves as a marker for ../main/authors.tex)
\def\AuthorDagieneV{} % \ifdefined\AuthorDagieneV \BrochureFlag{lt}{} Valentina Dagienė\fi
\def\AuthorKinciusV{} % \ifdefined\AuthorKinciusV \BrochureFlag{lt}{} Vaidotas Kinčius\fi
\def\AuthorPohlW{} % \ifdefined\AuthorPohlW \BrochureFlag{de}{} Wolfgang Pohl\fi
\def\AuthorBaumannL{} % \ifdefined\AuthorBaumannL \BrochureFlag{at}{} Liam Baumann\fi
\def\AuthorSchluterK{} % \ifdefined\AuthorSchluterK \BrochureFlag{de}{} Kirsten Schlüter\fi
\def\AuthorDatzkoThutS{} % \ifdefined\AuthorDatzkoThutS \BrochureFlag{de}{} Susanne Datzko-Thut\fi
\def\AuthorGiangC{} % \ifdefined\AuthorGiangC \BrochureFlag{ch}{} Christian Giang\fi

\newpage}{}
