% Definition of the meta information: task difficulties, task ID, task title, task country; definition of the variables as well as their scope is in commands.tex
\setcounter{taskAgeDifficulty3to4}{0}
\setcounter{taskAgeDifficulty5to6}{1}
\setcounter{taskAgeDifficulty7to8}{0}
\setcounter{taskAgeDifficulty9to10}{0}
\setcounter{taskAgeDifficulty11to13}{0}
\renewcommand{\taskTitle}{Photo}
\renewcommand{\taskCountry}{LT}

% include this task only if for the age groups being processed this task is relevant
\ifthenelse{
  \(\boolean{age3to4} \AND \(\value{taskAgeDifficulty3to4} > 0\)\) \OR
  \(\boolean{age5to6} \AND \(\value{taskAgeDifficulty5to6} > 0\)\) \OR
  \(\boolean{age7to8} \AND \(\value{taskAgeDifficulty7to8} > 0\)\) \OR
  \(\boolean{age9to10} \AND \(\value{taskAgeDifficulty9to10} > 0\)\) \OR
  \(\boolean{age11to13} \AND \(\value{taskAgeDifficulty11to13} > 0\)\)}{

\newchapter{\taskTitle}

% task body
{\centering%
\includesvg[width=180.4px]{\taskGraphicsFolder/graphics/2023-LT-02-task.svg}\par}

Le castor vient de prendre une photo.



% question (as \emph{})
{\em
Laquelle des quatre photos a-t-il prise?


}

% answer alternatives (as \begin{enumerate}[A)]) or interactivity
\begin{tabular}{ @{} c c c c @{} }
  \makecell[c]{\includesvg[width=108.2px]{\taskGraphicsFolder/graphics/2023-LT-02asw-A.svg}} & \makecell[c]{\includesvg[width=108.2px]{\taskGraphicsFolder/graphics/2023-LT-02asw-B.svg}} & \makecell[c]{\includesvg[width=108.2px]{\taskGraphicsFolder/graphics/2023-LT-02asw-C.svg}} & \makecell[c]{\includesvg[width=108.2px]{\taskGraphicsFolder/graphics/2023-LT-02asw-D.svg}} \\ 
  A) & B) & C) & D)
\end{tabular}



% from here on this is only included if solutions are processed
\ifthenelse{\boolean{solutions}}{
\newpage

% answer explanation
\section*{\BrochureSolution}
La bonne réponse est D). \raisebox{-0.5ex}{\includesvg[width=108.2px]{\taskGraphicsFolder/graphics/2023-LT-02asw-D.svg}}

Les troncs que le castor a photographiés sont arrangés en rond. Pour trouver quelle photo est la bonne, nous considérons l’ordre des troncs dans cet arrangemet. Nous choisissons un tronc (par exemple le tronc pointu) et lui donnons le numéro $1$. Nous regardons ensuite quel tronc se trouve à sa droite et lui donnons le numéro $2$. Nous continuons ainsi jusqu’à ce que chaque tronc ait un numéro. Dans la situation photographiée par le castor, les troncs sont arrangés dans l’ordre $1$ – tronc pointu, $2$ – tronc brun avec des feuilles, $3$ – tronc de bouleau, $4$ – gros tronc brunc.

{\centering%
\includesvg[scale=0.16]{\taskGraphicsFolder/graphics/2023-LT-02-explanation.svg}\par}

Nous regardons maintenant l’ordre des troncs sur les photos A à D. Comme plus haut, nous commençons par le tronc pointu numéro $1$ et allons vers la droite, dans le sens des aiguilles d’une montre:

\begin{itemize}
  \item Photo A: $1$ – $3$ – $2$ – 4
  \item Photo B: $1$ – $4$ – $3$ – 2
  \item Photo C: $1$ – $3$ – $4$ – 2
  \item Photo D: $1$ – $2$ – $3$ – 4
\end{itemize}

Seule la photo D montre les troncs dans le bon ordre.

{\centering%
\includesvg[scale=0.16]{\taskGraphicsFolder/graphics/2023-LT-02-explanationD.svg}\par}



% it's informatics
\section*{\BrochureItsInformatics}
Dans cet exercice du Castor, nous considérons l’ordre des troncs. Ce qui est visible à l’œil nu avec peu d’\emph{éléments} (ici quatre troncs) nécessite une méthode automatisée pour les problèmes ayant beaucoup d’éléments. Un programme devant traiter des éléments voisins peut utiliser une structure de données adaptée pour stocker les éléments, comme une liste chaînée:

{\centering%
\includesvg[width=360.8px]{\taskGraphicsFolder/graphics/2023-LT-02-linkedlist.svg}\par}

Dans une liste chaînée, chaque élément est stocké dans une cellule différente. En plus, un \emph{pointeur} vers la cellules suivante est aussi stocké dans chaque cellule. Si la dernière cellule contient un pointeur vers la première cellule, il s’agit d’une structure de données cyclique. C’est important dans notre exemple pour pouvoir commencer par n’importe quel tronc tout en parcourant la liste entière.



% keywords and websites (as \begin{itemize})
\section*{\BrochureWebsitesAndKeywords}
{\raggedright
\begin{itemize}
  \item Liste chaînée: \href{https://fr.wikipedia.org/wiki/Liste_cha\%C3\%AEn\%C3\%A9e}{\BrochureUrlText{https://fr.wikipedia.org/wiki/Liste\_chaînée}}
\end{itemize}


}

% end of ifthen for excluding the solutions
}{}

% all authors
% ATTENTION: you HAVE to make sure an according entry is in ../main/authors.tex.
% Syntax: \def\AuthorLastnameF{} (Lastname is last name, F is first letter of first name, this serves as a marker for ../main/authors.tex)
\def\AuthorDagieneV{} % \ifdefined\AuthorDagieneV \BrochureFlag{lt}{} Valentina Dagienė\fi
\def\AuthorKinciusV{} % \ifdefined\AuthorKinciusV \BrochureFlag{lt}{} Vaidotas Kinčius\fi
\def\AuthorPohlW{} % \ifdefined\AuthorPohlW \BrochureFlag{de}{} Wolfgang Pohl\fi
\def\AuthorBaumannL{} % \ifdefined\AuthorBaumannL \BrochureFlag{at}{} Liam Baumann\fi
\def\AuthorSchluterK{} % \ifdefined\AuthorSchluterK \BrochureFlag{de}{} Kirsten Schlüter\fi
\def\AuthorDatzkoThutS{} % \ifdefined\AuthorDatzkoThutS \BrochureFlag{de}{} Susanne Datzko-Thut\fi
\def\AuthorPelletE{} % \ifdefined\AuthorPelletE \BrochureFlag{ch}{} Elsa Pellet\fi

\newpage}{}
