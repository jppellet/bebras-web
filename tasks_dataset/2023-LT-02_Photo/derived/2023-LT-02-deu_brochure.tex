% Definition of the meta information: task difficulties, task ID, task title, task country; definition of the variables as well as their scope is in commands.tex
\setcounter{taskAgeDifficulty3to4}{0}
\setcounter{taskAgeDifficulty5to6}{1}
\setcounter{taskAgeDifficulty7to8}{0}
\setcounter{taskAgeDifficulty9to10}{0}
\setcounter{taskAgeDifficulty11to13}{0}
\renewcommand{\taskTitle}{Foto}
\renewcommand{\taskCountry}{LT}

% include this task only if for the age groups being processed this task is relevant
\ifthenelse{
  \(\boolean{age3to4} \AND \(\value{taskAgeDifficulty3to4} > 0\)\) \OR
  \(\boolean{age5to6} \AND \(\value{taskAgeDifficulty5to6} > 0\)\) \OR
  \(\boolean{age7to8} \AND \(\value{taskAgeDifficulty7to8} > 0\)\) \OR
  \(\boolean{age9to10} \AND \(\value{taskAgeDifficulty9to10} > 0\)\) \OR
  \(\boolean{age11to13} \AND \(\value{taskAgeDifficulty11to13} > 0\)\)}{

\newchapter{\taskTitle}

% task body
{\centering%
\includesvg[width=180.4px]{\taskGraphicsFolder/graphics/2023-LT-02-task.svg}\par}

Der Biber hat gerade ein Foto gemacht.



% question (as \emph{})
{\em
Welches der vier Fotos ist es?


}

% answer alternatives (as \begin{enumerate}[A)]) or interactivity
\begin{tabular}{ @{} c c c c @{} }
  \makecell[c]{\includesvg[width=108.2px]{\taskGraphicsFolder/graphics/2023-LT-02asw-A.svg}} & \makecell[c]{\includesvg[width=108.2px]{\taskGraphicsFolder/graphics/2023-LT-02asw-B.svg}} & \makecell[c]{\includesvg[width=108.2px]{\taskGraphicsFolder/graphics/2023-LT-02asw-C.svg}} & \makecell[c]{\includesvg[width=108.2px]{\taskGraphicsFolder/graphics/2023-LT-02asw-D.svg}} \\ 
  A) & B) & C) & D)
\end{tabular}



% from here on this is only included if solutions are processed
\ifthenelse{\boolean{solutions}}{
\newpage

% answer explanation
\section*{\BrochureSolution}
Die richtige Antwort ist D). \raisebox{-0.5ex}{\includesvg[width=108.2px]{\taskGraphicsFolder/graphics/2023-LT-02asw-D.svg}}

Die Baumstämme, die der Biber fotografiert hat, sind im Kreis angeordnet. Um herauszufinden, welches Foto das richtige ist, betrachten wir die Reihenfolge der Baumstämme in dieser Anordnung. Wir wählen einen Baumstamm aus (z.B. den angespitzten Baumstamm) und geben ihm die Nummer $1$. Dann bestimmen wir, welcher Baumstamm links daneben ist und geben ihm die Nummer $2$. Das machen wir solange, bis alle Baumstämme eine Nummer haben. In der Situation, die der Biber fotografiert hat, haben die Stämme also diese Reihenfolge: $1$ (angespitzter Stamm) – $2$ (brauner Stamm mit Blättern) – $3$ (Birkenstamm) – $4$ (dicker brauner Stamm).

{\centering%
\includesvg[scale=0.16]{\taskGraphicsFolder/graphics/2023-LT-02-explanation.svg}\par}

Nun betrachten wir die Reihenfolge der Stämme in den Fotos A bis D. Dabei beginnen wir wie oben mit dem angespitzten Baumstamm $1$ und gehen immer nach links:

\begin{itemize}
  \item Foto A: $1$ – $3$ – $2$ – 4
  \item Foto B: $1$ – $4$ – $3$ – 2
  \item Foto C: $1$ – $3$ – $4$ – 2
  \item Foto D: $1$ – $2$ – $3$ – 4
\end{itemize}

Nur Foto D zeigt die richtige Reihenfolge.

{\centering%
\includesvg[scale=0.16]{\taskGraphicsFolder/graphics/2023-LT-02-explanationD.svg}\par}



% it's informatics
\section*{\BrochureItsInformatics}
In dieser Biberaufgabe wird die Reihenfolge der Baumstämme betrachtet. Was bei wenigen \emph{Elementen} (hier vier Baumstämmen) durch einfaches “Hinsehen” und Vergleichen der Nachbarpaare möglich ist, erfordert bei Problemen mit viel mehr Elementen ein automatisiertes Vorgehen. In einem Computerprogramm, das benachbarte Elemente verarbeiten soll, könnten die Elemente in einer geeigneten Datenstruktur wie einer verketteten Liste gespeichert werden:

{\centering%
\includesvg[width=360.8px]{\taskGraphicsFolder/graphics/2023-LT-02-linkedlist.svg}\par}

In einer \emph{verketteten Liste} wird jedes Datenelement in einem einzelnen Knoten gespeichert. Zusätzlich ist in jedem Knoten ein \emph{Verweis} auf den nächsten Knoten in der Liste gespeichert. Enthält der letzte Knoten einen Verweis auf den ersten Knoten, so handelt es sich um eine ringförmige Datenstruktur. Das ist im Beispiel wichtig, damit man bei jedem beliebigen Baumstamm starten und die Liste durchlaufen kann.



% keywords and websites (as \begin{itemize})
\section*{\BrochureWebsitesAndKeywords}
{\raggedright
\begin{itemize}
  \item verkettete Liste: \href{https://de.wikipedia.org/wiki/Liste_(Datenstruktur)}{\BrochureUrlText{https://de.wikipedia.org/wiki/Liste\_(Datenstruktur)}}
\end{itemize}


}

% end of ifthen for excluding the solutions
}{}

% all authors
% ATTENTION: you HAVE to make sure an according entry is in ../main/authors.tex.
% Syntax: \def\AuthorLastnameF{} (Lastname is last name, F is first letter of first name, this serves as a marker for ../main/authors.tex)
\def\AuthorDagieneV{} % \ifdefined\AuthorDagieneV \BrochureFlag{lt}{} Valentina Dagienė\fi
\def\AuthorKinciusV{} % \ifdefined\AuthorKinciusV \BrochureFlag{lt}{} Vaidotas Kinčius\fi
\def\AuthorPohlW{} % \ifdefined\AuthorPohlW \BrochureFlag{de}{} Wolfgang Pohl\fi
\def\AuthorBaumannL{} % \ifdefined\AuthorBaumannL \BrochureFlag{at}{} Liam Baumann\fi
\def\AuthorSchluterK{} % \ifdefined\AuthorSchluterK \BrochureFlag{de}{} Kirsten Schlüter\fi
\def\AuthorDatzkoThutS{} % \ifdefined\AuthorDatzkoThutS \BrochureFlag{de}{} Susanne Datzko-Thut\fi

\newpage}{}
