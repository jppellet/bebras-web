\documentclass[a4paper,11pt]{report}
\usepackage[T1]{fontenc}
\usepackage[utf8]{inputenc}

\usepackage[french]{babel}
\frenchbsetup{ThinColonSpace=true}
\renewcommand*{\FBguillspace}{\hskip .4\fontdimen2\font plus .1\fontdimen3\font minus .3\fontdimen4\font \relax}
\AtBeginDocument{\def\labelitemi{$\bullet$}}

\usepackage{etoolbox}

\usepackage[margin=2cm]{geometry}
\usepackage{changepage}
\makeatletter
\renewenvironment{adjustwidth}[2]{%
    \begin{list}{}{%
    \partopsep\z@%
    \topsep\z@%
    \listparindent\parindent%
    \parsep\parskip%
    \@ifmtarg{#1}{\setlength{\leftmargin}{\z@}}%
                 {\setlength{\leftmargin}{#1}}%
    \@ifmtarg{#2}{\setlength{\rightmargin}{\z@}}%
                 {\setlength{\rightmargin}{#2}}%
    }
    \item[]}{\end{list}}
\makeatother

\newcommand{\BrochureUrlText}[1]{\texttt{#1}}
\usepackage{setspace}
\setstretch{1.15}

\usepackage{tabularx}
\usepackage{booktabs}
\usepackage{makecell}
\usepackage{multirow}
\renewcommand\theadfont{\bfseries}
\renewcommand{\tabularxcolumn}[1]{>{}m{#1}}
\newcolumntype{R}{>{\raggedleft\arraybackslash}X}
\newcolumntype{C}{>{\centering\arraybackslash}X}
\newcolumntype{L}{>{\raggedright\arraybackslash}X}
\newcolumntype{J}{>{\arraybackslash}X}

\newcommand{\BrochureInlineCode}[1]{{\ttfamily #1}}

\usepackage{amssymb}
\usepackage{amsmath}

\usepackage[babel=true,maxlevel=3]{csquotes}
\DeclareQuoteStyle{bebras-ch-eng}{“}[” ]{”}{‘}[”’ ]{’}\DeclareQuoteStyle{bebras-ch-deu}{«}[» ]{»}{“}[»› ]{”}
\DeclareQuoteStyle{bebras-ch-fra}{«\thinspace{}}[» ]{\thinspace{}»}{“}[»\thinspace{}› ]{”}
\DeclareQuoteStyle{bebras-ch-ita}{«}[» ]{»}{“}[»› ]{”}
\setquotestyle{bebras-ch-fra}

\usepackage{hyperref}
\usepackage{graphicx}
\usepackage{svg}
\svgsetup{inkscapeversion=1,inkscapearea=page}
\usepackage{wrapfig}

\usepackage{enumitem}
\setlist{nosep,itemsep=.5ex}

\setlength{\parindent}{0pt}
\setlength{\parskip}{2ex}
\raggedbottom

\usepackage{fancyhdr}
\usepackage{lastpage}
\pagestyle{fancy}

\fancyhf{}
\renewcommand{\headrulewidth}{0pt}
\renewcommand{\footrulewidth}{0.4pt}
\lfoot{\scriptsize © 2021 Bebras (CC BY-SA 4.0)}
\cfoot{\scriptsize\itshape 2021-TR-06 Les moulins de castor Max}
\rfoot{\scriptsize Page~\thepage{}/\pageref*{LastPage}}

\newcommand{\taskGraphicsFolder}{..}

\begin{document}

\section*{\centering{} 2021-TR-06 Les moulins de castor Max}


\subsection*{Body}

Le meunier Max a six moulins. Il doit encore fixer la roue de trois d’entre eux. Pour cela, il doit empêcher l’eau d’arriver à ces moulins. L’eau doit par contre continuer de couler jusqu’aux autres moulins.

L’eau ne peut couler que vers le bas. Un clapet fermé empêche l’eau de couler.

{\em


\subsection*{Question/Challenge - for the brochures}

Quels clapets faut-il fermer?

{\centering%
\includesvg[scale=0.25]{\taskGraphicsFolder/graphics/2021-TR-06-question.svg}\par}

}

\begingroup
\renewcommand{\arraystretch}{1.5}
\subsection*{Answer Options/Interactivity Description}



\endgroup

\subsection*{Answer Explanation}

La bonne réponse est qu’il faut fermer les trois clapets qui sont nommés D, F et H dans le dessin suivant.

{\centering%
\includesvg[scale=0.25]{\taskGraphicsFolder/graphics/2021-TR-06-answer-compatible.svg}\par}

C’est la seule possibilité permettant de couper l’eau aux moulins $2$, $4$ et $5$ tout en continuer d’alimenter en eau les trois moulins $1$, $3$ et $6$:

\begin{itemize}
  \item Les clapets A, G et I doivent rester ouverts, car sinon le moulin $1$ n’est pas alimenté en eau.
  \item Les clapets B et E doivent également rester ouverts pour que le moulin $6$ soit alimenté en eau.
  \item Comme les clapets B et E sont ouverts, le clapet H doit être fermé pour éviter que l’eau n’arrive au moulin $5$.
  \item Comme le clapet A est ouvert, le clapet F doit être fermé pour éviter que l’eau n’arrive au moulin $2$.
  \item Comme le clapet B est ouvert, le clapet D doit être fermé pour éviter que l’eau n’arrive au moulin $4$.
  \item Comme les clapets D et F sont fermés, le clapet C doit être ouvert pour que le moulin $3$ soit alimenté en eau.
\end{itemize}


\subsection*{It’s Informatics}

Dans cet exercice, l’écoulement de l’eau est régulé par des \emph{conditions}. Par exemple, l’eau ne coule jusqu’au moulin $6$ que si les deux clapets B et E sont ouverts. Voici un autre exemple un peu plus complexe: l’eau ne coule jusqu’au moulin $3$ que si au moins l’une des deux ou les deux conditions suivantes sont remplies:

\begin{itemize}
  \item Le clapet A est ouvert et l’un des deux clapets C ou F est ouvert.
  \item Les deux clapets B et D sont ouverts.
\end{itemize}

De telles combinaisons de conditions sont obtenues à l’aide des \emph{opérateurs logiques} ET (symbolisé par ${\wedge}$) ou OU (symbolisé par ${\vee}$). De tels opérateurs connectent des valeurs de vérité comme vrai et faux. Si A et B sont deux valeurs de vérité, on peut indiquer quelles valeurs de vérité les expressions “A ET B” et “A OU B” ont en fonction de A et B:

{\centering%
\begin{tabular}{ @{} c c c c @{} }
  {\setstretch{1.0}\thead[cb]{A}} & {\setstretch{1.0}\thead[cb]{B}} & {\setstretch{1.0}\thead[cb]{A ET B}} & {\setstretch{1.0}\thead[cb]{A OU B}} \\ 
\midrule
  faux & faux & faux & faux \\ 
  vrai & faux & faux & vrai \\ 
  faux & vrai & faux & vrai \\ 
  vrai & vrai & vrai & vrai
\end{tabular}

\par}

En informatique (et en mathématiques), l’expression “A OU B” est donc aussi considérée comme juste si A et B sont les deux justes.
L’affirmation “le moulin $6$ est alimenté” est équivalente à:

\begin{adjustwidth}{1.5em}{0em}
“le clapet B est ouvert” ET “le clapet E est ouvert”.
\end{adjustwidth}

Dans le deuxième exemple, l’affirmation “le moulin $3$ est alimenté” est équivalente à:

\begin{adjustwidth}{1.5em}{0em}
(“le clapet A et ouvert” ET (“le clapet C est ouvert” OU “le clapet F est ouvert”)) OU (“le clapet B est ouvert” ET “le clapet D est ouvert”).
\end{adjustwidth}

En programmation, il est important de formuler les conditions de manière exacte.
Chaque ET et chaque OU combine deux affirmations. Les parenthèses déterminent dans quel ordre les affirmations sont combinées.
Les combinaisons à l’aide d’opérateurs logiques et de parenthèses sont utiles pour formuler des conditions complexes. Des conditions sont utilisées aussi bien pour des branchements avec \BrochureInlineCode{if} que pour des boucles \BrochureInlineCode{while} afin de guider le déroulement d’un programme.

{\raggedright

\subsection*{Keywords and Websites}

\begin{itemize}
  \item Instruction conditionnelle: \href{https://fr.wikipedia.org/wiki/Instruction_conditionnelle_(programmation)}{\BrochureUrlText{https://fr.wikipedia.org/wiki/Instruction\_conditionnelle\_(programmation)}}
  \item Variable booléenne: \href{https://fr.wikipedia.org/wiki/Bool\%C3\%A9en}{\BrochureUrlText{https://fr.wikipedia.org/wiki/Booléen}}
  \item Algèbre de Boole: \href{https://fr.wikipedia.org/wiki/Alg\%C3\%A8bre_de_Boole_(logique)}{\BrochureUrlText{https://fr.wikipedia.org/wiki/Algèbre\_de\_Boole\_(logique)}}
\end{itemize}


}
\end{document}
