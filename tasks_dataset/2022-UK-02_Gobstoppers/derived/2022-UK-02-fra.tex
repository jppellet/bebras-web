\documentclass[a4paper,11pt]{report}
\usepackage[T1]{fontenc}
\usepackage[utf8]{inputenc}

\usepackage[french]{babel}
\frenchbsetup{ThinColonSpace=true}
\renewcommand*{\FBguillspace}{\hskip .4\fontdimen2\font plus .1\fontdimen3\font minus .3\fontdimen4\font \relax}
\AtBeginDocument{\def\labelitemi{$\bullet$}}

\usepackage{etoolbox}

\usepackage[margin=2cm]{geometry}
\usepackage{changepage}
\makeatletter
\renewenvironment{adjustwidth}[2]{%
    \begin{list}{}{%
    \partopsep\z@%
    \topsep\z@%
    \listparindent\parindent%
    \parsep\parskip%
    \@ifmtarg{#1}{\setlength{\leftmargin}{\z@}}%
                 {\setlength{\leftmargin}{#1}}%
    \@ifmtarg{#2}{\setlength{\rightmargin}{\z@}}%
                 {\setlength{\rightmargin}{#2}}%
    }
    \item[]}{\end{list}}
\makeatother

\newcommand{\BrochureUrlText}[1]{\texttt{#1}}
\usepackage{setspace}
\setstretch{1.15}

\usepackage{tabularx}
\usepackage{booktabs}
\usepackage{makecell}
\usepackage{multirow}
\renewcommand\theadfont{\bfseries}
\renewcommand{\tabularxcolumn}[1]{>{}m{#1}}
\newcolumntype{R}{>{\raggedleft\arraybackslash}X}
\newcolumntype{C}{>{\centering\arraybackslash}X}
\newcolumntype{L}{>{\raggedright\arraybackslash}X}
\newcolumntype{J}{>{\arraybackslash}X}

\newcommand{\BrochureInlineCode}[1]{{\ttfamily #1}}

\usepackage{amssymb}
\usepackage{amsmath}

\usepackage[babel=true,maxlevel=3]{csquotes}
\DeclareQuoteStyle{bebras-ch-eng}{“}[” ]{”}{‘}[”’ ]{’}\DeclareQuoteStyle{bebras-ch-deu}{«}[» ]{»}{“}[»› ]{”}
\DeclareQuoteStyle{bebras-ch-fra}{«\thinspace{}}[» ]{\thinspace{}»}{“}[»\thinspace{}› ]{”}
\DeclareQuoteStyle{bebras-ch-ita}{«}[» ]{»}{“}[»› ]{”}
\setquotestyle{bebras-ch-fra}

\usepackage{hyperref}
\usepackage{graphicx}
\usepackage{svg}
\svgsetup{inkscapeversion=1,inkscapearea=page}
\usepackage{wrapfig}

\usepackage{enumitem}
\setlist{nosep,itemsep=.5ex}

\setlength{\parindent}{0pt}
\setlength{\parskip}{2ex}
\raggedbottom

\usepackage{fancyhdr}
\usepackage{lastpage}
\pagestyle{fancy}

\fancyhf{}
\renewcommand{\headrulewidth}{0pt}
\renewcommand{\footrulewidth}{0.4pt}
\lfoot{\scriptsize © 2022 Bebras (CC BY-SA 4.0)}
\cfoot{\scriptsize\itshape 2022-UK-02 Bonbons préférés}
\rfoot{\scriptsize Page~\thepage{}/\pageref*{LastPage}}

\newcommand{\taskGraphicsFolder}{..}

\begin{document}

\section*{\centering{} 2022-UK-02 Bonbons préférés}


\subsection*{Body}

Anna met cinq bonbons dans un distributeur. Elle peut ensuite les manger les uns après les autres dans l’ordre dans lequel ils sortent du haut du distributeur.

Elle aimerait les manger dans cet ordre:

\raisebox{-0.5ex}{\includesvg[scale=0.3]{\taskGraphicsFolder/graphics/2022-UK-02-bonbon_pear.svg}} \raisebox{-0.5ex}{\includesvg[scale=0.3]{\taskGraphicsFolder/graphics/2022-UK-02-bonbon_cherry.svg}} \raisebox{-0.5ex}{\includesvg[scale=0.3]{\taskGraphicsFolder/graphics/2022-UK-02-bonbon_ananas.svg}} \raisebox{-0.5ex}{\includesvg[scale=0.3]{\taskGraphicsFolder/graphics/2022-UK-02-bonbon_grapes.svg}} \raisebox{-0.5ex}{\includesvg[scale=0.3]{\taskGraphicsFolder/graphics/2022-UK-02-bonbon_orange.svg}}

{\em


\subsection*{Question/Challenge - for the brochures}

Dans quel ordre doit-elle les mettre dans le distributeur?

{\centering%
\includesvg[scale=0.3]{\taskGraphicsFolder/graphics/2022-UK-02-question.svg}\par}

}


\subsection*{Interactivity Instructions}

Glisse les bonbons dans le distributeur et clique sur “Enregistrer la réponse” quand tu as fini.

\begingroup
\renewcommand{\arraystretch}{1.5}
\subsection*{Answer Options/Interactivity Description}



\endgroup

\subsection*{Answer Explanation}

Pour que les différents bonbons sortent du distributeur dans le bon ordre, c’est important de comprendre que le bonbon qui y a été mis en premier en sort en dernier. Cela veut dire que le distributeur doit être rempli de la manière suivante:

{\centering%
\includesvg[scale=0.3]{\taskGraphicsFolder/graphics/2022-UK-02-solution.svg}\par}


\subsection*{It’s Informatics}

Si Anna met les bonbons dans le distributeur dans le même ordre que celui dans lequel elle veut les manger, ils en ressortent exactement dans l’ordre inverse. C’est la raison pour laquelle elle décide d’abord dans quel ordre elle veut les manger, puis elle se représente comment elle doit remplir le distributeur pour obtenir les bonbons dans le bon ordre.

Pour les informaticiennes et informaticiens, c’est souvent important de réfléchir à l’ordre des choses, comme pour Anna. L’ordre utilisé dans cet exercice s’appelle “ordre de pile”. Les piles sont des structures de stockage dans lesquelles les objets peuvent être ajoutés et desquelles ils peuvent être sortis dans un ordre précis. Elle fonctionnent d’après le principe \emph{dernier arrivé, premier sorti} (LIFO de l’anglais “Last in, first out”), ce qui signifie que l’objet qui a été ajouté à la pile en dernier doit en être sorti en premier.

{\raggedright

\subsection*{Keywords and Websites}

\begin{itemize}
  \item Pile: \href{https://fr.wikipedia.org/wiki/Pile_(informatique)}{\BrochureUrlText{https://fr.wikipedia.org/wiki/Pile\_(informatique)}}
  \item Dernier arrivé, premier sorti (LIFO): \href{https://fr.wikipedia.org/wiki/Last_in,_first_out}{\BrochureUrlText{https://fr.wikipedia.org/wiki/Last\_in,\_first\_out}}
\end{itemize}


}
\end{document}
