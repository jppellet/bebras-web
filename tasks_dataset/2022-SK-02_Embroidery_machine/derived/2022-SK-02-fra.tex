\documentclass[a4paper,11pt]{report}
\usepackage[T1]{fontenc}
\usepackage[utf8]{inputenc}

\usepackage[french]{babel}
\frenchbsetup{ThinColonSpace=true}
\renewcommand*{\FBguillspace}{\hskip .4\fontdimen2\font plus .1\fontdimen3\font minus .3\fontdimen4\font \relax}
\AtBeginDocument{\def\labelitemi{$\bullet$}}

\usepackage{etoolbox}

\usepackage[margin=2cm]{geometry}
\usepackage{changepage}
\makeatletter
\renewenvironment{adjustwidth}[2]{%
    \begin{list}{}{%
    \partopsep\z@%
    \topsep\z@%
    \listparindent\parindent%
    \parsep\parskip%
    \@ifmtarg{#1}{\setlength{\leftmargin}{\z@}}%
                 {\setlength{\leftmargin}{#1}}%
    \@ifmtarg{#2}{\setlength{\rightmargin}{\z@}}%
                 {\setlength{\rightmargin}{#2}}%
    }
    \item[]}{\end{list}}
\makeatother

\newcommand{\BrochureUrlText}[1]{\texttt{#1}}
\usepackage{setspace}
\setstretch{1.15}

\usepackage{tabularx}
\usepackage{booktabs}
\usepackage{makecell}
\usepackage{multirow}
\renewcommand\theadfont{\bfseries}
\renewcommand{\tabularxcolumn}[1]{>{}m{#1}}
\newcolumntype{R}{>{\raggedleft\arraybackslash}X}
\newcolumntype{C}{>{\centering\arraybackslash}X}
\newcolumntype{L}{>{\raggedright\arraybackslash}X}
\newcolumntype{J}{>{\arraybackslash}X}

\newcommand{\BrochureInlineCode}[1]{{\ttfamily #1}}

\usepackage{amssymb}
\usepackage{amsmath}

\usepackage[babel=true,maxlevel=3]{csquotes}
\DeclareQuoteStyle{bebras-ch-eng}{“}[” ]{”}{‘}[”’ ]{’}\DeclareQuoteStyle{bebras-ch-deu}{«}[» ]{»}{“}[»› ]{”}
\DeclareQuoteStyle{bebras-ch-fra}{«\thinspace{}}[» ]{\thinspace{}»}{“}[»\thinspace{}› ]{”}
\DeclareQuoteStyle{bebras-ch-ita}{«}[» ]{»}{“}[»› ]{”}
\setquotestyle{bebras-ch-fra}

\usepackage{hyperref}
\usepackage{graphicx}
\usepackage{svg}
\svgsetup{inkscapeversion=1,inkscapearea=page}
\usepackage{wrapfig}

\usepackage{enumitem}
\setlist{nosep,itemsep=.5ex}

\setlength{\parindent}{0pt}
\setlength{\parskip}{2ex}
\raggedbottom

\usepackage{fancyhdr}
\usepackage{lastpage}
\pagestyle{fancy}

\fancyhf{}
\renewcommand{\headrulewidth}{0pt}
\renewcommand{\footrulewidth}{0.4pt}
\lfoot{\scriptsize © 2022 Bebras (CC BY-SA 4.0)}
\cfoot{\scriptsize\itshape 2022-SK-02 Broderie}
\rfoot{\scriptsize Page~\thepage{}/\pageref*{LastPage}}

\newcommand{\taskGraphicsFolder}{..}

\begin{document}

\section*{\centering{} 2022-SK-02 Broderie}


\subsection*{Body}

Lana a une machine à broder programmable. La machine peut broder deux sortes de points: \raisebox{-0.5ex}[0pt][0pt]{\includesvg[scale=0.07]{\taskGraphicsFolder/graphics/2022-SK-02-plus_embroidered.svg}} et \raisebox{-0.5ex}[0pt][0pt]{\includesvg[scale=0.07]{\taskGraphicsFolder/graphics/2022-SK-02-x_embroidered.svg}}. Pour générer en plus le point composé \raisebox{-0.5ex}[0pt][0pt]{\includesvg[scale=0.07]{\taskGraphicsFolder/graphics/2022-SK-02-star_embroidered.svg}}, les deux points \raisebox{-0.5ex}[0pt][0pt]{\includesvg[scale=0.07]{\taskGraphicsFolder/graphics/2022-SK-02-x_embroidered.svg}} et \raisebox{-0.5ex}[0pt][0pt]{\includesvg[scale=0.07]{\taskGraphicsFolder/graphics/2022-SK-02-plus_embroidered.svg}} sont nécessaires. Le tissu doit être retiré d’un point en arrière entre les deux points.

Lana peut programmer la machine à l’aide des trois touches suivantes:

\begin{tabularx}{\columnwidth}{ @{} l L @{} }
  \makecell[l]{\includesvg[scale=0.07]{\taskGraphicsFolder/graphics/2022-SK-02-plus.svg}} & La machine brode \raisebox{-0.5ex}[0pt][0pt]{\includesvg[scale=0.07]{\taskGraphicsFolder/graphics/2022-SK-02-plus_embroidered.svg}}. \\ 
  \makecell[l]{\includesvg[scale=0.07]{\taskGraphicsFolder/graphics/2022-SK-02-x.svg}} & La machine brode \raisebox{-0.5ex}[0pt][0pt]{\includesvg[scale=0.07]{\taskGraphicsFolder/graphics/2022-SK-02-x_embroidered.svg}}. \\ 
  \makecell[l]{\includesvg[scale=0.07]{\taskGraphicsFolder/graphics/2022-SK-02-arrow.svg}} & Le tissu est retiré d’un point en arrière.
\end{tabularx}

Un programme est créé grâce aux touches et est exécuté plusieurs fois par la machine à broder.

Par exemple, la machine à broder brode…

\begin{itemize}
  \item … ce motif: \raisebox{-0.5ex}{\includesvg[scale=0.07]{\taskGraphicsFolder/graphics/2022-SK-02-taskbody.svg}}
  \item … en exécutant ce programme: \raisebox{-0.5ex}{\includesvg[scale=0.07]{\taskGraphicsFolder/graphics/2022-SK-02-taskbody_programm.svg}}.
\end{itemize}

{\em


\subsection*{Question/Challenge - for the brochures}

Lequel des programmes suivants Lana a-t-elle utilisé pour broder ce motif?

{\centering%
\includesvg[scale=0.07]{\taskGraphicsFolder/graphics/2022-SK-02-question.svg}\par}

}


\subsection*{Interactivity Instructions}



\begingroup
\renewcommand{\arraystretch}{1.5}
\subsection*{Answer Options/Interactivity Description}

A) \raisebox{\dimexpr -0.5ex -1.3ex \relax}{\includesvg[scale=0.07]{\taskGraphicsFolder/graphics/2022-SK-02-answerA.svg}}

B) \raisebox{\dimexpr -0.5ex -1.3ex \relax}{\includesvg[scale=0.07]{\taskGraphicsFolder/graphics/2022-SK-02-answerB.svg}}

C) \raisebox{\dimexpr -0.5ex -1.3ex \relax}{\includesvg[scale=0.07]{\taskGraphicsFolder/graphics/2022-SK-02-answerC.svg}}

D) \raisebox{\dimexpr -0.5ex -1.3ex \relax}{\includesvg[scale=0.07]{\taskGraphicsFolder/graphics/2022-SK-02-answerD.svg}}

\endgroup

\subsection*{Answer Explanation}

La bonne réponse est C) \raisebox{-0.5ex}{\includesvg[width=119.1px]{\taskGraphicsFolder/graphics/2022-SK-02-answerC.svg}}.

Pour trouver le programme de Lana, nous commençons par chercher la partie du motif qui se répète: \raisebox{-0.5ex}{\includesvg[scale=0.07]{\taskGraphicsFolder/graphics/2022-SK-02-explanation1.svg}}. Les deux premiers points sont des \raisebox{-0.5ex}[0pt][0pt]{\includesvg[scale=0.07]{\taskGraphicsFolder/graphics/2022-SK-02-x_embroidered.svg}}. Pour cela, Lana utilise la touche \raisebox{-0.5ex}[0pt][0pt]{\includesvg[width=10.8px]{\taskGraphicsFolder/graphics/2022-SK-02-x.svg}}; il doit donc y avoir deux \raisebox{-0.5ex}[0pt][0pt]{\includesvg[width=10.8px]{\taskGraphicsFolder/graphics/2022-SK-02-x.svg}} au début de son programme. Comme le programme D commence par un \raisebox{-0.5ex}[0pt][0pt]{\includesvg[width=10.8px]{\taskGraphicsFolder/graphics/2022-SK-02-plus.svg}}, il est faux.

Le prochain point du motif est une \raisebox{-0.5ex}[0pt][0pt]{\includesvg[scale=0.07]{\taskGraphicsFolder/graphics/2022-SK-02-star_embroidered.svg}}. Pour le broder, la machine doit broder un point \raisebox{-0.5ex}[0pt][0pt]{\includesvg[scale=0.07]{\taskGraphicsFolder/graphics/2022-SK-02-plus_embroidered.svg}} et un point \raisebox{-0.5ex}[0pt][0pt]{\includesvg[scale=0.07]{\taskGraphicsFolder/graphics/2022-SK-02-x_embroidered.svg}} l’un sur l’autre; pour cela, le tissu doit être retiré d’un point entre deux. L’ordre dans lequel les points \raisebox{-0.5ex}[0pt][0pt]{\includesvg[scale=0.07]{\taskGraphicsFolder/graphics/2022-SK-02-plus_embroidered.svg}} et \raisebox{-0.5ex}[0pt][0pt]{\includesvg[scale=0.07]{\taskGraphicsFolder/graphics/2022-SK-02-x_embroidered.svg}} sont brodés est égal. On peut donc utiliser l’un des deux programmes \raisebox{-0.5ex}{\includesvg[width=36.1px]{\taskGraphicsFolder/graphics/2022-SK-02-explanation_starversion1.svg}} ou \raisebox{-0.5ex}{\includesvg[width=36.1px]{\taskGraphicsFolder/graphics/2022-SK-02-explanation_starversion2.svg}}.

Les quatre programmes génèrent les motifs suivants:

\begin{tabular}{ @{} c l l @{} }
  {\setstretch{1.0}\thead[cb]{}} & {\setstretch{1.0}\thead[lb]{programme}} & {\setstretch{1.0}\thead[lb]{motif brodé}} \\ 
\midrule
  \textbf{A} & \makecell[l]{\includesvg[scale=0.07]{\taskGraphicsFolder/graphics/2022-SK-02-answerA.svg}} & \makecell[l]{\includesvg[scale=0.07]{\taskGraphicsFolder/graphics/2022-SK-02-explanationA.svg}} \\ 
  \textbf{B} & \makecell[l]{\includesvg[scale=0.07]{\taskGraphicsFolder/graphics/2022-SK-02-answerB.svg}} & \makecell[l]{\includesvg[scale=0.07]{\taskGraphicsFolder/graphics/2022-SK-02-explanationB.svg}} \\ 
  \textbf{C} & \makecell[l]{\includesvg[scale=0.07]{\taskGraphicsFolder/graphics/2022-SK-02-answerC.svg}} & \makecell[l]{\includesvg[scale=0.07]{\taskGraphicsFolder/graphics/2022-SK-02-explanation1.svg}} \\ 
  \textbf{D} & \makecell[l]{\includesvg[scale=0.07]{\taskGraphicsFolder/graphics/2022-SK-02-answerD.svg}} & \makecell[l]{\includesvg[scale=0.07]{\taskGraphicsFolder/graphics/2022-SK-02-explanationD.svg}}
\end{tabular}

Les points des programmes B et D ne sont pas dans le bon ordre. Les motifs des programmes A et C sont les mêmes jusqu’au cinquième point, mais ensuite, le programme A ajoute deux \raisebox{-0.5ex}[0pt][0pt]{\includesvg[scale=0.07]{\taskGraphicsFolder/graphics/2022-SK-02-x_embroidered.svg}} après la deuxième étoile \raisebox{-0.5ex}[0pt][0pt]{\includesvg[scale=0.07]{\taskGraphicsFolder/graphics/2022-SK-02-star_embroidered.svg}}. Lorsque l’on répète le programme A, on obtient donc quatre \raisebox{-0.5ex}[0pt][0pt]{\includesvg[scale=0.07]{\taskGraphicsFolder/graphics/2022-SK-02-x_embroidered.svg}} au lieu de deux entre la deuxième étoile \raisebox{-0.5ex}[0pt][0pt]{\includesvg[scale=0.07]{\taskGraphicsFolder/graphics/2022-SK-02-star_embroidered.svg}} et la suivante.

Le programme C génère donc le bon motif.


\subsection*{It’s Informatics}

Dans cet exercice, un motif répétitif est généré grâce à une série d’instructions. En informatique, les problèmes compliqués sont aussi souvent divisés en problèmes plus petits qui sont plus facile à comprendre, à résoudre et à programmer. Une compétence importante pour cela est de savoir reconnaître ces séries de motifs répétés pour pouvoir réutiliser une solution. Ceci peut se faire en utilisant des \emph{boucles}, par exemple.

Le programme utilisé par la machine à broder est une liste d’instructions écrite dans un \emph{langage de programmation}. Une machine à broder programmable n’est en fait rien d’autre qu’un robot ou un ordinateur qui suit des instructions. À l’image d’une machine à broder qui brode les points, un ordinateur exécute exactement les instructions d’un programme. Le respect d’une suite d’instructions est un concept important en informatique. L’ordre des instructions est également important: lorsque nous changeons l’ordre des instructions, cela change en général aussi l’effet du programme. Dans notre cas, cela veut dire qu’un changement dans l’ordre des instructions cause un changement dans le motif brodé (sauf si nous brodons une étoile).

{\raggedright

\subsection*{Keywords and Websites}

\begin{itemize}
  \item Instruction: \href{https://fr.wikipedia.org/wiki/Instruction_informatique}{\BrochureUrlText{https://fr.wikipedia.org/wiki/Instruction\_informatique}}
  \item boucle: \href{https://fr.wikipedia.org/wiki/Structure_de_contr\%C3\%B4le\#Boucles}{\BrochureUrlText{https://fr.wikipedia.org/wiki/Structure\_de\_contrôle\#Boucles}}
  \item Commande: \href{https://fr.wikipedia.org/wiki/Commande_informatique}{\BrochureUrlText{https://fr.wikipedia.org/wiki/Commande\_informatique}}
  \item Langage de programmation: \href{https://fr.wikipedia.org/wiki/Langage_de_programmation}{\BrochureUrlText{https://fr.wikipedia.org/wiki/Langage\_de\_programmation}}
  \item Algorithme: \href{https://fr.wikipedia.org/wiki/Algorithme}{\BrochureUrlText{https://fr.wikipedia.org/wiki/Algorithme}}
  \item Broderie: \href{https://fr.wikipedia.org/wiki/Broderie}{\BrochureUrlText{https://fr.wikipedia.org/wiki/Broderie}}
\end{itemize}


}
\end{document}
