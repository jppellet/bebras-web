% Definition of the meta information: task difficulties, task ID, task title, task country; definition of the variables as well as their scope is in commands.tex
\setcounter{taskAgeDifficulty3to4}{0}
\setcounter{taskAgeDifficulty5to6}{3}
\setcounter{taskAgeDifficulty7to8}{2}
\setcounter{taskAgeDifficulty9to10}{1}
\setcounter{taskAgeDifficulty11to13}{0}
\renewcommand{\taskTitle}{Più caldo, più freddo}
\renewcommand{\taskCountry}{SK}

% include this task only if for the age groups being processed this task is relevant
\ifthenelse{
  \(\boolean{age3to4} \AND \(\value{taskAgeDifficulty3to4} > 0\)\) \OR
  \(\boolean{age5to6} \AND \(\value{taskAgeDifficulty5to6} > 0\)\) \OR
  \(\boolean{age7to8} \AND \(\value{taskAgeDifficulty7to8} > 0\)\) \OR
  \(\boolean{age9to10} \AND \(\value{taskAgeDifficulty9to10} > 0\)\) \OR
  \(\boolean{age11to13} \AND \(\value{taskAgeDifficulty11to13} > 0\)\)}{

\newchapter{\taskTitle}

% task body
Nina e Daniel giocano alla caccia al tesoro. Su una tavola con quadrati, Nina seleziona un quadrato e lo tiene a mente. Il tesoro è nascosto lì.

Daniel sceglie un campo di partenza. Da lì, sposta il suo pezzo da gioco \raisebox{-0.5ex}[0pt][0pt]{\includesvg[width=10.8px]{\taskGraphicsFolder/graphics/2023-SK-07-Daniel.svg}} di uno spazio alla volta: a sinistra, a destra, in alto o in basso.

\begin{tabularx}{\columnwidth}{ @{} l J @{} }
  \makecell[l]{\includesvg[scale=0.1]{\taskGraphicsFolder/graphics/2023-SK-07-example1.svg}} & Al primo tentativo, prendono un piccolo tabellone di gioco. Nina nasconde il tesoro nella casella con la stella \raisebox{-0.5ex}[0pt][0pt]{\includesvg[width=14.4px]{\taskGraphicsFolder/graphics/2023-SK-07_stern.svg}}. Daniel inizia in alto a destra e fa due passi lungo le frecce. Dopo ogni passo, Nina dice se Daniel è più vicino \raisebox{-0.5ex}[0pt][0pt]{\includesvg[width=8.7px]{\taskGraphicsFolder/graphics/2023-SK-07-warmer.svg}} al tesoro o più lontano \raisebox{-0.5ex}[0pt][0pt]{\includesvg[width=14.4px]{\taskGraphicsFolder/graphics/2023-SK-07-colder.svg}} dal tesoro rispetto a prima del passo. \\ 
  \makecell[l]{\includesvg[scale=0.1]{\taskGraphicsFolder/graphics/2023-SK-07-example3.svg}} & L’immagine a destra mostra le distanze di Daniel dal tesoro. La distanza dal tesoro è il minor numero di passi che Daniel potrebbe attualmente compiere per raggiungere il tesoro.
\end{tabularx}

Adesso prendono una tavola più grande.
Nina nasconde il tesoro in uno dei campi contrassegnati in blu.
L’immagine mostra nuovamente i passi di Daniel e ciò che Nina dice dopo ogni passo.



% question (as \emph{})
{\em
Dove è nascosto il tesoro?

{\centering%
\includesvg[scale=0.1]{\taskGraphicsFolder/graphics/2023-SK-07-question_symbols.svg}\par}


}

% answer alternatives (as \begin{enumerate}[A)]) or interactivity


% from here on this is only included if solutions are processed
\ifthenelse{\boolean{solutions}}{
\newpage

% answer explanation
\section*{\BrochureSolution}
La risposta corretta:

{\centering%
\includesvg[scale=0.1]{\taskGraphicsFolder/graphics/2023-SK-07-solution_compatible.svg}\par}

Seguiamo il percorso di Daniel e il riscontro di Nina. Daniel inizia nella riga $1$ del tabellone. Dopo il primo passo si trova nella riga $2$ e più vicino al tesoro rispetto alla riga $1$. Dopo il passo successivo si trova nella riga $3$ e di nuovo più lontano dal tesoro. Dato che è rimasto nella stessa colonna, il tesoro deve trovarsi su una casella della riga $2$, infatti non importa in quale colonna sia nascosto il tesoro: la via più breve per raggiungere il tesoro da un’altra colonna è quella di trovarsi nella stessa riga.

Ma in quale colonna è nascosto il tesoro? Continuando il suo cammino, Daniel si avvicina inizialmente al tesoro della riga $4$ facendo qualche passo verso sinistra; in particolare, è più vicino al tesoro della colonna $3$ che a quello della colonna $4$. Ma dopo l’ultimo passo della riga, Daniel si trova più lontano dal tesoro della colonna $2$ che da quello della colonna $3$. Quindi il tesoro deve trovarsi in un quadrato della colonna $3$, siccome quanto detto sopra per le colonne vale anche per le righe: la via più breve per raggiungere il tesoro da un’altra riga è quella di trovarsi nella stessa colonna.



% it's informatics
\section*{\BrochureItsInformatics}
Daniel cammina (con il suo pezzo di gioco) attraverso il tabellone. Da ogni casella su cui si trova attualmente, Nina misura la distanza dalla casella con il tesoro e la utilizza per il suo feedback.  Di solito, la distanza tra due punti viene misurata come la lunghezza del collegamento rettilineo tra i punti (distanza euclidea).  Ma i due campi non sono, in senso stretto, dei punti. Pertanto, Nina misura la distanza tra due campi nel numero di passi che Daniel dovrebbe fare per il percorso più breve da un campo all’altro.  Questa \emph{misura} può essere generalmente applicata alle griglie ed è nota in informatica come \emph{distanza di Manhattan}, derivata dalla pianta a griglia del quartiere newyorkese di Manhattan.

Gli informatici scelgono il modo di calcolare la distanza tra due oggetti a seconda del quesito che vogliono risolvere. Ad esempio, se si vuole misurare la distanza tra due parole della stessa lunghezza in un linguaggio naturale, si può contare il numero di punti in cui le parole differiscono; si tratta quindi della \emph{distanza di Hamming}.  Se le parole sono di lunghezza diversa, si può usare la \emph{distanza di Levenshtein}. Le distanze giocano spesso un ruolo nell’informatica quando si tratta di trovare soluzioni ottimali a un problema.  Non importa se la soluzione a un problema deve essere la più veloce, la più breve o la più economica:  spesso non è necessario cambiare l’algoritmo, ma solo la misura della distanza: durata, lunghezza o costo.



% keywords and websites (as \begin{itemize})
\section*{\BrochureWebsitesAndKeywords}
{\raggedright
\begin{itemize}
  \item Distanza di Manhattan: \href{https://it.wikipedia.org/wiki/Geometria_del_taxi}{\BrochureUrlText{https://it.wikipedia.org/wiki/Geometria\_del\_taxi}}
  \item Distanza di Hamming: \href{https://it.wikipedia.org/wiki/Distanza_di_Hamming}{\BrochureUrlText{https://it.wikipedia.org/wiki/Distanza\_di\_Hamming}}
  \item Distanza di Levenshtein: \href{https://it.wikipedia.org/wiki/Distanza_di_Levenshtein}{\BrochureUrlText{https://it.wikipedia.org/wiki/Distanza\_di\_Levenshtein}}
\end{itemize}


}

% end of ifthen for excluding the solutions
}{}

% all authors
% ATTENTION: you HAVE to make sure an according entry is in ../main/authors.tex.
% Syntax: \def\AuthorLastnameF{} (Lastname is last name, F is first letter of first name, this serves as a marker for ../main/authors.tex)
\def\AuthorDatzkoC{} % \ifdefined\AuthorDatzkoC \BrochureFlag{hu}{} Christian Datzko\fi
\def\AuthorWettsteinM{} % \ifdefined\AuthorWettsteinM \BrochureFlag{ch}{} Manuel Wettstein\fi
\def\AuthorSerafiniG{} % \ifdefined\AuthorSerafiniG \BrochureFlag{ch}{} Giovanni Serafini\fi
\def\AuthorKohJ{} % \ifdefined\AuthorKohJ \BrochureFlag{tw}{} Jia-Ling Koh\fi
\def\AuthorKohS{} % \ifdefined\AuthorKohS \BrochureFlag{sg}{} Sophie Koh\fi
\def\AuthorTomcsanyiovaM{} % \ifdefined\AuthorTomcsanyiovaM \BrochureFlag{sk}{} Monika Tomcsányiová\fi
\def\AuthorDatzkoThutS{} % \ifdefined\AuthorDatzkoThutS \BrochureFlag{de}{} Susanne Datzko-Thut\fi
\def\AuthorPohlW{} % \ifdefined\AuthorPohlW \BrochureFlag{de}{} Wolfgang Pohl\fi
\def\AuthorGiangC{} % \ifdefined\AuthorGiangC \BrochureFlag{ch}{} Christian Giang\fi

\newpage}{}
