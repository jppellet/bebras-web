\documentclass[a4paper,11pt]{report}
\usepackage[T1]{fontenc}
\usepackage[utf8]{inputenc}

\usepackage[english]{babel}
\AtBeginDocument{\def\labelitemi{$\bullet$}}

\usepackage{etoolbox}

\usepackage[margin=2cm]{geometry}
\usepackage{changepage}
\makeatletter
\renewenvironment{adjustwidth}[2]{%
    \begin{list}{}{%
    \partopsep\z@%
    \topsep\z@%
    \listparindent\parindent%
    \parsep\parskip%
    \@ifmtarg{#1}{\setlength{\leftmargin}{\z@}}%
                 {\setlength{\leftmargin}{#1}}%
    \@ifmtarg{#2}{\setlength{\rightmargin}{\z@}}%
                 {\setlength{\rightmargin}{#2}}%
    }
    \item[]}{\end{list}}
\makeatother

\newcommand{\BrochureUrlText}[1]{\texttt{#1}}
\usepackage{setspace}
\setstretch{1.15}

\usepackage{tabularx}
\usepackage{booktabs}
\usepackage{makecell}
\usepackage{multirow}
\renewcommand\theadfont{\bfseries}
\renewcommand{\tabularxcolumn}[1]{>{}m{#1}}
\newcolumntype{R}{>{\raggedleft\arraybackslash}X}
\newcolumntype{C}{>{\centering\arraybackslash}X}
\newcolumntype{L}{>{\raggedright\arraybackslash}X}
\newcolumntype{J}{>{\arraybackslash}X}

\newcommand{\BrochureInlineCode}[1]{{\ttfamily #1}}

\usepackage{amssymb}
\usepackage{amsmath}

\usepackage[babel=true,maxlevel=3]{csquotes}
\DeclareQuoteStyle{bebras-ch-eng}{“}[” ]{”}{‘}[”’ ]{’}\DeclareQuoteStyle{bebras-ch-deu}{«}[» ]{»}{“}[»› ]{”}
\DeclareQuoteStyle{bebras-ch-fra}{«\thinspace{}}[» ]{\thinspace{}»}{“}[»\thinspace{}› ]{”}
\DeclareQuoteStyle{bebras-ch-ita}{«}[» ]{»}{“}[»› ]{”}
\setquotestyle{bebras-ch-eng}

\usepackage{hyperref}
\usepackage{graphicx}
\usepackage{svg}
\svgsetup{inkscapeversion=1,inkscapearea=page}
\usepackage{wrapfig}

\usepackage{enumitem}
\setlist{nosep,itemsep=.5ex}

\setlength{\parindent}{0pt}
\setlength{\parskip}{2ex}
\raggedbottom

\usepackage{fancyhdr}
\usepackage{lastpage}
\pagestyle{fancy}

\fancyhf{}
\renewcommand{\headrulewidth}{0pt}
\renewcommand{\footrulewidth}{0.4pt}
\lfoot{\scriptsize © 2023 Bebras (CC BY-SA 4.0)}
\cfoot{\scriptsize\itshape 2023-DE-09 Domino}
\rfoot{\scriptsize Page~\thepage{}/\pageref*{LastPage}}

\newcommand{\taskGraphicsFolder}{..}

\begin{document}

\section*{\centering{} 2023-DE-09 Domino}


\subsection*{Body}

Domino tiles consist of two halves. Each half has between $1$ and $6$ dots. In this task, the following eight tiles are lying in front of you:

{\centering%
\includesvg[scale=0.48]{\taskGraphicsFolder/graphics/2023-DE-09_domino_steine.svg}\par}

These tiles now need to be put in a row such that:

\begin{enumerate}
  \item All eight tiles are used and
  \item The adjacent halves of two neighboring tiles always have the same number of dots
\end{enumerate}

{\centering%
\includesvg[scale=0.48]{\taskGraphicsFolder/graphics/2023-DE-09_ordnung.svg}\par}

A valid solution contains a row with all eight tiles in a specific order. The tiles can be placed in several different orderings and still produce a valid solution. Abstractly a solution looks like this:

{\centering%
\includesvg[width=288.6px]{\taskGraphicsFolder/graphics/2023-DE-09_domino_ordnung.svg}\par}

As you will notice, certain tiles cannot be place at the marked edges.

{\em


\subsection*{Question/Challenge - for the brochures}

Which tiles can be used at the edges?

}


\subsection*{Interactivity instruction - for the online challenge}

Click the Domino tiles to select them. Click them again to deselect them.

\begingroup
\renewcommand{\arraystretch}{1.5}
\subsection*{Answer Options/Interactivity Description}

Every Domino tile can be selected by clicking on it. Then the tile will be highlighted. By clicking again, the tile is deselected. Multiple tiles can be highlighted at the same time.

\endgroup

\subsection*{Answer Explanation}

Among the eight Domino tiles, there are five that can be considered as edge pieces:

{\centering%
\includesvg[width=288.6px]{\taskGraphicsFolder/graphics/2023-DE-09_domino_solution.svg}\par}

To start with, let’s count the number of dots on each Domino tile and examine how frequently each of these dot counts appears. In our scenario we have eight Domino tiles which translates to $16$ \emph{faces} (tile halves). The table below shows a summary along with an extra column showing whether the dot frequency is even or odd:

\begin{tabular}{ @{} l l l @{} }
  {\setstretch{1.0}\thead[lb]{Dots}} & {\setstretch{1.0}\thead[lb]{Frequency}} & {\setstretch{1.0}\thead[lb]{Even/Odd}} \\ 
\midrule
  \makecell[l]{\includesvg[width=21.6px]{\taskGraphicsFolder/graphics/2023-DE-09_1.svg}} & 3 & odd \\ 
  \makecell[l]{\includesvg[width=21.6px]{\taskGraphicsFolder/graphics/2023-DE-09_2.svg}} & 3 & odd \\ 
  \makecell[l]{\includesvg[width=21.6px]{\taskGraphicsFolder/graphics/2023-DE-09_3.svg}} & 2 & even \\ 
  \makecell[l]{\includesvg[width=21.6px]{\taskGraphicsFolder/graphics/2023-DE-09_4.svg}} & 2 & even \\ 
  \makecell[l]{\includesvg[width=21.6px]{\taskGraphicsFolder/graphics/2023-DE-09_5.svg}} & 4 & even \\ 
  \makecell[l]{\includesvg[width=21.6px]{\taskGraphicsFolder/graphics/2023-DE-09_6.svg}} & 2 & even
\end{tabular}

Note that the faces with $3$, $4$, $5$, and $6$ dots occur with an \emph{even frequency}, whereas the faces with $1$ and $2$ dots occur with an \emph{odd frequency}. This is relevant since adjacent faces of two tiles should always have the same number of dots and, consequently, occur in pairs in a given Domino sequence. Tiles with the faces $3$, $4$, $5$, and $6$ can be grouped into pairs without any remainder, whereas this is not the case with faces $1$ and $2$.

For the ordering of tiles, it is important to distinguish between tiles with even and odd frequencies. Specifically, faces with $1$ and $2$ (faces with odd frequencies) need to be carefully placed as they can lead to situations where the sequence cannot be continued despite having remaining tiles. For example, if we arrange the tiles so that all the tiles with one dot are used up, and the sequence ends in a tile with one dot, the sequence cannot be extended even if there are still tiles available.

{\centering%
\includesvg[width=324.7px]{\taskGraphicsFolder/graphics/2023-DE-09_domino_explanation_spare.svg}\par}

However, this problem can be solved by deliberately using a tile with $1$ dot as an edge tile. The same holds true for faces with two dots. The beginning and the end of the sequence are unique as they do not require a predecessor (in the case of the edge tile at the beginning) or a successor (in the case of the edge tile at the end).

Among the eight Domino tiles, five have the numbers $1$ or $2$ on one or both halves. Therefore, the following five Domino tiles are considered eligible edge tiles: ($1$,$2$), ($1$,$4$), ($1$,$5$), ($2$,$5$), ($2$,$6$).


\subsection*{This is Informatics}

There are multiple valid solutions that use all eight provided tiles and also fulfill the given rules. To gain a better overview of all possible solutions, we can visually represent the problem in a different way; computer scientists use so-called \emph{graphs}:

{\centering%
\includesvg[width=216.5px]{\taskGraphicsFolder/graphics/2023_DE-09-explanation.svg}\par}

In the graph above, squares (called \emph{nodes}) represent all six possible faces of Domino tiles. The lines (called \emph{edges}) represent Domino tiles where each line connects two halves of a Domino tile. In the task above, we were given eight specific Domino tiles and accordingly we also see exactly eight edges in the graph. For example, the tile 2$-6$ is represented by the following edge:

{\centering%
\includesvg[width=216.5px]{\taskGraphicsFolder/graphics/2023_DE-09-explanation-highlighted.svg}\par}

(Observation: there is no line between certain faces (e.g. 5$-6$) since this combination is not among the provided Domino tiles in our task.)

To find a valid arrangement using all eight Domino tiles, we need to find a sequence where the dots match up between adjacent segments. After placing the first tile, it is thus already clear with which number the second tile must begin because adjacent halves of two tiles should always have the same number of dots. In the graph, this is evident by the fact that tiles can be placed next to each other if and only if their edges meet at the same node. For example, the tiles ($2$,$6$) and ($6$,$3$) can be placed next to each other since they both contain the number $6$:

{\centering%
\includesvg[width=216.5px]{\taskGraphicsFolder/graphics/2023_DE-09-explanation-two-lines.svg}\par}

The arrangement of the tiles can be understood as a \emph{path} (a sequence of edges) through the graph. This path should visit each edge \emph{exactly once} to ensure that all eight tiles are used and none are used multiple times. A path that visits \emph{every edge exactly once} is called an \emph{Eulerian path}. The name originates from Leonhard Euler, a Swiss mathematician who invented the field of graph theory. Euler was able to prove that in a connected graph, an Eulerian path exists if and only if there are at most two nodes with an odd number of edges.


\subsection*{This is Computational Thinking}

Optional - not to be filled 2023


\subsection*{Informatics Keywords and Websites}

\begin{itemize}
  \item Graph, Nodes, Edges: \href{https://en.wikipedia.org/wiki/Graph_(abstract_data_type)}{\BrochureUrlText{https://en.wikipedia.org/wiki/Graph\_(abstract\_data\_type)}}
  \item Eulerian path: \href{https://en.wikipedia.org/wiki/Eulerian_path}{\BrochureUrlText{https://en.wikipedia.org/wiki/Eulerian\_path}}
\end{itemize}


\subsection*{Computational Thinking Keywords and Websites}

\begin{itemize}
  \item 
\end{itemize}


\end{document}
