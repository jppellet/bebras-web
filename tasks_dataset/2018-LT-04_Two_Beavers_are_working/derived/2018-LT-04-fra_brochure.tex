% Definition of the meta information: task difficulties, task ID, task title, task country; definition of the variables as well as their scope is in commands.tex
\setcounter{taskAgeDifficulty3to4}{0}
\setcounter{taskAgeDifficulty5to6}{0}
\setcounter{taskAgeDifficulty7to8}{0}
\setcounter{taskAgeDifficulty9to10}{0}
\setcounter{taskAgeDifficulty11to13}{2}
\renewcommand{\taskTitle}{Plan de travail}
\renewcommand{\taskCountry}{LT}

% include this task only if for the age groups being processed this task is relevant
\ifthenelse{
  \(\boolean{age3to4} \AND \(\value{taskAgeDifficulty3to4} > 0\)\) \OR
  \(\boolean{age5to6} \AND \(\value{taskAgeDifficulty5to6} > 0\)\) \OR
  \(\boolean{age7to8} \AND \(\value{taskAgeDifficulty7to8} > 0\)\) \OR
  \(\boolean{age9to10} \AND \(\value{taskAgeDifficulty9to10} > 0\)\) \OR
  \(\boolean{age11to13} \AND \(\value{taskAgeDifficulty11to13} > 0\)\)}{

\newchapter{\taskTitle}

% task body
\begin{wrapfigure}{R}{252.5px}
\raisebox{-.46cm}[\dimexpr \height-.92cm \relax][-.46cm]{\includesvg[width=252.5px]{\taskGraphicsFolder/graphics/2018-LT-04-body1.svg}}
\end{wrapfigure}
La construction d’un barrage par des castors peut être divisée en plusieurs tâches (abattre des arbres, enlever les branches, transporter les troncs jusqu’à la rivière, etc.). L’image de droite montre les huit tâches A, B, C, D, E, F, G et H et le nombre d’heures nécessaires à chacune pour construire un barrage. Les tâches ne sont pas complètement indépendantes les unes des autres: un flèche de X à Y signifie que la tâche X doit être terminée avec de pouvoir commencer la tâche Y.

La castore Ulla veut aider le castor Otso à construire le barrage plus vite. Ils se répartissent les tâches et font le plan de travail suivant en tenant compte des dépendances indiquées plus haut.

{\centering%
\includesvg[scale=0.7]{\taskGraphicsFolder/graphics/2018-LT-04-body2-compatible.svg}\par}

Le barrage serait ainsi terminé en $32$ heures. Mais c’est possible de le faire plus vite!



% question (as \emph{})
{\em
Fais un plan de travail qui permet de finir le barrage le plus vite possible.


}

% answer alternatives (as \begin{enumerate}[A)]) or interactivity


% from here on this is only included if solutions are processed
\ifthenelse{\boolean{solutions}}{
\newpage

% answer explanation
\section*{\BrochureSolution}
Dans le plan de travail de l’exercice, le premier castor a une longue pause ($10$ heures), et le deuxième castor a en tout $8$ heures de libre. Ils auraient terminé plus vite s’ils travaillaient en permanence.

On obtient un plan de travail plus rapide si l’on fait attention à ce que les deux tâches les plus longues E($10$) et F($9$) ne soient pas effectuées par le même castor. Voici un plan de travail qui ne dure que $23$ heures. Ce n’est pas possible de terminer le barrage plus vite, car les deux castors travaillent ici déjà sans pause.

{\centering%
\includesvg[scale=0.7]{\taskGraphicsFolder/graphics/2018-LT-04-explanation-compatible.svg}\par}



% it's informatics
\section*{\BrochureItsInformatics}
L’être humain est impatient de nature, c’est pourquoi il est souvent souhaitable de finir des travaux aussi rapidement que possible. Lorsqu’un travail peut être réparti (que ce soit entre différentes personnes, machines ou castors), la répartition des différentes tâches entre les travailleurs joue un rôle important dans la durée du travail. Similairement, les ordinateurs doivent souvent répartir leur tâches entre différents processeurs de manière optimale.

Il existe plusieurs stratégies bien étudiées pour de tels problèmes d’\emph{ordonnancement} en informatique. Pour le premier plan de travail de cet exercice, les plus longues tâches parmi toutes celles à réaliser ont été assignées au castor qui ne travaillait pas à ce moment-là – une mauvaise stratégie dans ce cas. Cela fonctionne souvent mieux lorsque les tâches les plus courtes sont effectuées en premier: la stratégie \emph{Shortest job first} (“le plus court processus en premier” en anglais) minimise aussi le temps d’attente moyen pour chaque tâche.



% keywords and websites (as \begin{itemize})
\section*{\BrochureWebsitesAndKeywords}
{\raggedright
\begin{itemize}
  \item Ordonnancement: \href{https://fr.wikipedia.org/wiki/Ordonnancement_de_travaux_informatiques}{\BrochureUrlText{https://fr.wikipedia.org/wiki/Ordonnancement\_de\_travaux\_informatiques}}
  \item Shortest job fext: \href{https://fr.wikipedia.org/wiki/Shortest_job_first}{\BrochureUrlText{https://fr.wikipedia.org/wiki/Shortest\_job\_first}}
\end{itemize}


}

% end of ifthen for excluding the solutions
}{}

% all authors
% ATTENTION: you HAVE to make sure an according entry is in ../main/authors.tex.
% Syntax: \def\AuthorLastnameF{} (Lastname is last name, F is first letter of first name, this serves as a marker for ../main/authors.tex)
\def\AuthorDagieneV{} % \ifdefined\AuthorDagieneV \BrochureFlag{lt}{} Valentina Dagienė\fi
\def\AuthorKinciusV{} % \ifdefined\AuthorKinciusV \BrochureFlag{lt}{} Vaidotas Kinčius\fi
\def\AuthorJaksicS{} % \ifdefined\AuthorJaksicS \BrochureFlag{rs}{} Svetlana Jaksic\fi
\def\AuthorNaughtonT{} % \ifdefined\AuthorNaughtonT \BrochureFlag{ie}{} Tom Naughton\fi
\def\AuthorWeigendM{} % \ifdefined\AuthorWeigendM \BrochureFlag{de}{} Michael Weigend\fi
\def\AuthorGuggisbergM{} % \ifdefined\AuthorGuggisbergM \BrochureFlag{ch}{} Martin Guggisberg\fi
\def\AuthorPohlW{} % \ifdefined\AuthorPohlW \BrochureFlag{de}{} Wolfgang Pohl\fi
\def\AuthorDatzkoS{} % \ifdefined\AuthorDatzkoS \BrochureFlag{ch}{} Susanne Datzko\fi
\def\AuthorFreiF{} % \ifdefined\AuthorFreiF \BrochureFlag{ch}{} Fabian Frei\fi
\def\AuthorTeamC{} % \ifdefined\AuthorTeamC \BrochureFlag{nl}{} Cuttle.org Team\fi
\def\AuthorPelletE{} % \ifdefined\AuthorPelletE \BrochureFlag{ch}{} Elsa Pellet\fi

\newpage}{}
