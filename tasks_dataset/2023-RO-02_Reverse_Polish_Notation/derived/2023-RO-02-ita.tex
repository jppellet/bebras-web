\documentclass[a4paper,11pt]{report}
\usepackage[T1]{fontenc}
\usepackage[utf8]{inputenc}

\usepackage[italian]{babel}
\AtBeginDocument{\def\labelitemi{$\bullet$}}

\usepackage{etoolbox}

\usepackage[margin=2cm]{geometry}
\usepackage{changepage}
\makeatletter
\renewenvironment{adjustwidth}[2]{%
    \begin{list}{}{%
    \partopsep\z@%
    \topsep\z@%
    \listparindent\parindent%
    \parsep\parskip%
    \@ifmtarg{#1}{\setlength{\leftmargin}{\z@}}%
                 {\setlength{\leftmargin}{#1}}%
    \@ifmtarg{#2}{\setlength{\rightmargin}{\z@}}%
                 {\setlength{\rightmargin}{#2}}%
    }
    \item[]}{\end{list}}
\makeatother

\newcommand{\BrochureUrlText}[1]{\texttt{#1}}
\usepackage{setspace}
\setstretch{1.15}

\usepackage{tabularx}
\usepackage{booktabs}
\usepackage{makecell}
\usepackage{multirow}
\renewcommand\theadfont{\bfseries}
\renewcommand{\tabularxcolumn}[1]{>{}m{#1}}
\newcolumntype{R}{>{\raggedleft\arraybackslash}X}
\newcolumntype{C}{>{\centering\arraybackslash}X}
\newcolumntype{L}{>{\raggedright\arraybackslash}X}
\newcolumntype{J}{>{\arraybackslash}X}

\newcommand{\BrochureInlineCode}[1]{{\ttfamily #1}}

\usepackage{amssymb}
\usepackage{amsmath}

\usepackage[babel=true,maxlevel=3]{csquotes}
\DeclareQuoteStyle{bebras-ch-eng}{“}[” ]{”}{‘}[”’ ]{’}\DeclareQuoteStyle{bebras-ch-deu}{«}[» ]{»}{“}[»› ]{”}
\DeclareQuoteStyle{bebras-ch-fra}{«\thinspace{}}[» ]{\thinspace{}»}{“}[»\thinspace{}› ]{”}
\DeclareQuoteStyle{bebras-ch-ita}{«}[» ]{»}{“}[»› ]{”}
\setquotestyle{bebras-ch-ita}

\usepackage{hyperref}
\usepackage{graphicx}
\usepackage{svg}
\svgsetup{inkscapeversion=1,inkscapearea=page}
\usepackage{wrapfig}

\usepackage{enumitem}
\setlist{nosep,itemsep=.5ex}

\setlength{\parindent}{0pt}
\setlength{\parskip}{2ex}
\raggedbottom

\usepackage{fancyhdr}
\usepackage{lastpage}
\pagestyle{fancy}

\fancyhf{}
\renewcommand{\headrulewidth}{0pt}
\renewcommand{\footrulewidth}{0.4pt}
\lfoot{\scriptsize © 2023 Bebras (CC BY-SA 4.0)}
\cfoot{\scriptsize\itshape 2023-RO-02 Notazione postfissa}
\rfoot{\scriptsize Page~\thepage{}/\pageref*{LastPage}}

\newcommand{\taskGraphicsFolder}{..}

\begin{document}

\section*{\centering{} 2023-RO-02 Notazione postfissa}


\subsection*{Body}

Un’espressione matematica è costituita da …

\begin{itemize}
  \item … un \emph{operatore}: \textbf{+}, \textbf{-}, \textbf{·} o \textbf{:}
  \item … e gli \emph{operandi}: numeri come $1$, $2$, …, lettere come a, b, … o ancora espressioni come ($1$ + $2$).
\end{itemize}

La struttura di un’espressione matematica può essere rappresentata come un \emph{albero strutturale}.
Questo diagramma di operatori e operandi è disegnato con un cerchio con l’operatore è collegato all’albero degli operandi da frecce.
Nel caso più semplice, si tratta di cerchi con un numero o una lettera.

Da un albero, a sua volta, si può leggere la \emph{notazione postfissa} di un’espressione matematica.
In questa notazione, per ogni espressione, gli operandi vengono scritti per primi, seguiti dall’operatore.

\begin{adjustwidth}{1.5em}{0em}
\begin{tabular}{ @{} l c c @{} }
  \textbf{Espressione matematica:} & a + b & (a + $1$) \ensuremath{\cdot} (b + c) \\ 
  \textbf{Albero strutturale:} & \makecell[c]{\includesvg[scale=0.32]{\taskGraphicsFolder/graphics/2023-RO-02-example1-compatible.svg}} & \makecell[c]{\includesvg[scale=0.32]{\taskGraphicsFolder/graphics/2023-RO-02-example2-compatible.svg}} \\ 
  \textbf{Notazione postfissa:} & a b + & a $1$ + b c + \ensuremath{\cdot}
\end{tabular}


\end{adjustwidth}

Ecco la notazione postfissa di un’altra espressione:

\begin{adjustwidth}{1.5em}{0em}
a $1$ + b $2$ + · $25$ c : +
\end{adjustwidth}

{\em


\subsection*{Question/Challenge - for the brochures}

Qual è l’albero strutturale di questa espressione?

}

\begingroup
\renewcommand{\arraystretch}{1.5}
\subsection*{Answer Options/Interactivity Description}

\begin{tabular}{ @{} c c c c @{} }
  \makecell[c]{\includesvg[scale=0.32]{\taskGraphicsFolder/graphics/2023-RO-02-answerA-compatible.svg}} & \makecell[c]{\includesvg[scale=0.32]{\taskGraphicsFolder/graphics/2023-RO-02-answerB-compatible.svg}} & \makecell[c]{\includesvg[scale=0.32]{\taskGraphicsFolder/graphics/2023-RO-02-answerC-compatible.svg}} & \makecell[c]{\includesvg[scale=0.32]{\taskGraphicsFolder/graphics/2023-RO-02-answerD-compatible.svg}} \\ 
  A) & B) & C) & D)
\end{tabular}

\endgroup

\subsection*{Answer Explanation}

La risposta C è corretta: \raisebox{-0.5ex}{\includesvg[scale=0.32]{\taskGraphicsFolder/graphics/2023-RO-02-answerC-compatible.svg}}

Come descritto nel compito, l’operatore centrale di un’espressione matematica si trova in cima all’albero (è la sua \emph{radice}) e alla fine nella notazione postfissa. Se si vuole trovare o creare l’albero di struttura di un’espressione in notazione postfissa, si deve cercare l’ultimo carattere della notazione postfissa in cima all’albero, in questo caso il +. Solo negli alberi delle risposte A e C si trova un + nella radice.

L’operatore + ha due operandi, uno a sinistra e uno a destra. Nella notazione postfissa, si può vedere direttamente (al penultimo carattere) che l’operando di destra dell’espressione è di nuovo un’espressione che ha l’operatore :. Nell’albero strutturale, quindi, deve esserci un : a destra sotto la radice. Questo è il caso solo dell’albero della risposta C. Quindi questa deve essere la risposta corretta.

Questo può essere dimostrato anche convertendo completamente l’albero della risposta C in notazione postfissa:

\begin{itemize}
  \item I tre sottoalberi più \enquote{piccoli}, ciascuno composto da $3$ elementi, diventano a $1$ +, b $2$ + e $25$ c :.
  \item I due sottoalberi di sinistra di questi tre \enquote{più piccoli} diventano l’operando di sinistra del \textbf{+} superiore, in modo che la trasformazione del sottoalbero di sinistra sia ora $1$ + b $2$ + ·. Il terzo dei sottoalberi \enquote{più piccoli} è già l’operando destro.
  \item Quindi la notazione postfissa della struttura ad albero della risposta C è complessivamente: a $1$ + b $2$ + · $25$ c : +. Questa è esattamente l’espressione data nel compito.
\end{itemize}


\subsection*{This is Informatics}

La \emph{notazione postfissa}, detta anche \emph{notazione polacca inversa}, è spesso utilizzata per formulare espressioni matematiche o di altro tipo (ad esempio nei linguaggi di programmazione) senza equivoci e in modo compatto. Se si dovesse scrivere l’espressione data dalla struttura ad albero della risposta C in notazione normale (cioè con gli operatori tra gli operandi, quindi chiamata anche \emph{notazione prefisso}), si dovrebbero mettere le parentesi (a + $1$) - (b + $2$) + $25$ : c, che non sono necessarie nella notazione postfissa. La notazione postfissa è stata introdotta per la prima volta da Jan Łukasiewicz (1878$-1956$), con l’operatore davanti agli operandi. È il modo in cui si scrive l’applicazione delle funzioni, tra le altre cose: in matematica ${f(x, y)}$, in programmazione \BrochureInlineCode{nomefunzione(argomento1, argomento2, argomento3)}.  Nel computer si usa, tra l’altro, quando si \emph{parsano} le espressioni di un linguaggio di programmazione.

Nel recente passato, molte persone hanno conosciuto la notazione postfissa soprattutto con l’uso delle prime calcolatrici scientifiche: con essa si potevano inserire e calcolare espressioni matematiche complesse in modo rapido e affidabile e, soprattutto, senza parentesi.  Ancora oggi, esiste una comunità che utilizza calcolatrici programmabili (come le HP 35s) con la notazione postfissa.


\subsection*{This is Computational Thinking}

Um die Aufgabe zu lösen, muss ein komplexer mathematischer Ausdruck durch \emph{Zerlegen} oder \emph{Dekomposition} in kleinere mathematische Ausdrücke verwandelt werden, die für sich im Kontext verständlich sind. Zudem muss ein Vorgang dahingehend \emph{analysiert} werden, dass seine Funktionsweise auch in rückwärtiger Reihenfolge verständlich wird. Im Grunde wird dabei dieselbe Information zwischen unterschiedlichen Darstellungsweisen konvertiert.


\subsection*{Informatics Keywords and Websites}

\begin{itemize}
  \item Notazione polacca inversa: \href{https://it.wikipedia.org/wiki/Notazione_polacca_inversa}{\BrochureUrlText{https://it.wikipedia.org/wiki/Notazione\_polacca\_inversa}}
  \item Albero sintttico: \href{https://it.wikipedia.org/wiki/Albero_sintattico}{\BrochureUrlText{https://it.wikipedia.org/wiki/Albero\_sintattico}}
  \item Jan Łukasiewicz: \href{https://it.wikipedia.org/wiki/Jan_\%C5\%81ukasiewicz}{\BrochureUrlText{https://it.wikipedia.org/wiki/Jan\_Łukasiewicz}}
  \item HP 35s: \href{https://it.wikipedia.org/wiki/HP_35s}{\BrochureUrlText{https://it.wikipedia.org/wiki/HP\_35s}}
\end{itemize}


\subsection*{Computational Thinking Keywords and Websites}

\begin{itemize}
  \item Zerlegen, Dekomposition: \href{https://de.wikipedia.org/wiki/Modell\#Modellbildung}{\BrochureUrlText{https://de.wikipedia.org/wiki/Modell\#Modellbildung}}
\end{itemize}


\end{document}
