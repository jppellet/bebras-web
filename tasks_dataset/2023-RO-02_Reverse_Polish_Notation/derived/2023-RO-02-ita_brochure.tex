% Definition of the meta information: task difficulties, task ID, task title, task country; definition of the variables as well as their scope is in commands.tex
\setcounter{taskAgeDifficulty3to4}{0}
\setcounter{taskAgeDifficulty5to6}{0}
\setcounter{taskAgeDifficulty7to8}{0}
\setcounter{taskAgeDifficulty9to10}{3}
\setcounter{taskAgeDifficulty11to13}{2}
\renewcommand{\taskTitle}{Notazione postfissa}
\renewcommand{\taskCountry}{RO}

% include this task only if for the age groups being processed this task is relevant
\ifthenelse{
  \(\boolean{age3to4} \AND \(\value{taskAgeDifficulty3to4} > 0\)\) \OR
  \(\boolean{age5to6} \AND \(\value{taskAgeDifficulty5to6} > 0\)\) \OR
  \(\boolean{age7to8} \AND \(\value{taskAgeDifficulty7to8} > 0\)\) \OR
  \(\boolean{age9to10} \AND \(\value{taskAgeDifficulty9to10} > 0\)\) \OR
  \(\boolean{age11to13} \AND \(\value{taskAgeDifficulty11to13} > 0\)\)}{

\newchapter{\taskTitle}

% task body
Un’espressione matematica è costituita da …

\begin{itemize}
  \item … un \emph{operatore}: \textbf{+}, \textbf{-}, \textbf{·} o \textbf{:}
  \item … e gli \emph{operandi}: numeri come $1$, $2$, …, lettere come a, b, … o ancora espressioni come ($1$ + $2$).
\end{itemize}

La struttura di un’espressione matematica può essere rappresentata come un \emph{albero strutturale}.
Questo diagramma di operatori e operandi è disegnato con un cerchio con l’operatore è collegato all’albero degli operandi da frecce.
Nel caso più semplice, si tratta di cerchi con un numero o una lettera.

Da un albero, a sua volta, si può leggere la \emph{notazione postfissa} di un’espressione matematica.
In questa notazione, per ogni espressione, gli operandi vengono scritti per primi, seguiti dall’operatore.

\begin{adjustwidth}{1.5em}{0em}
\begin{tabular}{ @{} l c c @{} }
  \textbf{Espressione matematica:} & a + b & (a + $1$) \ensuremath{\cdot} (b + c) \\ 
  \textbf{Albero strutturale:} & \makecell[c]{\includesvg[scale=0.32]{\taskGraphicsFolder/graphics/2023-RO-02-example1-compatible.svg}} & \makecell[c]{\includesvg[scale=0.32]{\taskGraphicsFolder/graphics/2023-RO-02-example2-compatible.svg}} \\ 
  \textbf{Notazione postfissa:} & a b + & a $1$ + b c + \ensuremath{\cdot}
\end{tabular}


\end{adjustwidth}

Ecco la notazione postfissa di un’altra espressione:

\begin{adjustwidth}{1.5em}{0em}
a $1$ + b $2$ + · $25$ c : +
\end{adjustwidth}



% question (as \emph{})
{\em
Qual è l’albero strutturale di questa espressione?


}

% answer alternatives (as \begin{enumerate}[A)]) or interactivity
\begin{tabular}{ @{} c c c c @{} }
  \makecell[c]{\includesvg[scale=0.32]{\taskGraphicsFolder/graphics/2023-RO-02-answerA-compatible.svg}} & \makecell[c]{\includesvg[scale=0.32]{\taskGraphicsFolder/graphics/2023-RO-02-answerB-compatible.svg}} & \makecell[c]{\includesvg[scale=0.32]{\taskGraphicsFolder/graphics/2023-RO-02-answerC-compatible.svg}} & \makecell[c]{\includesvg[scale=0.32]{\taskGraphicsFolder/graphics/2023-RO-02-answerD-compatible.svg}} \\ 
  A) & B) & C) & D)
\end{tabular}



% from here on this is only included if solutions are processed
\ifthenelse{\boolean{solutions}}{
\newpage

% answer explanation
\section*{\BrochureSolution}
La risposta C è corretta: \raisebox{-0.5ex}{\includesvg[scale=0.32]{\taskGraphicsFolder/graphics/2023-RO-02-answerC-compatible.svg}}

Come descritto nel compito, l’operatore centrale di un’espressione matematica si trova in cima all’albero (è la sua \emph{radice}) e alla fine nella notazione postfissa. Se si vuole trovare o creare l’albero di struttura di un’espressione in notazione postfissa, si deve cercare l’ultimo carattere della notazione postfissa in cima all’albero, in questo caso il +. Solo negli alberi delle risposte A e C si trova un + nella radice.

L’operatore + ha due operandi, uno a sinistra e uno a destra. Nella notazione postfissa, si può vedere direttamente (al penultimo carattere) che l’operando di destra dell’espressione è di nuovo un’espressione che ha l’operatore :. Nell’albero strutturale, quindi, deve esserci un : a destra sotto la radice. Questo è il caso solo dell’albero della risposta C. Quindi questa deve essere la risposta corretta.

Questo può essere dimostrato anche convertendo completamente l’albero della risposta C in notazione postfissa:

\begin{itemize}
  \item I tre sottoalberi più \enquote{piccoli}, ciascuno composto da $3$ elementi, diventano a $1$ +, b $2$ + e $25$ c :.
  \item I due sottoalberi di sinistra di questi tre \enquote{più piccoli} diventano l’operando di sinistra del \textbf{+} superiore, in modo che la trasformazione del sottoalbero di sinistra sia ora $1$ + b $2$ + ·. Il terzo dei sottoalberi \enquote{più piccoli} è già l’operando destro.
  \item Quindi la notazione postfissa della struttura ad albero della risposta C è complessivamente: a $1$ + b $2$ + · $25$ c : +. Questa è esattamente l’espressione data nel compito.
\end{itemize}



% it's informatics
\section*{\BrochureItsInformatics}
La \emph{notazione postfissa}, detta anche \emph{notazione polacca inversa}, è spesso utilizzata per formulare espressioni matematiche o di altro tipo (ad esempio nei linguaggi di programmazione) senza equivoci e in modo compatto. Se si dovesse scrivere l’espressione data dalla struttura ad albero della risposta C in notazione normale (cioè con gli operatori tra gli operandi, quindi chiamata anche \emph{notazione prefisso}), si dovrebbero mettere le parentesi (a + $1$) - (b + $2$) + $25$ : c, che non sono necessarie nella notazione postfissa. La notazione postfissa è stata introdotta per la prima volta da Jan Łukasiewicz (1878$-1956$), con l’operatore davanti agli operandi. È il modo in cui si scrive l’applicazione delle funzioni, tra le altre cose: in matematica ${f(x, y)}$, in programmazione \BrochureInlineCode{nomefunzione(argomento1, argomento2, argomento3)}.  Nel computer si usa, tra l’altro, quando si \emph{parsano} le espressioni di un linguaggio di programmazione.

Nel recente passato, molte persone hanno conosciuto la notazione postfissa soprattutto con l’uso delle prime calcolatrici scientifiche: con essa si potevano inserire e calcolare espressioni matematiche complesse in modo rapido e affidabile e, soprattutto, senza parentesi.  Ancora oggi, esiste una comunità che utilizza calcolatrici programmabili (come le HP 35s) con la notazione postfissa.



% keywords and websites (as \begin{itemize})
\section*{\BrochureWebsitesAndKeywords}
{\raggedright
\begin{itemize}
  \item Notazione polacca inversa: \href{https://it.wikipedia.org/wiki/Notazione_polacca_inversa}{\BrochureUrlText{https://it.wikipedia.org/wiki/Notazione\_polacca\_inversa}}
  \item Albero sintttico: \href{https://it.wikipedia.org/wiki/Albero_sintattico}{\BrochureUrlText{https://it.wikipedia.org/wiki/Albero\_sintattico}}
  \item Jan Łukasiewicz: \href{https://it.wikipedia.org/wiki/Jan_\%C5\%81ukasiewicz}{\BrochureUrlText{https://it.wikipedia.org/wiki/Jan\_Łukasiewicz}}
  \item HP 35s: \href{https://it.wikipedia.org/wiki/HP_35s}{\BrochureUrlText{https://it.wikipedia.org/wiki/HP\_35s}}
\end{itemize}


}

% end of ifthen for excluding the solutions
}{}

% all authors
% ATTENTION: you HAVE to make sure an according entry is in ../main/authors.tex.
% Syntax: \def\AuthorLastnameF{} (Lastname is last name, F is first letter of first name, this serves as a marker for ../main/authors.tex)
\def\AuthorUngureanuL{} % \ifdefined\AuthorUngureanuL \BrochureFlag{ro}{} Laura Ungureanu\fi
\def\AuthorConstantinescuR{} % \ifdefined\AuthorConstantinescuR \BrochureFlag{ro}{} Raluca Constantinescu\fi
\def\AuthorIoannouT{} % \ifdefined\AuthorIoannouT \BrochureFlag{cy}{} Thomas Ioannou\fi
\def\AuthorReyesO{} % \ifdefined\AuthorReyesO \BrochureFlag{pr}{} Omar Colon Reyes\fi
\def\AuthorAnwarK{} % \ifdefined\AuthorAnwarK \BrochureFlag{my}{} Khairul Anwar\fi
\def\AuthorKhachatryanG{} % \ifdefined\AuthorKhachatryanG \BrochureFlag{am}{} Gohar Khachatryan\fi
\def\AuthorDatzkoC{} % \ifdefined\AuthorDatzkoC \BrochureFlag{hu}{} Christian Datzko\fi
\def\AuthorPohlW{} % \ifdefined\AuthorPohlW \BrochureFlag{de}{} Wolfgang Pohl\fi
\def\AuthorDatzkoThutS{} % \ifdefined\AuthorDatzkoThutS \BrochureFlag{de}{} Susanne Datzko-Thut\fi
\def\AuthorGiangC{} % \ifdefined\AuthorGiangC \BrochureFlag{ch}{} Christian Giang\fi

\newpage}{}
