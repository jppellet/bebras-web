\documentclass[a4paper,11pt]{report}
\usepackage[T1]{fontenc}
\usepackage[utf8]{inputenc}

\usepackage[italian]{babel}
\AtBeginDocument{\def\labelitemi{$\bullet$}}

\usepackage{etoolbox}

\usepackage[margin=2cm]{geometry}
\usepackage{changepage}
\makeatletter
\renewenvironment{adjustwidth}[2]{%
    \begin{list}{}{%
    \partopsep\z@%
    \topsep\z@%
    \listparindent\parindent%
    \parsep\parskip%
    \@ifmtarg{#1}{\setlength{\leftmargin}{\z@}}%
                 {\setlength{\leftmargin}{#1}}%
    \@ifmtarg{#2}{\setlength{\rightmargin}{\z@}}%
                 {\setlength{\rightmargin}{#2}}%
    }
    \item[]}{\end{list}}
\makeatother

\newcommand{\BrochureUrlText}[1]{\texttt{#1}}
\usepackage{setspace}
\setstretch{1.15}

\usepackage{tabularx}
\usepackage{booktabs}
\usepackage{makecell}
\usepackage{multirow}
\renewcommand\theadfont{\bfseries}
\renewcommand{\tabularxcolumn}[1]{>{}m{#1}}
\newcolumntype{R}{>{\raggedleft\arraybackslash}X}
\newcolumntype{C}{>{\centering\arraybackslash}X}
\newcolumntype{L}{>{\raggedright\arraybackslash}X}
\newcolumntype{J}{>{\arraybackslash}X}

\newcommand{\BrochureInlineCode}[1]{{\ttfamily #1}}

\usepackage{amssymb}
\usepackage{amsmath}

\usepackage[babel=true,maxlevel=3]{csquotes}
\DeclareQuoteStyle{bebras-ch-eng}{“}[” ]{”}{‘}[”’ ]{’}\DeclareQuoteStyle{bebras-ch-deu}{«}[» ]{»}{“}[»› ]{”}
\DeclareQuoteStyle{bebras-ch-fra}{«\thinspace{}}[» ]{\thinspace{}»}{“}[»\thinspace{}› ]{”}
\DeclareQuoteStyle{bebras-ch-ita}{«}[» ]{»}{“}[»› ]{”}
\setquotestyle{bebras-ch-ita}

\usepackage{hyperref}
\usepackage{graphicx}
\usepackage{svg}
\svgsetup{inkscapeversion=1,inkscapearea=page}
\usepackage{wrapfig}

\usepackage{enumitem}
\setlist{nosep,itemsep=.5ex}

\setlength{\parindent}{0pt}
\setlength{\parskip}{2ex}
\raggedbottom

\usepackage{fancyhdr}
\usepackage{lastpage}
\pagestyle{fancy}

\fancyhf{}
\renewcommand{\headrulewidth}{0pt}
\renewcommand{\footrulewidth}{0.4pt}
\lfoot{\scriptsize © 2017 Bebras (CC BY-SA 4.0)}
\cfoot{\scriptsize\itshape 2017-RU-05 Decodifica}
\rfoot{\scriptsize Page~\thepage{}/\pageref*{LastPage}}

\newcommand{\taskGraphicsFolder}{..}

\begin{document}

\section*{\centering{} 2017-RU-05 Decodifica}


\subsection*{Body}

In un codice speciale per i testi, ogni lettera è codificata da un codice composto dalle cifre da $0$ a $9$.
Si applica questa regola: nessuna parola in codice può iniziare con la parola in codice di un’altra lettera.

Per esempio, la lettera \BrochureInlineCode{X} è codificata da $12$. Ora \BrochureInlineCode{Y} può essere codificata da $2$, perché $12$ non inizia con $2$ (e $2$ non inizia con $12$). Ora \BrochureInlineCode{Z} può essere codificata da $11$, perché né $12$ né $2$ iniziano con $11$ e $11$ non inizia né con $12$ né con $2$. Tuttavia, $21$ non sarebbe ammessa come parola chiave per \BrochureInlineCode{Z} perché inizia con $2$, che è la parola chiave di \BrochureInlineCode{Y}.

La parola \BrochureInlineCode{MEMORY} è codificata dalla sequenza di cifre $12112233321$.

{\em


\subsection*{Question/Challenge - for the brochures}

Divide la sequenza di numeri nelle parole in codice delle singole lettere!

}


\subsection*{Interactivity instruction - for the online challenge}

Sposta il puntatore del mouse negli spazi tra le cifre. Appare un <->. Fa clic per separare la sequenza di cifre. Clicca su \enquote{Reset} per eliminare tutti gli spazi vuoti. Al termine, fai clic su \enquote{Salva risposta}.

\begingroup
\renewcommand{\arraystretch}{1.5}
\subsection*{Answer Options/Interactivity Description}

Der Code wird angezeigt, mit ausreichend Abstand zwischen den Ziffern.  Geht man mit der Maus in einen solchen Abstand, wird der Mauspointer als \enquote{Trennungssymbol} angezeigt.  Ein Klick führt dann dazu, dass der Abstand zwischen den Ziffern vergrössert wird.  Ein Klick auf einen vergrösserten Abstand macht daraus wieder einen normalen Abstand. Es gibt einen Knopf um wieder von vorne anzufangen.

\endgroup

\subsection*{Answer Explanation}

Questo è la risposta corretta: $1$ $21$ $1$ $22$ $33$ 321

Si parte da sinistra all’inizio della sequenza di cifre. Se \BrochureInlineCode{M} fosse codificata da $12$, \BrochureInlineCode{E} avrebbe necessariamente la parola chiave $1$, perché dietro di essa ci sarebbe di nuovo $12$ per la seconda \BrochureInlineCode{M}. Tuttavia, ciò sarebbe in contraddizione con la regola: La parola in codice per \BrochureInlineCode{M} inizierebbe con $1$, la parola in codice per E. Anche i pezzi iniziali più lunghi della sequenza di cifre ($121$, $1211$, $12112$ ecc.) non possono essere la parola in codice per \BrochureInlineCode{M}, perché questa parola in codice deve comparire due volte, ma questi pezzi sono contenuti una sola volta nella sequenza di cifre. Di conseguenza, la parola in codice per `M’ è la cifra $1$.

Ora deve seguire la parola in codice per \BrochureInlineCode{E} - e dietro di essa ancora quella per M, cioè l’$1$. Quindi, la parola in codice per \BrochureInlineCode{E} può essere solo una delle seguenti sequenze di cifre: $2$, $21$ o $211223332$. Non può essere $2$ perché sennò la parola inizierebbe con MEMM. Non può essere $211223332$, siccome in tal caso la parola nel suo insieme sarebbe solo MEM. Di conseguenza, la parola in codice per `E’ è la sequenza di cifre $21$. Ora è chiaro che $1$ $21$ $1$ è la codifica per MEM.

Il resto della sequenza di cifre, cioè $2233321$, codifica le lettere ORY. Il $2$ da solo non può essere la parola in codice per \BrochureInlineCode{O}, altrimenti avremmo \BrochureInlineCode{OO} all’inizio. La parola in codice per \BrochureInlineCode{O} deve quindi contenere almeno il $22$. Alla fine, $1$ e $21$ sono già assegnati come parole in codice rispettivamente per \BrochureInlineCode{M} e \BrochureInlineCode{E}. La parola in codice per \BrochureInlineCode{Y} deve quindi essere almeno la sequenza $321$. Tra $22$ e $321$ c’è $33$, che deve essere la parola in codice per \BrochureInlineCode{R}: L’unica altra parola in codice possibile per \BrochureInlineCode{R} sarebbe $3$. Quindi $3321$ dovrebbe essere la parola in codice per \BrochureInlineCode{Y} - e inizierebbe con la parola in codice per \BrochureInlineCode{R}; questo è contrario alla regola. La divisione della parte posteriore è quindi $22$ $33$ $321$.


\subsection*{It’s Informatics}

Il codice descritto in questo task è un esempio di \emph{codice prefisso}. Un prefisso è una stringa di caratteri all’inizio di un’altra stringa di caratteri. In un codice prefisso, nessuna parola in codice può essere un prefisso di un’altra parola in codice. Cioè, nessuna parola in codice può iniziare con un’altra parola in codice.

Nei codici a prefisso, le parole in codice hanno lunghezze diverse. Il vantaggio della regola del prefisso è che non è necessario alcun simbolo di separazione tra le parole in codice. È sempre possibile vedere dove inizia la parola in codice successiva. Se si scelgono parole in codice brevi per le lettere che ricorrono frequentemente, è possibile codificare i testi in modo molto efficiente e memorizzare grandi quantità di testo in modo da risparmiare spazio.

La codifica Huffman è un metodo per trovare un prefisso ottimale. È ampiamente utilizzato ed è, ad esempio, alla base di noti formati di dati come JPEG e MP3.


\subsection*{This is Computational Thinking}

Optional - not to be filled 2023

{\raggedright

\subsection*{Keywords and Websites}

\begin{itemize}
  \item Codice prefisso: \href{https://it.wikipedia.org/wiki/Codice_prefisso}{\BrochureUrlText{https://it.wikipedia.org/wiki/Codice\_prefisso}}
  \item Codifica di Huffman: \href{https://it.wikipedia.org/wiki/Codifica_di_Huffman}{\BrochureUrlText{https://it.wikipedia.org/wiki/Codifica\_di\_Huffman}}
  \item Crittografia: \href{https://it.wikipedia.org/wiki/Crittografia}{\BrochureUrlText{https://it.wikipedia.org/wiki/Crittografia}}
  \item Crittoanalisi: \href{https://it.wikipedia.org/wiki/Crittoanalisi}{\BrochureUrlText{https://it.wikipedia.org/wiki/Crittoanalisi}}
\end{itemize}


}
\end{document}
