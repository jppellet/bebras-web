\documentclass[a4paper,11pt]{report}
\usepackage[T1]{fontenc}
\usepackage[utf8]{inputenc}

\usepackage[french]{babel}
\frenchbsetup{ThinColonSpace=true}
\renewcommand*{\FBguillspace}{\hskip .4\fontdimen2\font plus .1\fontdimen3\font minus .3\fontdimen4\font \relax}
\AtBeginDocument{\def\labelitemi{$\bullet$}}

\usepackage{etoolbox}

\usepackage[margin=2cm]{geometry}
\usepackage{changepage}
\makeatletter
\renewenvironment{adjustwidth}[2]{%
    \begin{list}{}{%
    \partopsep\z@%
    \topsep\z@%
    \listparindent\parindent%
    \parsep\parskip%
    \@ifmtarg{#1}{\setlength{\leftmargin}{\z@}}%
                 {\setlength{\leftmargin}{#1}}%
    \@ifmtarg{#2}{\setlength{\rightmargin}{\z@}}%
                 {\setlength{\rightmargin}{#2}}%
    }
    \item[]}{\end{list}}
\makeatother

\newcommand{\BrochureUrlText}[1]{\texttt{#1}}
\usepackage{setspace}
\setstretch{1.15}

\usepackage{tabularx}
\usepackage{booktabs}
\usepackage{makecell}
\usepackage{multirow}
\renewcommand\theadfont{\bfseries}
\renewcommand{\tabularxcolumn}[1]{>{}m{#1}}
\newcolumntype{R}{>{\raggedleft\arraybackslash}X}
\newcolumntype{C}{>{\centering\arraybackslash}X}
\newcolumntype{L}{>{\raggedright\arraybackslash}X}
\newcolumntype{J}{>{\arraybackslash}X}

\newcommand{\BrochureInlineCode}[1]{{\ttfamily #1}}

\usepackage{amssymb}
\usepackage{amsmath}

\usepackage[babel=true,maxlevel=3]{csquotes}
\DeclareQuoteStyle{bebras-ch-eng}{“}[” ]{”}{‘}[”’ ]{’}\DeclareQuoteStyle{bebras-ch-deu}{«}[» ]{»}{“}[»› ]{”}
\DeclareQuoteStyle{bebras-ch-fra}{«\thinspace{}}[» ]{\thinspace{}»}{“}[»\thinspace{}› ]{”}
\DeclareQuoteStyle{bebras-ch-ita}{«}[» ]{»}{“}[»› ]{”}
\setquotestyle{bebras-ch-fra}

\usepackage{hyperref}
\usepackage{graphicx}
\usepackage{svg}
\svgsetup{inkscapeversion=1,inkscapearea=page}
\usepackage{wrapfig}

\usepackage{enumitem}
\setlist{nosep,itemsep=.5ex}

\setlength{\parindent}{0pt}
\setlength{\parskip}{2ex}
\raggedbottom

\usepackage{fancyhdr}
\usepackage{lastpage}
\pagestyle{fancy}

\fancyhf{}
\renewcommand{\headrulewidth}{0pt}
\renewcommand{\footrulewidth}{0.4pt}
\lfoot{\scriptsize © 2017 Bebras (CC BY-SA 4.0)}
\cfoot{\scriptsize\itshape 2017-RU-05 Décryptage}
\rfoot{\scriptsize Page~\thepage{}/\pageref*{LastPage}}

\newcommand{\taskGraphicsFolder}{..}

\begin{document}

\section*{\centering{} 2017-RU-05 Décryptage}


\subsection*{Body}

Un code spécial pour les textes remplace chaque lettre par un mot composé de chiffres entre $0$ et $9$. La règle suivante s’applique: aucun mot du code ne peut commencer par un mot du code chiffrant une autre lettre.

Le lettre \BrochureInlineCode{X}, par exemple, est chiffrée par $12$. \BrochureInlineCode{Y} peut donc être chiffré par $2$, car $12$ ne commence pas par $2$ (et $2$ ne commence pas par $12$). \BrochureInlineCode{Z} peut être chiffré par $11$, car ni $12$, ni $2$ ne commence par $11$ et $11$ ne commence ni par $12$, ni par $2$. \BrochureInlineCode{Z} ne pourrait par contre par être chiffré par $21$, car $21$ commence par $2$, qui est le code de \BrochureInlineCode{Y}.

Le mot \BrochureInlineCode{MEMORY} est chiffré par la suite de chiffres $12112233321$.

{\em


\subsection*{Question/Challenge - for the brochures}

Sépare la suite de chiffres en mots représentant chacune des lettres.

}


\subsection*{Interactivity instruction - for the online challenge}

Bouge le curseur dans les espaces entre les chiffres. Le symbole <-> apparaît. Clique pour séparer la suite de chiffres à cet endroit. Clique sur “recommencer” pour enlever tous les espaces. Lorsque tu as fini, clique sur “Enregistrer la réponse”.

\begingroup
\renewcommand{\arraystretch}{1.5}
\subsection*{Answer Options/Interactivity Description}

Der Code wird angezeigt, mit ausreichend Abstand zwischen den Ziffern.  Geht man mit der Maus in einen solchen Abstand, wird der Mauspointer als “Trennungssymbol” angezeigt.  Ein Klick führt dann dazu, dass der Abstand zwischen den Ziffern vergrössert wird.  Ein Klick auf einen vergrösserten Abstand macht daraus wieder einen normalen Abstand. Es gibt einen Knopf um wieder von vorne anzufangen.

\endgroup

\subsection*{Answer Explanation}

La bonne réponse est $1$ $21$ $1$ $22$ $33$ $321$.

On commence par la gauche de la suite de chiffres. Si \BrochureInlineCode{M} était chiffré par le mot $12$, \BrochureInlineCode{E} devrait être chiffré par $1$, car $12$ revient tout de suite après pour le deuxième \BrochureInlineCode{M}. Ceci serait contraire à la règle: le code pour \BrochureInlineCode{M} commencerait par $1$, qui est le code pour \BrochureInlineCode{E}. De plus longs mots ($121$, $1211$, $12112$, etc.) ne peuvent pas coder le \BrochureInlineCode{M}, car ce mot chiffré doit apparaître deux fois dans le cryptogramme, ce qui n’est pas le cas de ces mots. Le mot chiffré pour \BrochureInlineCode{M} doit donc être le $1$.

Celui-ci doit être suivi du mot chiffré pour le \BrochureInlineCode{E}, puis à nouveau du \BrochureInlineCode{M} (donc du $1$). Le mot chiffré pour \BrochureInlineCode{E} doit donc être soit $2$, soit $21$, soit $211223332$. Le $2$ n’est pas possible, car le mot en clair commencerait par \BrochureInlineCode{MEMM}. $211223332$ n’est pas possible non plus, car le mot en clair serait alors \BrochureInlineCode{MEM}. Le mot chiffré pour \BrochureInlineCode{E} doit donc être $21$. $1$ $21$ $1$ est donc le code pour \BrochureInlineCode{MEM}.

Le reste de la suite de chiffres, c’est à dire $2233321$, code les lettres \BrochureInlineCode{ORY}. Le $2$ ne peut pas coder le \BrochureInlineCode{O}, sinon \BrochureInlineCode{MEM} serait suivi de \BrochureInlineCode{OO}. Le mot chiffrant le \BrochureInlineCode{O} doit donc contenir au moins $22$. À la fin, $1$ et $21$ sont déjà assignés à \BrochureInlineCode{M} et \BrochureInlineCode{E}, respectivement; le mot chiffré pour \BrochureInlineCode{Y} doit donc au moins être $321$. Entre $22$ et $321$ se trouve $33$, ce qui doit être le mot chiffré pour \BrochureInlineCode{R}: la seule autre possibilité serait le $3$. Le mot chiffré pour \BrochureInlineCode{Y} devrait alors être $3321$, et commencerait par $3$, le mot chiffré pour \BrochureInlineCode{R}, ce que la règle interdit. La séparation de la deuxième partie est donc $22$ $33$ $321$.


\subsection*{It’s Informatics}

Le code utilisé dans cet exercice est un exemple de \emph{code préfixe}. Un préfixe est une suite de symboles précédent une autre suite de symboles. Dans un code préfixe, aucun mot du code ne peut être le préfixe d’un autre mot du code. Cela veut dire qu’aucun mot du code ne peut commencer par un autre mot du code.

Les mots des codes préfixes ont des longueurs différentes. L’avantage de la règle des préfixes est qu’aucun symbole séparateur entre les mots du code n’est nécessaire: on peut toujours reconnaître à quelle position le prochain mot commence. En choisissant des mots courts pour les lettres fréquentes, on peut chiffrer des textes de manière efficace sans utiliser trop de place.

Le codage de Huffman est une méthode permettant de trouver un code préfixe idéal. Elle est très répandue et est utilisée, entre autres, pour les formats JPEG et MP3.


\subsection*{This is Computational Thinking}

Optional - not to be filled 2023

{\raggedright

\subsection*{Keywords and Websites}

\begin{itemize}
  \item Code préfixe: \href{https://fr.wikipedia.org/wiki/Code_pr\%C3\%A9fixe}{\BrochureUrlText{https://fr.wikipedia.org/wiki/Code\_préfixe}}
  \item Codage de Huffman: \href{https://fr.wikipedia.org/wiki/Codage_de_Huffman}{\BrochureUrlText{https://fr.wikipedia.org/wiki/Codage\_de\_Huffman}}
  \item Cryptographie: \href{https://fr.wikipedia.org/wiki/Cryptographie}{\BrochureUrlText{https://fr.wikipedia.org/wiki/Cryptographie}}
  \item Cryptanalyse: \href{https://fr.wikipedia.org/wiki/Cryptanalyse}{\BrochureUrlText{https://fr.wikipedia.org/wiki/Cryptanalyse}}
\end{itemize}


}
\end{document}
