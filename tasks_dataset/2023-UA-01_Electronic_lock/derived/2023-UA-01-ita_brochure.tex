% Definition of the meta information: task difficulties, task ID, task title, task country; definition of the variables as well as their scope is in commands.tex
\setcounter{taskAgeDifficulty3to4}{0}
\setcounter{taskAgeDifficulty5to6}{0}
\setcounter{taskAgeDifficulty7to8}{0}
\setcounter{taskAgeDifficulty9to10}{3}
\setcounter{taskAgeDifficulty11to13}{2}
\renewcommand{\taskTitle}{Serratura a combinazione}
\renewcommand{\taskCountry}{UA}

% include this task only if for the age groups being processed this task is relevant
\ifthenelse{
  \(\boolean{age3to4} \AND \(\value{taskAgeDifficulty3to4} > 0\)\) \OR
  \(\boolean{age5to6} \AND \(\value{taskAgeDifficulty5to6} > 0\)\) \OR
  \(\boolean{age7to8} \AND \(\value{taskAgeDifficulty7to8} > 0\)\) \OR
  \(\boolean{age9to10} \AND \(\value{taskAgeDifficulty9to10} > 0\)\) \OR
  \(\boolean{age11to13} \AND \(\value{taskAgeDifficulty11to13} > 0\)\)}{

\newchapter{\taskTitle}

% task body
Bob ha una serratura a combinazione sulla porta di casa.
Per aprirla, è necessario inserire un codice numerico.
Tutte le cifre del codice devono essere diverse.
Attualmente, il codice è composto da cinque cifre e si legge così:

{\centering%
\includesvg[scale=0.3]{\taskGraphicsFolder/graphics/2023-UA-01-code_compatible.svg}\par}

Bob si è scritto il codice, ma lo maschera un po’:
\emph{n} >\textcompwordmark{}> \emph{c} significa che nel codice ci sono esattamente \emph{n} cifre a sinistra della cifra \emph{c}
che sono maggiori di \emph{c}.  Ad esempio, Bob annota che con

$1$ >\textcompwordmark{}> 3

a sinistra della cifra $3$ c’è esattamente una cifra (cioè $4$) che è maggiore di $3$.
Ha scritto il codice numerico attuale in questo modo:

$0$ >\textcompwordmark{}> $0$ ; $3$ >\textcompwordmark{}> $1$ ; $0$ >\textcompwordmark{}> $2$ ; $1$ >\textcompwordmark{}> $3$ ; $0$ >\textcompwordmark{}> 4

Un codice di sole cinque cifre è troppo insicuro per Bob.
Pertanto, pensa a un nuovo codice, dalle cifre da $0$ a $7$.
Scrive il nuovo codice in questo modo:

$3$ >\textcompwordmark{}> $0$ ; $2$ >\textcompwordmark{}> $1$ ; $4$ >\textcompwordmark{}> $2$ ; $4$ >\textcompwordmark{}> $3$ ; $1$ >\textcompwordmark{}> $4$ ; $1$ >\textcompwordmark{}> $5$ ; $1$ >\textcompwordmark{}> $6$ ; $0$ >\textcompwordmark{}> 7



% question (as \emph{})
{\em
Qual è il nuovo codice?

{\centering%
\includesvg[scale=0.3]{\taskGraphicsFolder/graphics/2023-UA-01-question_brochure.svg}\par}


}

% answer alternatives (as \begin{enumerate}[A)]) or interactivity


% from here on this is only included if solutions are processed
\ifthenelse{\boolean{solutions}}{
\newpage

% answer explanation
\section*{\BrochureSolution}
Ecco la risposta corretta:

{\centering%
\includesvg[scale=0.3]{\taskGraphicsFolder/graphics/2023-UA-01-solution_compatible.svg}\par}

Per determinare il codice, analizziamo più da vicino la notazione di Bob, una per una per le cifre da $0$ a $7$.

\begin{itemize}
  \item $3$ >\textcompwordmark{}> $0$: ci sono esattamente $3$ cifre a sinistra di $0$ che sono maggiori di $0$.
La cifra $0$ deve quindi trovarsi nella quarta posizione del codice.
  \item $2$ >\textcompwordmark{}> $1$: ci sono esattamente $2$ cifre a sinistra di $1$ che sono superiori a $1$.
La cifra $1$ deve quindi trovarsi nella terza posizione del codice.
  \item $4$ >\textcompwordmark{}> $2$: ci sono esattamente $4$ cifre a sinistra di $2$ che sono maggiori di $2$.
Poiché le cifre più piccole $1$ e $0$ sono già al terzo e quarto posto,
le $4$ cifre più grandi devono essere al primo, secondo, quinto e sesto posto.
La cifra $2$ deve quindi trovarsi nella settima posizione del codice.
  \item $4$ >\textcompwordmark{}> $3$: ci sono esattamente $4$ cifre a sinistra di $3$ che sono maggiori di $3$.
La cifra $3$ deve quindi trovarsi nell’ottava e ultima posizione del codice.
  \item $1$ >\textcompwordmark{}> $4$: a sinistra di $4$ c’è esattamente $1$ cifra maggiore di $4$.
La cifra $4$ deve quindi trovarsi nella seconda delle cifre rimanenti; si tratta della seconda cifra del codice.
  \item $1$ >\textcompwordmark{}> $5$: a sinistra di $5$ c’è esattamente $1$ cifra maggiore di $5$.
La cifra $5$ deve quindi trovarsi nella seconda delle cifre rimaste; si tratta della quinta cifra del codice.
  \item $1$ >\textcompwordmark{}> $6$: a sinistra di $6$ c’è esattamente $1$ cifra maggiore di $6$.
La cifra $6$ deve quindi trovarsi nella seconda delle cifre rimaste; si tratta della sesta cifra del codice.
  \item $0$ >\textcompwordmark{}> $7$: non esiste una cifra superiore a $7$.
La cifra $7$ deve trovarsi nell’ultima posizione libera, cioè nella prima posizione del codice.
\end{itemize}



% it's informatics
\section*{\BrochureItsInformatics}
Bob descrive nella sua notazione come il codice si riferisce a una sequenza ordinata di cifre o numeri utilizzati.

Vediamo di nuovo il codice a cinque cifre: $0$ $2$ $4$ $3$ $1$.
Si crea prendendo i numeri ordinati $0$ $1$ $2$ $3$ $4$ e cambiandone la posizione.  Il risultato è chiamato anche \emph{permutazione} (dei numeri da $0$ a $4$).
In una permutazione, i numeri possono essere \enquote{stravolti} per quanto riguarda il loro ordinamento.
Ad esempio, nel codice, $4$ precede $3$, mentre nella sequenza ordinata $3$ precede $4$ (perché $3$ < $4$).
Quindi il $3$ è \enquote{sbagliato} rispetto all’ordinamento. In combinatoria, una branca della matematica, questo si chiama \emph{inversione} o \emph{errore}.

Il codice di Bob è quindi una permutazione, e la sua notazione indica per ogni numero quante volte è \enquote{invertito} in esso:
Lo $0$ è corretto, l’$1$ fa parte di $3$ inversioni ($3$ >\textcompwordmark{}> $1$: tre numeri più grandi sono davanti all’$1$),
il $2$ è corretto, il $3$ è invertito una volta, il $4$ è corretto.
La sequenza di questi numeri di inversione è chiamata \emph{sequenza di inversione}.
(A proposito, la somma dei numeri di inversione descrive il grado di non ordinabilità di una permutazione).

Ora abbiamo tre sequenze - il codice (o permutazione), la sequenza ordinata e la sequenza di inversione - e le riassumiamo in questa tabella:

\begin{adjustwidth}{1.5em}{0em}
\begin{tabular}{ @{} l l l l l l @{} }
  \textbf{Codice / Permutazione} & 0 & 2 & 4 & 3 & 1 \\ 
  \textbf{Sequenza ordinata} & 0 & 1 & 2 & 3 & 4 \\ 
  \textbf{Sequenza di inversione} & 0 & 3 & 0 & 1 & 0
\end{tabular}


\end{adjustwidth}

La descrizione della soluzione ha dimostrato che esiste un algoritmo efficiente che calcola la permutazione corrispondente dalla sequenza di inversione. È sufficiente ripercorrere una volta la sequenza di inversione. L’informatica si occupa spesso di problemi combinatori e conosce molti algoritmi per risolverli. Possono essere utilizzati per la soluzione automatica di rompicapo (come i Sudoku), ma anche per molti problemi \enquote{seri}. La maggior parte delle volte sono molto più complicati dell’algoritmo per la soluzione di questo compito del castoro.



% keywords and websites (as \begin{itemize})
\section*{\BrochureWebsitesAndKeywords}
{\raggedright
\begin{itemize}
  \item Permutazione: \href{https://it.wikipedia.org/wiki/Permutazione}{\BrochureUrlText{https://it.wikipedia.org/wiki/Permutazione}}
\end{itemize}


}

% end of ifthen for excluding the solutions
}{}

% all authors
% ATTENTION: you HAVE to make sure an according entry is in ../main/authors.tex.
% Syntax: \def\AuthorLastnameF{} (Lastname is last name, F is first letter of first name, this serves as a marker for ../main/authors.tex)
\def\AuthorShpakovychR{} % \ifdefined\AuthorShpakovychR \BrochureFlag{ua}{} Rostyslav Shpakovych\fi
\def\AuthorPohlW{} % \ifdefined\AuthorPohlW \BrochureFlag{de}{} Wolfgang Pohl\fi
\def\AuthorMaoY{} % \ifdefined\AuthorMaoY \BrochureFlag{cn}{} Yong Mao\fi
\def\AuthorKoleszarV{} % \ifdefined\AuthorKoleszarV \BrochureFlag{uy}{} Víctor Koleszar\fi
\def\AuthorDatzkoThutS{} % \ifdefined\AuthorDatzkoThutS \BrochureFlag{de}{} Susanne Datzko-Thut\fi
\def\AuthorGiangC{} % \ifdefined\AuthorGiangC \BrochureFlag{ch}{} Christian Giang\fi

\newpage}{}
