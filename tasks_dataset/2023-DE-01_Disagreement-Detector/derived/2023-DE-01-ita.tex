\documentclass[a4paper,11pt]{report}
\usepackage[T1]{fontenc}
\usepackage[utf8]{inputenc}

\usepackage[italian]{babel}
\AtBeginDocument{\def\labelitemi{$\bullet$}}

\usepackage{etoolbox}

\usepackage[margin=2cm]{geometry}
\usepackage{changepage}
\makeatletter
\renewenvironment{adjustwidth}[2]{%
    \begin{list}{}{%
    \partopsep\z@%
    \topsep\z@%
    \listparindent\parindent%
    \parsep\parskip%
    \@ifmtarg{#1}{\setlength{\leftmargin}{\z@}}%
                 {\setlength{\leftmargin}{#1}}%
    \@ifmtarg{#2}{\setlength{\rightmargin}{\z@}}%
                 {\setlength{\rightmargin}{#2}}%
    }
    \item[]}{\end{list}}
\makeatother

\newcommand{\BrochureUrlText}[1]{\texttt{#1}}
\usepackage{setspace}
\setstretch{1.15}

\usepackage{tabularx}
\usepackage{booktabs}
\usepackage{makecell}
\usepackage{multirow}
\renewcommand\theadfont{\bfseries}
\renewcommand{\tabularxcolumn}[1]{>{}m{#1}}
\newcolumntype{R}{>{\raggedleft\arraybackslash}X}
\newcolumntype{C}{>{\centering\arraybackslash}X}
\newcolumntype{L}{>{\raggedright\arraybackslash}X}
\newcolumntype{J}{>{\arraybackslash}X}

\newcommand{\BrochureInlineCode}[1]{{\ttfamily #1}}

\usepackage{amssymb}
\usepackage{amsmath}

\usepackage[babel=true,maxlevel=3]{csquotes}
\DeclareQuoteStyle{bebras-ch-eng}{“}[” ]{”}{‘}[”’ ]{’}\DeclareQuoteStyle{bebras-ch-deu}{«}[» ]{»}{“}[»› ]{”}
\DeclareQuoteStyle{bebras-ch-fra}{«\thinspace{}}[» ]{\thinspace{}»}{“}[»\thinspace{}› ]{”}
\DeclareQuoteStyle{bebras-ch-ita}{«}[» ]{»}{“}[»› ]{”}
\setquotestyle{bebras-ch-ita}

\usepackage{hyperref}
\usepackage{graphicx}
\usepackage{svg}
\svgsetup{inkscapeversion=1,inkscapearea=page}
\usepackage{wrapfig}

\usepackage{enumitem}
\setlist{nosep,itemsep=.5ex}

\setlength{\parindent}{0pt}
\setlength{\parskip}{2ex}
\raggedbottom

\usepackage{fancyhdr}
\usepackage{lastpage}
\pagestyle{fancy}

\fancyhf{}
\renewcommand{\headrulewidth}{0pt}
\renewcommand{\footrulewidth}{0.4pt}
\lfoot{\scriptsize © 2023 Bebras (CC BY-SA 4.0)}
\cfoot{\scriptsize\itshape 2023-DE-01 Rilevatore di conflitti}
\rfoot{\scriptsize Page~\thepage{}/\pageref*{LastPage}}

\newcommand{\taskGraphicsFolder}{..}

\begin{document}

\section*{\centering{} 2023-DE-01 Rilevatore di conflitti}


\subsection*{Body}

Anna e Ben vogliono costruire un \enquote{rilevatore di conflitti} che mostri se hanno un’opinione diversa.

Decidono di utilizzare delle unità che possono essere in due stati, Sì e No: due unità possono essere collegate tramite un cavo che può trasmettere un segnale.

I cavi sono impostati per trasmettere un segnale positivo (+) o negativo (-) all’unità collegata alla sua destra. Quando un’unità si trova nello stato:

\begin{itemize}
  \item Sì: trasmette un segnale attraverso tutti i cavi in uscita.
  \item No: non trasmette alcun segnale.
\end{itemize}

Un’unità collegata passa allo stato Si se riceve più segnali positivi che negativi, e allo stato No in caso contrario o se il numero di segnali positivi e negativi è lo stesso. Anna imposta lo stato dell’unità A e Ben imposta lo stato dell’unità B.

\begin{tabular}{ @{} l l @{} }
  Prima Anna e Ben & Notano che l’unità Z è Sì solo se A è sì e B è no. Questo non è ciò \\ 
  costruiscono & che vogliono: vorrebbero infatti che l’unità Z fosse Sì solo se A è sì \\ 
  questa macchina: & e B e no, ma anche quando A è no e B è sì. \\ 
  \makecell[l]{\includesvg[width=72.2px]{\taskGraphicsFolder/graphics/graphics-new/2023-DE-01-example_compatible.svg}} & \makecell[l]{\includesvg[width=346.3px]{\taskGraphicsFolder/graphics/graphics-new/-ita/2023-DE-01-example_explanation-compatible-ita.svg}}
\end{tabular}

Allora Anna e Ben costruiscono una macchina più grande (in basso nell’immagine) e sono sicuri che possa essere il rilevatore di conflitti corretto: che Z sia Sì solo quando A e B sono in stati diversi (Sì e No o No e Sì). Altrimenti, Z dovrebbe essere nello stato No. Ora non resta che impostare correttamente i cavi.

{\em


\subsection*{Question/Challenge - for the brochures}

Imposta per ciascun cavo la trasmissione di un segnale positivo (+) o negativo (-), in modo che il rilevatore di conflitti funzioni correttamente.

{\centering%
\includesvg[scale=0.3]{\taskGraphicsFolder/graphics/graphics-new/2023-DE-01-question_compatible.svg}\par}

}


\subsection*{Interactivity instruction - for the online challenge}

Fa clic sui cavi per modificare il segnale + e -. Al termine, fai clic su \enquote{Salva risposta}.

\begingroup
\renewcommand{\arraystretch}{1.5}
\subsection*{Answer Options/Interactivity Description}

In the picture of the network, each edge has a marker that can take values \enquote{+} and \enquote{–}. Clicking on the edge or the marker toggles between the two values. Initially, all edges have an empty grey box.  (DACH: We decided to have the markers preset to –.)

\endgroup

\subsection*{Answer Explanation}

Queste due risposte sono corrette:

{\centering%
\raisebox{-0.5ex}{\includesvg[width=108.2px]{\taskGraphicsFolder/graphics/graphics-new/2023-DE-01-solution01.svg}} \raisebox{-0.5ex}{\includesvg[width=108.2px]{\taskGraphicsFolder/graphics/graphics-new/2023-DE-01-solution02.svg}}\par}

Nel rilevatore di conflitti, l’unità di uscita deve essere Sì esattamente per due ingressi diversi (A=Sì e B=No oppure A=No e B=Sì). Z può essere Sì solo se attraverso i due cavi in ingresso arrivano più segnali positivi che negativi.  Almeno uno dei cavi deve quindi trasmettere un segnale positivo (+).  Supponiamo che solo il cavo superiore che porta a Z sia impostato su +.  Allora l’unità centrale superiore deve essere in grado di riconoscere entrambe le combinazioni di ingresso desiderate, cioè deve essere Sì in entrambi i casi.  Insieme alle unità di ingresso A e B, tuttavia, questa unità forma esattamente la macchina che Anna e Ben hanno costruito all’inizio. Può essere Sì solo in uno dei casi desiderati, cioè quando uno dei suoi cavi è impostato su + e l’altro su -:

{\centering%
\includesvg[width=108.2px]{\taskGraphicsFolder/graphics/graphics-new/2023-DE-01-explanation-wp.svg}\par}

Quindi, per ciascuno dei casi di ingresso desiderati è necessaria un’unità separata al centro, una per A=Sì e B=No, l’altra per A=No e B=Sì. I cavi alla prima unità devono essere impostati su + (cavo da A) e – (B), i cavi all’altra unità su – (A) e + (B).  Non è importante quale unità al centro scelga quale caso; pertanto, ci sono due possibilità per i cavi da A e B al centro.  Ora, se ogni unità al centro è Sì esattamente in un caso desiderato, entrambi i cavi dal centro in Z devono essere impostati su +; solo allora Z=Sì esattamente in due casi desiderati.

Per la prima risposta corretta, l’immagine sottostante mostra la funzione del rilevatore di conflitti. Si può vedere che l’unità superiore al centro rileva il caso A=Sì e B=No, quella inferiore il caso A=No e B=Sì. La rispettiva unità trasmette un segnale positivo a Z, e Z è Sì. Per gli altri ingressi (A=Sì e B=Sì così come A=No e B=No) entrambe le unità centrali sono No, Z non riceve alcun segnale positivo ed è quindi No.

{\centering%
\raisebox{-0.5ex}{\includesvg[width=108.2px]{\taskGraphicsFolder/graphics/graphics-new/-ita/2023-DE-01-explanation01_compatible-ita.svg}} \raisebox{-0.5ex}{\includesvg[width=108.2px]{\taskGraphicsFolder/graphics/graphics-new/-ita/2023-DE-01-explanation02_compatible-ita.svg}}

\raisebox{-0.5ex}{\includesvg[width=108.2px]{\taskGraphicsFolder/graphics/graphics-new/-ita/2023-DE-01-explanation03_compatible-ita.svg}} \raisebox{-0.5ex}{\includesvg[width=108.2px]{\taskGraphicsFolder/graphics/graphics-new/-ita/2023-DE-01-explanation04_compatible-ita.svg}}\par}


\subsection*{This is Informatics}

Il rilevatore di conflitti elabora due valori di ingresso (Sì o No) e restituisce l’uscita Sì esattamente quando i due valori di ingresso sono diversi. Questa funzione logica si chiama \enquote{OR esclusivo} (XOR, disgiunzione esclusiva). La prima macchina descritta in questo compito da Anna e Ben (due interruttori e un’unità di uscita) è una versione semplificata di un \emph{percettrone} descritto da Frank Rosenblatt nel $1957$. L’unità di uscita riproduce una cellula nervosa (neurone) in grado di elaborare i segnali di ingresso e produrre un segnale di uscita. Con un percettrone è possibile implementare le operazioni logiche AND e OR, ma non l’OR esclusivo. Per questo è necessario un altro strato di unità di commutazione, come nella soluzione di questo compito. Solo negli anni '$80$ è stato riconosciuto questo aspetto (ad esempio Rumelhart, Hinton \& Williams, $1986$) e in seguito è stato possibile programmare reti neurali artificiali che funzionano in modo simile al cervello umano e possono, ad esempio, valutare le immagini delle telecamere e riconoscere gli oggetti.  L’informatica ha sviluppato metodi per capire come reti neurali di grandi dimensioni con molti strati e unità possano eseguire i loro calcoli in modo efficiente.  Tali reti costituiscono la base di molti sistemi di IA (Intelligenza Artificiale) attuali.


\subsection*{This is Computational Thinking}

Dieser Abschnitt wird in diesem Jahr nicht bearbeitet.


\subsection*{Informatics Keywords and Websites}

\begin{itemize}
  \item Rumelhart, D. E., Hinton, G. E., \& Williams, R. J. ($1986$). Learning representations by back-propagating errors. Nature, $323$($6088$), 533$-536$: \href{http://www.cs.toronto.edu/~hinton/absps/naturebp.pdf}{\BrochureUrlText{http://www.cs.toronto.edu/\textasciitilde{}hinton/absps/naturebp.pdf}}
  \item Percettrone: \href{https://it.wikipedia.org/wiki/Percettrone}{\BrochureUrlText{https://it.wikipedia.org/wiki/Percettrone}}
  \item Disgiunzione esclusiva: \href{https://it.wikipedia.org/wiki/Disgiunzione_esclusiva}{\BrochureUrlText{https://it.wikipedia.org/wiki/Disgiunzione\_esclusiva}}
  \item Intelligenza artificiale: \href{https://it.wikipedia.org/wiki/Intelligenza_artificiale}{\BrochureUrlText{https://it.wikipedia.org/wiki/Intelligenza\_artificiale}}
\end{itemize}


\subsection*{Computational Thinking Keywords and Websites}

Dieser Abschnitt wird in diesem Jahr nicht bearbeitet.


\end{document}
