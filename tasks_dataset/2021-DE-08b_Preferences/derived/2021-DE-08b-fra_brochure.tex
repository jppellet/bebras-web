% Definition of the meta information: task difficulties, task ID, task title, task country; definition of the variables as well as their scope is in commands.tex
\setcounter{taskAgeDifficulty3to4}{0}
\setcounter{taskAgeDifficulty5to6}{0}
\setcounter{taskAgeDifficulty7to8}{2}
\setcounter{taskAgeDifficulty9to10}{1}
\setcounter{taskAgeDifficulty11to13}{0}
\renewcommand{\taskTitle}{Lieblingsgeschenk}
\renewcommand{\taskCountry}{DE}

% include this task only if for the age groups being processed this task is relevant
\ifthenelse{
  \(\boolean{age3to4} \AND \(\value{taskAgeDifficulty3to4} > 0\)\) \OR
  \(\boolean{age5to6} \AND \(\value{taskAgeDifficulty5to6} > 0\)\) \OR
  \(\boolean{age7to8} \AND \(\value{taskAgeDifficulty7to8} > 0\)\) \OR
  \(\boolean{age9to10} \AND \(\value{taskAgeDifficulty9to10} > 0\)\) \OR
  \(\boolean{age11to13} \AND \(\value{taskAgeDifficulty11to13} > 0\)\)}{

\newchapter{\taskTitle}

% task body
La famille castor a cinq cadeaux pour ses cinq enfants. Chaque enfant indique d’abord son cadeau favori, puis son second choix. Les cadeaux doivent être bien distribués:

\begin{enumerate}
  \item Le plus d’enfants possible doivent recevoir leur cadeau favori.
  \item Les autres enfants doivent recevoir leur second choix.
\end{enumerate}



% question (as \emph{})
{\em
Donne les bons cadeaux aux enfants.

{\centering%
\includesvg[width=288.6px]{\taskGraphicsFolder/graphics/2021-DE-08b-question-compatible.svg}\par}


}

% answer alternatives (as \begin{enumerate}[A)]) or interactivity


% from here on this is only included if solutions are processed
\ifthenelse{\boolean{solutions}}{
\newpage

% answer explanation
\section*{\BrochureSolution}
Voici la seule manière de distribuer les cadeaux en respectant les deux conditions.

{\centering%
\includesvg[width=288.6px]{\taskGraphicsFolder/graphics/2021-DE-08b-solution-compatible.svg}\par}

La distribution du graphique ci-dessus donne leur cadeau favori à quatre castors et son deuxième choix à un castor. Tous les castors ne peuvent pas recevoir leur cadeau favori, car deux castors ont le même. Il n’y a donc pas d’autre distribution donnant leur cadeau favori à plus de castors. Tu peux remarquer que si tu commences la distribution par le haut et donnes son cadeau favori au deuxième castor, le quatrième castor ne pourra avoir aucun de ses deux cadeaux préférés. Il ne suffit donc pas de donner à chaque castor le meilleur cadeau disponible dans cet exercice.

Une méthode de résolution consiste à distribuer d’abord tous les cadeaux qui ne sont le favori que d’un seul castor. Il ne reste ensuite que deux castors avec le même cadeau favori. On regarde ensuite lequel des deux deuxièmes choix est disponible, et on donne à l’autre castor son cadeau favori.



% it's informatics
\section*{\BrochureItsInformatics}
Dans cet exercice, nous avons affaire à un \emph{problème d’affectation} univoque: nous voulons affecter les cadeaux de manière à ce que tous les enfants recoivent un cadeau. Les enfants n’ont ici pas qu’un seul souhait, mais une liste de préférence. De tels problèmes d’affectation avec listes de préférence peuvent devenir très compliqués. L’informatique nous aide à résoudre de tels problèmes rapidement.

Une possibilité est de donner une valeur aux affectations: le cadeau favori a la valeur $1$ et le deuxième choix la valeur $2$. Un \emph{couplage} (\emph{matching} en anglais) est optimal s’il n’existe pas d’autre couplage avec plus de premiers choix distribués. Un tel couplage est appelé \emph{couplage parfait de poids minimum}. Il existe beaucoup de problèmes d’affectation. L’un deux est appelé \emph{problème des marriages stables} (\emph{Stable Marriage Problem} en anglais). Cela t’intéresse? Alors tu devrais étudier l’informatique!



% keywords and websites (as \begin{itemize})
\section*{\BrochureWebsitesAndKeywords}
{\raggedright
\begin{itemize}
  \item Problème d’affectation: \href{https://fr.wikipedia.org/wiki/Probl\%C3\%A8me_d\%27affectation}{\BrochureUrlText{https://fr.wikipedia.org/wiki/Problème\_d’affectation}}
  \item Couplage: \href{https://fr.wikipedia.org/wiki/Couplage_(th\%C3\%A9orie_des_graphes)}{\BrochureUrlText{https://fr.wikipedia.org/wiki/Couplage\_(théorie\_des\_graphes)}}
\end{itemize}


}

% end of ifthen for excluding the solutions
}{}

% all authors
% ATTENTION: you HAVE to make sure an according entry is in ../main/authors.tex.
% Syntax: \def\AuthorLastnameF{} (Lastname is last name, F is first letter of first name, this serves as a marker for ../main/authors.tex)
\def\AuthorPohlW{} % \ifdefined\AuthorPohlW \BrochureFlag{de}{} Wolfgang Pohl\fi
\def\AuthorVoborilF{} % \ifdefined\AuthorVoborilF \BrochureFlag{at}{} Florentina Voboril\fi
\def\AuthorKinciusV{} % \ifdefined\AuthorKinciusV \BrochureFlag{lt}{} Vaidotas Kinčius\fi
\def\AuthorPluharZ{} % \ifdefined\AuthorPluharZ \BrochureFlag{hu}{} Zsuzsa Pluhár\fi
\def\AuthorFreiF{} % \ifdefined\AuthorFreiF \BrochureFlag{ch}{} Fabian Frei\fi
\def\AuthorPelletE{} % \ifdefined\AuthorPelletE \BrochureFlag{ch}{} Elsa Pellet\fi

\newpage}{}
