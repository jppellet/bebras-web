\documentclass[a4paper,11pt]{report}
\usepackage[T1]{fontenc}
\usepackage[utf8]{inputenc}

\usepackage[french]{babel}
\frenchbsetup{ThinColonSpace=true}
\renewcommand*{\FBguillspace}{\hskip .4\fontdimen2\font plus .1\fontdimen3\font minus .3\fontdimen4\font \relax}
\AtBeginDocument{\def\labelitemi{$\bullet$}}

\usepackage{etoolbox}

\usepackage[margin=2cm]{geometry}
\usepackage{changepage}
\makeatletter
\renewenvironment{adjustwidth}[2]{%
    \begin{list}{}{%
    \partopsep\z@%
    \topsep\z@%
    \listparindent\parindent%
    \parsep\parskip%
    \@ifmtarg{#1}{\setlength{\leftmargin}{\z@}}%
                 {\setlength{\leftmargin}{#1}}%
    \@ifmtarg{#2}{\setlength{\rightmargin}{\z@}}%
                 {\setlength{\rightmargin}{#2}}%
    }
    \item[]}{\end{list}}
\makeatother

\newcommand{\BrochureUrlText}[1]{\texttt{#1}}
\usepackage{setspace}
\setstretch{1.15}

\usepackage{tabularx}
\usepackage{booktabs}
\usepackage{makecell}
\usepackage{multirow}
\renewcommand\theadfont{\bfseries}
\renewcommand{\tabularxcolumn}[1]{>{}m{#1}}
\newcolumntype{R}{>{\raggedleft\arraybackslash}X}
\newcolumntype{C}{>{\centering\arraybackslash}X}
\newcolumntype{L}{>{\raggedright\arraybackslash}X}
\newcolumntype{J}{>{\arraybackslash}X}

\newcommand{\BrochureInlineCode}[1]{{\ttfamily #1}}

\usepackage{amssymb}
\usepackage{amsmath}

\usepackage[babel=true,maxlevel=3]{csquotes}
\DeclareQuoteStyle{bebras-ch-eng}{“}[” ]{”}{‘}[”’ ]{’}\DeclareQuoteStyle{bebras-ch-deu}{«}[» ]{»}{“}[»› ]{”}
\DeclareQuoteStyle{bebras-ch-fra}{«\thinspace{}}[» ]{\thinspace{}»}{“}[»\thinspace{}› ]{”}
\DeclareQuoteStyle{bebras-ch-ita}{«}[» ]{»}{“}[»› ]{”}
\setquotestyle{bebras-ch-fra}

\usepackage{hyperref}
\usepackage{graphicx}
\usepackage{svg}
\svgsetup{inkscapeversion=1,inkscapearea=page}
\usepackage{wrapfig}

\usepackage{enumitem}
\setlist{nosep,itemsep=.5ex}

\setlength{\parindent}{0pt}
\setlength{\parskip}{2ex}
\raggedbottom

\usepackage{fancyhdr}
\usepackage{lastpage}
\pagestyle{fancy}

\fancyhf{}
\renewcommand{\headrulewidth}{0pt}
\renewcommand{\footrulewidth}{0.4pt}
\lfoot{\scriptsize © 2022 Bebras (CC BY-SA 4.0)}
\cfoot{\scriptsize\itshape 2022-CH-08 Carte au trésor}
\rfoot{\scriptsize Page~\thepage{}/\pageref*{LastPage}}

\newcommand{\taskGraphicsFolder}{..}

\begin{document}

\section*{\centering{} 2022-CH-08 Carte au trésor}


\subsection*{Body}

Bilbo le castor a deux bonnes cachettes pour ses réserves de nourriture. Sur une carte, il indique d’un \raisebox{\dimexpr -0.5ex +0.3ex \relax}[0pt][0pt]{\includesvg[width=10.8px]{\taskGraphicsFolder/graphics/2022-CH-08-inline_cross.svg}} les deux cases dans lequelles se trouvent ses cachette. Mais que faire si d’autres castors trouvent sa carte?

Pour brouiller les pistes, Bilbo marque d’autre cases d’un \raisebox{\dimexpr -0.5ex +0.3ex \relax}[0pt][0pt]{\includesvg[width=10.8px]{\taskGraphicsFolder/graphics/2022-CH-08-inline_cross.svg}}. Il le fait de telle manière que chaque ligne et chaque colonne de la carte contienne un nombre pair de cases avec un \raisebox{\dimexpr -0.5ex +0.3ex \relax}[0pt][0pt]{\includesvg[width=10.8px]{\taskGraphicsFolder/graphics/2022-CH-08-inline_cross.svg}}. Il efface ensuite les deux \raisebox{\dimexpr -0.5ex +0.3ex \relax}[0pt][0pt]{\includesvg[width=10.8px]{\taskGraphicsFolder/graphics/2022-CH-08-inline_cross.svg}} indiquant ses cachettes. Tu vois le résulat ci-dessous.

{\em


\subsection*{Question/Challenge - for the brochures}

Dans quelles cases se trouvent les cachettes de Bilbo?

{\centering%
\includesvg[scale=0.3]{\taskGraphicsFolder/graphics/2022-CH-08-taskbody.svg}\par}

}


\subsection*{Interactivity Instructions}

Clique sur les bonnes cases pour les marquer d’une croix. Clique à nouveau pour enlever la croix.

\begingroup
\renewcommand{\arraystretch}{1.5}
\subsection*{Answer Options/Interactivity Description}



\endgroup

\subsection*{Answer Explanation}

Voici les deux cachettes:

{\centering%
\includesvg[scale=0.3]{\taskGraphicsFolder/graphics/2022-CH-08-solution.svg}\par}

Pour les trouver, nous observons la carte de départ et constatons qu’il s’y trouve deux lignes et deux colonnes ne contenant pas un nombre pair de \raisebox{\dimexpr -0.5ex +0.3ex \relax}[0pt][0pt]{\includesvg[width=10.8px]{\taskGraphicsFolder/graphics/2022-CH-08-inline_cross.svg}}: les lignes $3$ et $6$ et les colonnes $3$ et $5$.

{\centering%
\includesvg[scale=0.3]{\taskGraphicsFolder/graphics/2022-CH-08-explanation1.svg}\par}

Les \raisebox{\dimexpr -0.5ex +0.3ex \relax}[0pt][0pt]{\includesvg[width=10.8px]{\taskGraphicsFolder/graphics/2022-CH-08-inline_cross.svg}} qui indiquent les cachettes ont été effacés. Nous savons qu’il doit y avoir un nombre pair de \raisebox{\dimexpr -0.5ex +0.3ex \relax}[0pt][0pt]{\includesvg[width=10.8px]{\taskGraphicsFolder/graphics/2022-CH-08-inline_cross.svg}} dans chaque ligne et chaque colonne une fois que les deux \raisebox{\dimexpr -0.5ex +0.3ex \relax}[0pt][0pt]{\includesvg[width=10.8px]{\taskGraphicsFolder/graphics/2022-CH-08-inline_cross.svg}} effacés y sont de nouveau dessinés.

Les lignes et colonnes concernées se croisent et ont quatre cases en commun (A, B, C et D). Ces “cases communes” sont spécialement intéressantes pour nous. Si nous mettions un \raisebox{\dimexpr -0.5ex +0.3ex \relax}[0pt][0pt]{\includesvg[width=10.8px]{\taskGraphicsFolder/graphics/2022-CH-08-inline_cross.svg}} dans un case autre qu’une case commune, nous pourrions obtenir un nombre pair de \raisebox{\dimexpr -0.5ex +0.3ex \relax}[0pt][0pt]{\includesvg[width=10.8px]{\taskGraphicsFolder/graphics/2022-CH-08-inline_cross.svg}} dans une colonne, mais un nombre impair dans une ligne et inversement. Les deux \raisebox{\dimexpr -0.5ex +0.3ex \relax}[0pt][0pt]{\includesvg[width=10.8px]{\taskGraphicsFolder/graphics/2022-CH-08-inline_cross.svg}} indiquant les cachettes doivent donc se trouver dans des cases communes.

La case commune B contient déjà un \raisebox{\dimexpr -0.5ex +0.3ex \relax}[0pt][0pt]{\includesvg[width=10.8px]{\taskGraphicsFolder/graphics/2022-CH-08-inline_cross.svg}} et ne peut donc pas être une cachette, car nous savons que Bilbo a effacé les \raisebox{\dimexpr -0.5ex +0.3ex \relax}[0pt][0pt]{\includesvg[width=10.8px]{\taskGraphicsFolder/graphics/2022-CH-08-inline_cross.svg}} des cases où se trouvent ses cachettes.

Pour obtenir un nombre pair de \raisebox{\dimexpr -0.5ex +0.3ex \relax}[0pt][0pt]{\includesvg[width=10.8px]{\taskGraphicsFolder/graphics/2022-CH-08-inline_cross.svg}} dans la ligne $2$, il doit donc y avoir un \raisebox{\dimexpr -0.5ex +0.3ex \relax}[0pt][0pt]{\includesvg[width=10.8px]{\taskGraphicsFolder/graphics/2022-CH-08-inline_cross.svg}} dans la case commune A. Une cachette s’y trouve. L’autre cachette ne peut pas se trouver dans la case commune C, car il y aurait alors trois \raisebox{\dimexpr -0.5ex +0.3ex \relax}[0pt][0pt]{\includesvg[width=10.8px]{\taskGraphicsFolder/graphics/2022-CH-08-inline_cross.svg}} dans la colonne correspondante. L’autre cachette se trouve donc dans la case commune D. Voici la carte avant que Bilbo n’efface les \raisebox{\dimexpr -0.5ex +0.3ex \relax}[0pt][0pt]{\includesvg[width=10.8px]{\taskGraphicsFolder/graphics/2022-CH-08-inline_cross.svg}} de ses cachettes, qui contient un nombre pair de \raisebox{\dimexpr -0.5ex +0.3ex \relax}[0pt][0pt]{\includesvg[width=10.8px]{\taskGraphicsFolder/graphics/2022-CH-08-inline_cross.svg}} dans chaque ligne et chaque colonne:

{\centering%
\includesvg[scale=0.3]{\taskGraphicsFolder/graphics/2022-CH-08-explanation2.svg}\par}


\subsection*{It’s Informatics}

Biblo utilise ici une astuce souvent utilisée en informatique: les \emph{bits de parité}. Ils font partie d’une série de techniques connues sous le nom de \emph{codes correcteurs}. Le principe est d’ajouter des bits à des données enregistrées ou transmises sous forme de séries de bits. Les bits ajoutés nous aident à determiner s’il y a eu des erreurs lors de la transmission ou de l’enregistrement de la série – typiquement, si un bit a été inversé, c’est à dire si un $1$ a été enregistré ou transmis au lieu d’un $0$ ou inversement.

Un exemple simple de code correcteur serait d’ajouter un bit de parité afin que le nombre de $1$ soit toujours pair. On ajouterait alors un $0$ à la série $0110101$, qui deviendrait $01101010$ (le nombre de $1$ resterait donc pair). Si le deuxième bit était inversé et la série $00101010$ était transmise, le message reçu ne remplirait plus la condition de parité (trois bits vaudraient $1$). Cette méthode ne peut cependant pas détecter les erreurs lorsque plusieurs bits sont inversée.

{\raggedright

\subsection*{Keywords and Websites}

\begin{itemize}
  \item Bit: \href{https://fr.wikipedia.org/wiki/Bit}{\BrochureUrlText{https://fr.wikipedia.org/wiki/Bit}}
  \item Bit de parité: \href{https://fr.wikipedia.org/wiki/Somme_de_contr\%C3\%B4le\#Exemple_:_bit_de_parit\%C3\%A9}{\BrochureUrlText{https://fr.wikipedia.org/wiki/Somme\_de\_contrôle\#Exemple\_:\_bit\_de\_parité}}
  \item Somme de contrôle: \href{https://fr.wikipedia.org/wiki/Somme_de_contr\%C3\%B4le}{\BrochureUrlText{https://fr.wikipedia.org/wiki/Somme\_de\_contrôle}}
  \item Code correcteur: \href{https://fr.wikipedia.org/wiki/Code_correcteur}{\BrochureUrlText{https://fr.wikipedia.org/wiki/Code\_correcteur}}
\end{itemize}


}
\end{document}
