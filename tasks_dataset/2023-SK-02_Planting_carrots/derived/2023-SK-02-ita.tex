\documentclass[a4paper,11pt]{report}
\usepackage[T1]{fontenc}
\usepackage[utf8]{inputenc}

\usepackage[italian]{babel}
\AtBeginDocument{\def\labelitemi{$\bullet$}}

\usepackage{etoolbox}

\usepackage[margin=2cm]{geometry}
\usepackage{changepage}
\makeatletter
\renewenvironment{adjustwidth}[2]{%
    \begin{list}{}{%
    \partopsep\z@%
    \topsep\z@%
    \listparindent\parindent%
    \parsep\parskip%
    \@ifmtarg{#1}{\setlength{\leftmargin}{\z@}}%
                 {\setlength{\leftmargin}{#1}}%
    \@ifmtarg{#2}{\setlength{\rightmargin}{\z@}}%
                 {\setlength{\rightmargin}{#2}}%
    }
    \item[]}{\end{list}}
\makeatother

\newcommand{\BrochureUrlText}[1]{\texttt{#1}}
\usepackage{setspace}
\setstretch{1.15}

\usepackage{tabularx}
\usepackage{booktabs}
\usepackage{makecell}
\usepackage{multirow}
\renewcommand\theadfont{\bfseries}
\renewcommand{\tabularxcolumn}[1]{>{}m{#1}}
\newcolumntype{R}{>{\raggedleft\arraybackslash}X}
\newcolumntype{C}{>{\centering\arraybackslash}X}
\newcolumntype{L}{>{\raggedright\arraybackslash}X}
\newcolumntype{J}{>{\arraybackslash}X}

\newcommand{\BrochureInlineCode}[1]{{\ttfamily #1}}

\usepackage{amssymb}
\usepackage{amsmath}

\usepackage[babel=true,maxlevel=3]{csquotes}
\DeclareQuoteStyle{bebras-ch-eng}{“}[” ]{”}{‘}[”’ ]{’}\DeclareQuoteStyle{bebras-ch-deu}{«}[» ]{»}{“}[»› ]{”}
\DeclareQuoteStyle{bebras-ch-fra}{«\thinspace{}}[» ]{\thinspace{}»}{“}[»\thinspace{}› ]{”}
\DeclareQuoteStyle{bebras-ch-ita}{«}[» ]{»}{“}[»› ]{”}
\setquotestyle{bebras-ch-ita}

\usepackage{hyperref}
\usepackage{graphicx}
\usepackage{svg}
\svgsetup{inkscapeversion=1,inkscapearea=page}
\usepackage{wrapfig}

\usepackage{enumitem}
\setlist{nosep,itemsep=.5ex}

\setlength{\parindent}{0pt}
\setlength{\parskip}{2ex}
\raggedbottom

\usepackage{fancyhdr}
\usepackage{lastpage}
\pagestyle{fancy}

\fancyhf{}
\renewcommand{\headrulewidth}{0pt}
\renewcommand{\footrulewidth}{0.4pt}
\lfoot{\scriptsize © 2023 Bebras (CC BY-SA 4.0)}
\cfoot{\scriptsize\itshape 2023-SK-02 Pianta di carote}
\rfoot{\scriptsize Page~\thepage{}/\pageref*{LastPage}}

\newcommand{\taskGraphicsFolder}{..}

\begin{document}

\section*{\centering{} 2023-SK-02 Pianta di carote}


\subsection*{Body}

Il robot coniglio può eseguire le seguenti istruzioni:

\raisebox{-0.5ex}{\includesvg[width=50.5px]{\taskGraphicsFolder/graphics/2023-SK-02_L.svg}} Salta a \textbf{sinistra} sulla collina successiva.

\raisebox{-0.5ex}{\includesvg[width=50.5px]{\taskGraphicsFolder/graphics/2023-SK-02_R.svg}} Salta a \textbf{destra} sulla collina successiva.

\raisebox{-0.5ex}{\includesvg[width=50.5px]{\taskGraphicsFolder/graphics/2023-SK-02_seed.svg}} \textbf{Pianta} un seme di carota sulla collina su cui ti trovi.

Il robot coniglio ha eseguito questa sequenza di istruzioni:

{\centering%
\includesvg[width=1\linewidth]{\taskGraphicsFolder/graphics/2023-SK-02_prog.svg}\par}

Nel corso del processo, il robot è salito su quattro colline.
Ma non sappiamo da quale collina sia partito.

{\em


\subsection*{Question/Challenge - for the brochures}

Su quali colline il robot ha piantato i semi di carota?

{\centering%
\includesvg[scale=0.3]{\taskGraphicsFolder/graphics/2023-SK-02_mounds.svg}\par}

}


\subsection*{Interactivity instruction - for the online challenge}

Pianta i semi sulle colline giuste. Al termine, fa clic su \enquote{Salva risposta}.

\begingroup
\renewcommand{\arraystretch}{1.5}
\subsection*{Answer Options/Interactivity Description}



\endgroup

\subsection*{Answer Explanation}

La risposta corretta è \raisebox{-0.5ex}{\includesvg[scale=0.3]{\taskGraphicsFolder/graphics/2023-SK-02-solution-compatible.svg}}

Per spiegare meglio la risposta corretta, diamo lettere alle colline (vedi sopra) e numeri alle istruzioni:

{\centering%
\includesvg[width=1\linewidth]{\taskGraphicsFolder/graphics/2023-SK-02_explanation-compatible.svg}\par}

Per prima cosa determiniamo il punto di partenza del robot: poiché il robot salta a sinistra per tre volte di seguito (istruzioni $3$, $5$, $6$), deve trovarsi prima sulla collina D. Prima di saltare tre volte a sinistra, salta una volta a destra (istruzione $1$). Il robot è quindi partito dalla collina C.
Di conseguenza, i semi di carota - secondo le istruzioni $2$, $4$ e $7$ - vengono piantati prima sulla collina D, poi sulla collina C e infine sulla collina A.


\subsection*{This is Informatics}

I robot reali hanno computer incorporati e sono \emph{programmati} proprio come il robot coniglio. Il programma di un computer è composto da molte singole \emph{istruzioni}.

Nel nostro caso, la sequenza di istruzioni per il computer robot è specificata con l’aiuto di blocchi immagine. Il risultato (\emph{output}) del programma dipende non solo dalla posizione di partenza (\emph{input}), ma anche dalla sequenza e dall’ordine delle istruzioni.

Questo compito mostra un esempio dell’uso dei robot in agricoltura. I robot possono non solo piantare, ma anche irrigare, impollinare o distribuire pesticidi in modo mirato.


\subsection*{This is Computational Thinking}

Optional - not to be filled 2023


\subsection*{Informatics Keywords and Websites}

\begin{itemize}
  \item Algoritmo: \href{https://it.wikipedia.org/wiki/Algoritmo}{\BrochureUrlText{https://it.wikipedia.org/wiki/Algoritmo}}
  \item Istruzione: \href{https://it.wikipedia.org/wiki/Istruzione_(informatica)}{\BrochureUrlText{https://it.wikipedia.org/wiki/Istruzione\_(informatica)}}
  \item Smart farming: \href{https://www.agroscope.admin.ch/agroscope/it/home/temi/economia-tecnologia/smart-farming.html}{\BrochureUrlText{https://www.agroscope.admin.ch/agroscope/it/home/temi/economia-tecnologia/smart-farming.html}}
  \item I robot in agricoltura: \href{https://cordis.europa.eu/article/id/441912-robots-help-farmers-say-goodbye-to-repetitive-tasks/it}{\BrochureUrlText{https://cordis.europa.eu/article/id/441912-robots-help-farmers-say-goodbye-to-repetitive-tasks/it}}
\end{itemize}


\subsection*{Computational Thinking Keywords and Websites}

\begin{itemize}
  \item Modelling and Simulation,
  \item Evaluation
\end{itemize}


\end{document}
