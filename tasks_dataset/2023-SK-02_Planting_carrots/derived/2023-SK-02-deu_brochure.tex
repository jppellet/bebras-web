% Definition of the meta information: task difficulties, task ID, task title, task country; definition of the variables as well as their scope is in commands.tex
\setcounter{taskAgeDifficulty3to4}{3}
\setcounter{taskAgeDifficulty5to6}{0}
\setcounter{taskAgeDifficulty7to8}{1}
\setcounter{taskAgeDifficulty9to10}{0}
\setcounter{taskAgeDifficulty11to13}{0}
\renewcommand{\taskTitle}{Rüebli pflanzen}
\renewcommand{\taskCountry}{SK}

% include this task only if for the age groups being processed this task is relevant
\ifthenelse{
  \(\boolean{age3to4} \AND \(\value{taskAgeDifficulty3to4} > 0\)\) \OR
  \(\boolean{age5to6} \AND \(\value{taskAgeDifficulty5to6} > 0\)\) \OR
  \(\boolean{age7to8} \AND \(\value{taskAgeDifficulty7to8} > 0\)\) \OR
  \(\boolean{age9to10} \AND \(\value{taskAgeDifficulty9to10} > 0\)\) \OR
  \(\boolean{age11to13} \AND \(\value{taskAgeDifficulty11to13} > 0\)\)}{

\newchapter{\taskTitle}

% task body
Der Kaninchenroboter kann folgende Anweisungen ausführen:

\raisebox{-0.5ex}{\includesvg[width=50.5px]{\taskGraphicsFolder/graphics/2023-SK-02_L.svg}} Springe nach \textbf{links} auf den nächsten Hügel.

\raisebox{-0.5ex}{\includesvg[width=50.5px]{\taskGraphicsFolder/graphics/2023-SK-02_R.svg}} Springe nach \textbf{rechts} auf den nächsten Hügel.

\raisebox{-0.5ex}{\includesvg[width=50.5px]{\taskGraphicsFolder/graphics/2023-SK-02_seed.svg}} \textbf{Pflanze} einen Karottensamen auf den Hügel, auf dem du stehst.

Der Kaninchenroboter hat diese Folge von Anweisungen ausgeführt:

{\centering%
\includesvg[width=1\linewidth]{\taskGraphicsFolder/graphics/2023-SK-02_prog.svg}\par}

Dabei ist der Roboter auf vier Hügeln gewesen.
Wir wissen aber nicht, auf welchem Hügel er angefangen hat.



% question (as \emph{})
{\em
Auf welche Hügel hat der Roboter die Rüeblisamen gepflanzt?

{\centering%
\includesvg[scale=0.3]{\taskGraphicsFolder/graphics/2023-SK-02_mounds.svg}\par}


}

% answer alternatives (as \begin{enumerate}[A)]) or interactivity




% from here on this is only included if solutions are processed
\ifthenelse{\boolean{solutions}}{
\newpage

% answer explanation
\section*{\BrochureSolution}
So ist es richtig: \raisebox{-0.5ex}{\includesvg[scale=0.3]{\taskGraphicsFolder/graphics/2023-SK-02-solution-compatible.svg}}

Um die richtige Antwort besser erklären zu können, geben wir den Hügeln Buchstaben (siehe oben) und den Anweisungen Nummern:

{\centering%
\includesvg[width=1\linewidth]{\taskGraphicsFolder/graphics/2023-SK-02_explanation-compatible.svg}\par}

Zuerst bestimmen wir den Startpunkt des Roboters: Da der Roboter dreimal hintereinander nach links springt (Anweisungen $3$, $5$, $6$), muss er vorher auf Hügel D stehen. Bevor er dreimal nach links springt, springt er einmal nach rechts (Anweisung $1$). Der Roboter hat also auf Hügel C angefangen.
Folglich werden die Rüeblisamen – den Anweisungen $2$, $4$ und $7$ entsprechend – zuerst auf Hügel D, dann auf Hügel C und zuletzt auf Hügel A gepflanzt.



% it's informatics
\section*{\BrochureItsInformatics}
Echte Roboter haben eingebaute Computer, und die werden so ähnlich \emph{programmiert} wie der Kaninchenroboter. Ein Computerprogramm besteht aus vielen einzelnen \emph{Anweisungen}.

In unserem Fall wird die Abfolge der Anweisungen für den Roboter-Computer mit Hilfe von Bildblöcken angegeben. Das Ergebnis (\emph{Output}) des Programms hängt nicht nur von der Startposition (\emph{Input}), sondern auch von der Folge und Reihenfolge der Anweisungen ab.

Diese Biberaufgabe zeigt ein Beispiel für den Einsatz von Robotern in der Landwirtschaft. Roboter können nicht nur pflanzen, sondern auch bewässern, bestäuben oder Pflanzenschutzmittel gezielt verteilen.



% keywords and websites (as \begin{itemize})
\section*{\BrochureWebsitesAndKeywords}
{\raggedright
\begin{itemize}
  \item Algorithmus: \href{https://de.wikipedia.org/wiki/Algorithmus}{\BrochureUrlText{https://de.wikipedia.org/wiki/Algorithmus}}
  \item Anweisungen: \href{https://de.wikipedia.org/wiki/Anweisung_(Programmierung)}{\BrochureUrlText{https://de.wikipedia.org/wiki/Anweisung\_(Programmierung)}}
  \item Smart Farming: \href{https://de.wikipedia.org/wiki/Smart_Farming}{\BrochureUrlText{https://de.wikipedia.org/wiki/Smart\_Farming}}, \href{https://www.agroscope.admin.ch/agroscope/de/home/themen/wirtschaft-technik/smart-farming.html}{\BrochureUrlText{https://www.agroscope.admin.ch/agroscope/de/home/themen/wirtschaft-technik/smart-farming.html}}
  \item Roboter in der Landwirtschaft: \href{https://cordis.europa.eu/article/id/441912-robots-help-farmers-say-goodbye-to-repetitive-tasks/de}{\BrochureUrlText{https://cordis.europa.eu/article/id/441912-robots-help-farmers-say-goodbye-to-repetitive-tasks/de}}
\end{itemize}


}

% end of ifthen for excluding the solutions
}{}

% all authors
% ATTENTION: you HAVE to make sure an according entry is in ../main/authors.tex.
% Syntax: \def\AuthorLastnameF{} (Lastname is last name, F is first letter of first name, this serves as a marker for ../main/authors.tex)
\def\AuthorTomcsanyiovaM{} % \ifdefined\AuthorTomcsanyiovaM \BrochureFlag{sk}{} Monika Tomcsányiová\fi
\def\AuthorStupurieneG{} % \ifdefined\AuthorStupurieneG \BrochureFlag{lt}{} Gabrielė Stupurienė\fi
\def\AuthorAlmajhadE{} % \ifdefined\AuthorAlmajhadE \BrochureFlag{sa}{} Esraa Almajhad\fi
\def\AuthorGuraS{} % \ifdefined\AuthorGuraS \BrochureFlag{sk}{} Štefan Gura\fi
\def\AuthorPluharZ{} % \ifdefined\AuthorPluharZ \BrochureFlag{hu}{} Zsuzsa Pluhár\fi
\def\AuthorDatzkoThutS{} % \ifdefined\AuthorDatzkoThutS \BrochureFlag{de}{} Susanne Datzko-Thut\fi
\def\AuthorSchluterK{} % \ifdefined\AuthorSchluterK \BrochureFlag{de}{} Kirsten Schlüter\fi

\newpage}{}
