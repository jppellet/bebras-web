\documentclass[a4paper,11pt]{report}
\usepackage[T1]{fontenc}
\usepackage[utf8]{inputenc}

\usepackage[german]{babel}
\AtBeginDocument{\def\labelitemi{$\bullet$}}

\usepackage{etoolbox}

\usepackage[margin=2cm]{geometry}
\usepackage{changepage}
\makeatletter
\renewenvironment{adjustwidth}[2]{%
    \begin{list}{}{%
    \partopsep\z@%
    \topsep\z@%
    \listparindent\parindent%
    \parsep\parskip%
    \@ifmtarg{#1}{\setlength{\leftmargin}{\z@}}%
                 {\setlength{\leftmargin}{#1}}%
    \@ifmtarg{#2}{\setlength{\rightmargin}{\z@}}%
                 {\setlength{\rightmargin}{#2}}%
    }
    \item[]}{\end{list}}
\makeatother

\newcommand{\BrochureUrlText}[1]{\texttt{#1}}
\usepackage{setspace}
\setstretch{1.15}

\usepackage{tabularx}
\usepackage{booktabs}
\usepackage{makecell}
\usepackage{multirow}
\renewcommand\theadfont{\bfseries}
\renewcommand{\tabularxcolumn}[1]{>{}m{#1}}
\newcolumntype{R}{>{\raggedleft\arraybackslash}X}
\newcolumntype{C}{>{\centering\arraybackslash}X}
\newcolumntype{L}{>{\raggedright\arraybackslash}X}
\newcolumntype{J}{>{\arraybackslash}X}

\newcommand{\BrochureInlineCode}[1]{{\ttfamily #1}}

\usepackage{amssymb}
\usepackage{amsmath}

\usepackage[babel=true,maxlevel=3]{csquotes}
\DeclareQuoteStyle{bebras-ch-eng}{“}[” ]{”}{‘}[”’ ]{’}\DeclareQuoteStyle{bebras-ch-deu}{«}[» ]{»}{“}[»› ]{”}
\DeclareQuoteStyle{bebras-ch-fra}{«\thinspace{}}[» ]{\thinspace{}»}{“}[»\thinspace{}› ]{”}
\DeclareQuoteStyle{bebras-ch-ita}{«}[» ]{»}{“}[»› ]{”}
\setquotestyle{bebras-ch-deu}

\usepackage{hyperref}
\usepackage{graphicx}
\usepackage{svg}
\svgsetup{inkscapeversion=1,inkscapearea=page}
\usepackage{wrapfig}

\usepackage{enumitem}
\setlist{nosep,itemsep=.5ex}

\setlength{\parindent}{0pt}
\setlength{\parskip}{2ex}
\raggedbottom

\usepackage{fancyhdr}
\usepackage{lastpage}
\pagestyle{fancy}

\fancyhf{}
\renewcommand{\headrulewidth}{0pt}
\renewcommand{\footrulewidth}{0.4pt}
\lfoot{\scriptsize © 2023 Bebras (CC BY-SA 4.0)}
\cfoot{\scriptsize\itshape 2023-SK-02 Rüebli pflanzen}
\rfoot{\scriptsize Page~\thepage{}/\pageref*{LastPage}}

\newcommand{\taskGraphicsFolder}{..}

\begin{document}

\section*{\centering{} 2023-SK-02 Rüebli pflanzen}


\subsection*{Body}

Der Kaninchenroboter kann folgende Anweisungen ausführen:

\raisebox{-0.5ex}{\includesvg[width=50.5px]{\taskGraphicsFolder/graphics/2023-SK-02_L.svg}} Springe nach \textbf{links} auf den nächsten Hügel.

\raisebox{-0.5ex}{\includesvg[width=50.5px]{\taskGraphicsFolder/graphics/2023-SK-02_R.svg}} Springe nach \textbf{rechts} auf den nächsten Hügel.

\raisebox{-0.5ex}{\includesvg[width=50.5px]{\taskGraphicsFolder/graphics/2023-SK-02_seed.svg}} \textbf{Pflanze} einen Karottensamen auf den Hügel, auf dem du stehst.

Der Kaninchenroboter hat diese Folge von Anweisungen ausgeführt:

{\centering%
\includesvg[width=1\linewidth]{\taskGraphicsFolder/graphics/2023-SK-02_prog.svg}\par}

Dabei ist der Roboter auf vier Hügeln gewesen.
Wir wissen aber nicht, auf welchem Hügel er angefangen hat.

{\em


\subsection*{Question/Challenge - for the brochures}

Auf welche Hügel hat der Roboter die Rüeblisamen gepflanzt?

{\centering%
\includesvg[scale=0.3]{\taskGraphicsFolder/graphics/2023-SK-02_mounds.svg}\par}

}


\subsection*{Interactivity instruction - for the online challenge}

Ziehe die Samen auf die richtigen Hügel. Wenn du fertig bist, klicke \enquote{Antwort speichern}.

\begingroup
\renewcommand{\arraystretch}{1.5}
\subsection*{Answer Options/Interactivity Description}



\endgroup

\subsection*{Answer Explanation}

So ist es richtig: \raisebox{-0.5ex}{\includesvg[scale=0.3]{\taskGraphicsFolder/graphics/2023-SK-02-solution-compatible.svg}}

Um die richtige Antwort besser erklären zu können, geben wir den Hügeln Buchstaben (siehe oben) und den Anweisungen Nummern:

{\centering%
\includesvg[width=1\linewidth]{\taskGraphicsFolder/graphics/2023-SK-02_explanation-compatible.svg}\par}

Zuerst bestimmen wir den Startpunkt des Roboters: Da der Roboter dreimal hintereinander nach links springt (Anweisungen $3$, $5$, $6$), muss er vorher auf Hügel D stehen. Bevor er dreimal nach links springt, springt er einmal nach rechts (Anweisung $1$). Der Roboter hat also auf Hügel C angefangen.
Folglich werden die Rüeblisamen – den Anweisungen $2$, $4$ und $7$ entsprechend – zuerst auf Hügel D, dann auf Hügel C und zuletzt auf Hügel A gepflanzt.


\subsection*{This is Informatics}

Echte Roboter haben eingebaute Computer, und die werden so ähnlich \emph{programmiert} wie der Kaninchenroboter. Ein Computerprogramm besteht aus vielen einzelnen \emph{Anweisungen}.

In unserem Fall wird die Abfolge der Anweisungen für den Roboter-Computer mit Hilfe von Bildblöcken angegeben. Das Ergebnis (\emph{Output}) des Programms hängt nicht nur von der Startposition (\emph{Input}), sondern auch von der Folge und Reihenfolge der Anweisungen ab.

Diese Biberaufgabe zeigt ein Beispiel für den Einsatz von Robotern in der Landwirtschaft. Roboter können nicht nur pflanzen, sondern auch bewässern, bestäuben oder Pflanzenschutzmittel gezielt verteilen.


\subsection*{This is Computational Thinking}

Optional - not to be filled 2023


\subsection*{Informatics Keywords and Websites}

\begin{itemize}
  \item Algorithmus: \href{https://de.wikipedia.org/wiki/Algorithmus}{\BrochureUrlText{https://de.wikipedia.org/wiki/Algorithmus}}
  \item Anweisungen: \href{https://de.wikipedia.org/wiki/Anweisung_(Programmierung)}{\BrochureUrlText{https://de.wikipedia.org/wiki/Anweisung\_(Programmierung)}}
  \item Smart Farming: \href{https://de.wikipedia.org/wiki/Smart_Farming}{\BrochureUrlText{https://de.wikipedia.org/wiki/Smart\_Farming}}, \href{https://www.agroscope.admin.ch/agroscope/de/home/themen/wirtschaft-technik/smart-farming.html}{\BrochureUrlText{https://www.agroscope.admin.ch/agroscope/de/home/themen/wirtschaft-technik/smart-farming.html}}
  \item Roboter in der Landwirtschaft: \href{https://cordis.europa.eu/article/id/441912-robots-help-farmers-say-goodbye-to-repetitive-tasks/de}{\BrochureUrlText{https://cordis.europa.eu/article/id/441912-robots-help-farmers-say-goodbye-to-repetitive-tasks/de}}
\end{itemize}


\subsection*{Computational Thinking Keywords and Websites}

\begin{itemize}
  \item Modelling and Simulation,
  \item Evaluation
\end{itemize}


\end{document}
