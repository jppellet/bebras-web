% Definition of the meta information: task difficulties, task ID, task title, task country; definition of the variables as well as their scope is in commands.tex
\setcounter{taskAgeDifficulty3to4}{3}
\setcounter{taskAgeDifficulty5to6}{0}
\setcounter{taskAgeDifficulty7to8}{1}
\setcounter{taskAgeDifficulty9to10}{0}
\setcounter{taskAgeDifficulty11to13}{0}
\renewcommand{\taskTitle}{Graines de carottes}
\renewcommand{\taskCountry}{SK}

% include this task only if for the age groups being processed this task is relevant
\ifthenelse{
  \(\boolean{age3to4} \AND \(\value{taskAgeDifficulty3to4} > 0\)\) \OR
  \(\boolean{age5to6} \AND \(\value{taskAgeDifficulty5to6} > 0\)\) \OR
  \(\boolean{age7to8} \AND \(\value{taskAgeDifficulty7to8} > 0\)\) \OR
  \(\boolean{age9to10} \AND \(\value{taskAgeDifficulty9to10} > 0\)\) \OR
  \(\boolean{age11to13} \AND \(\value{taskAgeDifficulty11to13} > 0\)\)}{

\newchapter{\taskTitle}

% task body
Le robot lapin peut exécuter les instructions suivantes:

\raisebox{-0.5ex}{\includesvg[width=50.5px]{\taskGraphicsFolder/graphics/2023-SK-02_L.svg}} Saute vers la \textbf{gauche} sur la prochaine colline

\raisebox{-0.5ex}{\includesvg[width=50.5px]{\taskGraphicsFolder/graphics/2023-SK-02_R.svg}} Saute vers la \textbf{droite} sur la prochaine colline

\raisebox{-0.5ex}{\includesvg[width=50.5px]{\taskGraphicsFolder/graphics/2023-SK-02_seed.svg}} \textbf{Plante} une graine de carotte sur la colline sur laquelle tu es

Le robot lapin a suivi la suite d’instructions suivante:

{\centering%
\includesvg[width=1\linewidth]{\taskGraphicsFolder/graphics/2023-SK-02_prog.svg}\par}

Il a passé sur quatre collines. Nous ne savons pas sur quelle colline il a commencé.



% question (as \emph{})
{\em
Sur quelles collines le robot a-t-il planté des graines de carottes?

{\centering%
\includesvg[scale=0.3]{\taskGraphicsFolder/graphics/2023-SK-02_mounds.svg}\par}


}

% answer alternatives (as \begin{enumerate}[A)]) or interactivity




% from here on this is only included if solutions are processed
\ifthenelse{\boolean{solutions}}{
\newpage

% answer explanation
\section*{\BrochureSolution}
Voici la bonne réponse: \raisebox{-0.5ex}{\includesvg[scale=0.3]{\taskGraphicsFolder/graphics/2023-SK-02-solution-compatible.svg}}

Pour mieux pouvoir expliquer la bonne réponse, nous donnons des lettres aux collines (voir ci-dessus) et des numéros aux instructions:

{\centering%
\includesvg[width=1\linewidth]{\taskGraphicsFolder/graphics/2023-SK-02_explanation-compatible.svg}\par}

Nous commençons par déterminer le point de départ du robot: avant de sauter trois fois de suite vers la gauche (instructions $3$, $5$ et $6$), il doit se trouver sur la colline D. Avant cela, il saute une fois vers la droite (instruction $1$). Le robot a donc commencé sur la colline C. Les graines de carottes sont donc plantées sur les collines D, puis C et finalement A, d’après les instructions $2$, $4$ et $7$.



% it's informatics
\section*{\BrochureItsInformatics}
Les vrais robots ont des ordinateurs intégrés, et ils sont programmés de manière similaire au robot lapin. Un programme informatique est constitué de plusieurs \emph{instructions} individuelles.

Dans notre cas, la suite d’instructions pour l’ordinateur du robot est donnée sous forme d’images. Le résultat (\emph{sortie}, \emph{output} en anglais) du programme ne dépend pas que de la position de départ (\emph{entrée}, \emph{input} en anglais), mais aussi des instructions et de l’ordre dans lequel elles sont données.

Cet exercice du Castor illustre l’utilisation de robot dans l’agriculture. Les robots ne peuvent pas seulement planter, mais aussi arroser, polliniser et répandre des produits de protection de manière ciblée.



% keywords and websites (as \begin{itemize})
\section*{\BrochureWebsitesAndKeywords}
{\raggedright
\begin{itemize}
  \item Algorithme: \href{https://fr.wikipedia.org/wiki/Algorithme}{\BrochureUrlText{https://fr.wikipedia.org/wiki/Algorithme}}
  \item Instruction: \href{https://fr.wikipedia.org/wiki/Instruction_informatique}{\BrochureUrlText{https://fr.wikipedia.org/wiki/Instruction\_informatique}}
  \item Smart Farming: \href{https://www.agroscope.admin.ch/agroscope/fr/home/themes/economie-technique/smart-farming.html}{\BrochureUrlText{https://www.agroscope.admin.ch/agroscope/fr/home/themes/economie-technique/smart-farming.html}}
  \item Les robots et l’agriculture: \href{https://cordis.europa.eu/article/id/441912-robots-help-farmers-say-goodbye-to-repetitive-tasks/fr}{\BrochureUrlText{https://cordis.europa.eu/article/id/441912-robots-help-farmers-say-goodbye-to-repetitive-tasks/fr}}
\end{itemize}


}

% end of ifthen for excluding the solutions
}{}

% all authors
% ATTENTION: you HAVE to make sure an according entry is in ../main/authors.tex.
% Syntax: \def\AuthorLastnameF{} (Lastname is last name, F is first letter of first name, this serves as a marker for ../main/authors.tex)
\def\AuthorTomcsanyiovaM{} % \ifdefined\AuthorTomcsanyiovaM \BrochureFlag{sk}{} Monika Tomcsányiová\fi
\def\AuthorStupurieneG{} % \ifdefined\AuthorStupurieneG \BrochureFlag{lt}{} Gabrielė Stupurienė\fi
\def\AuthorAlmajhadE{} % \ifdefined\AuthorAlmajhadE \BrochureFlag{sa}{} Esraa Almajhad\fi
\def\AuthorGuraS{} % \ifdefined\AuthorGuraS \BrochureFlag{sk}{} Štefan Gura\fi
\def\AuthorPluharZ{} % \ifdefined\AuthorPluharZ \BrochureFlag{hu}{} Zsuzsa Pluhár\fi
\def\AuthorDatzkoThutS{} % \ifdefined\AuthorDatzkoThutS \BrochureFlag{de}{} Susanne Datzko-Thut\fi
\def\AuthorSchluterK{} % \ifdefined\AuthorSchluterK \BrochureFlag{de}{} Kirsten Schlüter\fi
\def\AuthorPelletE{} % \ifdefined\AuthorPelletE \BrochureFlag{ch}{} Elsa Pellet\fi

\newpage}{}
