\documentclass[a4paper,11pt]{report}
\usepackage[T1]{fontenc}
\usepackage[utf8]{inputenc}

\usepackage[french]{babel}
\frenchbsetup{ThinColonSpace=true}
\renewcommand*{\FBguillspace}{\hskip .4\fontdimen2\font plus .1\fontdimen3\font minus .3\fontdimen4\font \relax}
\AtBeginDocument{\def\labelitemi{$\bullet$}}

\usepackage{etoolbox}

\usepackage[margin=2cm]{geometry}
\usepackage{changepage}
\makeatletter
\renewenvironment{adjustwidth}[2]{%
    \begin{list}{}{%
    \partopsep\z@%
    \topsep\z@%
    \listparindent\parindent%
    \parsep\parskip%
    \@ifmtarg{#1}{\setlength{\leftmargin}{\z@}}%
                 {\setlength{\leftmargin}{#1}}%
    \@ifmtarg{#2}{\setlength{\rightmargin}{\z@}}%
                 {\setlength{\rightmargin}{#2}}%
    }
    \item[]}{\end{list}}
\makeatother

\newcommand{\BrochureUrlText}[1]{\texttt{#1}}
\usepackage{setspace}
\setstretch{1.15}

\usepackage{tabularx}
\usepackage{booktabs}
\usepackage{makecell}
\usepackage{multirow}
\renewcommand\theadfont{\bfseries}
\renewcommand{\tabularxcolumn}[1]{>{}m{#1}}
\newcolumntype{R}{>{\raggedleft\arraybackslash}X}
\newcolumntype{C}{>{\centering\arraybackslash}X}
\newcolumntype{L}{>{\raggedright\arraybackslash}X}
\newcolumntype{J}{>{\arraybackslash}X}

\newcommand{\BrochureInlineCode}[1]{{\ttfamily #1}}

\usepackage{amssymb}
\usepackage{amsmath}

\usepackage[babel=true,maxlevel=3]{csquotes}
\DeclareQuoteStyle{bebras-ch-eng}{“}[” ]{”}{‘}[”’ ]{’}\DeclareQuoteStyle{bebras-ch-deu}{«}[» ]{»}{“}[»› ]{”}
\DeclareQuoteStyle{bebras-ch-fra}{«\thinspace{}}[» ]{\thinspace{}»}{“}[»\thinspace{}› ]{”}
\DeclareQuoteStyle{bebras-ch-ita}{«}[» ]{»}{“}[»› ]{”}
\setquotestyle{bebras-ch-fra}

\usepackage{hyperref}
\usepackage{graphicx}
\usepackage{svg}
\svgsetup{inkscapeversion=1,inkscapearea=page}
\usepackage{wrapfig}

\usepackage{enumitem}
\setlist{nosep,itemsep=.5ex}

\setlength{\parindent}{0pt}
\setlength{\parskip}{2ex}
\raggedbottom

\usepackage{fancyhdr}
\usepackage{lastpage}
\pagestyle{fancy}

\fancyhf{}
\renewcommand{\headrulewidth}{0pt}
\renewcommand{\footrulewidth}{0.4pt}
\lfoot{\scriptsize © 2023 Bebras (CC BY-SA 4.0)}
\cfoot{\scriptsize\itshape 2023-SK-02 Graines de carottes}
\rfoot{\scriptsize Page~\thepage{}/\pageref*{LastPage}}

\newcommand{\taskGraphicsFolder}{..}

\begin{document}

\section*{\centering{} 2023-SK-02 Graines de carottes}


\subsection*{Body}

Le robot lapin peut exécuter les instructions suivantes:

\raisebox{-0.5ex}{\includesvg[width=50.5px]{\taskGraphicsFolder/graphics/2023-SK-02_L.svg}} Saute vers la \textbf{gauche} sur la prochaine colline

\raisebox{-0.5ex}{\includesvg[width=50.5px]{\taskGraphicsFolder/graphics/2023-SK-02_R.svg}} Saute vers la \textbf{droite} sur la prochaine colline

\raisebox{-0.5ex}{\includesvg[width=50.5px]{\taskGraphicsFolder/graphics/2023-SK-02_seed.svg}} \textbf{Plante} une graine de carotte sur la colline sur laquelle tu es

Le robot lapin a suivi la suite d’instructions suivante:

{\centering%
\includesvg[width=1\linewidth]{\taskGraphicsFolder/graphics/2023-SK-02_prog.svg}\par}

Il a passé sur quatre collines. Nous ne savons pas sur quelle colline il a commencé.

{\em


\subsection*{Question/Challenge - for the brochures}

Sur quelles collines le robot a-t-il planté des graines de carottes?

{\centering%
\includesvg[scale=0.3]{\taskGraphicsFolder/graphics/2023-SK-02_mounds.svg}\par}

}


\subsection*{Interactivity instruction - for the online challenge}

Glisse les graines sur les bonnes collines. Quand tu as fini, clique sur “Enregistrer la réponse”.

\begingroup
\renewcommand{\arraystretch}{1.5}
\subsection*{Answer Options/Interactivity Description}



\endgroup

\subsection*{Answer Explanation}

Voici la bonne réponse: \raisebox{-0.5ex}{\includesvg[scale=0.3]{\taskGraphicsFolder/graphics/2023-SK-02-solution-compatible.svg}}

Pour mieux pouvoir expliquer la bonne réponse, nous donnons des lettres aux collines (voir ci-dessus) et des numéros aux instructions:

{\centering%
\includesvg[width=1\linewidth]{\taskGraphicsFolder/graphics/2023-SK-02_explanation-compatible.svg}\par}

Nous commençons par déterminer le point de départ du robot: avant de sauter trois fois de suite vers la gauche (instructions $3$, $5$ et $6$), il doit se trouver sur la colline D. Avant cela, il saute une fois vers la droite (instruction $1$). Le robot a donc commencé sur la colline C. Les graines de carottes sont donc plantées sur les collines D, puis C et finalement A, d’après les instructions $2$, $4$ et $7$.


\subsection*{This is Informatics}

Les vrais robots ont des ordinateurs intégrés, et ils sont programmés de manière similaire au robot lapin. Un programme informatique est constitué de plusieurs \emph{instructions} individuelles.

Dans notre cas, la suite d’instructions pour l’ordinateur du robot est donnée sous forme d’images. Le résultat (\emph{sortie}, \emph{output} en anglais) du programme ne dépend pas que de la position de départ (\emph{entrée}, \emph{input} en anglais), mais aussi des instructions et de l’ordre dans lequel elles sont données.

Cet exercice du Castor illustre l’utilisation de robot dans l’agriculture. Les robots ne peuvent pas seulement planter, mais aussi arroser, polliniser et répandre des produits de protection de manière ciblée.


\subsection*{This is Computational Thinking}

Optional - not to be filled 2023


\subsection*{Informatics Keywords and Websites}

\begin{itemize}
  \item Algorithme: \href{https://fr.wikipedia.org/wiki/Algorithme}{\BrochureUrlText{https://fr.wikipedia.org/wiki/Algorithme}}
  \item Instruction: \href{https://fr.wikipedia.org/wiki/Instruction_informatique}{\BrochureUrlText{https://fr.wikipedia.org/wiki/Instruction\_informatique}}
  \item Smart Farming: \href{https://www.agroscope.admin.ch/agroscope/fr/home/themes/economie-technique/smart-farming.html}{\BrochureUrlText{https://www.agroscope.admin.ch/agroscope/fr/home/themes/economie-technique/smart-farming.html}}
  \item Les robots et l’agriculture: \href{https://cordis.europa.eu/article/id/441912-robots-help-farmers-say-goodbye-to-repetitive-tasks/fr}{\BrochureUrlText{https://cordis.europa.eu/article/id/441912-robots-help-farmers-say-goodbye-to-repetitive-tasks/fr}}
\end{itemize}


\subsection*{Computational Thinking Keywords and Websites}

\begin{itemize}
  \item Modelling and Simulation,
  \item Evaluation
\end{itemize}


\end{document}
