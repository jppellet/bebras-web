\documentclass[a4paper,11pt]{report}
\usepackage[T1]{fontenc}
\usepackage[utf8]{inputenc}

\usepackage[german]{babel}
\AtBeginDocument{\def\labelitemi{$\bullet$}}

\usepackage{etoolbox}

\usepackage[margin=2cm]{geometry}
\usepackage{changepage}
\makeatletter
\renewenvironment{adjustwidth}[2]{%
    \begin{list}{}{%
    \partopsep\z@%
    \topsep\z@%
    \listparindent\parindent%
    \parsep\parskip%
    \@ifmtarg{#1}{\setlength{\leftmargin}{\z@}}%
                 {\setlength{\leftmargin}{#1}}%
    \@ifmtarg{#2}{\setlength{\rightmargin}{\z@}}%
                 {\setlength{\rightmargin}{#2}}%
    }
    \item[]}{\end{list}}
\makeatother

\newcommand{\BrochureUrlText}[1]{\texttt{#1}}
\usepackage{setspace}
\setstretch{1.15}

\usepackage{tabularx}
\usepackage{booktabs}
\usepackage{makecell}
\usepackage{multirow}
\renewcommand\theadfont{\bfseries}
\renewcommand{\tabularxcolumn}[1]{>{}m{#1}}
\newcolumntype{R}{>{\raggedleft\arraybackslash}X}
\newcolumntype{C}{>{\centering\arraybackslash}X}
\newcolumntype{L}{>{\raggedright\arraybackslash}X}
\newcolumntype{J}{>{\arraybackslash}X}

\newcommand{\BrochureInlineCode}[1]{{\ttfamily #1}}

\usepackage{amssymb}
\usepackage{amsmath}

\usepackage[babel=true,maxlevel=3]{csquotes}
\DeclareQuoteStyle{bebras-ch-eng}{“}[” ]{”}{‘}[”’ ]{’}\DeclareQuoteStyle{bebras-ch-deu}{«}[» ]{»}{“}[»› ]{”}
\DeclareQuoteStyle{bebras-ch-fra}{«\thinspace{}}[» ]{\thinspace{}»}{“}[»\thinspace{}› ]{”}
\DeclareQuoteStyle{bebras-ch-ita}{«}[» ]{»}{“}[»› ]{”}
\setquotestyle{bebras-ch-deu}

\usepackage{hyperref}
\usepackage{graphicx}
\usepackage{svg}
\svgsetup{inkscapeversion=1,inkscapearea=page}
\usepackage{wrapfig}

\usepackage{enumitem}
\setlist{nosep,itemsep=.5ex}

\setlength{\parindent}{0pt}
\setlength{\parskip}{2ex}
\raggedbottom

\usepackage{fancyhdr}
\usepackage{lastpage}
\pagestyle{fancy}

\fancyhf{}
\renewcommand{\headrulewidth}{0pt}
\renewcommand{\footrulewidth}{0.4pt}
\lfoot{\scriptsize © 2023 Bebras (CC BY-SA 4.0)}
\cfoot{\scriptsize\itshape 2023-CA-02 Riccas}
\rfoot{\scriptsize Page~\thepage{}/\pageref*{LastPage}}

\newcommand{\taskGraphicsFolder}{..}

\begin{document}

\section*{\centering{} 2023-CA-02 Riccas}


\subsection*{Body}

Evelyn hat fünf Bilder von Riccas. Sie beschreibt in Sätzen, wie Riccas aussehen.

\raisebox{-0.5ex}{\includesvg[width=68.5px]{\taskGraphicsFolder/graphics/2023-CA-02-ricca1.svg}}
\raisebox{-0.5ex}{\includesvg[width=119.1px]{\taskGraphicsFolder/graphics/2023-CA-02-ricca2.svg}}
\raisebox{-0.5ex}{\includesvg[width=83px]{\taskGraphicsFolder/graphics/2023-CA-02-ricca3.svg}}
\raisebox{-0.5ex}{\includesvg[width=83px]{\taskGraphicsFolder/graphics/2023-CA-02-ricca4.svg}}
\raisebox{-0.5ex}{\includesvg[width=97.4px]{\taskGraphicsFolder/graphics/2023-CA-02-ricca5.svg}}

Ihre Freundin Lydia zeigt ihr ein sechstes Bild von einem Ricca:

{\centering%
\includesvg[width=79.4px]{\taskGraphicsFolder/graphics/2023-CA-02-riccaException.svg}\par}

Nun stellt Evelyn fest: Einer ihrer Sätze über Riccas ist sicher falsch.

{\em


\subsection*{Question/Challenge - for the brochures}

Welcher dieser Sätze über Riccas ist nun sicher falsch?

}

\begingroup
\renewcommand{\arraystretch}{1.5}
\subsection*{Answer Options/Interactivity Description}

A) Alle Riccas haben Zähne.

B) Einige Riccas haben Flügel.

C) Riccas haben entweder Hörner oder drei Augen, aber nie Hörner \emph{und} drei Augen.

D) Wenn Riccas genau zwei Arme haben, dann haben sie auch genau zwei Beine.

\endgroup

\subsection*{Answer Explanation}

Antwort D) ist richtig: \emph{Wenn Riccas genau zwei Arme haben, dann haben sie auch genau zwei Beine.}

Antwort A) enthält eine Aussage, die für alle Riccas gelten muss. Wenn auch nur ein Ricca keine Zähne hätte, wäre sie falsch. Jedoch haben alle sechs Riccas, die Evelyn nun kennt, Zähne. Also kann Evelyns Satz nicht sicher falsch sein.

Antwort B) enthält eine Aussage, die nur für einige Riccas gelten soll.  Da eines der sechs Riccas, die Evelyn nun kennt, Flügel hat, ist dieser Satz für die sechs Riccas richtig.  Aber selbst wenn keines der sechs Riccas Flügel hätte, könnten andere, weitere Riccas Flügel haben, und der Satz könnte noch richtig sein.  Dieser Satz kann nur dann sicher falsch sein, wenn Evelyn alle Riccas kennen würde und keines Flügel hätte.

Antwort C) verknüpft zwei Aussagen mit \enquote{entweder}-\enquote{oder}. Diese verknüpfte Aussage ist genau dann wahr, wenn genau eine der beiden Aussagen wahr ist. Das ist für alle sechs Bilder der Fall: vier Riccas haben Hörner, aber keine drei Augen, die übrigen beiden Riccas haben keine Hörner, aber dafür drei Augen. Damit der Satz falsch wäre, müsste es mindestens ein Ricca mit drei Augen und Hörnern oder ein Ricca ohne Hörner und mit einer anderen Zahl als drei Augen geben.  Unter den sechs Riccas, die Evelyn nun kennt, ist kein solches Ricca. Also ist der Satz für die sechs Riccas richtig, und nicht sicher falsch.

Bleibt der Satz von Antwort D). Er ist in Form einer \enquote{Wenn}-\enquote{Dann}-Aussage formuliert. Nur wenn die Wenn-Bedingung stimmt, muss auch die Dann-Aussage wahr sein. Die Bedingung ist wahr für alle sechs Riccas, die Evelyn nun kennt: sie alle haben genau zwei Arme. Alle Riccas auf Evelyns ersten fünf Bildern haben ebenfalls genau zwei Beine; für sie ist Evelyns Satz also richtig. Jedoch hat das Ricca auf Lydias Bild mehr als zwei Beine, nämlich fünf. Deshalb ist dieser Satz nun sicher falsch.


\subsection*{This is Informatics}

Die Anzahl der Arme, Beine und Augen, und ob Riccas Zähne, Hörner oder Flügel haben, sind \emph{Eigenschaften} von Riccas. Wenn man Riccas beschreibt, formuliert man \emph{Aussagen} über diese Eigenschaften. Das führt zu einem \emph{Modell} davon, was Riccas sind.

Computer haben ebenfalls viele Modelle. Einige sind explizit formuliert wie beispielsweise ein Modell von Schülerinnen und Schülern, das aus Name, Geburtsdatum und Wohnadresse in einer Datenbank besteht. Andere Modelle bilden Computer aus Daten, wenn man ihnen beispielsweise Bilder zum Vergleich beim Training eines neuronalen Netzes gibt.

Evelyns Sätze –~also ihr Modell der Riccas in dieser Biberaufgabe – sind als \emph{logische Ausdrücke} formuliert. Einige haben \emph{Quantoren} (\enquote{(für) alle} oder \enquote{es gibt}/\enquote{einige}), andere nutzen \emph{logische Operatoren} (\enquote{entweder}-\enquote{oder} bzw. \enquote{wenn}-\enquote{dann}). Diese logischen Ausdrücke sind \emph{formalisiert}: Das heisst, dass es eine Festlegung gibt, wie man sie verwendet und was sie bedeuten.

\begin{samepage}
Anhand dieser Festlegungen

\nopagebreak

\begin{itemize}
  \item können (einfache) Ausdrücke mit Hilfe von Quantoren und Operatoren zu komplexeren Ausdrücken verknüpft werden.
  \item kann die Bedeutung der komplexeren Ausdrücke aus den einfacheren Ausdrücken berechnet werden.
\end{itemize}


\end{samepage}

Logische Ausdrücke sind eine in der Informatik verbreitete Methode, Modelle zu beschreiben.


\subsection*{This is Computational Thinking}

Der Vorgang, den Evelyn macht, nennt man \emph{Mustererkennung}. Sie sucht nach gemeinsamen Eigenschaften oder nach Aussagen, die für alle Riccas gelten. Solche Mustererkennung ist nicht nur wesentlich für Menschen, die mit Hilfe solcher Muster Begriffe bilden (wie z. B.: \enquote{hat vier Beine und knurrt \ensuremath{\rightarrow} muss ein Hund sein}), auch für Computer ist solche Mustererkennung wesentlich (z. B. autonom fahrende Autos, die Stoppschild zu unterschiedlichen Tages- und Nachtzeiten und bei unterschiedlichem Drumherum erkennen sollen).

Bei Computern nennt man diesen Vorgang \emph{maschinelles Lernen} (engl. \emph{Machine Learning}), und die meisten aufsehenerregenden \enquote{künstlichen Intelligenzen} machen genau das: Lernen, wie wahrscheinlich es ist, dass eine gewisse Eingabe (wie ein Kamerabild) eine gewisse Ausgabe erzeugt (wie die Interpretation, dass etwas ein Stoppschild ist). Das führt zu \emph{Klassifizierungen} von Objekten.

Genauso wie der Mensch ist auch der Computer nicht vor Vorurteilen gefeit. Hätte ein Mensch beispielsweise noch nie eine Katze gesehen und würde nun eine schnurrende Katze halten, könnte er sie als knurrenden Hund interpretieren. Daher müssen Trainingsdaten für \enquote{künstliche Intelligenzen} auch sorgfältig gewählt werden, sonst kommt es zu Fehlkategorisierungen. Deshalb muss vielen \enquote{künstliche Intelligenzen} heutzutage leider ein gewisser Hang zur Diskriminierung \enquote{abtrainiert} werden – sie spiegeln dabei jedoch letztlich nur ungefiltert unsere Gesellschaft wider.


\subsection*{Informatics Keywords and Websites}

Modellbildung: \href{https://de.wikipedia.org/wiki/Modell\#Modellbildung}{\BrochureUrlText{https://de.wikipedia.org/wiki/Modell\#Modellbildung}}, \href{https://de.wikipedia.org/wiki/Objektorientierte_Analyse_und_Design}{\BrochureUrlText{https://de.wikipedia.org/wiki/Objektorientierte\_Analyse\_und\_Design}}
Maschinelles Lernen: \href{https://de.wikipedia.org/wiki/Maschinelles_Lernen}{\BrochureUrlText{https://de.wikipedia.org/wiki/Maschinelles\_Lernen}}


\subsection*{Computational Thinking Keywords and Websites}

Mustererkennung: \href{https://de.wikipedia.org/wiki/Mustererkennung}{\BrochureUrlText{https://de.wikipedia.org/wiki/Mustererkennung}}


\end{document}
