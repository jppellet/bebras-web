% Definition of the meta information: task difficulties, task ID, task title, task country; definition of the variables as well as their scope is in commands.tex
\setcounter{taskAgeDifficulty3to4}{4}
\setcounter{taskAgeDifficulty5to6}{2}
\setcounter{taskAgeDifficulty7to8}{0}
\setcounter{taskAgeDifficulty9to10}{0}
\setcounter{taskAgeDifficulty11to13}{0}
\renewcommand{\taskTitle}{Ricca}
\renewcommand{\taskCountry}{CA}

% include this task only if for the age groups being processed this task is relevant
\ifthenelse{
  \(\boolean{age3to4} \AND \(\value{taskAgeDifficulty3to4} > 0\)\) \OR
  \(\boolean{age5to6} \AND \(\value{taskAgeDifficulty5to6} > 0\)\) \OR
  \(\boolean{age7to8} \AND \(\value{taskAgeDifficulty7to8} > 0\)\) \OR
  \(\boolean{age9to10} \AND \(\value{taskAgeDifficulty9to10} > 0\)\) \OR
  \(\boolean{age11to13} \AND \(\value{taskAgeDifficulty11to13} > 0\)\)}{

\newchapter{\taskTitle}

% task body
Evelyn ha cinque foto dei Ricca. Descrive con delle frasi il loro aspetto.

\raisebox{-0.5ex}{\includesvg[width=68.5px]{\taskGraphicsFolder/graphics/2023-CA-02-ricca1.svg}}
\raisebox{-0.5ex}{\includesvg[width=119.1px]{\taskGraphicsFolder/graphics/2023-CA-02-ricca2.svg}}
\raisebox{-0.5ex}{\includesvg[width=83px]{\taskGraphicsFolder/graphics/2023-CA-02-ricca3.svg}}
\raisebox{-0.5ex}{\includesvg[width=83px]{\taskGraphicsFolder/graphics/2023-CA-02-ricca4.svg}}
\raisebox{-0.5ex}{\includesvg[width=97.4px]{\taskGraphicsFolder/graphics/2023-CA-02-ricca5.svg}}

La sua amica Lydia le mostra una sesta foto di una Ricca:

{\centering%
\includesvg[width=79.4px]{\taskGraphicsFolder/graphics/2023-CA-02-riccaException.svg}\par}

Ora Evelyn si rende conto di una cosa: una delle sue frasi sui Ricca è sicuramente sbagliata.



% question (as \emph{})
{\em
Quale di queste frasi sui Ricca è sicuramente sbagliata?


}

% answer alternatives (as \begin{enumerate}[A)]) or interactivity
A) Tutti i Ricca hanno i denti.

B) Alcuni Ricca hanno le ali.

C) I Ricca hanno o corna o tre occhi, ma mai corna \emph{e} tre occhi.

D) Se i Ricca hanno esattamente due braccia, allora hanno anche esattamente due gambe.



% from here on this is only included if solutions are processed
\ifthenelse{\boolean{solutions}}{
\newpage

% answer explanation
\section*{\BrochureSolution}
La risposta D) è corretta: \emph{Se i Ricca hanno esattamente due braccia, allora hanno anche esattamente due gambe.}

La risposta A) contiene un’affermazione che deve valere per tutti i Ricca. Se anche un solo Ricca non avesse i denti, l’affermazione sarebbe falsa. Tuttavia, tuttii e sei i Ricca che Evelyn conosce ora hanno i denti. Quindi la frase di Evelyn non può essere sicuramente sbagliata.

La risposta B) contiene un’affermazione che dovrebbe valere solo per alcuni Ricca. Poiché uno dei sei Ricca che Evelyn conosce ora ha le ali, la frase è corretta per i sei Ricca. Ma anche se nessuno dei sei Ricca avesse le ali, altri Ricca potrebbero averle e la frase sarebbe comunque vera. La frase può essere sicuramente falsa solo se Evelyn conosceva tutti i Ricca e nessuno aveva le ali.

La risposta C) collega due affermazioni con \enquote{o} e \enquote{o}. L’affermazione collegata è vera se è vera esattamente una delle due affermazioni. Questo è il caso di tutti e sei i Ricca: quattro Ricca hanno le corna ma non tre occhi,gli altre due Ricca non hanno le corna ma tre occhi. Affinché la frase sia falsa, dovrebbe esserci almeno un Ricca con tre occhi e corna o un Ricca senza corna e con un numero diverso da tre occhi. Tra i sei Ricca che Evelyn conosce ora, non c’è nessun Ricca di questo tipo.  Quindi la frase è corretta per i sei Ricca, e non certamente sbagliata.

Rimane la frase della risposta D). È formulata nella forma di un’affermazione \enquote{se} e \enquote{allora}. Se la condizione \enquote{se} è vera, deve essere vera anche l’affermazione \enquote{allora}. La condizione è vera per tutte le sei Ricca che Evelyn conosce: tutti hanno esattamente due bracci.  Anche tutte le Ricche delle prime cinque immagini di Evelyn hanno esattamente due gambe; quindi per loro la frase di Evelyn è vera.  Tuttavia, la Ricca nella foto di Lydia ha più di due gambe, cioè cinque. Pertanto, la frase è sicuramente sbagliata.



% it's informatics
\section*{\BrochureItsInformatics}
Il numero di ali, braccia gambe e occhi e il fatto che i Ricca abbiano o meno i denti o le ali sono \emph{caratteristiche} dei Ricca. Quando si descrivono i Ricca, si formulano delle \enquote{affermazioni} su queste proprietà. Questo porta a un \emph{modello} di ciò che i Ricca sono.

Anche i computer hanno molti modelli. Alcuni sono formulati esplicitamente, come ad esempio un modello di studenti composto da nome, data di nascita e indirizzo di casa in un database. Altri modelli sono formati dai computer a partire dai dati, ad esempio quando vengono fornite immagini da confrontare per l’addestramento di una rete neurale.

Le frasi di Evelyn - cioè il suo modello dei Ricca in questo compito - sono formulate come \emph{espressioni logiche}. Alcune hanno dei \emph{quantificatori} (\enquote{(per) tutti} o \enquote{ci sono}/\enquote{alcuni}), altre usano degli \emph{operatori logici} (\enquote{o}-\enquote{o} o \enquote{se}-\enquote{allora}). Queste espressioni logiche sono \emph{formalizzate}: cioè, c’è una specificazione di come usarle e di cosa significano.

Sulla base di queste specifiche

\begin{itemize}
  \item Le espressioni (semplici) possono essere collegate a espressioni più complesse con l’aiuto di quantificatori e operatori.
  \item il significato delle espressioni più complesse può essere calcolato a partire da quelle più semplici.
\end{itemize}

Le espressioni logiche sono un metodo comune per descrivere i modelli in informatica.



% keywords and websites (as \begin{itemize})
\section*{\BrochureWebsitesAndKeywords}
{\raggedright
Apprendimento automatico: \href{https://it.wikipedia.org/wiki/Apprendimento_automatico}{\BrochureUrlText{https://it.wikipedia.org/wiki/Apprendimento\_automatico}}


}

% end of ifthen for excluding the solutions
}{}

% all authors
% ATTENTION: you HAVE to make sure an according entry is in ../main/authors.tex.
% Syntax: \def\AuthorLastnameF{} (Lastname is last name, F is first letter of first name, this serves as a marker for ../main/authors.tex)
\def\AuthorChanS{} % \ifdefined\AuthorChanS \BrochureFlag{ca}{} Sarah Chan\fi
\def\AuthorKhachatryanD{} % \ifdefined\AuthorKhachatryanD \BrochureFlag{am}{} David Khachatryan\fi
\def\AuthorBilbaoJ{} % \ifdefined\AuthorBilbaoJ \BrochureFlag{es}{} Javier Bilbao\fi
\def\AuthorDatzkoC{} % \ifdefined\AuthorDatzkoC \BrochureFlag{hu}{} Christian Datzko\fi
\def\AuthorBaumannL{} % \ifdefined\AuthorBaumannL \BrochureFlag{at}{} Liam Baumann\fi
\def\AuthorPohlW{} % \ifdefined\AuthorPohlW \BrochureFlag{de}{} Wolfgang Pohl\fi
\def\AuthorDatzkoThutS{} % \ifdefined\AuthorDatzkoThutS \BrochureFlag{de}{} Susanne Datzko-Thut\fi
\def\AuthorGiangC{} % \ifdefined\AuthorGiangC \BrochureFlag{ch}{} Christian Giang\fi

\newpage}{}
