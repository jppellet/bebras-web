% Definition of the meta information: task difficulties, task ID, task title, task country; definition of the variables as well as their scope is in commands.tex
\setcounter{taskAgeDifficulty3to4}{4}
\setcounter{taskAgeDifficulty5to6}{3}
\setcounter{taskAgeDifficulty7to8}{2}
\setcounter{taskAgeDifficulty9to10}{1}
\setcounter{taskAgeDifficulty11to13}{0}
\renewcommand{\taskTitle}{Orto di Lisa}
\renewcommand{\taskCountry}{AU}

% include this task only if for the age groups being processed this task is relevant
\ifthenelse{
  \(\boolean{age3to4} \AND \(\value{taskAgeDifficulty3to4} > 0\)\) \OR
  \(\boolean{age5to6} \AND \(\value{taskAgeDifficulty5to6} > 0\)\) \OR
  \(\boolean{age7to8} \AND \(\value{taskAgeDifficulty7to8} > 0\)\) \OR
  \(\boolean{age9to10} \AND \(\value{taskAgeDifficulty9to10} > 0\)\) \OR
  \(\boolean{age11to13} \AND \(\value{taskAgeDifficulty11to13} > 0\)\)}{

\newchapter{\taskTitle}

% task body
Lisa crea un orto. Vuole piantare cinque ortaggi diversi. Alcuni ortaggi vanno d’accordo tra loro \raisebox{-0.5ex}[0pt][0pt]{\includesvg[width=14.4px]{\taskGraphicsFolder/graphics/2023-AU-01-good.svg}}, altri no \raisebox{-0.5ex}[0pt][0pt]{\includesvg[width=25.3px]{\taskGraphicsFolder/graphics/2023-AU-01-bad.svg}}:

{\centering%
\includesvg[scale=0.3]{\taskGraphicsFolder/graphics/2023-AU-01-taskbody.svg}\par}

Lisa ha diviso l’orto in aree esagonali. Vuole piantare esattamente un ortaggio in ogni area.

Lisa ha già piantato porri \raisebox{-0.5ex}{\includesvg[scale=0.3]{\taskGraphicsFolder/graphics/2023-AU-01-leek.svg}} in tre aree.

{\centering%
\includesvg[scale=0.3]{\taskGraphicsFolder/graphics/2023-AU-01-question.svg}\par}

Quando si pianta, Lisa osserva la seguente regola: gli ortaggi che non vanno d’accordo non devono essere piantati in zone che si toccano.



% question (as \emph{})
{\em
Pianta tutte le aree ancora libere seguendo la regola di Lisa!


}

% answer alternatives (as \begin{enumerate}[A)]) or interactivity


% from here on this is only included if solutions are processed
\ifthenelse{\boolean{solutions}}{
\newpage

% answer explanation
\section*{\BrochureSolution}
La risposta gista:

{\centering%
\includesvg[scale=0.3]{\taskGraphicsFolder/graphics/2023-AU-01-solution.svg}\par}

\begin{tabularx}{\columnwidth}{ @{} J l @{} }
  Poiché i piselli non vanno d’accordo con i porri, Lisa non pianta i piselli nelle aree chiare. Solo le aree rimanenti rimangono per i piselli. & \makecell[l]{\includesvg[width=144.3px]{\taskGraphicsFolder/graphics/2023-AU-01-explanation01.svg}} \\ 
  Poiché i pomodori non vanno d’accordo con i piselli, Lisa non pianta i pomodori nelle aree chiare. Può piantare i pomodori nelle altre zone; i pomodori vanno d’accordo con i porri. & \makecell[l]{\includesvg[width=144.3px]{\taskGraphicsFolder/graphics/2023-AU-01-explanation02.svg}} \\ 
  Poiché i pomodori non vanno d’accordo con i finocchi, Lisa non li pianta nelle aree chiare. Può piantare il finocchio nelle due aree tra i porri e i piselli.  Può piantare la lattuga nelle aree chiare: Lisa non è a conoscenza di alcuna discordanza tra gli ortaggi vicini e la lattuga. & \makecell[l]{\includesvg[width=144.3px]{\taskGraphicsFolder/graphics/2023-AU-01-explanation03.svg}}
\end{tabularx}



% it's informatics
\section*{\BrochureItsInformatics}
Se si vuole piantare ortaggi in modo che il raccolto sia il più abbondante possibile, si deve osservare molte \emph{condizioni}: Ad esempio, le singole varietà hanno esigenze diverse in termini di spazio, nutrienti e luce. In questo compito consideriamo solo un tipo di condizione: la compatibilità tra le varietà di ortaggi.

Per trovare un piano per l’orto di Lisa che rispetti tutte le condizioni di compatibilità, si potrebbe procedere in questo modo: si provano sistematicamente tutte le combinazioni per disporre gli ortaggi sull’orto. Solo quando l’orto è pieno, si verifica se questa combinazione soddisfa tutte le condizioni ed è una soluzione al problema di Lisa. In informatica, tale prova di tutte le combinazioni è nota come metodo \emph{forza bruta}. Per problemi con molte combinazioni e poche soluzioni, procedere secondo questo metodo può richiedere molto tempo.

Pertanto, di solito è meglio procedere per gradi e considerare tutte le condizioni a ogni passo. In questo modo possiamo trovare la soluzione al problema di Lisa, una combinazione o una piantumazione \enquote{sbagliata} dell’orto infatti non può verificarsi.

Fortunatamente, la soluzione si può trovare in modo diretto: ci sono sempre aree in cui possiamo piantare alcuni degli ortaggi rimasti. Questo di solito non funziona sempre.

Se si cerca di assemblare la soluzione passo dopo passo, ci possono essere diverse possibilità di soddisfare tutte le condizioni in un unico passo A.

{\centering%
\includesvg[scale=0.3]{\taskGraphicsFolder/graphics/2023-AU-01b-itsinformatics-compatible.svg}\par}

A seconda della scelta, in una fase successiva B potrebbero non esserci più opzioni. Quindi si fanno gli ultimi passi indietro fino ad arrivare al passo A con diverse possibilità. A questo punto si sceglie un’altra possibilità e si cerca di trovare una soluzione.

In informatica, questo ritorno sui propri passi è noto come \emph{backtracking}.



% keywords and websites (as \begin{itemize})
\section*{\BrochureWebsitesAndKeywords}
{\raggedright
\begin{itemize}
  \item Metodo forza bruta: \href{https://it.wikipedia.org/wiki/Metodo_forza_bruta}{\BrochureUrlText{https://it.wikipedia.org/wiki/Metodo\_forza\_bruta}}
  \item Backtracking: \href{https://it.wikipedia.org/wiki/Backtracking}{\BrochureUrlText{https://it.wikipedia.org/wiki/Backtracking}}
\end{itemize}


}

% end of ifthen for excluding the solutions
}{}

% all authors
% ATTENTION: you HAVE to make sure an according entry is in ../main/authors.tex.
% Syntax: \def\AuthorLastnameF{} (Lastname is last name, F is first letter of first name, this serves as a marker for ../main/authors.tex)
\def\AuthorGatesE{} % \ifdefined\AuthorGatesE \BrochureFlag{au}{} Emily Gates\fi
\def\AuthorKingM{} % \ifdefined\AuthorKingM \BrochureFlag{au}{} Mhairi King\fi
\def\AuthorRamicE{} % \ifdefined\AuthorRamicE \BrochureFlag{ba}{} Estela Ramić\fi
\def\AuthorHironM{} % \ifdefined\AuthorHironM \BrochureFlag{fr}{} Mathias Hiron\fi
\def\AuthorDatzkoC{} % \ifdefined\AuthorDatzkoC \BrochureFlag{hu}{} Christian Datzko\fi
\def\AuthorPohlW{} % \ifdefined\AuthorPohlW \BrochureFlag{de}{} Wolfgang Pohl\fi
\def\AuthorBaumannW{} % \ifdefined\AuthorBaumannW \BrochureFlag{at}{} Wilfried Baumann\fi
\def\AuthorThutM{} % \ifdefined\AuthorThutM \BrochureFlag{ch}{} Marianne Thut\fi
\def\AuthorDatzkoThutS{} % \ifdefined\AuthorDatzkoThutS \BrochureFlag{de}{} Susanne Datzko-Thut\fi
\def\AuthorGiangC{} % \ifdefined\AuthorGiangC \BrochureFlag{ch}{} Christian Giang\fi

\newpage}{}
