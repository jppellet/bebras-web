% Definition of the meta information: task difficulties, task ID, task title, task country; definition of the variables as well as their scope is in commands.tex
\setcounter{taskAgeDifficulty3to4}{0}
\setcounter{taskAgeDifficulty5to6}{3}
\setcounter{taskAgeDifficulty7to8}{2}
\setcounter{taskAgeDifficulty9to10}{1}
\setcounter{taskAgeDifficulty11to13}{0}
\renewcommand{\taskTitle}{Martinas Dorf}
\renewcommand{\taskCountry}{PE}

% include this task only if for the age groups being processed this task is relevant
\ifthenelse{
  \(\boolean{age3to4} \AND \(\value{taskAgeDifficulty3to4} > 0\)\) \OR
  \(\boolean{age5to6} \AND \(\value{taskAgeDifficulty5to6} > 0\)\) \OR
  \(\boolean{age7to8} \AND \(\value{taskAgeDifficulty7to8} > 0\)\) \OR
  \(\boolean{age9to10} \AND \(\value{taskAgeDifficulty9to10} > 0\)\) \OR
  \(\boolean{age11to13} \AND \(\value{taskAgeDifficulty11to13} > 0\)\)}{

\newchapter{\taskTitle}

% task body
In Martinas Dorf gibt es sechs Häuser.
Ausserdem gibt es Wege, über die man von einem Haus zum nächsten gehen kann.
Für alle diese Wege benötigt Martina die gleiche Zeit.

Martina hat eine besondere Karte des Dorfs gezeichnet.
Sie hat darin Wege eingezeichnet, über die sie am schnellsten zu den anderen Häusern gehen kann.

{\centering%
\includesvg[scale=0.4]{\taskGraphicsFolder/graphics/2023-PE-02-taskbody.svg}\par}

Natürlich gibt es auch eine richtige Karte des Dorfs, mit allen Wegen.



% question (as \emph{})
{\em
Welche dieser Zeichnungen kann \emph{nicht} die richtige Karte sein?


}

% answer alternatives (as \begin{enumerate}[A)]) or interactivity
\begin{tabular}{ @{} c c @{} }
  \makecell[c]{\includesvg[scale=0.4]{\taskGraphicsFolder/graphics/2023-PE-02-answerA.svg}} & \makecell[c]{\includesvg[scale=0.4]{\taskGraphicsFolder/graphics/2023-PE-02-answerB.svg}} \\ 
  A) & B) \\ 
  \makecell[c]{\includesvg[scale=0.4]{\taskGraphicsFolder/graphics/2023-PE-02-answerC.svg}} & \makecell[c]{\includesvg[scale=0.4]{\taskGraphicsFolder/graphics/2023-PE-02-answerD.svg}} \\ 
  C) & D)
\end{tabular}



% from here on this is only included if solutions are processed
\ifthenelse{\boolean{solutions}}{
\newpage

% answer explanation
\section*{\BrochureSolution}
Antwort C ist richtig: \raisebox{-0.5ex}{\includesvg[scale=0.4]{\taskGraphicsFolder/graphics/2023-PE-02-answerC.svg}}

Martinas besondere Karte zeigt, dass sie zu dem Haus ganz rechts am schnellsten über drei Wege gehen kann. Wenn C die richtige Karte des Dorfes wäre, dann könnte Martina schneller zu diesem Haus gehen, nämlich über zwei Wege. Also kann C nicht die richtige Karte des Dorfes sein.

Bei den Karten A, B und D gibt es keine Möglichkeit, schneller zu einem der anderen Häuser zu gehen als über die Wege auf Martina besonderer Karte. Diese Karten könnten also richtige Karten des Dorfs sein.



% it's informatics
\section*{\BrochureItsInformatics}
Martina ist Informatikerin. Sie hat ihre Karte als \emph{Graph} gezeichnet. Graphen bestehen aus \emph{Knoten} (hier die Häuser), die durch \emph{Kanten} (hier die Wege) verbunden sein können. Sie sind in vielen Bereichen der Informatik geeignet, die Realität zu modellieren –~auch hier in dieser Biberaufgabe.

Martina weiss, dass es für Graphen eine ganze Reihe von Algorithmen gibt, beispielsweise die sogenannte Breitensuche, um Aufgaben zu lösen wie \enquote{Wie kommt man am schnellsten zu einem anderen Haus?}. Vielleicht hat sie ihre besondere Dorfkarte mit Hilfe einer Breitensuche in einem grösseren Graph, der richtigen Karte des Dorfes mit allen Wegen, erstellt.

In der Graphentheorie, die sich mit Graphen und Graph-Algorithmen beschäftigt, entspricht Martinas Karte einem Teilgraph der Gesamtkarte des Dorfes. Martinas Teilgraph hat zwei Besonderheiten:

\begin{itemize}
  \item Alle Knoten sind direkt (über eine Kante) oder indirekt (über mehreren Kanten) miteinander verbunden.
  \item Egal welche zwei Knoten man zufällig auswählt, es gibt immer nur genau einen Weg zwischen den beiden.
\end{itemize}

Ein Graph mit diesen Besonderheiten wird in der Informatik als \emph{Baum} bezeichnet. Martinas Haus stellt die \emph{Wurzel} des Baumes dar. Von der Wurzel aus kann Martina alle anderen Knoten (die anderen Häuser im Dorf) auf einem eindeutigen Weg erreichen.  Martinas Teilgraph ist also ein Baum; ausserdem enthält er alle Knoten des gesamten Graphen (der Gesamtkarte des Dorfes) – aber möglicherweise nicht alle Kanten. Ein Teilgraph mit diesen Eigenschaften wird als \emph{Spannbaum} des gesamten Graphen bezeichnet.

In der Informatik gibt es viele Anwendungen für Graph-Algorithmen, vor allem im Zusammenhang mit Netzwerken (Verkehrsnetze, Telekommunikationsnetze …), etwa bei der Berechnung von Routen in Navigationssystemen.  Spannbäume können beim Aufbau kostengünstiger Netze eingesetzt werden und hilfreich bei der Lösung besonders schwieriger Probleme sein.



% keywords and websites (as \begin{itemize})
\section*{\BrochureWebsitesAndKeywords}
{\raggedright
\begin{itemize}
  \item Graphentheorie: \href{https://de.wikipedia.org/wiki/Graph_(Graphentheorie)}{\BrochureUrlText{https://de.wikipedia.org/wiki/Graph\_(Graphentheorie)}}
  \item Baum: \href{https://de.wikipedia.org/wiki/Baum_(Graphentheorie)}{\BrochureUrlText{https://de.wikipedia.org/wiki/Baum\_(Graphentheorie)}}
  \item Breitensuche: \href{https://de.wikipedia.org/wiki/Breitensuche}{\BrochureUrlText{https://de.wikipedia.org/wiki/Breitensuche}}
  \item Spannbaum: \href{https://de.wikipedia.org/wiki/Spannbaum}{\BrochureUrlText{https://de.wikipedia.org/wiki/Spannbaum}}
\end{itemize}


}

% end of ifthen for excluding the solutions
}{}

% all authors
% ATTENTION: you HAVE to make sure an according entry is in ../main/authors.tex.
% Syntax: \def\AuthorLastnameF{} (Lastname is last name, F is first letter of first name, this serves as a marker for ../main/authors.tex)
\def\AuthorGutierrezJ{} % \ifdefined\AuthorGutierrezJ \BrochureFlag{pe}{} Juan Gutiérrez\fi
\def\AuthorLunaC{} % \ifdefined\AuthorLunaC \BrochureFlag{pe}{} Carlos Luna\fi
\def\AuthorSchluterK{} % \ifdefined\AuthorSchluterK \BrochureFlag{de}{} Kirsten Schlüter\fi
\def\AuthorSerafiniG{} % \ifdefined\AuthorSerafiniG \BrochureFlag{ch}{} Giovanni Serafini\fi
\def\AuthorIkramovA{} % \ifdefined\AuthorIkramovA \BrochureFlag{uz}{} Alisher Ikramov\fi
\def\AuthorDatzkoThutS{} % \ifdefined\AuthorDatzkoThutS \BrochureFlag{de}{} Susanne Datzko-Thut\fi

\newpage}{}
