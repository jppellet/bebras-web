\documentclass[a4paper,11pt]{report}
\usepackage[T1]{fontenc}
\usepackage[utf8]{inputenc}

\usepackage[french]{babel}
\frenchbsetup{ThinColonSpace=true}
\renewcommand*{\FBguillspace}{\hskip .4\fontdimen2\font plus .1\fontdimen3\font minus .3\fontdimen4\font \relax}
\AtBeginDocument{\def\labelitemi{$\bullet$}}

\usepackage{etoolbox}

\usepackage[margin=2cm]{geometry}
\usepackage{changepage}
\makeatletter
\renewenvironment{adjustwidth}[2]{%
    \begin{list}{}{%
    \partopsep\z@%
    \topsep\z@%
    \listparindent\parindent%
    \parsep\parskip%
    \@ifmtarg{#1}{\setlength{\leftmargin}{\z@}}%
                 {\setlength{\leftmargin}{#1}}%
    \@ifmtarg{#2}{\setlength{\rightmargin}{\z@}}%
                 {\setlength{\rightmargin}{#2}}%
    }
    \item[]}{\end{list}}
\makeatother

\newcommand{\BrochureUrlText}[1]{\texttt{#1}}
\usepackage{setspace}
\setstretch{1.15}

\usepackage{tabularx}
\usepackage{booktabs}
\usepackage{makecell}
\usepackage{multirow}
\renewcommand\theadfont{\bfseries}
\renewcommand{\tabularxcolumn}[1]{>{}m{#1}}
\newcolumntype{R}{>{\raggedleft\arraybackslash}X}
\newcolumntype{C}{>{\centering\arraybackslash}X}
\newcolumntype{L}{>{\raggedright\arraybackslash}X}
\newcolumntype{J}{>{\arraybackslash}X}

\newcommand{\BrochureInlineCode}[1]{{\ttfamily #1}}

\usepackage{amssymb}
\usepackage{amsmath}

\usepackage[babel=true,maxlevel=3]{csquotes}
\DeclareQuoteStyle{bebras-ch-eng}{“}[” ]{”}{‘}[”’ ]{’}\DeclareQuoteStyle{bebras-ch-deu}{«}[» ]{»}{“}[»› ]{”}
\DeclareQuoteStyle{bebras-ch-fra}{«\thinspace{}}[» ]{\thinspace{}»}{“}[»\thinspace{}› ]{”}
\DeclareQuoteStyle{bebras-ch-ita}{«}[» ]{»}{“}[»› ]{”}
\setquotestyle{bebras-ch-fra}

\usepackage{hyperref}
\usepackage{graphicx}
\usepackage{svg}
\svgsetup{inkscapeversion=1,inkscapearea=page}
\usepackage{wrapfig}

\usepackage{enumitem}
\setlist{nosep,itemsep=.5ex}

\setlength{\parindent}{0pt}
\setlength{\parskip}{2ex}
\raggedbottom

\usepackage{fancyhdr}
\usepackage{lastpage}
\pagestyle{fancy}

\fancyhf{}
\renewcommand{\headrulewidth}{0pt}
\renewcommand{\footrulewidth}{0.4pt}
\lfoot{\scriptsize © 2022 Bebras (CC BY-SA 4.0)}
\cfoot{\scriptsize\itshape 2022-DE-05 Hangar tournant}
\rfoot{\scriptsize Page~\thepage{}/\pageref*{LastPage}}

\newcommand{\taskGraphicsFolder}{..}

\begin{document}

\section*{\centering{} 2022-DE-05 Hangar tournant}


\subsection*{Body}

Six avions se parquent à l’aéroport de Castorville. Ils sont parqués dans un hangar, sur une plaque tournante qui a six places de parc. À l’extérieur se trouvent deux boutons fléchés \raisebox{-0.5ex}[0pt][0pt]{\includesvg[width=10.8px]{\taskGraphicsFolder/graphics/2022-DE-05-taskbody_buttonleft.svg}} \raisebox{-0.5ex}[0pt][0pt]{\includesvg[width=10.8px]{\taskGraphicsFolder/graphics/2022-DE-05-taskbody_buttonright.svg}}. En actionnant un bouton, on peut faire tourner la plaque d’exactement une place de parc vers la gauche ou la droite.

{\centering%
\includesvg[scale=0.7]{\taskGraphicsFolder/graphics/2022-DE-05-taskbody_compatible.svg}\par}

Lorsque les pilotes viennent chercher leurs avions le matin, la place de parc $1$ est toujours en face de la porte du hangar et l’avion parqué dessus peut tout de suite sortir. Dans le meilleur des cas, il ne faut appuyer que cinq fois sur les boutons fléchés pour permettre à tous les autres avions de sortir. Par exemple, si les pilotes veulent faire sortir les avions dans l’ordre $1$, $6$, $5$, $4$, $3$, $2$, il suffit d’appuyer cinq fois sur la touche \raisebox{-0.5ex}[0pt][0pt]{\includesvg[width=10.8px]{\taskGraphicsFolder/graphics/2022-DE-05-taskbody_buttonright.svg}}.

Mais quel est le pire des cas? Quelle séquence de sortie les avions doivent-ils avoir pour qu’il faille appuyer le plus souvent sur les boutons?

{\em


\subsection*{Question/Challenge - for the brochures}

Donne un exemple d’une telle séquence.

{\centering%
\includesvg[scale=0.7]{\taskGraphicsFolder/graphics/2022-DE-05-question_new_compatible.svg}\par}

}


\subsection*{Interactivity Instructions}

Glisse les numéros de place de parc pour former la séquence désirée.

{\centering%
\includesvg[scale=0.7]{\taskGraphicsFolder/graphics/2022-DE-05-question_new_compatible.svg}\par}

\begingroup
\renewcommand{\arraystretch}{1.5}
\subsection*{Answer Options/Interactivity Description}



\endgroup

\subsection*{Answer Explanation}

Il y a deux bonnes réponses possibles:

{\centering%
\includesvg[scale=0.7]{\taskGraphicsFolder/graphics/2022-DE-05-solution1_new_compatible.svg}\par}

{\centering%
\includesvg[scale=0.7]{\taskGraphicsFolder/graphics/2022-DE-05-solution2_new_compatible.svg}\par}

Pour arriver à une solution, il faut toujours choisir l’avion parqué sur la place la plus éloignée de la porte du hangar.

“\raisebox{-0.5ex}[0pt][0pt]{\includesvg[width=10.8px]{\taskGraphicsFolder/graphics/2022-DE-05-taskbody_buttonright.svg}} \raisebox{-0.5ex}[0pt][0pt]{\includesvg[width=10.8px]{\taskGraphicsFolder/graphics/2022-DE-05-taskbody_buttonright.svg}} \raisebox{-0.5ex}[0pt][0pt]{\includesvg[width=10.8px]{\taskGraphicsFolder/graphics/2022-DE-05-taskbody_buttonright.svg}}  $4$” signifie que l’avion parqué sur la place $4$ sort du hangar après que l’on a appuyé trois fois sur le bouton fléché \raisebox{-0.5ex}[0pt][0pt]{\includesvg[width=10.8px]{\taskGraphicsFolder/graphics/2022-DE-05-taskbody_buttonright.svg}}.

\textbf{4 1 3 6 2 5}:

{\centering%
\includesvg[width=339.1px]{\taskGraphicsFolder/graphics/2022-DE-05-explanation1_new_compatible.svg}\par}

\textbf{4 1 5 2 6 3}:

{\centering%
\includesvg[width=339.1px]{\taskGraphicsFolder/graphics/2022-DE-05-explanation2_new_compatible.svg}\par}

$16$ étapes sont nécessaires dans les deux cas.

Il ne peut pas y avoir plus de $16$ étapes, car il n’est possible d’avoir deux séries de trois étapes qu’au début; ensuite, il peut y avoir tout au plus des séries de deux et de trois étapes en alternance.


\subsection*{It’s Informatics}

Le hangar tournant a l’avantage de permettre de parquer des avions en économisant de la place, mais cela prend en général plus de temps d’aller chercher les avions.

L’\emph{efficacité} et la \emph{complexité} d’un processus forment un thème central de l’informatique, parce que c’est un critère d’évaluation des \emph{algorithmes} important. Souvent, l’efficacité concerne le \emph{temps d’exécution} de l’algorithme, mais ce n’est pas toujours le cas. La définition générale de la complexité d’un algorithme concerne toutes les ressources utilisées, par exemple l’\emph{espace de stockage} nécessaire.

Comme dans notre exemple du hangar, économiser une ressource cause un besoin plus grand d’autres ressources. Quelle ressource a la plus grande importance dépend de la relation concrète entre les ressources et de leurs différentes disponibilités.

Par exemple, les algorithmes \emph{Timsort} et du \emph{tri à bulles} sont deux algorithmes permettant de trier une liste d’objets. Le tri à bulles trie les objets en un temps proportionnel à leur nombre au carré (${\mathcal{O}(n^2)}$), mais n’a besoin que de peu d’espace supplémentaire et cet espace n’augmente pas avec le nombre d’objets.

Timsort trie beaucoup plus rapidement que le tri à bulles (${\mathcal{O}(n\log{}n)}$), mais l’espace dont il a besoin et proportionnel à la taille de la liste. Si une utilisation concrète de tri demande de trier de longues listes rapidement, Timsort est la meilleur méthode; par contre, s’il est plus important de minimiser l’espace de stockage nécessaire, le tri à bulles est une meilleure méthode.

{\raggedright

\subsection*{Keywords and Websites}

\begin{itemize}
  \item Complexité (efficacité): \href{https://fr.wikipedia.org/wiki/Analyse_de_la_complexit\%C3\%A9_des_algorithmes}{\BrochureUrlText{https://fr.wikipedia.org/wiki/Analyse\_de\_la\_complexité\_des\_algorithmes}}
  \item Algorithme: \href{https://de.wikipedia.org/wiki/Algorithmus}{\BrochureUrlText{https://de.wikipedia.org/wiki/Algorithmus}}
  \item Tri à bulles: \href{https://fr.wikipedia.org/wiki/Tri_\%C3\%A0_bulles}{\BrochureUrlText{https://fr.wikipedia.org/wiki/Tri\_à\_bulles}}
  \item Timsort: \href{https://fr.wikipedia.org/wiki/Timsort}{\BrochureUrlText{https://fr.wikipedia.org/wiki/Timsort}}
  \item Notation de Landau: \href{https://fr.wikipedia.org/wiki/Comparaison_asymptotique\#La_famille_de_notations_de_Landau_O,_o,_\%CE\%A9,_\%CF\%89,_\%CE\%98,_~}{\BrochureUrlText{https://fr.wikipedia.org/wiki/Comparaison\_asymptotique\#La\_famille\_de\_notations\_de\_Landau}}
\end{itemize}


}
\end{document}
