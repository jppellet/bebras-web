% Definition of the meta information: task difficulties, task ID, task title, task country; definition of the variables as well as their scope is in commands.tex
\setcounter{taskAgeDifficulty3to4}{0}
\setcounter{taskAgeDifficulty5to6}{0}
\setcounter{taskAgeDifficulty7to8}{3}
\setcounter{taskAgeDifficulty9to10}{2}
\setcounter{taskAgeDifficulty11to13}{1}
\renewcommand{\taskTitle}{Robots tamponneurs}
\renewcommand{\taskCountry}{CZ}

% include this task only if for the age groups being processed this task is relevant
\ifthenelse{
  \(\boolean{age3to4} \AND \(\value{taskAgeDifficulty3to4} > 0\)\) \OR
  \(\boolean{age5to6} \AND \(\value{taskAgeDifficulty5to6} > 0\)\) \OR
  \(\boolean{age7to8} \AND \(\value{taskAgeDifficulty7to8} > 0\)\) \OR
  \(\boolean{age9to10} \AND \(\value{taskAgeDifficulty9to10} > 0\)\) \OR
  \(\boolean{age11to13} \AND \(\value{taskAgeDifficulty11to13} > 0\)\)}{

\newchapter{\taskTitle}

% task body
Les robots tamponneurs sont des robots très simples. Ils roulent sur un plateau de jeu divisé en cases.

{\centering%
\includesvg[scale=0.36]{\taskGraphicsFolder/graphics/2023-CZ-01-example1_compatible.svg}\par}

Pour les diriger, on commence par sélectionner un robot tamponneur. On l’envoie ensuite dans une certaine direction à l’aide d’une flèche-commande: en haut \raisebox{-0.5ex}[0pt][0pt]{\includesvg[scale=0.36]{\taskGraphicsFolder/graphics/2023-CZ-01-arrow-up.svg}}, en bas \raisebox{-0.5ex}[0pt][0pt]{\includesvg[scale=0.36]{\taskGraphicsFolder/graphics/2023-CZ-01-arrow-down.svg}}, à gauche \raisebox{-0.5ex}[0pt][0pt]{\includesvg[scale=0.36]{\taskGraphicsFolder/graphics/2023-CZ-01-arrow-left.svg}} ou à droite \raisebox{-0.5ex}[0pt][0pt]{\includesvg[scale=0.36]{\taskGraphicsFolder/graphics/2023-CZ-01-arrow-right.svg}}. Le robot tamponneurs roule ensuite tout droit dans la direction de la commande jusqu’à ce qu’il rencontre un obstacle \raisebox{-0.5ex}[0pt][0pt]{\includesvg[scale=0.36]{\taskGraphicsFolder/graphics/2023-CZ-01-square.svg}} ou un autre robot. Il s’arrête alors et ne bouge plus jusqu’à ce qu’il reçoive une nouvelle commande.

En utilisant un suite de commandes adaptée, tu dois faire en sorte que le robot tamponneur \raisebox{-0.5ex}[0pt][0pt]{\includesvg[scale=0.36]{\taskGraphicsFolder/graphics/2023-CZ-01-robot1.svg}} atteigne le but \raisebox{-0.5ex}[0pt][0pt]{\includesvg[scale=0.36]{\taskGraphicsFolder/graphics/2023-CZ-01-star.svg}} et y reste.

Le robot tamponneur \raisebox{-0.5ex}[0pt][0pt]{\includesvg[scale=0.36]{\taskGraphicsFolder/graphics/2023-CZ-01-robot1.svg}} atteint le but \raisebox{-0.5ex}[0pt][0pt]{\includesvg[scale=0.36]{\taskGraphicsFolder/graphics/2023-CZ-01-star.svg}} en suivant la suite de commandes suivante:

{\centering%
\raisebox{-0.5ex}[0pt][0pt]{\includesvg[scale=0.36]{\taskGraphicsFolder/graphics/2023-CZ-01-robot3.svg}} \raisebox{-0.5ex}[0pt][0pt]{\includesvg[scale=0.36]{\taskGraphicsFolder/graphics/2023-CZ-01-arrow-right.svg}} \raisebox{-0.5ex}[0pt][0pt]{\includesvg[scale=0.36]{\taskGraphicsFolder/graphics/2023-CZ-01-robot1.svg}} \raisebox{-0.5ex}[0pt][0pt]{\includesvg[scale=0.36]{\taskGraphicsFolder/graphics/2023-CZ-01-arrow-up.svg}} \raisebox{-0.5ex}[0pt][0pt]{\includesvg[scale=0.36]{\taskGraphicsFolder/graphics/2023-CZ-01-robot1.svg}} \raisebox{-0.5ex}[0pt][0pt]{\includesvg[scale=0.36]{\taskGraphicsFolder/graphics/2023-CZ-01-arrow-left.svg}}

{\centering%
\includesvg[scale=0.36]{\taskGraphicsFolder/graphics/2023-CZ-01-example2_compatible.svg}\par}\par}



% question (as \emph{})
{\em
Crée une suite de commandes avec quatre flèches permettant au robot tamponneur \raisebox{-0.5ex}[0pt][0pt]{\includesvg[scale=0.36]{\taskGraphicsFolder/graphics/2023-CZ-01-robot1.svg}} d’atteindre le but \raisebox{-0.5ex}[0pt][0pt]{\includesvg[scale=0.36]{\taskGraphicsFolder/graphics/2023-CZ-01-star.svg}}.

{\centering%
{\centering%
\includesvg[scale=0.36]{\taskGraphicsFolder/graphics/2023-CZ-01-question2.svg}\par}

{\centering%
\includesvg[scale=0.36]{\taskGraphicsFolder/graphics/2023-CZ-01-question_compatible.svg}\par}\par}


}

% answer alternatives (as \begin{enumerate}[A)]) or interactivity


% from here on this is only included if solutions are processed
\ifthenelse{\boolean{solutions}}{
\newpage

% answer explanation
\section*{\BrochureSolution}
Voici la bonne suite de commandes:
\raisebox{-0.5ex}[0pt][0pt]{\includesvg[scale=0.36]{\taskGraphicsFolder/graphics/2023-CZ-01-robot3.svg}} \raisebox{-0.5ex}[0pt][0pt]{\includesvg[scale=0.36]{\taskGraphicsFolder/graphics/2023-CZ-01-arrow-up.svg}}  \raisebox{-0.5ex}[0pt][0pt]{\includesvg[scale=0.36]{\taskGraphicsFolder/graphics/2023-CZ-01-robot2.svg}} \raisebox{-0.5ex}[0pt][0pt]{\includesvg[scale=0.36]{\taskGraphicsFolder/graphics/2023-CZ-01-arrow-left.svg}}  \raisebox{-0.5ex}[0pt][0pt]{\includesvg[scale=0.36]{\taskGraphicsFolder/graphics/2023-CZ-01-robot1.svg}} \raisebox{-0.5ex}[0pt][0pt]{\includesvg[scale=0.36]{\taskGraphicsFolder/graphics/2023-CZ-01-arrow-down.svg}}  \raisebox{-0.5ex}[0pt][0pt]{\includesvg[scale=0.36]{\taskGraphicsFolder/graphics/2023-CZ-01-robot1.svg}} \raisebox{-0.5ex}[0pt][0pt]{\includesvg[scale=0.36]{\taskGraphicsFolder/graphics/2023-CZ-01-arrow-right.svg}}

Afin que le robot \raisebox{-0.5ex}[0pt][0pt]{\includesvg[scale=0.36]{\taskGraphicsFolder/graphics/2023-CZ-01-robot1.svg}} atteigne le but grâce à une suite de commandes à quatre flèches, trois robots tamponneurs doivent coopérer. D’abord, \raisebox{-0.5ex}[0pt][0pt]{\includesvg[scale=0.36]{\taskGraphicsFolder/graphics/2023-CZ-01-robot3.svg}} roule vers le haut jusqu’à ce qu’il rencontre un obstacle. Il devient ainsi lui-même un obstacle pour \raisebox{-0.5ex}[0pt][0pt]{\includesvg[scale=0.36]{\taskGraphicsFolder/graphics/2023-CZ-01-robot2.svg}}, qui roule vers la gauche. \raisebox{-0.5ex}[0pt][0pt]{\includesvg[scale=0.36]{\taskGraphicsFolder/graphics/2023-CZ-01-robot2.svg}} va maintenant arrêter \raisebox{-0.5ex}[0pt][0pt]{\includesvg[scale=0.36]{\taskGraphicsFolder/graphics/2023-CZ-01-robot1.svg}} sur son chemin vers le bas à la hauteur du but. \raisebox{-0.5ex}[0pt][0pt]{\includesvg[scale=0.36]{\taskGraphicsFolder/graphics/2023-CZ-01-robot1.svg}} va aller à droite après \raisebox{-0.5ex}[0pt][0pt]{\includesvg[scale=0.36]{\taskGraphicsFolder/graphics/2023-CZ-01-robot2.svg}} et s’arrêter devant l’obstacle, juste sur la case du but.

{\centering%
\includesvg[scale=0.36]{\taskGraphicsFolder/graphics/2023-CZ-01-explanation_compatible.svg}\par}

Comment trouver la bonne suite de commandes? Nous pouvons commencer par la fin et réfléchir à quel doit être le dernier mouvement de \raisebox{-0.5ex}[0pt][0pt]{\includesvg[scale=0.36]{\taskGraphicsFolder/graphics/2023-CZ-01-robot1.svg}} avant le but. Il n’y a que deux possibilités: a) il vient de la gauche, comme dans notre exemple; b) il vient d’en haut. Dans ce cas, le robot tamponneur \raisebox{-0.5ex}[0pt][0pt]{\includesvg[scale=0.36]{\taskGraphicsFolder/graphics/2023-CZ-01-robot4.svg}} devrait être déplacé vers le haut à droite à l’aide de trois commandes afin d’être un obstacle pour \raisebox{-0.5ex}[0pt][0pt]{\includesvg[scale=0.36]{\taskGraphicsFolder/graphics/2023-CZ-01-robot1.svg}}. Nous aurions alors besoin de ${3 + 2 = 5}$ flèches-commandes.

Nous cherchons cependant une solution avec $4$ flèches-commandes, donc la solution a), dans laquelle \raisebox{-0.5ex}[0pt][0pt]{\includesvg[scale=0.36]{\taskGraphicsFolder/graphics/2023-CZ-01-robot1.svg}} vient de la gauche, doit être la bonne. L’avant-dernier mouvement de \raisebox{-0.5ex}[0pt][0pt]{\includesvg[scale=0.36]{\taskGraphicsFolder/graphics/2023-CZ-01-robot1.svg}} est alors de haut en bas. Pour qu’il s’arrête au bon endroit, il faut d’abord que les robots \raisebox{-0.5ex}[0pt][0pt]{\includesvg[scale=0.36]{\taskGraphicsFolder/graphics/2023-CZ-01-robot2.svg}} et \raisebox{-0.5ex}[0pt][0pt]{\includesvg[scale=0.36]{\taskGraphicsFolder/graphics/2023-CZ-01-robot3.svg}} se déplacent comme montré sur l’image.



% it's informatics
\section*{\BrochureItsInformatics}
Dans cet exercice du Castor, plusieurs robots ont travaillé ensemble pour atteindre un but. Ils avaient des tâches différentes: le robot bleu devait atteindre le but alors que les autres servaient d’obstacles.

La répartition des tâches est un aspect important de la robotique. Par exemple, dans un entrepôt automatisé, plusieurs robots différents travaillent ensemble pour ranger des marchandises, les touver et les transporter. Toutes les activités sont coordonnées de manière à ce que le moins de pauses inutiles aient lieu, à ce que les chemins de transport soient les plus courts possible, à ce que le moins d’énergie possible soit utilisé et donc à ce que l’entrepôt marche de manière aussi efficace que possible.

La \emph{robotique en essaim} est un domaine particulier de la robotique. Les robots y sont, comme les robots tamponneurs, des machines simples qui effectuent une tâche en groupe. En agriculture, des robots travaillant en essaim peuvent semer le maïs, observer le développement des plantes et la qualité du sol et même récolter les céréales. Chaque robot de l’essaim est petit et construit de manière simple, mais l’essaim dans son ensemble est un outil puissant. Ce principe est aussi valable pour les systèmes multi-agents: ce sont de simples unités logicielles qui peuvent ensemble résoudre des problèmes complexes. Le rôle de l’informatique est de développer des algorithmes permmettant une coordination et coopération optimales des agents – qu’il s’agisse de logiciel ou de matériel – composant ces systèmes.



% keywords and websites (as \begin{itemize})
\section*{\BrochureWebsitesAndKeywords}
{\raggedright
\begin{itemize}
  \item Robotique: \href{https://fr.wikipedia.org/wiki/Robotique}{\BrochureUrlText{https://fr.wikipedia.org/wiki/Robotique}}
  \item Robotique en essaim: \href{https://fr.wikipedia.org/wiki/Robotique_en_essaim}{\BrochureUrlText{https://fr.wikipedia.org/wiki/Robotique\_en\_essaim}}
  \item Robotique industrielle: \href{https://fr.wikipedia.org/wiki/Robotique_industrielle}{\BrochureUrlText{https://fr.wikipedia.org/wiki/Robotique\_industrielle}}
  \item Système multi-agents: \href{https://fr.wikipedia.org/wiki/Syst\%C3\%A8me_multi-agents}{\BrochureUrlText{https://fr.wikipedia.org/wiki/Système\_multi-agents}}
\end{itemize}


}

% end of ifthen for excluding the solutions
}{}

% all authors
% ATTENTION: you HAVE to make sure an according entry is in ../main/authors.tex.
% Syntax: \def\AuthorLastnameF{} (Lastname is last name, F is first letter of first name, this serves as a marker for ../main/authors.tex)
\def\AuthorVanicekJ{} % \ifdefined\AuthorVanicekJ \BrochureFlag{cz}{} Jiří Vaníček\fi
\def\AuthorMaoY{} % \ifdefined\AuthorMaoY \BrochureFlag{cn}{} Yong Mao\fi
\def\AuthorKoleszarV{} % \ifdefined\AuthorKoleszarV \BrochureFlag{uy}{} Víctor Koleszar\fi
\def\AuthorAhmedA{} % \ifdefined\AuthorAhmedA \BrochureFlag{sa}{} Akram Ahmed\fi
\def\AuthorWeigendM{} % \ifdefined\AuthorWeigendM \BrochureFlag{de}{} Michael Weigend\fi
\def\AuthorDatzkoThutS{} % \ifdefined\AuthorDatzkoThutS \BrochureFlag{de}{} Susanne Datzko-Thut\fi
\def\AuthorPelletE{} % \ifdefined\AuthorPelletE \BrochureFlag{ch}{} Elsa Pellet\fi

\newpage}{}
