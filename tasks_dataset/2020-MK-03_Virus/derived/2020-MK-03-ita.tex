\documentclass[a4paper,11pt]{report}
\usepackage[T1]{fontenc}
\usepackage[utf8]{inputenc}

\usepackage[italian]{babel}
\AtBeginDocument{\def\labelitemi{$\bullet$}}

\usepackage{etoolbox}

\usepackage[margin=2cm]{geometry}
\usepackage{changepage}
\makeatletter
\renewenvironment{adjustwidth}[2]{%
    \begin{list}{}{%
    \partopsep\z@%
    \topsep\z@%
    \listparindent\parindent%
    \parsep\parskip%
    \@ifmtarg{#1}{\setlength{\leftmargin}{\z@}}%
                 {\setlength{\leftmargin}{#1}}%
    \@ifmtarg{#2}{\setlength{\rightmargin}{\z@}}%
                 {\setlength{\rightmargin}{#2}}%
    }
    \item[]}{\end{list}}
\makeatother

\newcommand{\BrochureUrlText}[1]{\texttt{#1}}
\usepackage{setspace}
\setstretch{1.15}

\usepackage{tabularx}
\usepackage{booktabs}
\usepackage{makecell}
\usepackage{multirow}
\renewcommand\theadfont{\bfseries}
\renewcommand{\tabularxcolumn}[1]{>{}m{#1}}
\newcolumntype{R}{>{\raggedleft\arraybackslash}X}
\newcolumntype{C}{>{\centering\arraybackslash}X}
\newcolumntype{L}{>{\raggedright\arraybackslash}X}
\newcolumntype{J}{>{\arraybackslash}X}

\newcommand{\BrochureInlineCode}[1]{{\ttfamily #1}}

\usepackage{amssymb}
\usepackage{amsmath}

\usepackage[babel=true,maxlevel=3]{csquotes}
\DeclareQuoteStyle{bebras-ch-eng}{“}[” ]{”}{‘}[”’ ]{’}\DeclareQuoteStyle{bebras-ch-deu}{«}[» ]{»}{“}[»› ]{”}
\DeclareQuoteStyle{bebras-ch-fra}{«\thinspace{}}[» ]{\thinspace{}»}{“}[»\thinspace{}› ]{”}
\DeclareQuoteStyle{bebras-ch-ita}{«}[» ]{»}{“}[»› ]{”}
\setquotestyle{bebras-ch-ita}

\usepackage{hyperref}
\usepackage{graphicx}
\usepackage{svg}
\svgsetup{inkscapeversion=1,inkscapearea=page}
\usepackage{wrapfig}

\usepackage{enumitem}
\setlist{nosep,itemsep=.5ex}

\setlength{\parindent}{0pt}
\setlength{\parskip}{2ex}
\raggedbottom

\usepackage{fancyhdr}
\usepackage{lastpage}
\pagestyle{fancy}

\fancyhf{}
\renewcommand{\headrulewidth}{0pt}
\renewcommand{\footrulewidth}{0.4pt}
\lfoot{\scriptsize © 2020 Bebras (CC BY-SA 4.0)}
\cfoot{\scriptsize\itshape 2020-MK-03 Rete di comunicazione}
\rfoot{\scriptsize Page~\thepage{}/\pageref*{LastPage}}

\newcommand{\taskGraphicsFolder}{..}

\begin{document}

\section*{\centering{} 2020-MK-03 Rete di comunicazione}


\subsection*{Body}

Ai castori piace diffondere notizie tra di loro. A tale scopo utilizzano la rete di comunicazione qui sotto. Quando un castoro riceve un nuovo messaggio, lo inoltra a tutti coloro con cui è collegato da un canale di comunicazione diretta (una linea bianca). L’invio dei messaggi si effettua a turni. C’è sempre un turno tra l’invio e la ricezione.

{\em

\subsection*{Question/Challenge}

Da quale castoro un messaggio raggiunge tutti gli altri castori più velocemente, cioè nel minor numero di turni?

{\centering%
\includesvg[width=252.5px]{\taskGraphicsFolder/graphics/2020-MK-03_taskbody-compatible.svg}\par}

}\begingroup
\renewcommand{\arraystretch}{1.5}
\subsection*{Answer Options/Interactivity Description}



\endgroup

\subsection*{Answer Explanation}

La risposta corretta è il castoro B. Può diffondere un messaggio a tutti gli altri castori in due turni.

Nel primo turno, il castoro B invia il messaggio ai suoi vicini, cioè i castori A, D e J collegati ad un canale di comunicazione diretta. L’immagine sottostante mostra chi ha il messaggio dopo questo primo turno.

{\centering%
\includesvg[width=252.5px]{\taskGraphicsFolder/graphics/2020-MK-03_explanation1-compatible.svg}\par}

Nel secondo turno i castori A, D e J inviano il messaggio ai loro vicini: - Il castoro A invia il messaggio ai castori E e H;

\begin{itemize}
  \item Il castoro D invia il messaggio ai castori I e K;
  \item Il castoro J invia il messaggio ai castori C, F, G e L.
\end{itemize}

Inoltre, il castoro B riceve il messaggio tre volte, perché anche lui è un vicino dei tre castori A, D e J. Poiché questo non è un messaggio nuovo per lui, il castoro B non lo invierà nei prossimi turni. Anche i castori A e D si invieranno di nuovo il messaggio attraverso il loro canale di comunicazione diretta, ma dopo di che non lo invieranno più perché non è più nuovo per loro.

\begin{samepage}
L’immagine sottostante mostra la situazione dopo il secondo turno.

\nopagebreak

{\centering%
\includesvg[width=252.5px]{\taskGraphicsFolder/graphics/2020-MK-03_explanation2-compatible.svg}\par}
\end{samepage}

Così la notizia ha raggiunto tutti i castori in soli due turni.

Non c’è un modo più veloce, perché altrimenti un castoro dovrebbe essere collegato a tutti gli altri castori con una linea bianca per inviare il messaggio direttamente a tutti gli altri castori in un unico turno.

Il castoro B è anche l’unico dal quale un messaggio raggiunge tutti gli altri castori in soli due turni: per i castori C, E, F, G, H, J e L, il castoro I non sarebbe raggiungibile in due turni. E per i castori A, D, E, E, H, I e K, il castoro L non può essere raggiunto in due turni.


\subsection*{It’s Informatics}

La rete di comunicazione dei castori può essere descritta da un \emph{grafo}. Ogni castoro si trova in un cosiddetto \emph{nodo}, che in questo caso è nominato con una lettera. Le linee bianche sono chiamate \emph{archi}, ognuna di esse collega due nodi. I messaggi si diffondono nella rete di comunicazione attraverso turni \emph{sincronizzati}, cioè tutti i castori inviano contemporaneamente. In un turno, ogni castoro invia nuovi messaggi a tutti i suoi vicini. Quello che i castori fanno qui è quello che gli informatici chiamano \emph{broadcasting} (inglese per “trasmettere”)  in una \emph{rete di comunicazioni}. Nel compito di cui sopra, abbiamo studiato la velocità con cui una tale trasmissione può essere completata, cioè la velocità con cui un nuovo messaggio può raggiungere tutti i partecipanti.

Un compito ancora più complesso è quello di progettare le reti in modo tale che sia possibile una trasmissione veloce da tutti i nodi, ma con un numero di connessioni non troppo elevato.
Il nodo del ricercato castoro B è chiamato poi il centro del grafo. In astratto, il centro è un nodo che riduce al minimo la distanza dai nodi più lontani. Non c’è quindi nessun altro nodo che avrebbe una distanza minore rispetto a tutti gli altri nodi. Nel presente compito c’è un solo centro. Tuttavia, a seconda del grafo, possono esserci diversi nodi, in modo che ognuno di essi minimizzi la distanza dai nodi più lontani da esso; quindi, un grafo può avere diversi centri.

Trovare un centro non è sempre facile come in questo compito. Per prima cosa, potrebbe essere che ci vogliano diversi turni per il trasferimento tra alcuni nodi direttamente collegati. D’altra parte, i grafi possono essere semplicemente molto più grandi e complessi. Per tali grafi, si potrebbe, ad esempio, utilizzare l’algoritmo di Floyd-Warshall per trovare in modo efficiente un centro.

{\raggedright

\subsection*{Keywords and Websites}

\begin{itemize}
  \item Grafo: \href{https://it.wikipedia.org/wiki/Grafo}{\BrochureUrlText{https://it.wikipedia.org/wiki/Grafo}}
  \item Algoritmo di Floyd-Warshall: \href{https://it.wikipedia.org/wiki/Algoritmo_di_Floyd-Warshall}{\BrochureUrlText{https://it.wikipedia.org/wiki/Algoritmo\_di\_Floyd-Warshall}}
\end{itemize}


}
\end{document}
