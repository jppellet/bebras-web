% Definition of the meta information: task difficulties, task ID, task title, task country; definition of the variables as well as their scope is in commands.tex
\setcounter{taskAgeDifficulty3to4}{1}
\setcounter{taskAgeDifficulty5to6}{0}
\setcounter{taskAgeDifficulty7to8}{0}
\setcounter{taskAgeDifficulty9to10}{0}
\setcounter{taskAgeDifficulty11to13}{0}
\renewcommand{\taskTitle}{Cappelli nuovi}
\renewcommand{\taskCountry}{LT}

% include this task only if for the age groups being processed this task is relevant
\ifthenelse{
  \(\boolean{age3to4} \AND \(\value{taskAgeDifficulty3to4} > 0\)\) \OR
  \(\boolean{age5to6} \AND \(\value{taskAgeDifficulty5to6} > 0\)\) \OR
  \(\boolean{age7to8} \AND \(\value{taskAgeDifficulty7to8} > 0\)\) \OR
  \(\boolean{age9to10} \AND \(\value{taskAgeDifficulty9to10} > 0\)\) \OR
  \(\boolean{age11to13} \AND \(\value{taskAgeDifficulty11to13} > 0\)\)}{

\newchapter{\taskTitle}

% task body
I castori hanno nuovi cappelli.

{\centering%
\includesvg[scale=0.5]{\taskGraphicsFolder/graphics/2023-LT-01-taskbody.svg}\par}



% question (as \emph{})
{\em
Ordina i cappelli in base alle dimensioni.


}

% answer alternatives (as \begin{enumerate}[A)]) or interactivity


% from here on this is only included if solutions are processed
\ifthenelse{\boolean{solutions}}{
\newpage

% answer explanation
\section*{\BrochureSolution}
In questo modo i cappelli sono ordinati correttamente:

{\centering%
\raisebox{-0.5ex}{\includesvg[width=144.3px]{\taskGraphicsFolder/graphics/2023-LT-01-answer01.svg}} \raisebox{-0.5ex}{\includesvg[width=144.3px]{\taskGraphicsFolder/graphics/2023-LT-01-answer02.svg}}\par}

Ci sono due soluzioni corrette, i cappelli sono da sinistra a destra

\begin{itemize}
  \item sempre più grandi o
  \item sempre più piccoli.
\end{itemize}

Quando ordiniamo i castori, prestiamo attenzione solo ai cappelli. In questo modo è molto più facile ordinarli in base alla taglia.

{\centering%
\raisebox{-0.5ex}{\includesvg[width=144.3px]{\taskGraphicsFolder/graphics/2023-LT-01-explanation.svg}} \raisebox{-0.5ex}{\includesvg[width=144.3px]{\taskGraphicsFolder/graphics/2023-LT-01-explanation_alt.svg}}\par}



% it's informatics
\section*{\BrochureItsInformatics}
Molte cose nel nostro ambiente sono ordinate per facilitare la scelta dei singoli oggetti: se gli utensili sono ordinati per dimensione, è più facile trovare l’utensile giusto. Poiché le voci di un dizionario sono ordinate alfabeticamente, è possibile trovare più rapidamente la pagina con la parola cercata.

In questo compito si deve ordinare i castori in base alla dimensione dei cappelli. La difficoltà, tuttavia, è che la \emph{proprietà} \enquote{dimensione dei cappelli} non è facilmente identificabile. Potremmo ordinare in base ad almeno tre dimensioni diverse:

\begin{itemize}
  \item Dimensione del castoro (\raisebox{-0.5ex}[0pt][0pt]{\includesvg[width=13px]{\taskGraphicsFolder/graphics/2023-LT-01-informatics03.svg}})
  \item Dimensione dei cappelli (\raisebox{-0.5ex}[0pt][0pt]{\includesvg[width=8.7px]{\taskGraphicsFolder/graphics/2023-LT-01-informatics02.svg}})
  \item dimensione totale (\raisebox{-0.5ex}[0pt][0pt]{\includesvg[width=8.7px]{\taskGraphicsFolder/graphics/2023-LT-01-informatics02.svg}} + \raisebox{-0.5ex}[0pt][0pt]{\includesvg[width=13px]{\taskGraphicsFolder/graphics/2023-LT-01-informatics03.svg}})
\end{itemize}

{\centering%
\includesvg[width=144.3px]{\taskGraphicsFolder/graphics/2023-LT-01-informatics01-compatible.svg}\par}

La classificazione dei castori è diversa per ciascuna delle tre caratteristiche dimensionali.

{\centering%
\begin{tabular}{ @{} l c c c @{} }
  {\setstretch{1.0}\thead[lb]{Castoro}} & {\setstretch{1.0}\thead[cb]{${~~~}$\includesvg[width=8.7px]{\taskGraphicsFolder/graphics/2023-LT-01-informatics02.svg}${~~~}$}} & {\setstretch{1.0}\thead[cb]{${~~~}$\includesvg[width=13px]{\taskGraphicsFolder/graphics/2023-LT-01-informatics03.svg}${~~~}$}} & {\setstretch{1.0}\thead[cb]{\raisebox{-0.5ex}[0pt][0pt]{\includesvg[width=8.7px]{\taskGraphicsFolder/graphics/2023-LT-01-informatics02.svg}} + \raisebox{-0.5ex}[0pt][0pt]{\includesvg[width=13px]{\taskGraphicsFolder/graphics/2023-LT-01-informatics03.svg}}}} \\ 
\midrule
  A & 3 & 9 & 12 \\ 
  B & 6 & 3 & 9 \\ 
  C & 2 & 4 & 6 \\ 
  D & 4 & 5 & 9 \\ 
  E & 5 & 7 & 12
\end{tabular}

\par}

Per l’ordinamento, è quindi importante innanzitutto definire esattamente la proprietà in base alla quale deve avvenire l’ordinamento.  In secondo luogo, i valori di questa proprietà devono essere ordinabili.  Possiamo ordinare in base a proprietà espresse in numeri (come dimensione, lunghezza, peso, …): per due numeri diversi, possiamo dire quale numero è il più piccolo.  Le parole possono essere ordinate perché l’ordine delle lettere dell’alfabeto è fisso e quindi per due parole diverse è chiaro quale deve essere in testa al dizionario. In generale, possiamo dire che una proprietà è ordinabile se possiamo specificare una relazione unica \enquote{meno di} (un \emph{ordine}) per i suoi singoli valori.

I computer vengono utilizzati per gestire grandi quantità di dati. Per poter trovare rapidamente i singoli dati, è necessario ordinarli. L’informatica conosce molti metodi di ordinamento rapido ed è ben studiata in quali casi si debbano utilizzare tali metodi.



% keywords and websites (as \begin{itemize})
\section*{\BrochureWebsitesAndKeywords}
{\raggedright
\begin{itemize}
  \item Ordinamento: \href{https://it.wikipedia.org/wiki/Algoritmo_di_ordinamento}{\BrochureUrlText{https://it.wikipedia.org/wiki/Algoritmo\_di\_ordinamento}}
  \item Relazione d’ordine: \href{https://it.wikipedia.org/wiki/Relazione_d\%27ordine}{\BrochureUrlText{https://it.wikipedia.org/wiki/Relazione\_d’ordine}}
  \item Algoritmo di ricerca: \href{https://it.wikipedia.org/wiki/Algoritmo_di_ricerca}{\BrochureUrlText{https://it.wikipedia.org/wiki/Algoritmo\_di\_ricerca}}
\end{itemize}


}

% end of ifthen for excluding the solutions
}{}

% all authors
% ATTENTION: you HAVE to make sure an according entry is in ../main/authors.tex.
% Syntax: \def\AuthorLastnameF{} (Lastname is last name, F is first letter of first name, this serves as a marker for ../main/authors.tex)
\def\AuthorDagieneV{} % \ifdefined\AuthorDagieneV \BrochureFlag{lt}{} Valentina Dagienė\fi
\def\AuthorFutschekG{} % \ifdefined\AuthorFutschekG \BrochureFlag{at}{} Gerald Futschek\fi
\def\AuthorPohlW{} % \ifdefined\AuthorPohlW \BrochureFlag{de}{} Wolfgang Pohl\fi
\def\AuthorDasovicD{} % \ifdefined\AuthorDasovicD \BrochureFlag{hr}{} Darija Dasović\fi
\def\AuthorKinciusV{} % \ifdefined\AuthorKinciusV \BrochureFlag{lt}{} Vaidotas Kinčius\fi
\def\AuthorDatzkoThutS{} % \ifdefined\AuthorDatzkoThutS \BrochureFlag{de}{} Susanne Datzko-Thut\fi
\def\AuthorBaumannW{} % \ifdefined\AuthorBaumannW \BrochureFlag{at}{} Wilfried Baumann\fi
\def\AuthorPluharZ{} % \ifdefined\AuthorPluharZ \BrochureFlag{hu}{} Zsuzsa Pluhár\fi
\def\AuthorGiangC{} % \ifdefined\AuthorGiangC \BrochureFlag{ch}{} Christian Giang\fi

\newpage}{}
