\documentclass[a4paper,11pt]{report}
\usepackage[T1]{fontenc}
\usepackage[utf8]{inputenc}

\usepackage[italian]{babel}
\AtBeginDocument{\def\labelitemi{$\bullet$}}

\usepackage{etoolbox}

\usepackage[margin=2cm]{geometry}
\usepackage{changepage}
\makeatletter
\renewenvironment{adjustwidth}[2]{%
    \begin{list}{}{%
    \partopsep\z@%
    \topsep\z@%
    \listparindent\parindent%
    \parsep\parskip%
    \@ifmtarg{#1}{\setlength{\leftmargin}{\z@}}%
                 {\setlength{\leftmargin}{#1}}%
    \@ifmtarg{#2}{\setlength{\rightmargin}{\z@}}%
                 {\setlength{\rightmargin}{#2}}%
    }
    \item[]}{\end{list}}
\makeatother

\newcommand{\BrochureUrlText}[1]{\texttt{#1}}
\usepackage{setspace}
\setstretch{1.15}

\usepackage{tabularx}
\usepackage{booktabs}
\usepackage{makecell}
\usepackage{multirow}
\renewcommand\theadfont{\bfseries}
\renewcommand{\tabularxcolumn}[1]{>{}m{#1}}
\newcolumntype{R}{>{\raggedleft\arraybackslash}X}
\newcolumntype{C}{>{\centering\arraybackslash}X}
\newcolumntype{L}{>{\raggedright\arraybackslash}X}
\newcolumntype{J}{>{\arraybackslash}X}

\newcommand{\BrochureInlineCode}[1]{{\ttfamily #1}}

\usepackage{amssymb}
\usepackage{amsmath}

\usepackage[babel=true,maxlevel=3]{csquotes}
\DeclareQuoteStyle{bebras-ch-eng}{“}[” ]{”}{‘}[”’ ]{’}\DeclareQuoteStyle{bebras-ch-deu}{«}[» ]{»}{“}[»› ]{”}
\DeclareQuoteStyle{bebras-ch-fra}{«\thinspace{}}[» ]{\thinspace{}»}{“}[»\thinspace{}› ]{”}
\DeclareQuoteStyle{bebras-ch-ita}{«}[» ]{»}{“}[»› ]{”}
\setquotestyle{bebras-ch-ita}

\usepackage{hyperref}
\usepackage{graphicx}
\usepackage{svg}
\svgsetup{inkscapeversion=1,inkscapearea=page}
\usepackage{wrapfig}

\usepackage{enumitem}
\setlist{nosep,itemsep=.5ex}

\setlength{\parindent}{0pt}
\setlength{\parskip}{2ex}
\raggedbottom

\usepackage{fancyhdr}
\usepackage{lastpage}
\pagestyle{fancy}

\fancyhf{}
\renewcommand{\headrulewidth}{0pt}
\renewcommand{\footrulewidth}{0.4pt}
\lfoot{\scriptsize © 2023 Bebras (CC BY-SA 4.0)}
\cfoot{\scriptsize\itshape 2023-LT-01 Cappelli nuovi}
\rfoot{\scriptsize Page~\thepage{}/\pageref*{LastPage}}

\newcommand{\taskGraphicsFolder}{..}

\begin{document}

\section*{\centering{} 2023-LT-01 Cappelli nuovi}


\subsection*{Body}

I castori hanno nuovi cappelli.

{\centering%
\includesvg[scale=0.5]{\taskGraphicsFolder/graphics/2023-LT-01-taskbody.svg}\par}

{\em


\subsection*{Question/Challenge - for the brochures}

Ordina i cappelli in base alle dimensioni.

}


\subsection*{Interactivity instruction - for the online challenge}

Trascina i castori nell’ordine giusto. Al termine, fa clic su \enquote{Salva risposta}.

\begingroup
\renewcommand{\arraystretch}{1.5}
\subsection*{Answer Options/Interactivity Description}

\begin{itemize}
  \item The beavers with the hats can be dragged from right to left. Dragging the beaver between $2$ other beavers is possible. The interacitvity opens a space between the $2$ beavers.
\end{itemize}

\endgroup

\subsection*{Answer Explanation}

In questo modo i cappelli sono ordinati correttamente:

{\centering%
\raisebox{-0.5ex}{\includesvg[width=144.3px]{\taskGraphicsFolder/graphics/2023-LT-01-answer01.svg}} \raisebox{-0.5ex}{\includesvg[width=144.3px]{\taskGraphicsFolder/graphics/2023-LT-01-answer02.svg}}\par}

Ci sono due soluzioni corrette, i cappelli sono da sinistra a destra

\begin{itemize}
  \item sempre più grandi o
  \item sempre più piccoli.
\end{itemize}

Quando ordiniamo i castori, prestiamo attenzione solo ai cappelli. In questo modo è molto più facile ordinarli in base alla taglia.

{\centering%
\raisebox{-0.5ex}{\includesvg[width=144.3px]{\taskGraphicsFolder/graphics/2023-LT-01-explanation.svg}} \raisebox{-0.5ex}{\includesvg[width=144.3px]{\taskGraphicsFolder/graphics/2023-LT-01-explanation_alt.svg}}\par}


\subsection*{This is Informatics}

Molte cose nel nostro ambiente sono ordinate per facilitare la scelta dei singoli oggetti: se gli utensili sono ordinati per dimensione, è più facile trovare l’utensile giusto. Poiché le voci di un dizionario sono ordinate alfabeticamente, è possibile trovare più rapidamente la pagina con la parola cercata.

In questo compito si deve ordinare i castori in base alla dimensione dei cappelli. La difficoltà, tuttavia, è che la \emph{proprietà} \enquote{dimensione dei cappelli} non è facilmente identificabile. Potremmo ordinare in base ad almeno tre dimensioni diverse:

\begin{itemize}
  \item Dimensione del castoro (\raisebox{-0.5ex}[0pt][0pt]{\includesvg[width=13px]{\taskGraphicsFolder/graphics/2023-LT-01-informatics03.svg}})
  \item Dimensione dei cappelli (\raisebox{-0.5ex}[0pt][0pt]{\includesvg[width=8.7px]{\taskGraphicsFolder/graphics/2023-LT-01-informatics02.svg}})
  \item dimensione totale (\raisebox{-0.5ex}[0pt][0pt]{\includesvg[width=8.7px]{\taskGraphicsFolder/graphics/2023-LT-01-informatics02.svg}} + \raisebox{-0.5ex}[0pt][0pt]{\includesvg[width=13px]{\taskGraphicsFolder/graphics/2023-LT-01-informatics03.svg}})
\end{itemize}

{\centering%
\includesvg[width=144.3px]{\taskGraphicsFolder/graphics/2023-LT-01-informatics01-compatible.svg}\par}

La classificazione dei castori è diversa per ciascuna delle tre caratteristiche dimensionali.

{\centering%
\begin{tabular}{ @{} l c c c @{} }
  {\setstretch{1.0}\thead[lb]{Castoro}} & {\setstretch{1.0}\thead[cb]{${~~~}$\includesvg[width=8.7px]{\taskGraphicsFolder/graphics/2023-LT-01-informatics02.svg}${~~~}$}} & {\setstretch{1.0}\thead[cb]{${~~~}$\includesvg[width=13px]{\taskGraphicsFolder/graphics/2023-LT-01-informatics03.svg}${~~~}$}} & {\setstretch{1.0}\thead[cb]{\raisebox{-0.5ex}[0pt][0pt]{\includesvg[width=8.7px]{\taskGraphicsFolder/graphics/2023-LT-01-informatics02.svg}} + \raisebox{-0.5ex}[0pt][0pt]{\includesvg[width=13px]{\taskGraphicsFolder/graphics/2023-LT-01-informatics03.svg}}}} \\ 
\midrule
  A & 3 & 9 & 12 \\ 
  B & 6 & 3 & 9 \\ 
  C & 2 & 4 & 6 \\ 
  D & 4 & 5 & 9 \\ 
  E & 5 & 7 & 12
\end{tabular}

\par}

Per l’ordinamento, è quindi importante innanzitutto definire esattamente la proprietà in base alla quale deve avvenire l’ordinamento.  In secondo luogo, i valori di questa proprietà devono essere ordinabili.  Possiamo ordinare in base a proprietà espresse in numeri (come dimensione, lunghezza, peso, …): per due numeri diversi, possiamo dire quale numero è il più piccolo.  Le parole possono essere ordinate perché l’ordine delle lettere dell’alfabeto è fisso e quindi per due parole diverse è chiaro quale deve essere in testa al dizionario. In generale, possiamo dire che una proprietà è ordinabile se possiamo specificare una relazione unica \enquote{meno di} (un \emph{ordine}) per i suoi singoli valori.

I computer vengono utilizzati per gestire grandi quantità di dati. Per poter trovare rapidamente i singoli dati, è necessario ordinarli. L’informatica conosce molti metodi di ordinamento rapido ed è ben studiata in quali casi si debbano utilizzare tali metodi.


\subsection*{This is Computational Thinking}

Optional - not to be filled 2023


\subsection*{Informatics Keywords and Websites}

\begin{itemize}
  \item Ordinamento: \href{https://it.wikipedia.org/wiki/Algoritmo_di_ordinamento}{\BrochureUrlText{https://it.wikipedia.org/wiki/Algoritmo\_di\_ordinamento}}
  \item Relazione d’ordine: \href{https://it.wikipedia.org/wiki/Relazione_d\%27ordine}{\BrochureUrlText{https://it.wikipedia.org/wiki/Relazione\_d’ordine}}
  \item Algoritmo di ricerca: \href{https://it.wikipedia.org/wiki/Algoritmo_di_ricerca}{\BrochureUrlText{https://it.wikipedia.org/wiki/Algoritmo\_di\_ricerca}}
\end{itemize}


\subsection*{Computational Thinking Keywords and Websites}

\begin{itemize}
  \item 
\end{itemize}


\end{document}
