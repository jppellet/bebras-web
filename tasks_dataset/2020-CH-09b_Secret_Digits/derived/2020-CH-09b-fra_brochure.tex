% Definition of the meta information: task difficulties, task ID, task title, task country; definition of the variables as well as their scope is in commands.tex
\setcounter{taskAgeDifficulty3to4}{2}
\setcounter{taskAgeDifficulty5to6}{1}
\setcounter{taskAgeDifficulty7to8}{0}
\setcounter{taskAgeDifficulty9to10}{0}
\setcounter{taskAgeDifficulty11to13}{0}
\renewcommand{\taskTitle}{Chiffres secrets}
\renewcommand{\taskCountry}{CH}

% include this task only if for the age groups being processed this task is relevant
\ifthenelse{
  \(\boolean{age3to4} \AND \(\value{taskAgeDifficulty3to4} > 0\)\) \OR
  \(\boolean{age5to6} \AND \(\value{taskAgeDifficulty5to6} > 0\)\) \OR
  \(\boolean{age7to8} \AND \(\value{taskAgeDifficulty7to8} > 0\)\) \OR
  \(\boolean{age9to10} \AND \(\value{taskAgeDifficulty9to10} > 0\)\) \OR
  \(\boolean{age11to13} \AND \(\value{taskAgeDifficulty11to13} > 0\)\)}{

\newchapter{\taskTitle}

% task body
L’année de construction de chaque hutte de castor est écrite sur un panneau en dessus de l’entrée. Les castors utilisent leurs propres symboles pour représenter les chiffres. La table à droite montre comment les symboles des castors sont assemblés à partir des chiffres:

{\centering%
\includesvg[width=166px]{\taskGraphicsFolder/graphics/2020-CH-09_taskbody1-compatible.svg}\par}

Par exemple, les castors représentent le chiffre “$5$” par le nouveau symbole \raisebox{-0.5ex}[0pt][0pt]{\includesvg[width=21.6px]{\taskGraphicsFolder/graphics/2020-CH-09_taskbody2.svg}}, qui est assemblé comme ça:

{\centering%
\includesvg[width=166px]{\taskGraphicsFolder/graphics/2020-CH-09_taskbody3-compatible.svg}\par}

Voici la hutte de Cleverias:

{\centering%
\includesvg[width=274.2px]{\taskGraphicsFolder/graphics/2020-CH-09_taskbody4.svg}\par}



% question (as \emph{})
{\em
En quelle année la hutte de Cleverias a-t-elle été construite?


}

% answer alternatives (as \begin{enumerate}[A)]) or interactivity
\begin{tabular}{ @{} r l @{} }
  A) & 0978 \\ 
  B) & 1574 \\ 
  C) & 1923 \\ 
  D) & 1973 \\ 
  E) & 1993 \\ 
  F) & 2973 \\ 
  G) & 6378
\end{tabular}



% from here on this is only included if solutions are processed
\ifthenelse{\boolean{solutions}}{
\newpage

% answer explanation
\section*{\BrochureSolution}
Tu peux trouver l’année de construction de la hutte en déterminant la ligne et la colonne correspondant à chaque symbole. Le chiffre recherché se trouve à l’intersection de la ligne et de la colonne.

Comme il y a quatre symboles, tu fais cela quatre fois.

\begin{tabularx}{\columnwidth}{ @{} C C C C @{} }
  \makecell[c]{\includesvg[width=28.1px]{\taskGraphicsFolder/graphics/2020-CH-09b_explanation1.svg}} & \makecell[c]{\includesvg[width=36.1px]{\taskGraphicsFolder/graphics/2020-CH-09b_explanation2.svg}} & \makecell[c]{\includesvg[width=36.1px]{\taskGraphicsFolder/graphics/2020-CH-09b_explanation3.svg}} & \makecell[c]{\includesvg[width=36.1px]{\taskGraphicsFolder/graphics/2020-CH-09b_explanation4.svg}} \\ 
  \makecell[c]{\includesvg[width=97.4px]{\taskGraphicsFolder/graphics/2020-CH-09_explanation_digit1.svg}} & \makecell[c]{\includesvg[width=97.4px]{\taskGraphicsFolder/graphics/2020-CH-09_explanation_digit9.svg}} & \makecell[c]{\includesvg[width=97.4px]{\taskGraphicsFolder/graphics/2020-CH-09_explanation_digit7.svg}} & \makecell[c]{\includesvg[width=97.4px]{\taskGraphicsFolder/graphics/2020-CH-09_explanation_digit3.svg}}
\end{tabularx}

Les quatre chiffres dans le bon ordre donnent le nombre $1973$.



% it's informatics
\section*{\BrochureItsInformatics}
Garder des informations secrètes ou protéger des données est une tâche vielle de $4000$ ans. D’innombrables écritures secrètes ont été développées et utilisées dans ce but. Aujourd’hui, la sécurité des données est l’un des thèmes majeurs de l’informatique. Une des méthodes pour empêcher la lecture non autorisée de données est de les \emph{chiffrer}. Le chiffrement transforme un \emph{texte clair} en \emph{cryptogramme}. La reconstruction du texte clair à partir du cryptogramme s’appelle \emph{déchiffrement}. L’étude des cryptogrammes s’appelle \emph{cryptologie}.

Les cultures antiques utilisaient le plus souvent des écritures secrètes remplaçant des lettres par d’autres lettres ou de tout nouveaux symboles. L’écriture secrète utilisée ici a été développée spécialement pour le Castor Informatique, mais se base sur un concept venant de la Palestine antique. À l’époque, la règle de sécurité était que seules des écriture secrètes faciles à apprendre par cœur pouvaient être utilisées. C’était considéré comme un trop grand risque de garder une description écrite de l’écriture secrète. Une table comme celle utilisée ici est facile à apprendre par cœur. Le célèbre chiffre des francs-maçon se base sur ce principe.

Au lieu d’assembler de nouveaux symboles seulement pour les chiffres, on peut également inventer une écriture secrète pour les textes. Pour cela, on écrit toutes les lettres dans une table, puis on invente de nouveaux symboles pour les lignes et les colonnes. Ainsi, on obtient un nouveau symbole pour chaque lettre.



% keywords and websites (as \begin{itemize})
\section*{\BrochureWebsitesAndKeywords}
{\raggedright
\begin{itemize}
  \item Cryptographie: \href{https://fr.wikipedia.org/wiki/Cryptographie}{\BrochureUrlText{https://fr.wikipedia.org/wiki/Cryptographie}}
  \item Cryptogramme
\end{itemize}


}

% end of ifthen for excluding the solutions
}{}

% all authors
% ATTENTION: you HAVE to make sure an according entry is in ../main/authors.tex.
% Syntax: \def\AuthorLastnameF{} (Lastname is last name, F is first letter of first name, this serves as a marker for ../main/authors.tex)
\def\AuthorHromkovicJ{} % \ifdefined\AuthorHromkovicJ \BrochureFlag{ch}{} Juraj Hromkovič\fi
\def\AuthorLacherR{} % \ifdefined\AuthorLacherR \BrochureFlag{ch}{} Regula Lacher\fi
\def\AuthorDatzkoS{} % \ifdefined\AuthorDatzkoS \BrochureFlag{ch}{} Susanne Datzko\fi
\def\AuthorChanS{} % \ifdefined\AuthorChanS \BrochureFlag{ca}{} Sarah Chan\fi
\def\AuthorBergsveinsdottirL{} % \ifdefined\AuthorBergsveinsdottirL \BrochureFlag{is}{} Linda Björk Bergsveinsdóttir\fi
\def\AuthorNaughtonT{} % \ifdefined\AuthorNaughtonT \BrochureFlag{ie}{} Tom Naughton\fi
\def\AuthorPelletE{} % \ifdefined\AuthorPelletE \BrochureFlag{ch}{} Elsa Pellet\fi

\newpage}{}
