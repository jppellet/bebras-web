\documentclass[a4paper,11pt]{report}
\usepackage[T1]{fontenc}
\usepackage[utf8]{inputenc}

\usepackage[italian]{babel}
\AtBeginDocument{\def\labelitemi{$\bullet$}}

\usepackage{etoolbox}

\usepackage[margin=2cm]{geometry}
\usepackage{changepage}
\makeatletter
\renewenvironment{adjustwidth}[2]{%
    \begin{list}{}{%
    \partopsep\z@%
    \topsep\z@%
    \listparindent\parindent%
    \parsep\parskip%
    \@ifmtarg{#1}{\setlength{\leftmargin}{\z@}}%
                 {\setlength{\leftmargin}{#1}}%
    \@ifmtarg{#2}{\setlength{\rightmargin}{\z@}}%
                 {\setlength{\rightmargin}{#2}}%
    }
    \item[]}{\end{list}}
\makeatother

\newcommand{\BrochureUrlText}[1]{\texttt{#1}}
\usepackage{setspace}
\setstretch{1.15}

\usepackage{tabularx}
\usepackage{booktabs}
\usepackage{makecell}
\usepackage{multirow}
\renewcommand\theadfont{\bfseries}
\renewcommand{\tabularxcolumn}[1]{>{}m{#1}}
\newcolumntype{R}{>{\raggedleft\arraybackslash}X}
\newcolumntype{C}{>{\centering\arraybackslash}X}
\newcolumntype{L}{>{\raggedright\arraybackslash}X}
\newcolumntype{J}{>{\arraybackslash}X}

\newcommand{\BrochureInlineCode}[1]{{\ttfamily #1}}

\usepackage{amssymb}
\usepackage{amsmath}

\usepackage[babel=true,maxlevel=3]{csquotes}
\DeclareQuoteStyle{bebras-ch-eng}{“}[” ]{”}{‘}[”’ ]{’}\DeclareQuoteStyle{bebras-ch-deu}{«}[» ]{»}{“}[»› ]{”}
\DeclareQuoteStyle{bebras-ch-fra}{«\thinspace{}}[» ]{\thinspace{}»}{“}[»\thinspace{}› ]{”}
\DeclareQuoteStyle{bebras-ch-ita}{«}[» ]{»}{“}[»› ]{”}
\setquotestyle{bebras-ch-ita}

\usepackage{hyperref}
\usepackage{graphicx}
\usepackage{svg}
\svgsetup{inkscapeversion=1,inkscapearea=page}
\usepackage{wrapfig}

\usepackage{enumitem}
\setlist{nosep,itemsep=.5ex}

\setlength{\parindent}{0pt}
\setlength{\parskip}{2ex}
\raggedbottom

\usepackage{fancyhdr}
\usepackage{lastpage}
\pagestyle{fancy}

\fancyhf{}
\renewcommand{\headrulewidth}{0pt}
\renewcommand{\footrulewidth}{0.4pt}
\lfoot{\scriptsize © 2023 Bebras (CC BY-SA 4.0)}
\cfoot{\scriptsize\itshape 2023-CH-01 L'Ombrello di Anna}
\rfoot{\scriptsize Page~\thepage{}/\pageref*{LastPage}}

\newcommand{\taskGraphicsFolder}{..}

\begin{document}

\section*{\centering{} 2023-CH-01 L’Ombrello di Anna}


\subsection*{Body}

Questo è l’ombrello di Anna:
\raisebox{-0.5ex}{\includesvg[scale=0.2]{\taskGraphicsFolder/graphics/2023-CH-01-taskbody.svg}}

{\em


\subsection*{Question/Challenge - for the brochures}

Una delle quattro immagini mostra l’ombrello di Anna. Quale?

}


\subsection*{Interactivity instruction - for the online challenge}

Fa clic sull’immagine giusta.

\begingroup
\renewcommand{\arraystretch}{1.5}
\subsection*{Answer Options/Interactivity Description}

\begin{tabular}{ @{} c c c c @{} }
  \makecell[c]{\includesvg[scale=0.2]{\taskGraphicsFolder/graphics/2023-CH-01-A.svg}} & \makecell[c]{\includesvg[scale=0.2]{\taskGraphicsFolder/graphics/2023-CH-01-B.svg}} & \makecell[c]{\includesvg[scale=0.2]{\taskGraphicsFolder/graphics/2023-CH-01-C.svg}} & \makecell[c]{\includesvg[scale=0.2]{\taskGraphicsFolder/graphics/2023-CH-01-D.svg}} \\ 
  A) & B) & C) & D)
\end{tabular}

\endgroup

\subsection*{Answer Explanation}

Ogni motivo sull’ombrello di Anna si ripete esattamente una volta.

{\centering%
\includesvg[scale=0.2]{\taskGraphicsFolder/graphics/2023-CH-01-explanation-umbrella_numbered_patterns_compatible.svg}\par}

Per trovare l’immagine corretta, confrontiamo a turno ciascuna immagine con l’ombrello di Anna:

\begin{itemize}
  \item scegliamo il motivo più a sinistra e troviamo la sua posizione sull’ombrello di Anna.
  \item verifichiamo che i motivi adiacenti siano uguali a quelli dell’ombrello di Anna.

\begin{tabular}{ @{} l c c c c @{} }
  {\setstretch{1.0}\thead[lb]{}} & {\setstretch{1.0}\thead[cb]{A)}} & {\setstretch{1.0}\thead[cb]{B)}} & {\setstretch{1.0}\thead[cb]{C)}} & {\setstretch{1.0}\thead[cb]{D)}} \\ 
\midrule
  Immagini di risposta & \makecell[c]{\includesvg[width=64.9px]{\taskGraphicsFolder/graphics/2023-CH-01-A.svg}} & \makecell[c]{\includesvg[width=64.9px]{\taskGraphicsFolder/graphics/2023-CH-01-B.svg}} & \makecell[c]{\includesvg[width=64.9px]{\taskGraphicsFolder/graphics/2023-CH-01-C.svg}} & \makecell[c]{\includesvg[width=64.9px]{\taskGraphicsFolder/graphics/2023-CH-01-D.svg}} \\ 
  Ombrello di Anna & \makecell[c]{\includesvg[width=64.9px]{\taskGraphicsFolder/graphics/2023-CH-02-explanation-A.svg}} & \makecell[c]{\includesvg[width=64.9px]{\taskGraphicsFolder/graphics/2023-CH-02-explanation-B.svg}} & \makecell[c]{\includesvg[width=64.9px]{\taskGraphicsFolder/graphics/2023-CH-02-explanation-C.svg}} & \makecell[c]{\includesvg[width=64.9px]{\taskGraphicsFolder/graphics/2023-CH-02-explanation-D.svg}}
\end{tabular}


\end{itemize}

Ciascuna delle quattro immagini mostra una sequenza di soli cinque motivi e non di tutti e dieci. Non possiamo sapere se la sequenza di motivi di una delle quattro immagini corrisponde alla sequenza completa di tutti e dieci i motivi dell’ombrello di Anna.

L’immagine C è l’unica con una sequenza di cinque motivi che corrisponde completamente a quelli dell’ombrello di Anna. Dunque solo l’immagine C può mostrare l’ombrello di Anna. Tutte le altre immagini hanno sequenze di motivi che non corrispondono o corrispondono solo parzialmente a quelli dell’ombrello di Anna. Quindi queste immagini non possono rappresentare l’ombrello di Anna.


\subsection*{This is Informatics}

In ciascuna delle opzioni di risposta è mostrata solo una parte della sequenza di modelli. Anche se contengono solo \emph{informazioni parziali}, possiamo determinare quale delle quattro immagini mostra l’ombrello di Anna: un’immagine mostra l’ombrello di Anna solo se la sequenza di modelli si verifica completamente nella sequenza di modelli dell’ombrello di Anna.

Per la ricerca in un documento di testo si applica lo stesso principio della \enquote{ricerca a ombrello}. Il computer cerca le \emph{stringhe di caratteri} corrispondenti nel documento con le informazioni parziali fornite (parola di ricerca). Una stringa è una sequenza di caratteri (ad esempio lettere, numeri, caratteri speciali).
Per la ricerca vale quanto segue:

\begin{itemize}
  \item Più lunga è la parola di ricerca, minore è il numero di possibili corrispondenze e maggiore è la possibilità di trovare il punto del documento che si sta cercando.
  \item Più breve è la parola chiave, maggiore è il numero di possibili corrispondenze che la ricerca produrrà e meno accurata sarà la ricerca.
\end{itemize}

Per migliorare la ricerca, sono state sviluppate diverse procedure di ricerca (o \emph{Algoritmi di ricerca}). Sono progettate per eseguire una ricerca accurata nel più breve tempo possibile e fornire un risultato adeguato. Questi algoritmi di ricerca vengono costantemente sviluppati e sono in grado di cercare enormi quantità di dati in tempi molto brevi (ad esempio, i motori di ricerca su Internet utilizzano tali algoritmi di ricerca).


\subsection*{This is Computational Thinking}

–


\subsection*{Informatics Keywords and Websites}

\begin{itemize}
  \item Stringa: \href{https://it.wikipedia.org/wiki/Stringa_(informatica)}{\BrochureUrlText{https://it.wikipedia.org/wiki/Stringa\_(informatica)}}
  \item Algoritmo di ricerca: \href{https://it.wikipedia.org/wiki/Algoritmo_di_ricerca}{\BrochureUrlText{https://it.wikipedia.org/wiki/Algoritmo\_di\_ricerca}}
\end{itemize}


\subsection*{Computational Thinking Keywords and Websites}

–


\end{document}
