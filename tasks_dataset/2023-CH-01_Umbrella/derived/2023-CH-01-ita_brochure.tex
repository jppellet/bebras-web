% Definition of the meta information: task difficulties, task ID, task title, task country; definition of the variables as well as their scope is in commands.tex
\setcounter{taskAgeDifficulty3to4}{2}
\setcounter{taskAgeDifficulty5to6}{1}
\setcounter{taskAgeDifficulty7to8}{0}
\setcounter{taskAgeDifficulty9to10}{0}
\setcounter{taskAgeDifficulty11to13}{0}
\renewcommand{\taskTitle}{L'Ombrello di Anna}
\renewcommand{\taskCountry}{CH}

% include this task only if for the age groups being processed this task is relevant
\ifthenelse{
  \(\boolean{age3to4} \AND \(\value{taskAgeDifficulty3to4} > 0\)\) \OR
  \(\boolean{age5to6} \AND \(\value{taskAgeDifficulty5to6} > 0\)\) \OR
  \(\boolean{age7to8} \AND \(\value{taskAgeDifficulty7to8} > 0\)\) \OR
  \(\boolean{age9to10} \AND \(\value{taskAgeDifficulty9to10} > 0\)\) \OR
  \(\boolean{age11to13} \AND \(\value{taskAgeDifficulty11to13} > 0\)\)}{

\newchapter{\taskTitle}

% task body
Questo è l’ombrello di Anna:
\raisebox{-0.5ex}{\includesvg[scale=0.2]{\taskGraphicsFolder/graphics/2023-CH-01-taskbody.svg}}



% question (as \emph{})
{\em
Una delle quattro immagini mostra l’ombrello di Anna. Quale?


}

% answer alternatives (as \begin{enumerate}[A)]) or interactivity
\begin{tabular}{ @{} c c c c @{} }
  \makecell[c]{\includesvg[scale=0.2]{\taskGraphicsFolder/graphics/2023-CH-01-A.svg}} & \makecell[c]{\includesvg[scale=0.2]{\taskGraphicsFolder/graphics/2023-CH-01-B.svg}} & \makecell[c]{\includesvg[scale=0.2]{\taskGraphicsFolder/graphics/2023-CH-01-C.svg}} & \makecell[c]{\includesvg[scale=0.2]{\taskGraphicsFolder/graphics/2023-CH-01-D.svg}} \\ 
  A) & B) & C) & D)
\end{tabular}



% from here on this is only included if solutions are processed
\ifthenelse{\boolean{solutions}}{
\newpage

% answer explanation
\section*{\BrochureSolution}
Ogni motivo sull’ombrello di Anna si ripete esattamente una volta.

{\centering%
\includesvg[scale=0.2]{\taskGraphicsFolder/graphics/2023-CH-01-explanation-umbrella_numbered_patterns_compatible.svg}\par}

Per trovare l’immagine corretta, confrontiamo a turno ciascuna immagine con l’ombrello di Anna:

\begin{itemize}
  \item scegliamo il motivo più a sinistra e troviamo la sua posizione sull’ombrello di Anna.
  \item verifichiamo che i motivi adiacenti siano uguali a quelli dell’ombrello di Anna.

\begin{tabular}{ @{} l c c c c @{} }
  {\setstretch{1.0}\thead[lb]{}} & {\setstretch{1.0}\thead[cb]{A)}} & {\setstretch{1.0}\thead[cb]{B)}} & {\setstretch{1.0}\thead[cb]{C)}} & {\setstretch{1.0}\thead[cb]{D)}} \\ 
\midrule
  Immagini di risposta & \makecell[c]{\includesvg[width=64.9px]{\taskGraphicsFolder/graphics/2023-CH-01-A.svg}} & \makecell[c]{\includesvg[width=64.9px]{\taskGraphicsFolder/graphics/2023-CH-01-B.svg}} & \makecell[c]{\includesvg[width=64.9px]{\taskGraphicsFolder/graphics/2023-CH-01-C.svg}} & \makecell[c]{\includesvg[width=64.9px]{\taskGraphicsFolder/graphics/2023-CH-01-D.svg}} \\ 
  Ombrello di Anna & \makecell[c]{\includesvg[width=64.9px]{\taskGraphicsFolder/graphics/2023-CH-02-explanation-A.svg}} & \makecell[c]{\includesvg[width=64.9px]{\taskGraphicsFolder/graphics/2023-CH-02-explanation-B.svg}} & \makecell[c]{\includesvg[width=64.9px]{\taskGraphicsFolder/graphics/2023-CH-02-explanation-C.svg}} & \makecell[c]{\includesvg[width=64.9px]{\taskGraphicsFolder/graphics/2023-CH-02-explanation-D.svg}}
\end{tabular}


\end{itemize}

Ciascuna delle quattro immagini mostra una sequenza di soli cinque motivi e non di tutti e dieci. Non possiamo sapere se la sequenza di motivi di una delle quattro immagini corrisponde alla sequenza completa di tutti e dieci i motivi dell’ombrello di Anna.

L’immagine C è l’unica con una sequenza di cinque motivi che corrisponde completamente a quelli dell’ombrello di Anna. Dunque solo l’immagine C può mostrare l’ombrello di Anna. Tutte le altre immagini hanno sequenze di motivi che non corrispondono o corrispondono solo parzialmente a quelli dell’ombrello di Anna. Quindi queste immagini non possono rappresentare l’ombrello di Anna.



% it's informatics
\section*{\BrochureItsInformatics}
In ciascuna delle opzioni di risposta è mostrata solo una parte della sequenza di modelli. Anche se contengono solo \emph{informazioni parziali}, possiamo determinare quale delle quattro immagini mostra l’ombrello di Anna: un’immagine mostra l’ombrello di Anna solo se la sequenza di modelli si verifica completamente nella sequenza di modelli dell’ombrello di Anna.

Per la ricerca in un documento di testo si applica lo stesso principio della \enquote{ricerca a ombrello}. Il computer cerca le \emph{stringhe di caratteri} corrispondenti nel documento con le informazioni parziali fornite (parola di ricerca). Una stringa è una sequenza di caratteri (ad esempio lettere, numeri, caratteri speciali).
Per la ricerca vale quanto segue:

\begin{itemize}
  \item Più lunga è la parola di ricerca, minore è il numero di possibili corrispondenze e maggiore è la possibilità di trovare il punto del documento che si sta cercando.
  \item Più breve è la parola chiave, maggiore è il numero di possibili corrispondenze che la ricerca produrrà e meno accurata sarà la ricerca.
\end{itemize}

Per migliorare la ricerca, sono state sviluppate diverse procedure di ricerca (o \emph{Algoritmi di ricerca}). Sono progettate per eseguire una ricerca accurata nel più breve tempo possibile e fornire un risultato adeguato. Questi algoritmi di ricerca vengono costantemente sviluppati e sono in grado di cercare enormi quantità di dati in tempi molto brevi (ad esempio, i motori di ricerca su Internet utilizzano tali algoritmi di ricerca).



% keywords and websites (as \begin{itemize})
\section*{\BrochureWebsitesAndKeywords}
{\raggedright
\begin{itemize}
  \item Stringa: \href{https://it.wikipedia.org/wiki/Stringa_(informatica)}{\BrochureUrlText{https://it.wikipedia.org/wiki/Stringa\_(informatica)}}
  \item Algoritmo di ricerca: \href{https://it.wikipedia.org/wiki/Algoritmo_di_ricerca}{\BrochureUrlText{https://it.wikipedia.org/wiki/Algoritmo\_di\_ricerca}}
\end{itemize}


}

% end of ifthen for excluding the solutions
}{}

% all authors
% ATTENTION: you HAVE to make sure an according entry is in ../main/authors.tex.
% Syntax: \def\AuthorLastnameF{} (Lastname is last name, F is first letter of first name, this serves as a marker for ../main/authors.tex)
\def\AuthorLoyoA{} % \ifdefined\AuthorLoyoA \BrochureFlag{ch}{} Angélica Herrera Loyo\fi
\def\AuthorDagieneV{} % \ifdefined\AuthorDagieneV \BrochureFlag{lt}{} Valentina Dagienė\fi
\def\AuthorEscherleN{} % \ifdefined\AuthorEscherleN \BrochureFlag{ch}{} Nora A.~Escherle\fi
\def\AuthorPluharZ{} % \ifdefined\AuthorPluharZ \BrochureFlag{hu}{} Zsuzsa Pluhár\fi
\def\AuthorKinciusV{} % \ifdefined\AuthorKinciusV \BrochureFlag{lt}{} Vaidotas Kinčius\fi
\def\AuthorDatzkoThutS{} % \ifdefined\AuthorDatzkoThutS \BrochureFlag{de}{} Susanne Datzko-Thut\fi
\def\AuthorSerafiniG{} % \ifdefined\AuthorSerafiniG \BrochureFlag{ch}{} Giovanni Serafini\fi
\def\AuthorGiangC{} % \ifdefined\AuthorGiangC \BrochureFlag{ch}{} Christian Giang\fi

\newpage}{}
