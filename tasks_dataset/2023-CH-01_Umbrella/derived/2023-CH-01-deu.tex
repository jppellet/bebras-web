\documentclass[a4paper,11pt]{report}
\usepackage[T1]{fontenc}
\usepackage[utf8]{inputenc}

\usepackage[german]{babel}
\AtBeginDocument{\def\labelitemi{$\bullet$}}

\usepackage{etoolbox}

\usepackage[margin=2cm]{geometry}
\usepackage{changepage}
\makeatletter
\renewenvironment{adjustwidth}[2]{%
    \begin{list}{}{%
    \partopsep\z@%
    \topsep\z@%
    \listparindent\parindent%
    \parsep\parskip%
    \@ifmtarg{#1}{\setlength{\leftmargin}{\z@}}%
                 {\setlength{\leftmargin}{#1}}%
    \@ifmtarg{#2}{\setlength{\rightmargin}{\z@}}%
                 {\setlength{\rightmargin}{#2}}%
    }
    \item[]}{\end{list}}
\makeatother

\newcommand{\BrochureUrlText}[1]{\texttt{#1}}
\usepackage{setspace}
\setstretch{1.15}

\usepackage{tabularx}
\usepackage{booktabs}
\usepackage{makecell}
\usepackage{multirow}
\renewcommand\theadfont{\bfseries}
\renewcommand{\tabularxcolumn}[1]{>{}m{#1}}
\newcolumntype{R}{>{\raggedleft\arraybackslash}X}
\newcolumntype{C}{>{\centering\arraybackslash}X}
\newcolumntype{L}{>{\raggedright\arraybackslash}X}
\newcolumntype{J}{>{\arraybackslash}X}

\newcommand{\BrochureInlineCode}[1]{{\ttfamily #1}}

\usepackage{amssymb}
\usepackage{amsmath}

\usepackage[babel=true,maxlevel=3]{csquotes}
\DeclareQuoteStyle{bebras-ch-eng}{“}[” ]{”}{‘}[”’ ]{’}\DeclareQuoteStyle{bebras-ch-deu}{«}[» ]{»}{“}[»› ]{”}
\DeclareQuoteStyle{bebras-ch-fra}{«\thinspace{}}[» ]{\thinspace{}»}{“}[»\thinspace{}› ]{”}
\DeclareQuoteStyle{bebras-ch-ita}{«}[» ]{»}{“}[»› ]{”}
\setquotestyle{bebras-ch-deu}

\usepackage{hyperref}
\usepackage{graphicx}
\usepackage{svg}
\svgsetup{inkscapeversion=1,inkscapearea=page}
\usepackage{wrapfig}

\usepackage{enumitem}
\setlist{nosep,itemsep=.5ex}

\setlength{\parindent}{0pt}
\setlength{\parskip}{2ex}
\raggedbottom

\usepackage{fancyhdr}
\usepackage{lastpage}
\pagestyle{fancy}

\fancyhf{}
\renewcommand{\headrulewidth}{0pt}
\renewcommand{\footrulewidth}{0.4pt}
\lfoot{\scriptsize © 2023 Bebras (CC BY-SA 4.0)}
\cfoot{\scriptsize\itshape 2023-CH-01 Annas Regenschirm}
\rfoot{\scriptsize Page~\thepage{}/\pageref*{LastPage}}

\newcommand{\taskGraphicsFolder}{..}

\begin{document}

\section*{\centering{} 2023-CH-01 Annas Regenschirm}


\subsection*{Body}

Das ist Annas Regenschirm:
\raisebox{-0.5ex}{\includesvg[scale=0.2]{\taskGraphicsFolder/graphics/2023-CH-01-taskbody.svg}}

{\em


\subsection*{Question/Challenge - for the brochures}

Eines der vier Bilder zeigt Annas Regenschirm. Welches?

}


\subsection*{Interactivity instruction - for the online challenge}

Klicke auf das richtige Bild.

\begingroup
\renewcommand{\arraystretch}{1.5}
\subsection*{Answer Options/Interactivity Description}

\begin{tabular}{ @{} c c c c @{} }
  \makecell[c]{\includesvg[scale=0.2]{\taskGraphicsFolder/graphics/2023-CH-01-A.svg}} & \makecell[c]{\includesvg[scale=0.2]{\taskGraphicsFolder/graphics/2023-CH-01-B.svg}} & \makecell[c]{\includesvg[scale=0.2]{\taskGraphicsFolder/graphics/2023-CH-01-C.svg}} & \makecell[c]{\includesvg[scale=0.2]{\taskGraphicsFolder/graphics/2023-CH-01-D.svg}} \\ 
  A) & B) & C) & D)
\end{tabular}

\endgroup

\subsection*{Answer Explanation}

Jedes Muster auf Annas Regenschirm kommt genau einmal vor.

{\centering%
\includesvg[scale=0.2]{\taskGraphicsFolder/graphics/2023-CH-01-explanation-umbrella_numbered_patterns_compatible.svg}\par}

Um das korrekte Bild zu finden, vergleichen wir nacheinander jedes der Bilder mit Annas Regenschirm:

\begin{itemize}
  \item wir wählen das Muster, welches am weitesten links ist, und suchen dessen Position auf Annas Regenschirm
  \item wir prüfen, ob die angrenzenden Muster dieselben sind wie die auf Annas Regenschirm.

\begin{tabular}{ @{} l c c c c @{} }
  {\setstretch{1.0}\thead[lb]{}} & {\setstretch{1.0}\thead[cb]{A)}} & {\setstretch{1.0}\thead[cb]{B)}} & {\setstretch{1.0}\thead[cb]{C)}} & {\setstretch{1.0}\thead[cb]{D)}} \\ 
\midrule
  Antwortbild & \makecell[c]{\includesvg[width=64.9px]{\taskGraphicsFolder/graphics/2023-CH-01-A.svg}} & \makecell[c]{\includesvg[width=64.9px]{\taskGraphicsFolder/graphics/2023-CH-01-B.svg}} & \makecell[c]{\includesvg[width=64.9px]{\taskGraphicsFolder/graphics/2023-CH-01-C.svg}} & \makecell[c]{\includesvg[width=64.9px]{\taskGraphicsFolder/graphics/2023-CH-01-D.svg}} \\ 
  Annas Regenschirm & \makecell[c]{\includesvg[width=64.9px]{\taskGraphicsFolder/graphics/2023-CH-02-explanation-A.svg}} & \makecell[c]{\includesvg[width=64.9px]{\taskGraphicsFolder/graphics/2023-CH-02-explanation-B.svg}} & \makecell[c]{\includesvg[width=64.9px]{\taskGraphicsFolder/graphics/2023-CH-02-explanation-C.svg}} & \makecell[c]{\includesvg[width=64.9px]{\taskGraphicsFolder/graphics/2023-CH-02-explanation-D.svg}}
\end{tabular}


\end{itemize}

Jedes der vier Bilder zeigt eine Folge von nur fünf Mustern und nicht alle zehn. Wir können nicht wissen, ob die Musterfolge von einem der vier Bilder mit der vollständigen Reihenfolge aller zehn Muster von Annas Regenschirm übereinstimmt.

Bild C hat als einziges eine Folge von fünf Mustern, die vollständig mit denen auf Annas Regenschirm übereinstimmt. Aus diesem Grund kann nur Bild C Annas Regenschirm zeigen. Alle anderen Bilder weisen Musterfolgen auf, die nicht oder nur teilweise mit denen von Annas Regenschirm übereinstimmen. Diese Bilder können also nicht Annas Regenschirm zeigen.


\subsection*{This is Informatics}

In den Antwortmöglichkeiten ist jeweils nur ein Teil der Musterfolge abgebildet. Obwohl sie nur eine \emph{Teilinformation} enthalten, können wir feststellen, welches der vier Bilder Annas Regenschirm zeigt: Ein Bild zeigt nur dann Annas Regenschirm, wenn die Musterfolge vollständig in der Musterfolge von Annas Regenschirm vorkommt.

Das gleiche Prinzip wie bei der \enquote{Regenschirm-Mustersuche} wird bei der Suche in einem Textdokument angewendet. Der Computer sucht mit gegebenen Teilinformationen (Suchwort) nach passenden \emph{Zeichenketten} im Dokument. Eine Zeichenkette ist eine Folge von Zeichen (z.B. Buchstaben, Ziffern, Sonderzeichen).
Dabei gilt bei der Suche:

\begin{itemize}
  \item Je länger das Suchwort, desto weniger mögliche Übereinstimmungen gibt es und desto grösser ist die Chance, die gesuchte Stelle im Dokument zu finden.
  \item Je kürzer das Suchwort, desto mehr mögliche Übereinstimmungen ergibt die Suche und desto ungenauer ist die Suche.
\end{itemize}

Um das Durchsuchen zu verbessern, wurden verschiedene Suchverfahren (oder \emph{Suchalgorthimen}) entwickelt. Sie sollen möglichst schnell eine genaue Suche durchführen, und ein passendes Resultat liefern. Diese Suchalgorthimen werden ständig weiterentwickelt und können riesige Datenmengen in sehr kurzer Zeit durchsuchen (z.B. Internetsuchmaschinen verwenden solche Suchalgorthimen).


\subsection*{This is Computational Thinking}

–


\subsection*{Informatics Keywords and Websites}

\begin{itemize}
  \item Zeichenkette, String: \href{https://de.wikipedia.org/wiki/Zeichenkette}{\BrochureUrlText{https://de.wikipedia.org/wiki/Zeichenkette}}
  \item Suchverfahren: \href{https://de.wikipedia.org/wiki/Suchverfahren}{\BrochureUrlText{https://de.wikipedia.org/wiki/Suchverfahren}}
\end{itemize}


\subsection*{Computational Thinking Keywords and Websites}

–


\end{document}
