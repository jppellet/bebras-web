\documentclass[a4paper,11pt]{report}
\usepackage[T1]{fontenc}
\usepackage[utf8]{inputenc}

\usepackage[french]{babel}
\frenchbsetup{ThinColonSpace=true}
\renewcommand*{\FBguillspace}{\hskip .4\fontdimen2\font plus .1\fontdimen3\font minus .3\fontdimen4\font \relax}
\AtBeginDocument{\def\labelitemi{$\bullet$}}

\usepackage{etoolbox}

\usepackage[margin=2cm]{geometry}
\usepackage{changepage}
\makeatletter
\renewenvironment{adjustwidth}[2]{%
    \begin{list}{}{%
    \partopsep\z@%
    \topsep\z@%
    \listparindent\parindent%
    \parsep\parskip%
    \@ifmtarg{#1}{\setlength{\leftmargin}{\z@}}%
                 {\setlength{\leftmargin}{#1}}%
    \@ifmtarg{#2}{\setlength{\rightmargin}{\z@}}%
                 {\setlength{\rightmargin}{#2}}%
    }
    \item[]}{\end{list}}
\makeatother

\newcommand{\BrochureUrlText}[1]{\texttt{#1}}
\usepackage{setspace}
\setstretch{1.15}

\usepackage{tabularx}
\usepackage{booktabs}
\usepackage{makecell}
\usepackage{multirow}
\renewcommand\theadfont{\bfseries}
\renewcommand{\tabularxcolumn}[1]{>{}m{#1}}
\newcolumntype{R}{>{\raggedleft\arraybackslash}X}
\newcolumntype{C}{>{\centering\arraybackslash}X}
\newcolumntype{L}{>{\raggedright\arraybackslash}X}
\newcolumntype{J}{>{\arraybackslash}X}

\newcommand{\BrochureInlineCode}[1]{{\ttfamily #1}}

\usepackage{amssymb}
\usepackage{amsmath}

\usepackage[babel=true,maxlevel=3]{csquotes}
\DeclareQuoteStyle{bebras-ch-eng}{“}[” ]{”}{‘}[”’ ]{’}\DeclareQuoteStyle{bebras-ch-deu}{«}[» ]{»}{“}[»› ]{”}
\DeclareQuoteStyle{bebras-ch-fra}{«\thinspace{}}[» ]{\thinspace{}»}{“}[»\thinspace{}› ]{”}
\DeclareQuoteStyle{bebras-ch-ita}{«}[» ]{»}{“}[»› ]{”}
\setquotestyle{bebras-ch-fra}

\usepackage{hyperref}
\usepackage{graphicx}
\usepackage{svg}
\svgsetup{inkscapeversion=1,inkscapearea=page}
\usepackage{wrapfig}

\usepackage{enumitem}
\setlist{nosep,itemsep=.5ex}

\setlength{\parindent}{0pt}
\setlength{\parskip}{2ex}
\raggedbottom

\usepackage{fancyhdr}
\usepackage{lastpage}
\pagestyle{fancy}

\fancyhf{}
\renewcommand{\headrulewidth}{0pt}
\renewcommand{\footrulewidth}{0.4pt}
\lfoot{\scriptsize © 2023 Bebras (CC BY-SA 4.0)}
\cfoot{\scriptsize\itshape 2023-BR-05 Fontaine}
\rfoot{\scriptsize Page~\thepage{}/\pageref*{LastPage}}

\newcommand{\taskGraphicsFolder}{..}

\begin{document}

\section*{\centering{} 2023-BR-05 Fontaine}


\subsection*{Body}

L’été est chaud dans la ville. La maire fait installer des fontaines d’eau potable.

Les fontaines doivent être installées de manière à ce qu’il ne faille jamais parcourir plus de deux tronçons de rue pour atteindre une fontaine depuis n’importe quel coin de rue. La maire sera alors satisfaite.

Voici un plan de la ville. Les lignes sont les tronçons de rue et les points les coins de rue. Il y a déjà des fontaines \raisebox{-0.5ex}[0pt][0pt]{\includesvg[width=15.9px]{\taskGraphicsFolder/graphics/2023-BR-05-fountain.svg}} à trois coins de rue.

{\centering%
\includesvg[scale=0.4]{\taskGraphicsFolder/graphics/2023-BR-05-challenge.svg}\par}

{\em


\subsection*{Question/Challenge - for the brochures}

Installe une fontaine supplémentaire pour satisfaire la maire.

}


\subsection*{Interactivity instruction - for the online challenge}

Clique sur un coin de rue pour y installer une fontaine. Quand tu as fini, clique sur “Enregistrer la réponse”.

\begingroup
\renewcommand{\arraystretch}{1.5}
\subsection*{Answer Options/Interactivity Description}

Wenn eine unbesetzte Strassenecke angeklickt wird, wird dort ein Brunnen angezeigt. Ein evtl. vorher durch Klick aufgestellter Brunnen verschwindet durch einen Klick auf die gleiche oder eine andere Strassenecke wieder. Strassenecken mit vorab aufgestellten Brunnen können nicht angeklickt werden.

\endgroup

\subsection*{Answer Explanation}

Voici la bonne réponse:

{\centering%
\includesvg[scale=0.4]{\taskGraphicsFolder/graphics/2023-BR-05-solution.svg}\par}

Si une fontaine supplémentaire est intallée en bas au centre, il faut parcourir au maxium deux tronçons de rue pour atteindre une fontaine. La maire est alors satisfaite.

Comment pouvons-nous déterminer à quel coin de rue installer une fontaine supplémentaire? Sur le plan, nous indiquons d’un \raisebox{-0.5ex}[0pt][0pt]{\includesvg[width=15.9px]{\taskGraphicsFolder/graphics/2023-BR-05-mark.svg}} tous les coins de rue se trouvant déjà à deux tronçons de rue d’une fontaine au maximum. La maire peut déjà être satisfaite de ces coins de rue.

{\centering%
\includesvg[scale=0.4]{\taskGraphicsFolder/graphics/2023-BR-05-explanation_compatible.svg}\par}

Pour les cinq coins de rue A, B, C, D et E restants, nous ajoutons une fontaine au coin de rue C. Comme ça, ces coins-là sont aussi loin de deux tronçons au maximum d’une fontaine.

Le coin C est le seul endroit où installer une nouvelle fontaine permettant de satisfaire la maire:
Si l’on considère les coins de rue se trouvant à deux tronçons des coins A et E (entourés d’une ligne sur l’image), seul le coin C remplit cette condition pour les coins A \emph{et} E.


\subsection*{This is Informatics}

Le plan de la ville peut être représenté par un \emph{graphe}. C’est un outil important en informatique qui permet de modéliser les relations entre des objets et de répondre à des questions sur ces relations. Ici, on peut représenter les coins de rue comme des objets et donc des \emph{nœuds} du graphe. Les relations entre deux objets sont modélisées par des \emph{arêtes} qui relient deux nœuds. Ici, une arête entre deux coins de rue veut dire qu’ils sont reliés par un tronçon de rue. On peut appeler cette relation “voisinage”. Les arêtes peuvent aussi représenter d’autres relations, comme l’amitié.

Dans cet exercice du Castor, il faut trouver un sous-ensemble de nœuds (pour installer une fontaine) de manière à ce que chaque nœud à l’extérieur de ce sous-ensemble soit relié à un “nœud fontaine” par un chemin de deux arêtes de long au maximum. En langage technique informatique, on parlerait de la recherche d’un ensemble 2-dominant (\emph{distance$-2$ dominating set} en anglais). En général (pour toutes les longueurs de chemin ${d \geq 1}$), cette recherche d’un sous-ensemble de taille minimale fait partie des problèmes les plus difficiles rencontrés en informatique.

Les “ensembles \emph{d}-dominants” jouent un rôle croissant actuellement, en particulier dans le domaine du \emph{social computing}: pour traiter des données venant de réseaux sociaux de manière automatique (par exemple pour détecter la diffusion de \emph{fake news}), les relations entre les utilisateurs (fan, follower, ami) sont modélisées sous forme de graphes. Ces graphes peuvent être si grands que seule une sélection représentative (aussi petite que possible) peut être prise en considération – par exemple, un set 3-dominant. Comme la sélection la plus petite possible ne peut pas être calculée efficacement, on développe des méthodes informatiques qui permettent de rapidement déterminer de petites sélections (mais pas forcément \emph{la} plus petite).


\subsection*{This is Computational Thinking}

Optional - not to be filled 2023


\subsection*{Informatics Keywords and Websites}

\begin{itemize}
  \item Ensemble dominant: \href{https://fr.wikipedia.org/wiki/Ensemble_dominant}{\BrochureUrlText{https://fr.wikipedia.org/wiki/Ensemble\_dominant}}
  \item Théorie des graphes: \href{https://fr.wikipedia.org/wiki/Th\%C3\%A9orie_des_graphes}{\BrochureUrlText{https://fr.wikipedia.org/wiki/Théorie\_des\_graphes}}
  \item Analyse des réseaux sociaux: \href{https://fr.wikipedia.org/wiki/Analyse_des_r\%C3\%A9seaux_sociaux}{\BrochureUrlText{https://fr.wikipedia.org/wiki/Analyse\_des\_réseaux\_sociaux}}
\end{itemize}


\subsection*{Computational Thinking Keywords and Websites}

Optional - not to be filled 2023


\end{document}
