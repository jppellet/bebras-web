\documentclass[a4paper,11pt]{report}
\usepackage[T1]{fontenc}
\usepackage[utf8]{inputenc}

\usepackage[french]{babel}
\frenchbsetup{ThinColonSpace=true}
\renewcommand*{\FBguillspace}{\hskip .4\fontdimen2\font plus .1\fontdimen3\font minus .3\fontdimen4\font \relax}
\AtBeginDocument{\def\labelitemi{$\bullet$}}

\usepackage{etoolbox}

\usepackage[margin=2cm]{geometry}
\usepackage{changepage}
\makeatletter
\renewenvironment{adjustwidth}[2]{%
    \begin{list}{}{%
    \partopsep\z@%
    \topsep\z@%
    \listparindent\parindent%
    \parsep\parskip%
    \@ifmtarg{#1}{\setlength{\leftmargin}{\z@}}%
                 {\setlength{\leftmargin}{#1}}%
    \@ifmtarg{#2}{\setlength{\rightmargin}{\z@}}%
                 {\setlength{\rightmargin}{#2}}%
    }
    \item[]}{\end{list}}
\makeatother

\newcommand{\BrochureUrlText}[1]{\texttt{#1}}
\usepackage{setspace}
\setstretch{1.15}

\usepackage{tabularx}
\usepackage{booktabs}
\usepackage{makecell}
\usepackage{multirow}
\renewcommand\theadfont{\bfseries}
\renewcommand{\tabularxcolumn}[1]{>{}m{#1}}
\newcolumntype{R}{>{\raggedleft\arraybackslash}X}
\newcolumntype{C}{>{\centering\arraybackslash}X}
\newcolumntype{L}{>{\raggedright\arraybackslash}X}
\newcolumntype{J}{>{\arraybackslash}X}

\newcommand{\BrochureInlineCode}[1]{{\ttfamily #1}}

\usepackage{amssymb}
\usepackage{amsmath}

\usepackage[babel=true,maxlevel=3]{csquotes}
\DeclareQuoteStyle{bebras-ch-eng}{“}[” ]{”}{‘}[”’ ]{’}\DeclareQuoteStyle{bebras-ch-deu}{«}[» ]{»}{“}[»› ]{”}
\DeclareQuoteStyle{bebras-ch-fra}{«\thinspace{}}[» ]{\thinspace{}»}{“}[»\thinspace{}› ]{”}
\DeclareQuoteStyle{bebras-ch-ita}{«}[» ]{»}{“}[»› ]{”}
\setquotestyle{bebras-ch-fra}

\usepackage{hyperref}
\usepackage{graphicx}
\usepackage{svg}
\svgsetup{inkscapeversion=1,inkscapearea=page}
\usepackage{wrapfig}

\usepackage{enumitem}
\setlist{nosep,itemsep=.5ex}

\setlength{\parindent}{0pt}
\setlength{\parskip}{2ex}
\raggedbottom

\usepackage{fancyhdr}
\usepackage{lastpage}
\pagestyle{fancy}

\fancyhf{}
\renewcommand{\headrulewidth}{0pt}
\renewcommand{\footrulewidth}{0.4pt}
\lfoot{\scriptsize © 2021 Bebras (CC BY-SA 4.0)}
\cfoot{\scriptsize\itshape 2021-KR-02 Jeu de balles}
\rfoot{\scriptsize Page~\thepage{}/\pageref*{LastPage}}

\newcommand{\taskGraphicsFolder}{..}

\begin{document}

\section*{\centering{} 2021-KR-02 Jeu de balles}


\subsection*{Body}

Les castors aimeraient trier des balles par couleur. À la fin, toutes les balles doivent se trouver dans deux verres. Toutes les balles qui se trouvent dans un verre doivent avoir la même couleur. Ils doivent suivre les règles suivantes:

\raisebox{\dimexpr -0.5ex -.3ex \relax}{\includesvg[scale=0.3]{\taskGraphicsFolder/graphics/2021-KR-03-rule01.svg}} Règle $1$: Seule la balle la plus haute d’un verre peut être déplacée dans un autre verre.

\raisebox{\dimexpr -0.5ex -.3ex \relax}{\includesvg[scale=0.3]{\taskGraphicsFolder/graphics/2021-KR-03-rule02.svg}} Règle $2$: Une balle peut toujours être mise dans un verre vide.

\raisebox{\dimexpr -0.5ex -.3ex \relax}{\includesvg[scale=0.3]{\taskGraphicsFolder/graphics/2021-KR-03-rule03.svg}} Règle $3$: Une balle peut être mise dans un verre non vide uniquement s’il y reste de la place et que la balle en dessous a la même couleur que la balle déplacée.

{\centering%
\includesvg[scale=0.3]{\taskGraphicsFolder/graphics/2021-KR-03-taskbody.svg}\par}

{\em


\subsection*{Question/Challenge - for the brochures}

Trie les balles en les déplaçant d’après les trois règles.

{\centering%
\includesvg[width=144.3px]{\taskGraphicsFolder/graphics/2021-KR-03-question.svg}\par}

}

\begingroup
\renewcommand{\arraystretch}{1.5}
\subsection*{Answer Options/Interactivity Description}



\endgroup

\subsection*{Answer Explanation}

Les balles peuvent être déplacées dans l’ordre suivant:

{\centering%
\includesvg[scale=0.3]{\taskGraphicsFolder/graphics/2021-KR-03-solution.svg}\par}

Il faut au moins six étapes pour réarranger les balles. Il existe d’autres possibilités de réarranger les balles en seulement six étapes.


\subsection*{It’s Informatics}

Dans cet exercice, tu déplaces les balles comme un ordinateur gère les données enregistrées dans une \emph{pile}: il ne peut qu’\emph{empiler} (\emph{push} en anglais) un élément en haut de la pile et \emph{dépiler} (\emph{pop} en anlgais) l’élément du haut de la pile.

{\centering%
\includesvg[width=144.3px]{\taskGraphicsFolder/graphics/2021-KR-03-itsinformatics.svg}\par}

L’ordinateur ne peut accéder aux éléments du bas que si les balles du dessus ont d’abord été retirées. L’élément qui a été enregistré en dernier va être retiré en premier par l’ordinateur. Ce principe est appelé \emph{dernier arrivé, premier sorti} en informatique.

{\raggedright

\subsection*{Keywords and Websites}

\begin{itemize}
  \item Pile: \href{https://fr.wikipedia.org/wiki/Pile_(informatique)}{\BrochureUrlText{https://fr.wikipedia.org/wiki/Pile\_(informatique)}}
\end{itemize}


}
\end{document}
