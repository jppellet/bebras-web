% Definition of the meta information: task difficulties, task ID, task title, task country; definition of the variables as well as their scope is in commands.tex
\setcounter{taskAgeDifficulty3to4}{3}
\setcounter{taskAgeDifficulty5to6}{0}
\setcounter{taskAgeDifficulty7to8}{1}
\setcounter{taskAgeDifficulty9to10}{0}
\setcounter{taskAgeDifficulty11to13}{0}
\renewcommand{\taskTitle}{Jeu de balles}
\renewcommand{\taskCountry}{KR}

% include this task only if for the age groups being processed this task is relevant
\ifthenelse{
  \(\boolean{age3to4} \AND \(\value{taskAgeDifficulty3to4} > 0\)\) \OR
  \(\boolean{age5to6} \AND \(\value{taskAgeDifficulty5to6} > 0\)\) \OR
  \(\boolean{age7to8} \AND \(\value{taskAgeDifficulty7to8} > 0\)\) \OR
  \(\boolean{age9to10} \AND \(\value{taskAgeDifficulty9to10} > 0\)\) \OR
  \(\boolean{age11to13} \AND \(\value{taskAgeDifficulty11to13} > 0\)\)}{

\newchapter{\taskTitle}

% task body
Les castors aimeraient trier des balles par couleur. À la fin, toutes les balles doivent se trouver dans deux verres. Toutes les balles qui se trouvent dans un verre doivent avoir la même couleur. Ils doivent suivre les règles suivantes:

\raisebox{\dimexpr -0.5ex -.3ex \relax}{\includesvg[scale=0.3]{\taskGraphicsFolder/graphics/2021-KR-03-rule01.svg}} Règle $1$: Seule la balle la plus haute d’un verre peut être déplacée dans un autre verre.

\raisebox{\dimexpr -0.5ex -.3ex \relax}{\includesvg[scale=0.3]{\taskGraphicsFolder/graphics/2021-KR-03-rule02.svg}} Règle $2$: Une balle peut toujours être mise dans un verre vide.

\raisebox{\dimexpr -0.5ex -.3ex \relax}{\includesvg[scale=0.3]{\taskGraphicsFolder/graphics/2021-KR-03-rule03.svg}} Règle $3$: Une balle peut être mise dans un verre non vide uniquement s’il y reste de la place et que la balle en dessous a la même couleur que la balle déplacée.

{\centering%
\includesvg[scale=0.3]{\taskGraphicsFolder/graphics/2021-KR-03-taskbody.svg}\par}



% question (as \emph{})
{\em
Trie les balles en les déplaçant d’après les trois règles.

{\centering%
\includesvg[width=144.3px]{\taskGraphicsFolder/graphics/2021-KR-03-question.svg}\par}


}

% answer alternatives (as \begin{enumerate}[A)]) or interactivity


% from here on this is only included if solutions are processed
\ifthenelse{\boolean{solutions}}{
\newpage

% answer explanation
\section*{\BrochureSolution}
Les balles peuvent être déplacées dans l’ordre suivant:

{\centering%
\includesvg[scale=0.3]{\taskGraphicsFolder/graphics/2021-KR-03-solution.svg}\par}

Il faut au moins six étapes pour réarranger les balles. Il existe d’autres possibilités de réarranger les balles en seulement six étapes.



% it's informatics
\section*{\BrochureItsInformatics}
Dans cet exercice, tu déplaces les balles comme un ordinateur gère les données enregistrées dans une \emph{pile}: il ne peut qu’\emph{empiler} (\emph{push} en anglais) un élément en haut de la pile et \emph{dépiler} (\emph{pop} en anlgais) l’élément du haut de la pile.

{\centering%
\includesvg[width=144.3px]{\taskGraphicsFolder/graphics/2021-KR-03-itsinformatics.svg}\par}

L’ordinateur ne peut accéder aux éléments du bas que si les balles du dessus ont d’abord été retirées. L’élément qui a été enregistré en dernier va être retiré en premier par l’ordinateur. Ce principe est appelé \emph{dernier arrivé, premier sorti} en informatique.



% keywords and websites (as \begin{itemize})
\section*{\BrochureWebsitesAndKeywords}
{\raggedright
\begin{itemize}
  \item Pile: \href{https://fr.wikipedia.org/wiki/Pile_(informatique)}{\BrochureUrlText{https://fr.wikipedia.org/wiki/Pile\_(informatique)}}
\end{itemize}


}

% end of ifthen for excluding the solutions
}{}

% all authors
% ATTENTION: you HAVE to make sure an according entry is in ../main/authors.tex.
% Syntax: \def\AuthorLastnameF{} (Lastname is last name, F is first letter of first name, this serves as a marker for ../main/authors.tex)
\def\AuthorJunS{} % \ifdefined\AuthorJunS \BrochureFlag{kr}{} Soojin Jun\fi
\def\AuthorYehH{} % \ifdefined\AuthorYehH \BrochureFlag{kr}{} Hongjin Yeh\fi
\def\AuthorJeonY{} % \ifdefined\AuthorJeonY \BrochureFlag{kr}{} YongJu Jeon\fi
\def\AuthorKimD{} % \ifdefined\AuthorKimD \BrochureFlag{kr}{} Dong Yoon Kim\fi
\def\AuthorKimJ{} % \ifdefined\AuthorKimJ \BrochureFlag{kr}{} Jihye Kim\fi
\def\AuthorVoborilF{} % \ifdefined\AuthorVoborilF \BrochureFlag{at}{} Florentina Voboril\fi
\def\AuthorSpielerB{} % \ifdefined\AuthorSpielerB \BrochureFlag{ch}{} Bernadette Spieler\fi
\def\AuthorWillekesK{} % \ifdefined\AuthorWillekesK \BrochureFlag{nl}{} Kyra Willekes\fi
\def\AuthorDatzkoS{} % \ifdefined\AuthorDatzkoS \BrochureFlag{ch}{} Susanne Datzko\fi
\def\AuthorPelletE{} % \ifdefined\AuthorPelletE \BrochureFlag{ch}{} Elsa Pellet\fi

\newpage}{}
