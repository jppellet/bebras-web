\documentclass[a4paper,11pt]{report}
\usepackage[T1]{fontenc}
\usepackage[utf8]{inputenc}

\usepackage[german]{babel}
\AtBeginDocument{\def\labelitemi{$\bullet$}}

\usepackage{etoolbox}

\usepackage[margin=2cm]{geometry}
\usepackage{changepage}
\makeatletter
\renewenvironment{adjustwidth}[2]{%
    \begin{list}{}{%
    \partopsep\z@%
    \topsep\z@%
    \listparindent\parindent%
    \parsep\parskip%
    \@ifmtarg{#1}{\setlength{\leftmargin}{\z@}}%
                 {\setlength{\leftmargin}{#1}}%
    \@ifmtarg{#2}{\setlength{\rightmargin}{\z@}}%
                 {\setlength{\rightmargin}{#2}}%
    }
    \item[]}{\end{list}}
\makeatother

\newcommand{\BrochureUrlText}[1]{\texttt{#1}}
\usepackage{setspace}
\setstretch{1.15}

\usepackage{tabularx}
\usepackage{booktabs}
\usepackage{makecell}
\usepackage{multirow}
\renewcommand\theadfont{\bfseries}
\renewcommand{\tabularxcolumn}[1]{>{}m{#1}}
\newcolumntype{R}{>{\raggedleft\arraybackslash}X}
\newcolumntype{C}{>{\centering\arraybackslash}X}
\newcolumntype{L}{>{\raggedright\arraybackslash}X}
\newcolumntype{J}{>{\arraybackslash}X}

\newcommand{\BrochureInlineCode}[1]{{\ttfamily #1}}

\usepackage{amssymb}
\usepackage{amsmath}

\usepackage[babel=true,maxlevel=3]{csquotes}
\DeclareQuoteStyle{bebras-ch-eng}{“}[” ]{”}{‘}[”’ ]{’}\DeclareQuoteStyle{bebras-ch-deu}{«}[» ]{»}{“}[»› ]{”}
\DeclareQuoteStyle{bebras-ch-fra}{«\thinspace{}}[» ]{\thinspace{}»}{“}[»\thinspace{}› ]{”}
\DeclareQuoteStyle{bebras-ch-ita}{«}[» ]{»}{“}[»› ]{”}
\setquotestyle{bebras-ch-deu}

\usepackage{hyperref}
\usepackage{graphicx}
\usepackage{svg}
\svgsetup{inkscapeversion=1,inkscapearea=page}
\usepackage{wrapfig}

\usepackage{enumitem}
\setlist{nosep,itemsep=.5ex}

\setlength{\parindent}{0pt}
\setlength{\parskip}{2ex}
\raggedbottom

\usepackage{fancyhdr}
\usepackage{lastpage}
\pagestyle{fancy}

\fancyhf{}
\renewcommand{\headrulewidth}{0pt}
\renewcommand{\footrulewidth}{0.4pt}
\lfoot{\scriptsize © 2023 Bebras (CC BY-SA 4.0)}
\cfoot{\scriptsize\itshape 2023-IE-02b Ogham}
\rfoot{\scriptsize Page~\thepage{}/\pageref*{LastPage}}

\newcommand{\taskGraphicsFolder}{..}

\begin{document}

\section*{\centering{} 2023-IE-02b Ogham}


\subsection*{Body}

Sue kennt das alte irische Alphabet Ogham.
Jeder Buchstabe besteht aus einem oder mehreren Strichen, die entlang einer langen Linie angeordnet sind.
Zwei aufeinander folgende Buchstaben werden durch einen Zwischenraum getrennt.

Sue benutzt Ogham als Code.  Sie kodiert vier Wörter – ihre liebsten Fruchtsorten: \\
ANANAS, BANANE, MELONE und ORANGE.

{\em


\subsection*{Question/Challenge - for the brochures}

Welches Wort passt zu welchem Ogham-Code?

{\centering%
\includesvg[scale=0.5]{\taskGraphicsFolder/graphics/2023-IE-02b-question-deu_compatible.svg}\par}

}


\subsection*{Interactivity instruction - for the online challenge}

Ziehe die Wörter auf die richtigen Felder. Wenn du fertig bist, klicke \enquote{Antwort speichern}.

\begingroup
\renewcommand{\arraystretch}{1.5}
\subsection*{Answer Options/Interactivity Description}

The yellow squares with the fruitnames are draggables. To be dragged into the gray containers under the Ogham-Code.

\endgroup

\subsection*{Answer Explanation}

So ist es richtig:

{\centering%
\includesvg[scale=0.5]{\taskGraphicsFolder/graphics/2023-IE-02b-explanation-deu_compatible.svg}\par}

Es gibt verschiedene Möglichkeiten, die richtige Zuordnung zu bestimmen. Auf jeden Fall aber muss herausgefunden werden, in welcher Richtung die Buchstaben entlang der Linie geschrieben werden. Dabei hilft das besonders markante Wort ANANAS. Darin kommt der Buchstabe A dreimal vor, mit jeweils einem anderen Buchstaben dazwischen.

Nur im Ogham-Code $4$ kommt ein Buchstabe dreimal vor, und auch dort ist jeweils ein Buchstabe dazwischen. Code $4$ ist also der einzige, zu dem das Wort ANANAS passt. So erkennt man, dass man in Ogham Wörter von unten nach oben schreibt und dass der dreifach vorkommende Buchstabe A in Ogham als horizontaler Strich durch die Linie geschrieben wird.

Dieser Ogham-Buchstabe A kommt nur im Code $2$ zweimal vor.  Auch wegen der aus ANANAS bekannten Kodierung von N (fünf horizontale Striche rechts von der Linie) und der Anordnung der weiteren Buchstaben passt nur BANANE zu diesem Code. ORANGE passt nur zum Code $1$; unter anderem, weil man dort den Ogham-Buchstaben A genau einmal findet.  Nun ist nur noch Code $3$ übrig; er muss also das Ogham-Wort für MELONE sein und enthält die von den übrigen Wörtern bekannten Ogham-Buchstaben E und N an den passenden Stellen.


\subsection*{This is Informatics}

In dieser Biberaufgabe muss ein unbekannter Text entschlüsselt bzw. dechiffriert werden.  Das ist hier nicht sehr schwierig, weil der entschlüsselte \emph{Klartext} bekannt ist. Ausserdem ist der unbekannte Text auf gleiche Weise in Buchstaben und Wörter eingeteilt wie der bekannte Text. Beim Dechiffrieren eines geheimen Textes bzw. eines Textes in unbekannter Schrift, dessen Klartext nicht bekannt ist, hilft es in diesem Fall oft, sich Gedanken über die Häufigkeit von Buchstaben und Wörtern zu machen und auf dieser Grundlage zu versuchen, sie im Text zu finden. Auf diese Weise sind einige antike Alphabete und Schriften entschlüsselt worden. Schwierig wird es aber, wenn die Schriftzeichen im unbekannten Text nicht so einfach den Buchstaben und Wörtern der bekannten Sprache zuzuordnen sind wie im Fall von Ogham. Dann hilft oft nur der Abgleich mit bekannten Texten oder Schriften, wie in dieser Biberaufgabe. Zum Beispiel wurden die ägyptischen Hieroglyphen jahrhundertelang nicht entschlüsselt, bis durch einen Zufall ein Stein mit Hieroglyphen und zwei bekannten Schriften gefunden wurde, der Stein von Rosetta. Auf dem Stein fand sich dreimal der gleiche Text. Der war zwar in verschiedenen Sprachen geschrieben, enthielt aber immer dieselben Namen. So konnten wesentliche Elemente der Hieroglyphen entschlüsselt werden. Das gilt aber nicht für alle Schriften: Noch immer sind die etwa $650$ Schriftzeichen der Maya-Kultur nicht vollständig entschlüsselt, genau so wenig wie die Schriften Linearschrift A und Linearschrift B aus der Mittelmeerregion.

Auch in der Informatik werden Schriftzeichen und Texte entschlüsselt – nachdem sie vorher zur abhörsicheren Datenübertragung verschlüsselt wurden.  Dazu werden aber ganz andere Verfahren verwendet als bei der Kodierung von Wörtern in anderen Schriften.  Solche einfachen Kodierungen sind insbesondere mit Hilfe von Computern zu leicht zu entschlüsseln, meist mit Hilfe der oben schon genannten Überlegungen zur Häufigkeit von Buchstaben und Wörtern.


\subsection*{This is Computational Thinking}

Wenn man diese Aufgabe löst, sucht man letztlich nach Mustern, die in dem codierten Text und im Klartext wiederzufinden sind, wie beispielsweise die Position des Buchstaben A. Solche Muster zu erkennen kommt häufig in der Informatik vor, insbesondere wenn komplexe Probleme auf andere zurückgeführt werden, die bereits gelöst sind.


\subsection*{Informatics Keywords and Websites}

\begin{itemize}
  \item Kryptographie: \href{https://de.wikipedia.org/wiki/Kryptographie}{\BrochureUrlText{https://de.wikipedia.org/wiki/Kryptographie}}
  \item Kryptoanalyse: \href{https://de.wikipedia.org/wiki/Kryptoanalyse}{\BrochureUrlText{https://de.wikipedia.org/wiki/Kryptoanalyse}}
  \item Ogham: \href{https://de.wikipedia.org/wiki/Ogham}{\BrochureUrlText{https://de.wikipedia.org/wiki/Ogham}}
\end{itemize}


\subsection*{Computational Thinking Keywords and Websites}

\begin{itemize}
  \item \emph{Pattern recognition:} \href{https://www.bbc.co.uk/bitesize/guides/zxxbgk7/revision/1}{\BrochureUrlText{https://www.bbc.co.uk/bitesize/guides/zxxbgk7/revision/1}}
  \item \emph{Decomposition:} \href{https://www.bbc.co.uk/bitesize/guides/zqqfyrd/revision/1}{\BrochureUrlText{https://www.bbc.co.uk/bitesize/guides/zqqfyrd/revision/1}}
\end{itemize}


\end{document}
