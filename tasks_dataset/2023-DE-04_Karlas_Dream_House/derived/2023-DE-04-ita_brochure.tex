% Definition of the meta information: task difficulties, task ID, task title, task country; definition of the variables as well as their scope is in commands.tex
\setcounter{taskAgeDifficulty3to4}{3}
\setcounter{taskAgeDifficulty5to6}{2}
\setcounter{taskAgeDifficulty7to8}{1}
\setcounter{taskAgeDifficulty9to10}{0}
\setcounter{taskAgeDifficulty11to13}{0}
\renewcommand{\taskTitle}{La casa dei sogni di Karla}
\renewcommand{\taskCountry}{DE}

% include this task only if for the age groups being processed this task is relevant
\ifthenelse{
  \(\boolean{age3to4} \AND \(\value{taskAgeDifficulty3to4} > 0\)\) \OR
  \(\boolean{age5to6} \AND \(\value{taskAgeDifficulty5to6} > 0\)\) \OR
  \(\boolean{age7to8} \AND \(\value{taskAgeDifficulty7to8} > 0\)\) \OR
  \(\boolean{age9to10} \AND \(\value{taskAgeDifficulty9to10} > 0\)\) \OR
  \(\boolean{age11to13} \AND \(\value{taskAgeDifficulty11to13} > 0\)\)}{

\newchapter{\taskTitle}

% task body
Karla ha tre mappe che mostrano esattamente la stessa area. Una mappa mostra le foreste, una i fiumi e una le case della zona. La casa dei sogni di Karla si trova nella foresta e vicino a un fiume.

{\centering%
\begin{tabular}{ @{} c c c @{} }
  \makecell[c]{\includesvg[scale=0.54]{\taskGraphicsFolder/./graphics/-fra_ita/2023-DE-04-Map_Forest-compatible.svg}} & \makecell[c]{\includesvg[scale=0.54]{\taskGraphicsFolder/./graphics/-fra_ita/2023-DE-04-Map_Rivers-compatible.svg}} & \makecell[c]{\includesvg[scale=0.54]{\taskGraphicsFolder/./graphics/-fra_ita/2023-DE-04-Map_Houses-compatible.svg}} \\ 
  Mappa delle foreste & Mappa dei fiumi & Mappa delle case
\end{tabular}

\par}



% question (as \emph{})
{\em
Qual è la casa dei sogni di Karla?


}

% answer alternatives (as \begin{enumerate}[A)]) or interactivity


% from here on this is only included if solutions are processed
\ifthenelse{\boolean{solutions}}{
\newpage

% answer explanation
\section*{\BrochureSolution}
La casa in alto a sinistra della mappa è la casa dei sogni di Karla:

{\centering%
\includesvg[width=115.4px]{\taskGraphicsFolder/./graphics/-fra_ita/2023-DE-04-solution-compatible.svg}\par}

Per trovare la casa dei sogni di Karla, è necessario valutare le informazioni contenute in tutte e tre le mappe. La casa dei sogni deve trovarsi in una zona boscosa e vicino a un fiume. Questo vale solo per la casa in alto a sinistra. Questo è facile da vedere quando le carte vengono messe una sopra l’altra:

{\centering%
\includesvg[width=115.4px]{\taskGraphicsFolder/./graphics/-fra_ita/2023-DE-04-Maps_Overlay-compatible.svg}\par}



% it's informatics
\section*{\BrochureItsInformatics}
Quando le informazioni sulle foreste, sui fiumi e sulle case sono presentate su un’unica mappa è facile trovare la casa che si sta cercando.

Un \emph{geoinformation system} (GIS, sistema informativo geografico) riunisce una serie di informazioni spaziali (ad esempio, foreste, strade, confini nazionali, stazioni di servizio, municipi, pianure alluvionali, ecc.). Un GIS serve quindi alla visualizzazione e all’analisi dei cosiddetti \emph{geodati}. Con l’aiuto di un GIS è possibile, ad esempio, per gli addetti al controllo delle catastrofi, compilare informazioni per i piani di evacuazione.

L’uso di più livelli con diverse informazioni sull’immagine è noto anche nei programmi di grafica. Una questione importante è sempre quale livello, con gli oggetti che contiene, è il più alto e quindi viene visualizzato in primo piano. Nell’esempio, la mappa delle case dovrebbe essere il livello superiore, in modo che le case non siano nascoste dalle aree boschive.



% keywords and websites (as \begin{itemize})
\section*{\BrochureWebsitesAndKeywords}
{\raggedright
\begin{itemize}
  \item GIS: \href{https://it.wikipedia.org/wiki/Geographic_information_system}{\BrochureUrlText{https://it.wikipedia.org/wiki/Geographic\_information\_system}}
\end{itemize}


}

% end of ifthen for excluding the solutions
}{}

% all authors
% ATTENTION: you HAVE to make sure an according entry is in ../main/authors.tex.
% Syntax: \def\AuthorLastnameF{} (Lastname is last name, F is first letter of first name, this serves as a marker for ../main/authors.tex)
\def\AuthorSchluterK{} % \ifdefined\AuthorSchluterK \BrochureFlag{de}{} Kirsten Schlüter\fi
\def\AuthorParviainenM{} % \ifdefined\AuthorParviainenM \BrochureFlag{fi}{} Marika Parviainen\fi
\def\AuthorMatsuzawaY{} % \ifdefined\AuthorMatsuzawaY \BrochureFlag{jp}{} Yoshiaki Matsuzawa\fi
\def\AuthorJeonH{} % \ifdefined\AuthorJeonH \BrochureFlag{kr}{} Hyun-seok Jeon\fi
\def\AuthorChoudaryM{} % \ifdefined\AuthorChoudaryM \BrochureFlag{pk}{} Marios Omar Choudary\fi
\def\AuthorPluharZ{} % \ifdefined\AuthorPluharZ \BrochureFlag{hu}{} Zsuzsa Pluhár\fi
\def\AuthorDatzkoThutS{} % \ifdefined\AuthorDatzkoThutS \BrochureFlag{de}{} Susanne Datzko-Thut\fi
\def\AuthorGiangC{} % \ifdefined\AuthorGiangC \BrochureFlag{ch}{} Christian Giang\fi

\newpage}{}
