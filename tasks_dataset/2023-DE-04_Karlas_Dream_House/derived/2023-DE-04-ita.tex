\documentclass[a4paper,11pt]{report}
\usepackage[T1]{fontenc}
\usepackage[utf8]{inputenc}

\usepackage[italian]{babel}
\AtBeginDocument{\def\labelitemi{$\bullet$}}

\usepackage{etoolbox}

\usepackage[margin=2cm]{geometry}
\usepackage{changepage}
\makeatletter
\renewenvironment{adjustwidth}[2]{%
    \begin{list}{}{%
    \partopsep\z@%
    \topsep\z@%
    \listparindent\parindent%
    \parsep\parskip%
    \@ifmtarg{#1}{\setlength{\leftmargin}{\z@}}%
                 {\setlength{\leftmargin}{#1}}%
    \@ifmtarg{#2}{\setlength{\rightmargin}{\z@}}%
                 {\setlength{\rightmargin}{#2}}%
    }
    \item[]}{\end{list}}
\makeatother

\newcommand{\BrochureUrlText}[1]{\texttt{#1}}
\usepackage{setspace}
\setstretch{1.15}

\usepackage{tabularx}
\usepackage{booktabs}
\usepackage{makecell}
\usepackage{multirow}
\renewcommand\theadfont{\bfseries}
\renewcommand{\tabularxcolumn}[1]{>{}m{#1}}
\newcolumntype{R}{>{\raggedleft\arraybackslash}X}
\newcolumntype{C}{>{\centering\arraybackslash}X}
\newcolumntype{L}{>{\raggedright\arraybackslash}X}
\newcolumntype{J}{>{\arraybackslash}X}

\newcommand{\BrochureInlineCode}[1]{{\ttfamily #1}}

\usepackage{amssymb}
\usepackage{amsmath}

\usepackage[babel=true,maxlevel=3]{csquotes}
\DeclareQuoteStyle{bebras-ch-eng}{“}[” ]{”}{‘}[”’ ]{’}\DeclareQuoteStyle{bebras-ch-deu}{«}[» ]{»}{“}[»› ]{”}
\DeclareQuoteStyle{bebras-ch-fra}{«\thinspace{}}[» ]{\thinspace{}»}{“}[»\thinspace{}› ]{”}
\DeclareQuoteStyle{bebras-ch-ita}{«}[» ]{»}{“}[»› ]{”}
\setquotestyle{bebras-ch-ita}

\usepackage{hyperref}
\usepackage{graphicx}
\usepackage{svg}
\svgsetup{inkscapeversion=1,inkscapearea=page}
\usepackage{wrapfig}

\usepackage{enumitem}
\setlist{nosep,itemsep=.5ex}

\setlength{\parindent}{0pt}
\setlength{\parskip}{2ex}
\raggedbottom

\usepackage{fancyhdr}
\usepackage{lastpage}
\pagestyle{fancy}

\fancyhf{}
\renewcommand{\headrulewidth}{0pt}
\renewcommand{\footrulewidth}{0.4pt}
\lfoot{\scriptsize © 2023 Bebras (CC BY-SA 4.0)}
\cfoot{\scriptsize\itshape 2023-DE-04 La casa dei sogni di Karla}
\rfoot{\scriptsize Page~\thepage{}/\pageref*{LastPage}}

\newcommand{\taskGraphicsFolder}{..}

\begin{document}

\section*{\centering{} 2023-DE-04 La casa dei sogni di Karla}


\subsection*{Body}

Karla ha tre mappe che mostrano esattamente la stessa area. Una mappa mostra le foreste, una i fiumi e una le case della zona. La casa dei sogni di Karla si trova nella foresta e vicino a un fiume.

{\centering%
\begin{tabular}{ @{} c c c @{} }
  \makecell[c]{\includesvg[scale=0.54]{\taskGraphicsFolder/./graphics/-fra_ita/2023-DE-04-Map_Forest-compatible.svg}} & \makecell[c]{\includesvg[scale=0.54]{\taskGraphicsFolder/./graphics/-fra_ita/2023-DE-04-Map_Rivers-compatible.svg}} & \makecell[c]{\includesvg[scale=0.54]{\taskGraphicsFolder/./graphics/-fra_ita/2023-DE-04-Map_Houses-compatible.svg}} \\ 
  Mappa delle foreste & Mappa dei fiumi & Mappa delle case
\end{tabular}

\par}

{\em


\subsection*{Question/Challenge - for the brochures}

Qual è la casa dei sogni di Karla?

}


\subsection*{Interactivity instruction - for the online challenge}

Fa clic sulla casa giusta nella mappa delle case. Al termine, fa clic su \enquote{Salva risposta}.

\begingroup
\renewcommand{\arraystretch}{1.5}
\subsection*{Answer Options/Interactivity Description}

Karla’s dream house shall be selected by click on the rightmost map \href{./graphics/-fra_ita/2023-DE-04-Map_Houses-compatible.svg}{\BrochureUrlText{maph}}. All houses are clickable on that map, and the selection can be seen. Next click on the house or a click on another house will remove the selection.

\endgroup

\subsection*{Answer Explanation}

La casa in alto a sinistra della mappa è la casa dei sogni di Karla:

{\centering%
\includesvg[width=115.4px]{\taskGraphicsFolder/./graphics/-fra_ita/2023-DE-04-solution-compatible.svg}\par}

Per trovare la casa dei sogni di Karla, è necessario valutare le informazioni contenute in tutte e tre le mappe. La casa dei sogni deve trovarsi in una zona boscosa e vicino a un fiume. Questo vale solo per la casa in alto a sinistra. Questo è facile da vedere quando le carte vengono messe una sopra l’altra:

{\centering%
\includesvg[width=115.4px]{\taskGraphicsFolder/./graphics/-fra_ita/2023-DE-04-Maps_Overlay-compatible.svg}\par}


\subsection*{This is Informatics}

Quando le informazioni sulle foreste, sui fiumi e sulle case sono presentate su un’unica mappa è facile trovare la casa che si sta cercando.

Un \emph{geoinformation system} (GIS, sistema informativo geografico) riunisce una serie di informazioni spaziali (ad esempio, foreste, strade, confini nazionali, stazioni di servizio, municipi, pianure alluvionali, ecc.). Un GIS serve quindi alla visualizzazione e all’analisi dei cosiddetti \emph{geodati}. Con l’aiuto di un GIS è possibile, ad esempio, per gli addetti al controllo delle catastrofi, compilare informazioni per i piani di evacuazione.

L’uso di più livelli con diverse informazioni sull’immagine è noto anche nei programmi di grafica. Una questione importante è sempre quale livello, con gli oggetti che contiene, è il più alto e quindi viene visualizzato in primo piano. Nell’esempio, la mappa delle case dovrebbe essere il livello superiore, in modo che le case non siano nascoste dalle aree boschive.


\subsection*{This is Computational Thinking}

Optional - not to be filled 2023


\subsection*{Informatics Keywords and Websites}

\begin{itemize}
  \item GIS: \href{https://it.wikipedia.org/wiki/Geographic_information_system}{\BrochureUrlText{https://it.wikipedia.org/wiki/Geographic\_information\_system}}
\end{itemize}


\subsection*{Computational Thinking Keywords and Websites}

\begin{itemize}
  \item Abstraction
  \item Data structure analysis
\end{itemize}


\end{document}
