% Definition of the meta information: task difficulties, task ID, task title, task country; definition of the variables as well as their scope is in commands.tex
\setcounter{taskAgeDifficulty3to4}{0}
\setcounter{taskAgeDifficulty5to6}{0}
\setcounter{taskAgeDifficulty7to8}{2}
\setcounter{taskAgeDifficulty9to10}{1}
\setcounter{taskAgeDifficulty11to13}{0}
\renewcommand{\taskTitle}{Mattoni di castoro}
\renewcommand{\taskCountry}{AU}

% include this task only if for the age groups being processed this task is relevant
\ifthenelse{
  \(\boolean{age3to4} \AND \(\value{taskAgeDifficulty3to4} > 0\)\) \OR
  \(\boolean{age5to6} \AND \(\value{taskAgeDifficulty5to6} > 0\)\) \OR
  \(\boolean{age7to8} \AND \(\value{taskAgeDifficulty7to8} > 0\)\) \OR
  \(\boolean{age9to10} \AND \(\value{taskAgeDifficulty9to10} > 0\)\) \OR
  \(\boolean{age11to13} \AND \(\value{taskAgeDifficulty11to13} > 0\)\)}{

\newchapter{\taskTitle}

% task body
I mattoni di castoro Otto si differenziano per quattro caratteristiche:

\begin{tabularx}{\columnwidth}{ @{} J l @{} }
  \begin{enumerate}
  \item larghezza: stretta, media, larga
  \item altezza: piccola, media, grande
  \item numero di sporgenze sulla parte superiore: zero, uno, due
  \item numero di scanalature nella parte inferiore: zero, una, due.
\end{enumerate}

 & \makecell[l]{\includesvg[scale=1.1]{\taskGraphicsFolder/graphics/-ita/2023-AU-05-taskbody_compatible-ita.svg}}
\end{tabularx}

\begin{wrapfigure}{R}{134.20000000000002px}
\raisebox{-.46cm}[\dimexpr \height-.92cm \relax][-.46cm]{\includesvg[scale=1.1]{\taskGraphicsFolder/graphics/2023-AU-05_example.svg}}
\end{wrapfigure}

Otto divide i mattoni in gruppi di tre. Lo fa in modo che per ogni gruppo valga quanto segue: i tre mattoni hanno per ciascuna delle quattro proprietà …

\begin{itemize}
  \item … o tutte con lo stesso valore …
  \item … o tutte con tre valori diversi.
\end{itemize}

A destra, uno dei gruppi di Otto.

Perché questi tre mattoni hanno tutti

\begin{itemize}
  \item la stessa larghezza,
  \item diverse altezze,
  \item numero diverso di sporgenze e
  \item numero diverso di scanalature.
\end{itemize}



% question (as \emph{})
{\em
Dividi questi mattoni in gruppi di tre, come farebbe Otto.

{\centering%
\includesvg[width=1\linewidth]{\taskGraphicsFolder/graphics/2023-AU-05_interactive_task.svg}\par}


}

% answer alternatives (as \begin{enumerate}[A)]) or interactivity


% from here on this is only included if solutions are processed
\ifthenelse{\boolean{solutions}}{
\newpage

% answer explanation
\section*{\BrochureSolution}
La risposta giusta:

{\centering%
\includesvg[scale=1.1]{\taskGraphicsFolder/graphics/2023-AU-05_interactive_task_solution-compatible.svg}\par}

I mattoni sono divisi in gruppi secondo le regole di Otto.
La tabella mostra i tre gruppi, per i quali i valori delle proprietà sono tutti diversi o tutti uguali.

{\centering%
\begin{tabular}{ @{} l c c c @{} }
  {\setstretch{1.0}\thead[lb]{Proprietà}} & {\setstretch{1.0}\thead[cb]{Gruppo A}} & {\setstretch{1.0}\thead[cb]{Gruppo B}} & {\setstretch{1.0}\thead[cb]{Gruppo C}} \\ 
\midrule
  Larghezza & diversa & uguale & diversa \\ 
  Altezza & uguale & diversa & diversa \\ 
  Sporgenze & uguali & uguali & diversi \\ 
  Scanalature & diverse & diverse & uguali
\end{tabular}

\par}

Ma è questo l’unico modo per dividere i mattoni come farebbe Otto?

Si può considerare che se i valori di una proprietà devono essere diversi in tutti i gruppi, i valori diversi devono verificarsi in tutti i mattoni tante volte quanti sono i gruppi. In caso contrario, deve esistere almeno un gruppo in cui i valori di questa proprietà sono tutti uguali.

Un’analisi più attenta di tutti i mattoni mostra che i valori di larghezza di $2$ e $4$ unità si verificano solo due volte. Deve quindi esistere un gruppo in cui tutti i mattoni hanno una larghezza di tre unità.

Dei cinque mattoni di larghezza di tre unità, nessuno ha una sola sporgenza. Pertanto, non è possibile formare un gruppo con un numero diverso di sporgenze. Ma ci sono tre mattoni con zero sporgenze - e tutti hanno altezze diverse e un numero diverso di scanalature. Pertanto, il gruppo B è l’unico gruppo possibile di mattoni con larghezza di tre unità.

Negli altri due gruppi, le larghezze devono essere tutte diverse.

Nei sei mattoni rimanenti, i valori di altezza di tre unità e due unità si verifica solo una volta. Deve quindi esistere un gruppo in cui tutti i mattoni hanno un’altezza di un’unità. Il gruppo A è l’unico gruppo possibile di tre mattoni con altezza di un’unità che corrisponde alle idee di Otto. Restano i tre mattoni del gruppo C. Anch’essi formano un gruppo di tre, proprio come vorrebbe Otto.



% it's informatics
\section*{\BrochureItsInformatics}
In questo compito i mattoni sono descritti usando quattro \emph{proprietà} (o \emph{attributi}).

Per poter dividere i mattoni in gruppi di tre come vuole Otto, è necessario conoscere i valori delle proprietà di ciascun mattone.

Per questo, è sufficiente guardare ciascun elemento costitutivo. Un programma informatico che deve creare i gruppi di tre non può \enquote{vedere} e ha bisogno di una descrizione in una \emph{struttura di dati}.

Ad esempio, i mattoni in una \emph{base di dati} possono essere descritti come righe di una tabella. Le colonne della tabella corrispondono alle proprietà e ogni riga (chiamata anche \emph{dataset}) contiene i valori di un mattone nelle colonne appropriate:

{\centering%
\begin{tabular}{ @{} l l l l l @{} }
  {\setstretch{1.0}\thead[lb]{Mattone n.}} & {\setstretch{1.0}\thead[lb]{Larghezza}} & {\setstretch{1.0}\thead[lb]{Altezza}} & {\setstretch{1.0}\thead[lb]{Sporgenze}} & {\setstretch{1.0}\thead[lb]{Scanalature}} \\ 
\midrule
  1 & 1 & 3 & 1 & 0 \\ 
  2 & 2 & 2 & 2 & 0 \\ 
  … & … & … & … & …
\end{tabular}

\par}

La progettazione di tabelle di basi di dati è un’attività comune per gli informatici.

È necessario essere scrupolosi e considerare quali proprietà degli oggetti sono importanti per l’elaborazione da parte di un programma informatico. Le modifiche successive non sono così facili, soprattutto se i dati di molti oggetti sono già memorizzati.



% keywords and websites (as \begin{itemize})
\section*{\BrochureWebsitesAndKeywords}
{\raggedright
\begin{itemize}
  \item Struttura dati: \href{https://it.wikipedia.org/wiki/Struttura_dati}{\BrochureUrlText{https://it.wikipedia.org/wiki/Struttura\_dati}}
  \item Basi di dati: \href{https://it.wikipedia.org/wiki/Tabella_(basi_di_dati)}{\BrochureUrlText{https://it.wikipedia.org/wiki/Tabella\_(basi\_di\_dati)}}
\end{itemize}


}

% end of ifthen for excluding the solutions
}{}

% all authors
% ATTENTION: you HAVE to make sure an according entry is in ../main/authors.tex.
% Syntax: \def\AuthorLastnameF{} (Lastname is last name, F is first letter of first name, this serves as a marker for ../main/authors.tex)
\def\AuthorGrodeckA{} % \ifdefined\AuthorGrodeckA \BrochureFlag{au}{} Adam Grodeck\fi
\def\AuthorLeonardM{} % \ifdefined\AuthorLeonardM \BrochureFlag{fr}{} Marielle Léonard\fi
\def\AuthorPohlW{} % \ifdefined\AuthorPohlW \BrochureFlag{de}{} Wolfgang Pohl\fi
\def\AuthorEscherleN{} % \ifdefined\AuthorEscherleN \BrochureFlag{ch}{} Nora A.~Escherle\fi
\def\AuthorDatzkoThutS{} % \ifdefined\AuthorDatzkoThutS \BrochureFlag{de}{} Susanne Datzko-Thut\fi
\def\AuthorGiangC{} % \ifdefined\AuthorGiangC \BrochureFlag{ch}{} Christian Giang\fi

\newpage}{}
