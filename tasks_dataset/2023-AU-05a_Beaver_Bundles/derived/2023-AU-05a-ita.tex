\documentclass[a4paper,11pt]{report}
\usepackage[T1]{fontenc}
\usepackage[utf8]{inputenc}

\usepackage[italian]{babel}
\AtBeginDocument{\def\labelitemi{$\bullet$}}

\usepackage{etoolbox}

\usepackage[margin=2cm]{geometry}
\usepackage{changepage}
\makeatletter
\renewenvironment{adjustwidth}[2]{%
    \begin{list}{}{%
    \partopsep\z@%
    \topsep\z@%
    \listparindent\parindent%
    \parsep\parskip%
    \@ifmtarg{#1}{\setlength{\leftmargin}{\z@}}%
                 {\setlength{\leftmargin}{#1}}%
    \@ifmtarg{#2}{\setlength{\rightmargin}{\z@}}%
                 {\setlength{\rightmargin}{#2}}%
    }
    \item[]}{\end{list}}
\makeatother

\newcommand{\BrochureUrlText}[1]{\texttt{#1}}
\usepackage{setspace}
\setstretch{1.15}

\usepackage{tabularx}
\usepackage{booktabs}
\usepackage{makecell}
\usepackage{multirow}
\renewcommand\theadfont{\bfseries}
\renewcommand{\tabularxcolumn}[1]{>{}m{#1}}
\newcolumntype{R}{>{\raggedleft\arraybackslash}X}
\newcolumntype{C}{>{\centering\arraybackslash}X}
\newcolumntype{L}{>{\raggedright\arraybackslash}X}
\newcolumntype{J}{>{\arraybackslash}X}

\newcommand{\BrochureInlineCode}[1]{{\ttfamily #1}}

\usepackage{amssymb}
\usepackage{amsmath}

\usepackage[babel=true,maxlevel=3]{csquotes}
\DeclareQuoteStyle{bebras-ch-eng}{“}[” ]{”}{‘}[”’ ]{’}\DeclareQuoteStyle{bebras-ch-deu}{«}[» ]{»}{“}[»› ]{”}
\DeclareQuoteStyle{bebras-ch-fra}{«\thinspace{}}[» ]{\thinspace{}»}{“}[»\thinspace{}› ]{”}
\DeclareQuoteStyle{bebras-ch-ita}{«}[» ]{»}{“}[»› ]{”}
\setquotestyle{bebras-ch-ita}

\usepackage{hyperref}
\usepackage{graphicx}
\usepackage{svg}
\svgsetup{inkscapeversion=1,inkscapearea=page}
\usepackage{wrapfig}

\usepackage{enumitem}
\setlist{nosep,itemsep=.5ex}

\setlength{\parindent}{0pt}
\setlength{\parskip}{2ex}
\raggedbottom

\usepackage{fancyhdr}
\usepackage{lastpage}
\pagestyle{fancy}

\fancyhf{}
\renewcommand{\headrulewidth}{0pt}
\renewcommand{\footrulewidth}{0.4pt}
\lfoot{\scriptsize © 2023 Bebras (CC BY-SA 4.0)}
\cfoot{\scriptsize\itshape 2023-AU-05a Mattoni di castoro}
\rfoot{\scriptsize Page~\thepage{}/\pageref*{LastPage}}

\newcommand{\taskGraphicsFolder}{..}

\begin{document}

\section*{\centering{} 2023-AU-05a Mattoni di castoro}


\subsection*{Body}

I mattoni di castoro Otto si differenziano per quattro caratteristiche:

\begin{tabularx}{\columnwidth}{ @{} J l @{} }
  \begin{enumerate}
  \item larghezza: stretta, media, larga
  \item altezza: piccola, media, grande
  \item numero di sporgenze sulla parte superiore: zero, uno, due
  \item numero di scanalature nella parte inferiore: zero, una, due.
\end{enumerate}

 & \makecell[l]{\includesvg[scale=1.1]{\taskGraphicsFolder/graphics/-ita/2023-AU-05-taskbody_compatible-ita.svg}}
\end{tabularx}

\begin{wrapfigure}{R}{134.20000000000002px}
\raisebox{-.46cm}[\dimexpr \height-.92cm \relax][-.46cm]{\includesvg[scale=1.1]{\taskGraphicsFolder/graphics/2023-AU-05_example.svg}}
\end{wrapfigure}

Otto divide i mattoni in gruppi di tre. Lo fa in modo che per ogni gruppo valga quanto segue: i tre mattoni hanno per ciascuna delle quattro proprietà …

\begin{itemize}
  \item … o tutte con lo stesso valore …
  \item … o tutte con tre valori diversi.
\end{itemize}

A destra, uno dei gruppi di Otto.

Perché questi tre mattoni hanno tutti

\begin{itemize}
  \item la stessa larghezza,
  \item diverse altezze,
  \item numero diverso di sporgenze e
  \item numero diverso di scanalature.
\end{itemize}

{\em


\subsection*{Question/Challenge - for the brochures}

Dividi questi mattoni in gruppi di tre, come farebbe Otto.

{\centering%
\includesvg[width=1\linewidth]{\taskGraphicsFolder/graphics/2023-AU-05_interactive_task.svg}\par}

}


\subsection*{Interactivity instruction - for the online challenge}

Trascina i mattoni nelle caselle. Al termine, fai clic su \enquote{Salva risposta}.

\begingroup
\renewcommand{\arraystretch}{1.5}
\subsection*{Answer Options/Interactivity Description}

The blocks are draggable. The script should check if the draggbles in the group fit the contraints in the taskbody.

\endgroup

\subsection*{Answer Explanation}

La risposta giusta:

{\centering%
\includesvg[scale=1.1]{\taskGraphicsFolder/graphics/2023-AU-05_interactive_task_solution-compatible.svg}\par}

I mattoni sono divisi in gruppi secondo le regole di Otto.
La tabella mostra i tre gruppi, per i quali i valori delle proprietà sono tutti diversi o tutti uguali.

{\centering%
\begin{tabular}{ @{} l c c c @{} }
  {\setstretch{1.0}\thead[lb]{Proprietà}} & {\setstretch{1.0}\thead[cb]{Gruppo A}} & {\setstretch{1.0}\thead[cb]{Gruppo B}} & {\setstretch{1.0}\thead[cb]{Gruppo C}} \\ 
\midrule
  Larghezza & diversa & uguale & diversa \\ 
  Altezza & uguale & diversa & diversa \\ 
  Sporgenze & uguali & uguali & diversi \\ 
  Scanalature & diverse & diverse & uguali
\end{tabular}

\par}

Ma è questo l’unico modo per dividere i mattoni come farebbe Otto?

Si può considerare che se i valori di una proprietà devono essere diversi in tutti i gruppi, i valori diversi devono verificarsi in tutti i mattoni tante volte quanti sono i gruppi. In caso contrario, deve esistere almeno un gruppo in cui i valori di questa proprietà sono tutti uguali.

Un’analisi più attenta di tutti i mattoni mostra che i valori di larghezza di $2$ e $4$ unità si verificano solo due volte. Deve quindi esistere un gruppo in cui tutti i mattoni hanno una larghezza di tre unità.

Dei cinque mattoni di larghezza di tre unità, nessuno ha una sola sporgenza. Pertanto, non è possibile formare un gruppo con un numero diverso di sporgenze. Ma ci sono tre mattoni con zero sporgenze - e tutti hanno altezze diverse e un numero diverso di scanalature. Pertanto, il gruppo B è l’unico gruppo possibile di mattoni con larghezza di tre unità.

Negli altri due gruppi, le larghezze devono essere tutte diverse.

Nei sei mattoni rimanenti, i valori di altezza di tre unità e due unità si verifica solo una volta. Deve quindi esistere un gruppo in cui tutti i mattoni hanno un’altezza di un’unità. Il gruppo A è l’unico gruppo possibile di tre mattoni con altezza di un’unità che corrisponde alle idee di Otto. Restano i tre mattoni del gruppo C. Anch’essi formano un gruppo di tre, proprio come vorrebbe Otto.


\subsection*{This is Informatics}

In questo compito i mattoni sono descritti usando quattro \emph{proprietà} (o \emph{attributi}).

Per poter dividere i mattoni in gruppi di tre come vuole Otto, è necessario conoscere i valori delle proprietà di ciascun mattone.

Per questo, è sufficiente guardare ciascun elemento costitutivo. Un programma informatico che deve creare i gruppi di tre non può \enquote{vedere} e ha bisogno di una descrizione in una \emph{struttura di dati}.

Ad esempio, i mattoni in una \emph{base di dati} possono essere descritti come righe di una tabella. Le colonne della tabella corrispondono alle proprietà e ogni riga (chiamata anche \emph{dataset}) contiene i valori di un mattone nelle colonne appropriate:

{\centering%
\begin{tabular}{ @{} l l l l l @{} }
  {\setstretch{1.0}\thead[lb]{Mattone n.}} & {\setstretch{1.0}\thead[lb]{Larghezza}} & {\setstretch{1.0}\thead[lb]{Altezza}} & {\setstretch{1.0}\thead[lb]{Sporgenze}} & {\setstretch{1.0}\thead[lb]{Scanalature}} \\ 
\midrule
  1 & 1 & 3 & 1 & 0 \\ 
  2 & 2 & 2 & 2 & 0 \\ 
  … & … & … & … & …
\end{tabular}

\par}

La progettazione di tabelle di basi di dati è un’attività comune per gli informatici.

È necessario essere scrupolosi e considerare quali proprietà degli oggetti sono importanti per l’elaborazione da parte di un programma informatico. Le modifiche successive non sono così facili, soprattutto se i dati di molti oggetti sono già memorizzati.


\subsection*{This is Computational Thinking}

Optional - not to be filled 2023


\subsection*{Informatics Keywords and Websites}

\begin{itemize}
  \item Struttura dati: \href{https://it.wikipedia.org/wiki/Struttura_dati}{\BrochureUrlText{https://it.wikipedia.org/wiki/Struttura\_dati}}
  \item Basi di dati: \href{https://it.wikipedia.org/wiki/Tabella_(basi_di_dati)}{\BrochureUrlText{https://it.wikipedia.org/wiki/Tabella\_(basi\_di\_dati)}}
\end{itemize}


\subsection*{Computational Thinking Keywords and Websites}

–


\end{document}
