\documentclass[a4paper,11pt]{report}
\usepackage[T1]{fontenc}
\usepackage[utf8]{inputenc}

\usepackage[italian]{babel}
\AtBeginDocument{\def\labelitemi{$\bullet$}}

\usepackage{etoolbox}

\usepackage[margin=2cm]{geometry}
\usepackage{changepage}
\makeatletter
\renewenvironment{adjustwidth}[2]{%
    \begin{list}{}{%
    \partopsep\z@%
    \topsep\z@%
    \listparindent\parindent%
    \parsep\parskip%
    \@ifmtarg{#1}{\setlength{\leftmargin}{\z@}}%
                 {\setlength{\leftmargin}{#1}}%
    \@ifmtarg{#2}{\setlength{\rightmargin}{\z@}}%
                 {\setlength{\rightmargin}{#2}}%
    }
    \item[]}{\end{list}}
\makeatother

\newcommand{\BrochureUrlText}[1]{\texttt{#1}}
\usepackage{setspace}
\setstretch{1.15}

\usepackage{tabularx}
\usepackage{booktabs}
\usepackage{makecell}
\usepackage{multirow}
\renewcommand\theadfont{\bfseries}
\renewcommand{\tabularxcolumn}[1]{>{}m{#1}}
\newcolumntype{R}{>{\raggedleft\arraybackslash}X}
\newcolumntype{C}{>{\centering\arraybackslash}X}
\newcolumntype{L}{>{\raggedright\arraybackslash}X}
\newcolumntype{J}{>{\arraybackslash}X}

\newcommand{\BrochureInlineCode}[1]{{\ttfamily #1}}

\usepackage{amssymb}
\usepackage{amsmath}

\usepackage[babel=true,maxlevel=3]{csquotes}
\DeclareQuoteStyle{bebras-ch-eng}{“}[” ]{”}{‘}[”’ ]{’}\DeclareQuoteStyle{bebras-ch-deu}{«}[» ]{»}{“}[»› ]{”}
\DeclareQuoteStyle{bebras-ch-fra}{«\thinspace{}}[» ]{\thinspace{}»}{“}[»\thinspace{}› ]{”}
\DeclareQuoteStyle{bebras-ch-ita}{«}[» ]{»}{“}[»› ]{”}
\setquotestyle{bebras-ch-ita}

\usepackage{hyperref}
\usepackage{graphicx}
\usepackage{svg}
\svgsetup{inkscapeversion=1,inkscapearea=page}
\usepackage{wrapfig}

\usepackage{enumitem}
\setlist{nosep,itemsep=.5ex}

\setlength{\parindent}{0pt}
\setlength{\parskip}{2ex}
\raggedbottom

\usepackage{fancyhdr}
\usepackage{lastpage}
\pagestyle{fancy}

\fancyhf{}
\renewcommand{\headrulewidth}{0pt}
\renewcommand{\footrulewidth}{0.4pt}
\lfoot{\scriptsize © 2023 Bebras (CC BY-SA 4.0)}
\cfoot{\scriptsize\itshape 2023-DE-02 Bouquet}
\rfoot{\scriptsize Page~\thepage{}/\pageref*{LastPage}}

\newcommand{\taskGraphicsFolder}{..}

\begin{document}

\section*{\centering{} 2023-DE-02 Bouquet}


\subsection*{Body}

{\centering%
\includesvg[scale=0.5]{\taskGraphicsFolder/graphics/2023-DE-02-taskbody.svg}\par}

Florian vende mazzi di fiori. Florian lega ogni bouquet secondo queste istruzioni:

\begin{enumerate}
  \item Prendere un primo fiore dal secchio A.
  \item Se il primo fiore è una margherita \raisebox{\dimexpr -0.5ex -1.0ex \relax}[0pt][0pt]{\includesvg[width=14.4px]{\taskGraphicsFolder/graphics/2023-DE-02-taskbody-flower.svg}}, prendere un’altra margherita \raisebox{\dimexpr -0.5ex -1.0ex \relax}[0pt][0pt]{\includesvg[width=14.4px]{\taskGraphicsFolder/graphics/2023-DE-02-taskbody-flower.svg}}.
  \item Poi prendere un rametto \raisebox{\dimexpr -0.5ex -0.5ex \relax}[0pt][0pt]{\includesvg[width=13px]{\taskGraphicsFolder/graphics/2023-DE-02-taskbody-sprig.svg}} dal secchio B fino a formare un bouquet di $4$ parti. Fatto!
\end{enumerate}

{\em


\subsection*{Question/Challenge - for the brochures}

Aiuta Florian: segui le istruzioni e scegli fiori e rami per un bouquet.

{\centering%
\includesvg[width=1\linewidth]{\taskGraphicsFolder/graphics/2023-DE-02-question-interactive.svg}\par}

}


\subsection*{Interactivity instruction - for the online challenge}

Trascina le parti selezionate sul foglio verde. Al termine, fa clic su \enquote{Salva risposta}.

\begingroup
\renewcommand{\arraystretch}{1.5}
\subsection*{Answer Options/Interactivity Description}

Every sprig and flower are a draggable. ($4$ from each type). The squares are the containers for the flowers and sprigs.

\endgroup

\subsection*{Answer Explanation}

Esistono due soluzioni corrette:

{\centering%
\raisebox{-0.5ex}{\includesvg[scale=0.5]{\taskGraphicsFolder/graphics/2023-DE-02-answer01.svg}}
\raisebox{-0.5ex}{\includesvg[scale=0.5]{\taskGraphicsFolder/graphics/2023-DE-02-answer02.svg}}\par}

Per legare correttamente i bouquet, Florian deve seguire le istruzioni. Possiamo anche descrivere le istruzioni con un diagramma:

{\centering%
\includesvg[width=216.5px]{\taskGraphicsFolder/graphics/2023-DE-02-explanation-ita-compatible.svg}\par}

Dopo che Florian ha scelto il primo fiore dal secchio A, segue una decisione che dipende dal primo fiore. O prende un’altra margherita (\raisebox{\dimexpr -0.5ex -1.0ex \relax}[0pt][0pt]{\includesvg[width=14.4px]{\taskGraphicsFolder/graphics/2023-DE-02-taskbody-flower.svg}}), o segue la freccia \enquote{no} e prende un rametto \raisebox{\dimexpr -0.5ex -0.5ex \relax}[0pt][0pt]{\includesvg[width=13px]{\taskGraphicsFolder/graphics/2023-DE-02-taskbody-sprig.svg}}.

Poi controlla se ha già quattro parti.
In caso contrario, segue la freccia \enquote{no} e deve prendere un altro rametto e poi controllare di nuovo il numero di parti.

Quindi, se prende prima una margherita \raisebox{\dimexpr -0.5ex -1.0ex \relax}[0pt][0pt]{\includesvg[width=14.4px]{\taskGraphicsFolder/graphics/2023-DE-02-taskbody-flower.svg}}, prenderà un’altra margherita e poi prenderà due volte un rametto \raisebox{\dimexpr -0.5ex -0.5ex \relax}[0pt][0pt]{\includesvg[width=13px]{\taskGraphicsFolder/graphics/2023-DE-02-taskbody-sprig.svg}}. Ma se prende prima un tulipano \raisebox{-0.5ex}[0pt][0pt]{\includesvg[width=14.4px]{\taskGraphicsFolder/graphics/2023-DE-02-tulpe.svg}}, andrà direttamente a \enquote{scegliere dal secchio B} e prenderà un rametto \raisebox{\dimexpr -0.5ex -0.5ex \relax}[0pt][0pt]{\includesvg[width=13px]{\taskGraphicsFolder/graphics/2023-DE-02-taskbody-sprig.svg}} dal secchio B fino ad avere $4$ pezzi - quindi prenderà $3$ rametti in totale.


\subsection*{This is Informatics}

Le \emph{istruzioni} per legare il bouquet sono chiare e potrebbero essere eseguite da una macchina. In informatica si parla di \emph{algoritmo}. Le istruzioni utilizzano alcune istruzioni che sono comuni anche nei programmi per computer:

\begin{itemize}
  \item La prima istruzione è una selezione casuale da un insieme di oggetti.
  \item La seconda istruzione si chiama \emph{struttura condizionale} o \emph{selezione}: perché si deve scegliere tra due o più possibilità.
  \item La terza istruzione sembra relativamente semplice, ma deve essere ben strutturata in un programma per computer. La parte interna dell’istruzione (essa stessa un’istruzione: \enquote{Prendi un rametto dal secchio B}) deve essere eseguita più volte finché il bouquet non è composto da $4$ parti. L’esecuzione dell’istruzione interna viene quindi ripetuta finché non viene soddisfatta la condizione \enquote{Il bouquet è composto da $4$ parti}. Una tale \emph{iterazione} è chiamata anche \emph{loop}.
\end{itemize}

Un algoritmo può essere rappresentato in modi diversi.  In questo compito, l’algoritmo del \enquote{bouquet di fiori} di Florian è formulato come istruzioni in linguaggio naturale. Nella spiegazione della soluzione, viene presentato come un \enquote{diagramma di flusso del programma}.

Il fiorista è un mestiere. Esistono tradizioni e regole su come legare un bouquet o una corona di fiori. Questo è un esempio di come le istruzioni o gli algoritmi esistano in molti settori della vita, non solo nell’informatica.


\subsection*{This is Computational Thinking}

–


\subsection*{Informatics Keywords and Websites}

\begin{itemize}
  \item Selezione: \href{https://it.wikipedia.org/wiki/Selezione_(informatica)}{\BrochureUrlText{https://it.wikipedia.org/wiki/Selezione\_(informatica)}}
  \item Iterazione: \href{https://it.wikipedia.org/wiki/Iterazione}{\BrochureUrlText{https://it.wikipedia.org/wiki/Iterazione}}
  \item Pseudocodice: \href{https://it.wikipedia.org/wiki/Pseudocodice}{\BrochureUrlText{https://it.wikipedia.org/wiki/Pseudocodice}}
  \item Diagramma di flusso: \href{https://it.wikipedia.org/wiki/Diagramma_di_flusso}{\BrochureUrlText{https://it.wikipedia.org/wiki/Diagramma\_di\_flusso}}
\end{itemize}


\subsection*{Computational Thinking Keywords and Websites}

\begin{itemize}
  \item Algorithmisches Denken
  \item Evaluation
\end{itemize}


\end{document}
