% Definition of the meta information: task difficulties, task ID, task title, task country; definition of the variables as well as their scope is in commands.tex
\setcounter{taskAgeDifficulty3to4}{2}
\setcounter{taskAgeDifficulty5to6}{1}
\setcounter{taskAgeDifficulty7to8}{0}
\setcounter{taskAgeDifficulty9to10}{0}
\setcounter{taskAgeDifficulty11to13}{0}
\renewcommand{\taskTitle}{Bouquet}
\renewcommand{\taskCountry}{DE}

% include this task only if for the age groups being processed this task is relevant
\ifthenelse{
  \(\boolean{age3to4} \AND \(\value{taskAgeDifficulty3to4} > 0\)\) \OR
  \(\boolean{age5to6} \AND \(\value{taskAgeDifficulty5to6} > 0\)\) \OR
  \(\boolean{age7to8} \AND \(\value{taskAgeDifficulty7to8} > 0\)\) \OR
  \(\boolean{age9to10} \AND \(\value{taskAgeDifficulty9to10} > 0\)\) \OR
  \(\boolean{age11to13} \AND \(\value{taskAgeDifficulty11to13} > 0\)\)}{

\newchapter{\taskTitle}

% task body
{\centering%
\includesvg[scale=0.5]{\taskGraphicsFolder/graphics/2023-DE-02-taskbody.svg}\par}

Florian vende mazzi di fiori. Florian lega ogni bouquet secondo queste istruzioni:

\begin{enumerate}
  \item Prendere un primo fiore dal secchio A.
  \item Se il primo fiore è una margherita \raisebox{\dimexpr -0.5ex -1.0ex \relax}[0pt][0pt]{\includesvg[width=14.4px]{\taskGraphicsFolder/graphics/2023-DE-02-taskbody-flower.svg}}, prendere un’altra margherita \raisebox{\dimexpr -0.5ex -1.0ex \relax}[0pt][0pt]{\includesvg[width=14.4px]{\taskGraphicsFolder/graphics/2023-DE-02-taskbody-flower.svg}}.
  \item Poi prendere un rametto \raisebox{\dimexpr -0.5ex -0.5ex \relax}[0pt][0pt]{\includesvg[width=13px]{\taskGraphicsFolder/graphics/2023-DE-02-taskbody-sprig.svg}} dal secchio B fino a formare un bouquet di $4$ parti. Fatto!
\end{enumerate}



% question (as \emph{})
{\em
Aiuta Florian: segui le istruzioni e scegli fiori e rami per un bouquet.

{\centering%
\includesvg[width=1\linewidth]{\taskGraphicsFolder/graphics/2023-DE-02-question-interactive.svg}\par}


}

% answer alternatives (as \begin{enumerate}[A)]) or interactivity


% from here on this is only included if solutions are processed
\ifthenelse{\boolean{solutions}}{
\newpage

% answer explanation
\section*{\BrochureSolution}
Esistono due soluzioni corrette:

{\centering%
\raisebox{-0.5ex}{\includesvg[scale=0.5]{\taskGraphicsFolder/graphics/2023-DE-02-answer01.svg}}
\raisebox{-0.5ex}{\includesvg[scale=0.5]{\taskGraphicsFolder/graphics/2023-DE-02-answer02.svg}}\par}

Per legare correttamente i bouquet, Florian deve seguire le istruzioni. Possiamo anche descrivere le istruzioni con un diagramma:

{\centering%
\includesvg[width=216.5px]{\taskGraphicsFolder/graphics/2023-DE-02-explanation-ita-compatible.svg}\par}

Dopo che Florian ha scelto il primo fiore dal secchio A, segue una decisione che dipende dal primo fiore. O prende un’altra margherita (\raisebox{\dimexpr -0.5ex -1.0ex \relax}[0pt][0pt]{\includesvg[width=14.4px]{\taskGraphicsFolder/graphics/2023-DE-02-taskbody-flower.svg}}), o segue la freccia \enquote{no} e prende un rametto \raisebox{\dimexpr -0.5ex -0.5ex \relax}[0pt][0pt]{\includesvg[width=13px]{\taskGraphicsFolder/graphics/2023-DE-02-taskbody-sprig.svg}}.

Poi controlla se ha già quattro parti.
In caso contrario, segue la freccia \enquote{no} e deve prendere un altro rametto e poi controllare di nuovo il numero di parti.

Quindi, se prende prima una margherita \raisebox{\dimexpr -0.5ex -1.0ex \relax}[0pt][0pt]{\includesvg[width=14.4px]{\taskGraphicsFolder/graphics/2023-DE-02-taskbody-flower.svg}}, prenderà un’altra margherita e poi prenderà due volte un rametto \raisebox{\dimexpr -0.5ex -0.5ex \relax}[0pt][0pt]{\includesvg[width=13px]{\taskGraphicsFolder/graphics/2023-DE-02-taskbody-sprig.svg}}. Ma se prende prima un tulipano \raisebox{-0.5ex}[0pt][0pt]{\includesvg[width=14.4px]{\taskGraphicsFolder/graphics/2023-DE-02-tulpe.svg}}, andrà direttamente a \enquote{scegliere dal secchio B} e prenderà un rametto \raisebox{\dimexpr -0.5ex -0.5ex \relax}[0pt][0pt]{\includesvg[width=13px]{\taskGraphicsFolder/graphics/2023-DE-02-taskbody-sprig.svg}} dal secchio B fino ad avere $4$ pezzi - quindi prenderà $3$ rametti in totale.



% it's informatics
\section*{\BrochureItsInformatics}
Le \emph{istruzioni} per legare il bouquet sono chiare e potrebbero essere eseguite da una macchina. In informatica si parla di \emph{algoritmo}. Le istruzioni utilizzano alcune istruzioni che sono comuni anche nei programmi per computer:

\begin{itemize}
  \item La prima istruzione è una selezione casuale da un insieme di oggetti.
  \item La seconda istruzione si chiama \emph{struttura condizionale} o \emph{selezione}: perché si deve scegliere tra due o più possibilità.
  \item La terza istruzione sembra relativamente semplice, ma deve essere ben strutturata in un programma per computer. La parte interna dell’istruzione (essa stessa un’istruzione: \enquote{Prendi un rametto dal secchio B}) deve essere eseguita più volte finché il bouquet non è composto da $4$ parti. L’esecuzione dell’istruzione interna viene quindi ripetuta finché non viene soddisfatta la condizione \enquote{Il bouquet è composto da $4$ parti}. Una tale \emph{iterazione} è chiamata anche \emph{loop}.
\end{itemize}

Un algoritmo può essere rappresentato in modi diversi.  In questo compito, l’algoritmo del \enquote{bouquet di fiori} di Florian è formulato come istruzioni in linguaggio naturale. Nella spiegazione della soluzione, viene presentato come un \enquote{diagramma di flusso del programma}.

Il fiorista è un mestiere. Esistono tradizioni e regole su come legare un bouquet o una corona di fiori. Questo è un esempio di come le istruzioni o gli algoritmi esistano in molti settori della vita, non solo nell’informatica.



% keywords and websites (as \begin{itemize})
\section*{\BrochureWebsitesAndKeywords}
{\raggedright
\begin{itemize}
  \item Selezione: \href{https://it.wikipedia.org/wiki/Selezione_(informatica)}{\BrochureUrlText{https://it.wikipedia.org/wiki/Selezione\_(informatica)}}
  \item Iterazione: \href{https://it.wikipedia.org/wiki/Iterazione}{\BrochureUrlText{https://it.wikipedia.org/wiki/Iterazione}}
  \item Pseudocodice: \href{https://it.wikipedia.org/wiki/Pseudocodice}{\BrochureUrlText{https://it.wikipedia.org/wiki/Pseudocodice}}
  \item Diagramma di flusso: \href{https://it.wikipedia.org/wiki/Diagramma_di_flusso}{\BrochureUrlText{https://it.wikipedia.org/wiki/Diagramma\_di\_flusso}}
\end{itemize}


}

% end of ifthen for excluding the solutions
}{}

% all authors
% ATTENTION: you HAVE to make sure an according entry is in ../main/authors.tex.
% Syntax: \def\AuthorLastnameF{} (Lastname is last name, F is first letter of first name, this serves as a marker for ../main/authors.tex)
\def\AuthorWeigendM{} % \ifdefined\AuthorWeigendM \BrochureFlag{de}{} Michael Weigend\fi
\def\AuthorMaoY{} % \ifdefined\AuthorMaoY \BrochureFlag{cn}{} Yong Mao\fi
\def\AuthorKoleszarV{} % \ifdefined\AuthorKoleszarV \BrochureFlag{uy}{} Víctor Koleszar\fi
\def\AuthorDatzkoThutS{} % \ifdefined\AuthorDatzkoThutS \BrochureFlag{de}{} Susanne Datzko-Thut\fi
\def\AuthorPluharZ{} % \ifdefined\AuthorPluharZ \BrochureFlag{hu}{} Zsuzsa Pluhár\fi
\def\AuthorGiangC{} % \ifdefined\AuthorGiangC \BrochureFlag{ch}{} Christian Giang\fi

\newpage}{}
