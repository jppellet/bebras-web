% Definition of the meta information: task difficulties, task ID, task title, task country; definition of the variables as well as their scope is in commands.tex
\setcounter{taskAgeDifficulty3to4}{2}
\setcounter{taskAgeDifficulty5to6}{1}
\setcounter{taskAgeDifficulty7to8}{0}
\setcounter{taskAgeDifficulty9to10}{0}
\setcounter{taskAgeDifficulty11to13}{0}
\renewcommand{\taskTitle}{Blumenstrauss}
\renewcommand{\taskCountry}{DE}

% include this task only if for the age groups being processed this task is relevant
\ifthenelse{
  \(\boolean{age3to4} \AND \(\value{taskAgeDifficulty3to4} > 0\)\) \OR
  \(\boolean{age5to6} \AND \(\value{taskAgeDifficulty5to6} > 0\)\) \OR
  \(\boolean{age7to8} \AND \(\value{taskAgeDifficulty7to8} > 0\)\) \OR
  \(\boolean{age9to10} \AND \(\value{taskAgeDifficulty9to10} > 0\)\) \OR
  \(\boolean{age11to13} \AND \(\value{taskAgeDifficulty11to13} > 0\)\)}{

\newchapter{\taskTitle}

% task body
{\centering%
\includesvg[scale=0.5]{\taskGraphicsFolder/graphics/2023-DE-02-taskbody.svg}\par}

Florian verkauft Blumensträusse. Jeden Blumenstrauss bindet Florian nach dieser Anleitung:

\begin{enumerate}
  \item Nimm die erste Blume aus Eimer A.
  \item Wenn die erste Blume eine Margarite \raisebox{\dimexpr -0.5ex -1.0ex \relax}[0pt][0pt]{\includesvg[width=14.4px]{\taskGraphicsFolder/graphics/2023-DE-02-taskbody-flower.svg}} ist, nimm noch eine Margarite \raisebox{\dimexpr -0.5ex -1.0ex \relax}[0pt][0pt]{\includesvg[width=14.4px]{\taskGraphicsFolder/graphics/2023-DE-02-taskbody-flower.svg}}.
  \item Nun nimm solange einen Zweig \raisebox{\dimexpr -0.5ex -0.5ex \relax}[0pt][0pt]{\includesvg[width=13px]{\taskGraphicsFolder/graphics/2023-DE-02-taskbody-sprig.svg}} aus Eimer B, bis der Blumenstrauss $4$ Teile hat. Fertig!
\end{enumerate}



% question (as \emph{})
{\em
Hilf Florian: Folge der Anleitung und wähle Blumen und Zweige für einen Strauss aus.

{\centering%
\includesvg[width=1\linewidth]{\taskGraphicsFolder/graphics/2023-DE-02-question-interactive.svg}\par}


}

% answer alternatives (as \begin{enumerate}[A)]) or interactivity


% from here on this is only included if solutions are processed
\ifthenelse{\boolean{solutions}}{
\newpage

% answer explanation
\section*{\BrochureSolution}
Es gibt zwei richtige Lösungen:

{\centering%
\raisebox{-0.5ex}{\includesvg[scale=0.5]{\taskGraphicsFolder/graphics/2023-DE-02-answer01.svg}}
\raisebox{-0.5ex}{\includesvg[scale=0.5]{\taskGraphicsFolder/graphics/2023-DE-02-answer02.svg}}\par}

Um die Blumensträusse korrekt zu binden, muss Florian die Anleitung befolgen. Wir können die Anleitung auch mit einem Diagramm beschreiben:

{\centering%
\includesvg[width=288.6px]{\taskGraphicsFolder/graphics/2023-DE-02-explanation-deu-compatible.svg}\par}

Nachdem Florian die erste Blume aus Eimer A gewählt hat, folgt eine Entscheidung, die abhängig von der ersten Blume ist. Entweder nimmt er noch eine Margarite (\raisebox{\dimexpr -0.5ex -1.0ex \relax}[0pt][0pt]{\includesvg[width=14.4px]{\taskGraphicsFolder/graphics/2023-DE-02-taskbody-flower.svg}}) oder er folgt dem \enquote{nein}-Pfeil und nimmt einen Zweig \raisebox{\dimexpr -0.5ex -0.5ex \relax}[0pt][0pt]{\includesvg[width=13px]{\taskGraphicsFolder/graphics/2023-DE-02-taskbody-sprig.svg}}.

Dann überprüft er, ob er schon vier Teile hat.
Wenn nicht, folgt er dem \enquote{nein}-Pfeil und muss einen weiteren Zweig \raisebox{\dimexpr -0.5ex -0.5ex \relax}[0pt][0pt]{\includesvg[width=13px]{\taskGraphicsFolder/graphics/2023-DE-02-taskbody-sprig.svg}} nehmen und dann die Anzahl der Teile wieder überprüfen.

Wenn er also zuerst eine Margarite \raisebox{\dimexpr -0.5ex -1.0ex \relax}[0pt][0pt]{\includesvg[width=14.4px]{\taskGraphicsFolder/graphics/2023-DE-02-taskbody-flower.svg}} nimmt, wird er noch eine Margarite nehmen und dann zweimal einen Zweig \raisebox{\dimexpr -0.5ex -0.5ex \relax}[0pt][0pt]{\includesvg[width=13px]{\taskGraphicsFolder/graphics/2023-DE-02-taskbody-sprig.svg}} nehmen. Wenn er aber zuerst eine Tulpe \raisebox{-0.5ex}[0pt][0pt]{\includesvg[width=14.4px]{\taskGraphicsFolder/graphics/2023-DE-02-tulpe.svg}} nimmt, wird er danach direkt zu \enquote{wähle aus Eimer B} gehen und aus Eimer B solange einen Zweig \raisebox{\dimexpr -0.5ex -0.5ex \relax}[0pt][0pt]{\includesvg[width=13px]{\taskGraphicsFolder/graphics/2023-DE-02-taskbody-sprig.svg}} nehmen, bis er $4$ Teile hat – also insgesamt $3$ Zweige nehmen.



% it's informatics
\section*{\BrochureItsInformatics}
Die \emph{Anleitung} fürs Binden des Blumenstrausses sind klar und könnten von einer Maschine ausgeführt werden. In der Informatik nennt man dies einen \emph{Algorithmus}. Die Anleitung benutzt einige Anweisungen, die auch in Computerprogrammen üblich sind:

\begin{itemize}
  \item Die erste Anweisung ist eine zufällige Auswahl aus einer Menge von Objekten.
  \item Die zweite Anweisung nennt man eine \emph{bedingte Anweisung} (engl. \emph{if-statement}) oder eine \emph{Verzweigung}: Denn du musst aus zwei oder mehr Möglichkeiten auswählen.
  \item Die dritte Anweisung sieht relativ einfach aus, muss aber in einem Computerprogramm gut strukturiert werden. Der innere Teil der Anweisung (selbst wieder eine Anweisung: \enquote{Nimm einen Zweig aus Eimer B}) muss mehrmals ausgeführt werden, bis der Blumenstrauss aus $4$ Teilen besteht. Die Ausführung der inneren Anweisung wird also solange wiederholt, bis die Bedingung \enquote{Der Blumenstrauss besteht aus $4$ Teilen.} erfüllt ist. Eine solche \emph{Wiederholungs-Anweisung} nennt man auch \emph{Schleife}.
\end{itemize}

Ein Algorithmus kann unterschiedlich dargestellt werden.  In dieser Biberaufgabe ist Florians \enquote{Blumenstrauss-Algorithmus} als Anleitung in natürlicher Sprache formuliert. In der Lösungserklärung ist er als \emph{Programmablaufplan} dargestellt.

Floristik ist eine Handwerkskunst. Es existieren Traditionen und Regeln, wie ein Blumenstrauss oder ein Kranz gebunden wird. Dies ist ein Beispiel dafür, dass Anleitungen oder Algorithmen in vielen Lebensbereichen vorkommen, nicht nur in der Informatik.



% keywords and websites (as \begin{itemize})
\section*{\BrochureWebsitesAndKeywords}
{\raggedright
\begin{itemize}
  \item bedingte Anweisungen und Verzweigungen: \href{https://de.wikipedia.org/wiki/Bedingte_Anweisung_und_Verzweigung}{\BrochureUrlText{https://de.wikipedia.org/wiki/Bedingte\_Anweisung\_und\_Verzweigung}}
  \item Schleife: \href{https://de.wikipedia.org/wiki/Schleife_(Programmierung)}{\BrochureUrlText{https://de.wikipedia.org/wiki/Schleife\_(Programmierung)}}
  \item Pseudocode: \href{https://de.wikipedia.org/wiki/Pseudocode}{\BrochureUrlText{https://de.wikipedia.org/wiki/Pseudocode}}
  \item Programmablaufplan: \href{https://de.wikipedia.org/wiki/Programmablaufplan}{\BrochureUrlText{https://de.wikipedia.org/wiki/Programmablaufplan}}
  \item Floristik: \href{https://de.wikipedia.org/wiki/Floristik_(Handwerk)}{\BrochureUrlText{https://de.wikipedia.org/wiki/Floristik\_(Handwerk)}}
\end{itemize}


}

% end of ifthen for excluding the solutions
}{}

% all authors
% ATTENTION: you HAVE to make sure an according entry is in ../main/authors.tex.
% Syntax: \def\AuthorLastnameF{} (Lastname is last name, F is first letter of first name, this serves as a marker for ../main/authors.tex)
\def\AuthorWeigendM{} % \ifdefined\AuthorWeigendM \BrochureFlag{de}{} Michael Weigend\fi
\def\AuthorMaoY{} % \ifdefined\AuthorMaoY \BrochureFlag{cn}{} Yong Mao\fi
\def\AuthorKoleszarV{} % \ifdefined\AuthorKoleszarV \BrochureFlag{uy}{} Víctor Koleszar\fi
\def\AuthorDatzkoThutS{} % \ifdefined\AuthorDatzkoThutS \BrochureFlag{de}{} Susanne Datzko-Thut\fi
\def\AuthorPluharZ{} % \ifdefined\AuthorPluharZ \BrochureFlag{hu}{} Zsuzsa Pluhár\fi

\newpage}{}
