% Definition of the meta information: task difficulties, task ID, task title, task country; definition of the variables as well as their scope is in commands.tex
\setcounter{taskAgeDifficulty3to4}{0}
\setcounter{taskAgeDifficulty5to6}{0}
\setcounter{taskAgeDifficulty7to8}{3}
\setcounter{taskAgeDifficulty9to10}{2}
\setcounter{taskAgeDifficulty11to13}{1}
\renewcommand{\taskTitle}{Morpion}
\renewcommand{\taskCountry}{LV}

% include this task only if for the age groups being processed this task is relevant
\ifthenelse{
  \(\boolean{age3to4} \AND \(\value{taskAgeDifficulty3to4} > 0\)\) \OR
  \(\boolean{age5to6} \AND \(\value{taskAgeDifficulty5to6} > 0\)\) \OR
  \(\boolean{age7to8} \AND \(\value{taskAgeDifficulty7to8} > 0\)\) \OR
  \(\boolean{age9to10} \AND \(\value{taskAgeDifficulty9to10} > 0\)\) \OR
  \(\boolean{age11to13} \AND \(\value{taskAgeDifficulty11to13} > 0\)\)}{

\newchapter{\taskTitle}

% task body
Le morpion est un jeu pour deux personnes.

Les deux joueurs remplissent une grille de ${3 \times 3}$ cases à tour de rôle: un joueur dessine des \raisebox{-0.5ex}[0pt][0pt]{\includesvg[width=7.9px]{\taskGraphicsFolder/graphics/2022-LV-03-taskbodyX.svg}} et l’autre des \raisebox{-0.5ex}[0pt][0pt]{\includesvg[width=11.5px]{\taskGraphicsFolder/graphics/2022-LV-03-taskbodyO.svg}} dans les cases. Le premier joueur qui place ses trois symboles dans trois cases alignées dans une colonne, une ligne ou une diagonale a gagné et le jeu est terminé. Si personne n’a gagné et que toutes les cases sont remplies, c’est un match nul.

Tu vois ici les étapes possibles d’un jeu: les quatre premiers tours et le dernier tour. Le joueurs avec les \raisebox{-0.5ex}[0pt][0pt]{\includesvg[width=7.9px]{\taskGraphicsFolder/graphics/2022-LV-03-taskbodyX.svg}} a gagné.

{\centering%
\includesvg[scale=0.15]{\taskGraphicsFolder/graphics/2022-LV-03-taskbody.svg}\par}

Nous appelons la dernière étape du jeu “état final”. Les règles du jeu déterminent exactement de quelle manière les cases peuvent être remplies de \raisebox{-0.5ex}[0pt][0pt]{\includesvg[width=7.9px]{\taskGraphicsFolder/graphics/2022-LV-03-taskbodyX.svg}} et de \raisebox{-0.5ex}[0pt][0pt]{\includesvg[width=11.5px]{\taskGraphicsFolder/graphics/2022-LV-03-taskbodyO.svg}} et quand le jeu se termine.



% question (as \emph{})
{\em
Une seule des images suivantes montre l’état final d’un jeu de morpion. Laquelle?


}

% answer alternatives (as \begin{enumerate}[A)]) or interactivity
\begin{tabularx}{\columnwidth}{ @{} r L r L r L r L @{} }
  A) & \makecell[l]{\includesvg[scale=0.15]{\taskGraphicsFolder/graphics/2022-LV-03-answerA.svg}} & B) & \makecell[l]{\includesvg[scale=0.15]{\taskGraphicsFolder/graphics/2022-LV-03-answerB.svg}} & C) & \makecell[l]{\includesvg[scale=0.15]{\taskGraphicsFolder/graphics/2022-LV-03-answerC.svg}} & D) & \makecell[l]{\includesvg[scale=0.15]{\taskGraphicsFolder/graphics/2022-LV-03-answerD.svg}}
\end{tabularx}



% from here on this is only included if solutions are processed
\ifthenelse{\boolean{solutions}}{
\newpage

% answer explanation
\section*{\BrochureSolution}
La réponse C est juste: \raisebox{-0.5ex}{\includesvg[scale=0.15]{\taskGraphicsFolder/graphics/2022-LV-03-answerC.svg}}

\begin{tabularx}{\columnwidth}{ @{} r J @{} }
  \makecell[r]{\includesvg[scale=0.15]{\taskGraphicsFolder/graphics/2022-LV-03-explanationC.svg}} & La réponse C est correcte, car un joueur a gagné (trois \raisebox{-0.5ex}[0pt][0pt]{\includesvg[width=11.5px]{\taskGraphicsFolder/graphics/2022-LV-03-taskbodyO.svg}} en diagonale) et le jeu s’est terminé. \\ 
  \makecell[r]{\includesvg[scale=0.15]{\taskGraphicsFolder/graphics/2022-LV-03-explanationA.svg}} & La réponse A n’est pas correcte. Le joueur \raisebox{-0.5ex}[0pt][0pt]{\includesvg[width=7.9px]{\taskGraphicsFolder/graphics/2022-LV-03-taskbodyX.svg}} a gagné, mais le joueur \raisebox{-0.5ex}[0pt][0pt]{\includesvg[width=11.5px]{\taskGraphicsFolder/graphics/2022-LV-03-taskbodyO.svg}} a continué de remplir des cases. Comme le gagnant remplit toujours la dernière case, il ne peut jamais dessiner moins de symboles que le perdant. \\ 
  \makecell[r]{\includesvg[scale=0.15]{\taskGraphicsFolder/graphics/2022-LV-03-explanationB.svg}} & La réponse B n’est pas correcte, car cinq cases ont des \raisebox{-0.5ex}[0pt][0pt]{\includesvg[width=7.9px]{\taskGraphicsFolder/graphics/2022-LV-03-taskbodyX.svg}} alors seulement trois ont des \raisebox{-0.5ex}[0pt][0pt]{\includesvg[width=11.5px]{\taskGraphicsFolder/graphics/2022-LV-03-taskbodyO.svg}}. Ce n’est pas possible, car le nombre de \raisebox{-0.5ex}[0pt][0pt]{\includesvg[width=7.9px]{\taskGraphicsFolder/graphics/2022-LV-03-taskbodyX.svg}} et de \raisebox{-0.5ex}[0pt][0pt]{\includesvg[width=11.5px]{\taskGraphicsFolder/graphics/2022-LV-03-taskbodyO.svg}} ne peuvent avoir qu’un point de différence. \\ 
  \makecell[r]{\includesvg[scale=0.15]{\taskGraphicsFolder/graphics/2022-LV-03-explanationD.svg}} & Le réponse D n’est pas correcte, car elle ne montre pas un état final. il n’y a pas de gagnant et les cases ne sont pas toutes remplies.
\end{tabularx}



% it's informatics
\section*{\BrochureItsInformatics}
Pour résoudre l’exercice, nous avons vérifié pour chacune des images si elle correspondait à un état final valide. On peut déduire de nouvelles règles définissant un état final à partir des règles du morpion, par exemple celles-ci:

\begin{enumerate}
  \item La différence entre le nombre de \raisebox{-0.5ex}[0pt][0pt]{\includesvg[width=7.9px]{\taskGraphicsFolder/graphics/2022-LV-03-taskbodyX.svg}} et de \raisebox{-0.5ex}[0pt][0pt]{\includesvg[width=11.5px]{\taskGraphicsFolder/graphics/2022-LV-03-taskbodyO.svg}} doit être égale à $0$, –$1$ ou $1$.
  \item Si aucun joueur n’a gagné, toutes les cases doivent être remplies.
  \item Le perdant peut avoir rempli au maximum le même nombre de cases que le gagnant.
  \item Il ne peut pas y avoir plus d’une seule suite de trois symboles alignés.
\end{enumerate}

Ces nouvelles règles ne sont pas des règles de jeu, mais servent à vérifier si une grille remplie représente un état final. Si une image entre en conflit avec une de ces règles, elle ne peut pas représenter d’état final.

Les règles sont très inmportantes en informatique. Un \emph{interprète} qui exécute un programme vérifie si le texte du programme correspond aux règles de syntaxe du langage de programmation.

En programmation, on utilise des règles appelées assertions pour pouvoir vérifier qu’un programme est correct pendant son exécution.



% keywords and websites (as \begin{itemize})
\section*{\BrochureWebsitesAndKeywords}
{\raggedright
\begin{itemize}
  \item Morpion: \href{https://fr.wikipedia.org/wiki/Tic-tac-toe}{\BrochureUrlText{https://fr.wikipedia.org/wiki/Tic-tac-toe}}
  \item Interprète: \href{https://fr.wikipedia.org/wiki/Interpr\%C3\%A8te_(informatique)}{\BrochureUrlText{https://fr.wikipedia.org/wiki/Interprète\_(informatique)}}
  \item Langage de programmation: \href{https://fr.wikipedia.org/wiki/Langage_de_programmation}{\BrochureUrlText{https://fr.wikipedia.org/wiki/Langage\_de\_programmation}}
\end{itemize}


}

% end of ifthen for excluding the solutions
}{}

% all authors
% ATTENTION: you HAVE to make sure an according entry is in ../main/authors.tex.
% Syntax: \def\AuthorLastnameF{} (Lastname is last name, F is first letter of first name, this serves as a marker for ../main/authors.tex)
\def\AuthorOpmanisM{} % \ifdefined\AuthorOpmanisM \BrochureFlag{lv}{} Mārtiņš Opmanis\fi
\def\AuthorNilandereI{} % \ifdefined\AuthorNilandereI \BrochureFlag{lv}{} Ilze Nilandere\fi
\def\AuthorWeigendM{} % \ifdefined\AuthorWeigendM \BrochureFlag{de}{} Michael Weigend\fi
\def\AuthorPohlW{} % \ifdefined\AuthorPohlW \BrochureFlag{de}{} Wolfgang Pohl\fi
\def\AuthorBaumannW{} % \ifdefined\AuthorBaumannW \BrochureFlag{at}{} Wilfried Baumann\fi
\def\AuthorDatzkoS{} % \ifdefined\AuthorDatzkoS \BrochureFlag{ch}{} Susanne Datzko\fi
\def\AuthorPelletE{} % \ifdefined\AuthorPelletE \BrochureFlag{ch}{} Elsa Pellet\fi

\newpage}{}
