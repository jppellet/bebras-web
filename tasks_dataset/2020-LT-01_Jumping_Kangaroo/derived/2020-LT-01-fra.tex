\documentclass[a4paper,11pt]{report}
\usepackage[T1]{fontenc}
\usepackage[utf8]{inputenc}

\usepackage[french]{babel}
\frenchbsetup{ThinColonSpace=true}
\renewcommand*{\FBguillspace}{\hskip .4\fontdimen2\font plus .1\fontdimen3\font minus .3\fontdimen4\font \relax}
\AtBeginDocument{\def\labelitemi{$\bullet$}}

\usepackage{etoolbox}

\usepackage[margin=2cm]{geometry}
\usepackage{changepage}
\makeatletter
\renewenvironment{adjustwidth}[2]{%
    \begin{list}{}{%
    \partopsep\z@%
    \topsep\z@%
    \listparindent\parindent%
    \parsep\parskip%
    \@ifmtarg{#1}{\setlength{\leftmargin}{\z@}}%
                 {\setlength{\leftmargin}{#1}}%
    \@ifmtarg{#2}{\setlength{\rightmargin}{\z@}}%
                 {\setlength{\rightmargin}{#2}}%
    }
    \item[]}{\end{list}}
\makeatother

\newcommand{\BrochureUrlText}[1]{\texttt{#1}}
\usepackage{setspace}
\setstretch{1.15}

\usepackage{tabularx}
\usepackage{booktabs}
\usepackage{makecell}
\usepackage{multirow}
\renewcommand\theadfont{\bfseries}
\renewcommand{\tabularxcolumn}[1]{>{}m{#1}}
\newcolumntype{R}{>{\raggedleft\arraybackslash}X}
\newcolumntype{C}{>{\centering\arraybackslash}X}
\newcolumntype{L}{>{\raggedright\arraybackslash}X}
\newcolumntype{J}{>{\arraybackslash}X}

\newcommand{\BrochureInlineCode}[1]{{\ttfamily #1}}

\usepackage{amssymb}
\usepackage{amsmath}

\usepackage[babel=true,maxlevel=3]{csquotes}
\DeclareQuoteStyle{bebras-ch-eng}{“}[” ]{”}{‘}[”’ ]{’}\DeclareQuoteStyle{bebras-ch-deu}{«}[» ]{»}{“}[»› ]{”}
\DeclareQuoteStyle{bebras-ch-fra}{«\thinspace{}}[» ]{\thinspace{}»}{“}[»\thinspace{}› ]{”}
\DeclareQuoteStyle{bebras-ch-ita}{«}[» ]{»}{“}[»› ]{”}
\setquotestyle{bebras-ch-fra}

\usepackage{hyperref}
\usepackage{graphicx}
\usepackage{svg}
\svgsetup{inkscapeversion=1,inkscapearea=page}
\usepackage{wrapfig}

\usepackage{enumitem}
\setlist{nosep,itemsep=.5ex}

\setlength{\parindent}{0pt}
\setlength{\parskip}{2ex}
\raggedbottom

\usepackage{fancyhdr}
\usepackage{lastpage}
\pagestyle{fancy}

\fancyhf{}
\renewcommand{\headrulewidth}{0pt}
\renewcommand{\footrulewidth}{0.4pt}
\lfoot{\scriptsize © 2020 Bebras (CC BY-SA 4.0)}
\cfoot{\scriptsize\itshape 2020-LT-01 Kangourou bondissant}
\rfoot{\scriptsize Page~\thepage{}/\pageref*{LastPage}}

\newcommand{\taskGraphicsFolder}{..}

\begin{document}

\section*{\centering{} 2020-LT-01 Kangourou bondissant}


\subsection*{Body}

\begin{wrapfigure}{R}{184px}
\raisebox{-.46cm}[\dimexpr \height-.92cm \relax][-.46cm]{\includesvg[width=184px]{\taskGraphicsFolder/graphics/2020-LT-01_taskbody2-compatible.svg}}
\end{wrapfigure}

Un kangourou saute jusqu’à la maison \raisebox{-0.5ex}[0pt][0pt]{\includesvg[width=14.4px]{\taskGraphicsFolder/graphics/2020-LT-01_taskbody3-compatible.svg}}. Il ne peut sauter que sur le chemin et atteint le croisement suivant d’un grand saut. À un croisement, il peut sauter soit à gauche, soit à droite, soit vers le haut, soit vers le bas. Il n’arrive pas à sautant au dessus d’un tas de $3$ cailloux.

Le kangourou aimerait rentrer à la maison par le chemin le plus court.

{\centering%
\includesvg[width=324.7px]{\taskGraphicsFolder/graphics/2020-LT-01_taskbody1-compatible.svg}\par}

{\em

\subsection*{Question/Challenge}

Combien de sauts le kangourou doit-il faire s’il rentre à la maison par le chemin le plus court?

}\begingroup
\renewcommand{\arraystretch}{1.5}
\subsection*{Answer Options/Interactivity Description}

\begin{tabular}{ @{} r l @{} }
  A) & $10$ sauts \\ 
  B) & $11$ sauts \\ 
  C) & $12$ sauts \\ 
  D) & $13$ sauts \\ 
  E) & $14$ sauts \\ 
  F) & $15$ sauts \\ 
  G) & $16$ sauts \\ 
  H) & $17$ sauts \\ 
  I) & $18$ sauts \\ 
  J) & $19$ sauts \\ 
  K) & $20$ sauts
\end{tabular}

\endgroup

\subsection*{Answer Explanation}

La bonne réponse est E) $14$ sauts:

{\centering%
\includesvg[width=324.7px]{\taskGraphicsFolder/graphics/2020-LT-01_explanation-compatible.svg}\par}

Le plus simple est de commencer la recherche par la fin. On voit rapidement qu’il n’y a qu’un chemin possible sur une grande distance depuis l’arrivée, à savoir $9$ sauts jusqu’au point D. Maintenant, on ne doit plus que trouver le chemin le plus court depuis le départ jusqu’au point D. En deux sauts, le kangourou arrive au point A. Il ne peut pas sauter directement du point A au point D, car il y a un tas de trois cailloux entre deux. Le détour le plus court pour aller de A à D passe par B et C, le kangourou doit pour cela faire $3$ sauts. Le kangourou doit donc faire ${2 + 3 + 9 = 14}$ sauts en tout; tous les autres chemins sont plus longs.


\subsection*{It’s Informatics}

On peut procéder de la manière suivante pour trouver n’importe quel chemin: on suit un chemin au choix pas à pas. Dès que l’on arrive à une impasse où toutes les directions sont bloquées ou que l’on arrive à un endroit déjà visité du chemin, on revient en arrière jusqu’à trouver une autre direction possible, et on essaie ensuite dans cette direction.

En informatique, cette méthode de résolution s’appelle \emph{retour sur trace} ou \emph{retour arrière} (“\emph{backtrack}” en anglais). Elle est utilisée de manière variée dans différents algorithmes. Elle peut être utilisée pour trouver la solution de puzzles, de sudokus et d’autres problèmes d’optimisation combinatoire.

Cet exercice montre qu’il est parfois plus efficace de commencer par la fin pour trouver une solution. On parle alors d’une \emph{recherche en arrière}. Dans notre cas, on a besoin de faire moins de retour sur trace, car on a plus d’options du tout à la fin du chemin. On ne peut pas dire de manière générale qu’une recherche en arrière ou une recherche en avant est mieux, cela dépend du problème concret.

{\raggedright

\subsection*{Keywords and Websites}

\begin{itemize}
  \item Retour sur trace: \href{https://fr.wikipedia.org/wiki/Retour_sur_trace}{\BrochureUrlText{https://fr.wikipedia.org/wiki/Retour\_sur\_trace}}
\end{itemize}


}
\end{document}
