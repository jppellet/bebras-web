% Definition of the meta information: task difficulties, task ID, task title, task country; definition of the variables as well as their scope is in commands.tex
\setcounter{taskAgeDifficulty3to4}{2}
\setcounter{taskAgeDifficulty5to6}{1}
\setcounter{taskAgeDifficulty7to8}{0}
\setcounter{taskAgeDifficulty9to10}{0}
\setcounter{taskAgeDifficulty11to13}{0}
\renewcommand{\taskTitle}{Timber!}
\renewcommand{\taskCountry}{UY}

% include this task only if for the age groups being processed this task is relevant
\ifthenelse{
  \(\boolean{age3to4} \AND \(\value{taskAgeDifficulty3to4} > 0\)\) \OR
  \(\boolean{age5to6} \AND \(\value{taskAgeDifficulty5to6} > 0\)\) \OR
  \(\boolean{age7to8} \AND \(\value{taskAgeDifficulty7to8} > 0\)\) \OR
  \(\boolean{age9to10} \AND \(\value{taskAgeDifficulty9to10} > 0\)\) \OR
  \(\boolean{age11to13} \AND \(\value{taskAgeDifficulty11to13} > 0\)\)}{

\newchapter{\taskTitle}

% task body
Un castor aimerait construire un barrage. Afin de toujours abattre les bons arbres, il s’est fixé deux règles. Il n’abat un arbre que si:

\begin{itemize}
  \item un arbre plus petit pousse directement à sa gauche et
  \item un arbre plus grand pousse directement à sa droite.
\end{itemize}



% question (as \emph{})
{\em
Quels arbres le castor va-t-il abattre?

{\centering%
\includesvg[scale=0.075]{\taskGraphicsFolder/graphics/2021-UY-11-question.svg}\par}


}

% answer alternatives (as \begin{enumerate}[A)]) or interactivity


% from here on this is only included if solutions are processed
\ifthenelse{\boolean{solutions}}{
\newpage

% answer explanation
\section*{\BrochureSolution}
Seuls les arbres à la quatrième et septième positions remplissent les conditions données: il y a un arbre plus petit directement à gauche ET un arbre plus grand directement à droite.

{\centering%
\includesvg[scale=0.075]{\taskGraphicsFolder/graphics/2021-UY-11-solution.svg}\par}



% it's informatics
\section*{\BrochureItsInformatics}
En informatique, il faut souvent résoudre des problèmes qui sont spécifiés par une série de \emph{contraintes} logiques. La tâche consiste à trouver une solution qui respecte toutes les contraintes. Des problème plus complexes que celui-ci peuvent être résolus en utilisant des \emph{opérateurs logiques} pour combiner les contraintes. La conjonction (${\wedge}$ ou encore opérateur ET) donne par exemple le résultat “vrai” pour l’expression A ${\wedge}$ B lorsque les deux contraintes A et B sont vraies. Dans cet exercice, ce serait donc: “l’arbre de gauche est plus petit” ${\wedge}$ “l’arbre de droite est plus grand”. On retrouve ce principe fondamental dans presque tous les domaines de l’informatique, par exemple dans beaucoup d’algorithmes de tri comme le \emph{tri à bulles} lors duquel la satisfaction aux contraintes de plusieurs objets d’une liste est évaluée avant de les déplacer, si nécessaire, à une autre position. Ce procédé est répété jusqu’à ce que la liste soit complètement triée.



% keywords and websites (as \begin{itemize})
\section*{\BrochureWebsitesAndKeywords}
{\raggedright
\begin{itemize}
  \item Pensée algorithmique (\emph{algorithmic thinking})
  \item Opérateur logique: \href{https://fr.wikipedia.org/wiki/Connecteur_logique}{\BrochureUrlText{https://fr.wikipedia.org/wiki/Connecteur\_logique}}
  \item Tri à bulles: \href{https://fr.wikipedia.org/wiki/Tri_\%C3\%A0_bulles}{\BrochureUrlText{https://fr.wikipedia.org/wiki/Tri\_à\_bulles}}
  \item Tri: \href{https://sorting.at/}{\BrochureUrlText{https://sorting.at/}}
  \item Problème de satisfaction de contraintes: \href{https://fr.wikipedia.org/wiki/Probl\%C3\%A8me_de_satisfaction_de_contraintes}{\BrochureUrlText{https://fr.wikipedia.org/wiki/Problème\_de\_satisfaction\_de\_contraintes}}
\end{itemize}


}

% end of ifthen for excluding the solutions
}{}

% all authors
% ATTENTION: you HAVE to make sure an according entry is in ../main/authors.tex.
% Syntax: \def\AuthorLastnameF{} (Lastname is last name, F is first letter of first name, this serves as a marker for ../main/authors.tex)
\def\AuthorOyhenardG{} % \ifdefined\AuthorOyhenardG \BrochureFlag{uy}{} Graciela Oyhenard\fi
\def\AuthorSchunkR{} % \ifdefined\AuthorSchunkR \BrochureFlag{uy}{} Rosario Schunk\fi
\def\AuthorKoleszarV{} % \ifdefined\AuthorKoleszarV \BrochureFlag{uy}{} Víctor Koleszar\fi
\def\AuthorHirschB{} % \ifdefined\AuthorHirschB \BrochureFlag{at}{} Benjamin Hirsch\fi
\def\AuthorKandlhoferM{} % \ifdefined\AuthorKandlhoferM \BrochureFlag{at}{} Martin Kandlhofer\fi
\def\AuthorDatzkoC{} % \ifdefined\AuthorDatzkoC \BrochureFlag{hu}{} Christian Datzko\fi
\def\AuthorFreiF{} % \ifdefined\AuthorFreiF \BrochureFlag{ch}{} Fabian Frei\fi
\def\AuthorDatzkoS{} % \ifdefined\AuthorDatzkoS \BrochureFlag{ch}{} Susanne Datzko\fi
\def\AuthorPelletE{} % \ifdefined\AuthorPelletE \BrochureFlag{ch}{} Elsa Pellet\fi

\newpage}{}
