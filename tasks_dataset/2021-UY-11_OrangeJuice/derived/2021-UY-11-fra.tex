\documentclass[a4paper,11pt]{report}
\usepackage[T1]{fontenc}
\usepackage[utf8]{inputenc}

\usepackage[french]{babel}
\frenchbsetup{ThinColonSpace=true}
\renewcommand*{\FBguillspace}{\hskip .4\fontdimen2\font plus .1\fontdimen3\font minus .3\fontdimen4\font \relax}
\AtBeginDocument{\def\labelitemi{$\bullet$}}

\usepackage{etoolbox}

\usepackage[margin=2cm]{geometry}
\usepackage{changepage}
\makeatletter
\renewenvironment{adjustwidth}[2]{%
    \begin{list}{}{%
    \partopsep\z@%
    \topsep\z@%
    \listparindent\parindent%
    \parsep\parskip%
    \@ifmtarg{#1}{\setlength{\leftmargin}{\z@}}%
                 {\setlength{\leftmargin}{#1}}%
    \@ifmtarg{#2}{\setlength{\rightmargin}{\z@}}%
                 {\setlength{\rightmargin}{#2}}%
    }
    \item[]}{\end{list}}
\makeatother

\newcommand{\BrochureUrlText}[1]{\texttt{#1}}
\usepackage{setspace}
\setstretch{1.15}

\usepackage{tabularx}
\usepackage{booktabs}
\usepackage{makecell}
\usepackage{multirow}
\renewcommand\theadfont{\bfseries}
\renewcommand{\tabularxcolumn}[1]{>{}m{#1}}
\newcolumntype{R}{>{\raggedleft\arraybackslash}X}
\newcolumntype{C}{>{\centering\arraybackslash}X}
\newcolumntype{L}{>{\raggedright\arraybackslash}X}
\newcolumntype{J}{>{\arraybackslash}X}

\newcommand{\BrochureInlineCode}[1]{{\ttfamily #1}}

\usepackage{amssymb}
\usepackage{amsmath}

\usepackage[babel=true,maxlevel=3]{csquotes}
\DeclareQuoteStyle{bebras-ch-eng}{“}[” ]{”}{‘}[”’ ]{’}\DeclareQuoteStyle{bebras-ch-deu}{«}[» ]{»}{“}[»› ]{”}
\DeclareQuoteStyle{bebras-ch-fra}{«\thinspace{}}[» ]{\thinspace{}»}{“}[»\thinspace{}› ]{”}
\DeclareQuoteStyle{bebras-ch-ita}{«}[» ]{»}{“}[»› ]{”}
\setquotestyle{bebras-ch-fra}

\usepackage{hyperref}
\usepackage{graphicx}
\usepackage{svg}
\svgsetup{inkscapeversion=1,inkscapearea=page}
\usepackage{wrapfig}

\usepackage{enumitem}
\setlist{nosep,itemsep=.5ex}

\setlength{\parindent}{0pt}
\setlength{\parskip}{2ex}
\raggedbottom

\usepackage{fancyhdr}
\usepackage{lastpage}
\pagestyle{fancy}

\fancyhf{}
\renewcommand{\headrulewidth}{0pt}
\renewcommand{\footrulewidth}{0.4pt}
\lfoot{\scriptsize © 2021 Bebras (CC BY-SA 4.0)}
\cfoot{\scriptsize\itshape 2021-UY-11 Timber!}
\rfoot{\scriptsize Page~\thepage{}/\pageref*{LastPage}}

\newcommand{\taskGraphicsFolder}{..}

\begin{document}

\section*{\centering{} 2021-UY-11 Timber!}


\subsection*{Body}

Un castor aimerait construire un barrage. Afin de toujours abattre les bons arbres, il s’est fixé deux règles. Il n’abat un arbre que si:

\begin{itemize}
  \item un arbre plus petit pousse directement à sa gauche et
  \item un arbre plus grand pousse directement à sa droite.
\end{itemize}

{\em


\subsection*{Question/Challenge - for the brochures}

Quels arbres le castor va-t-il abattre?

{\centering%
\includesvg[scale=0.075]{\taskGraphicsFolder/graphics/2021-UY-11-question.svg}\par}

}

\begingroup
\renewcommand{\arraystretch}{1.5}
\subsection*{Answer Options/Interactivity Description}



\endgroup

\subsection*{Answer Explanation}

Seuls les arbres à la quatrième et septième positions remplissent les conditions données: il y a un arbre plus petit directement à gauche ET un arbre plus grand directement à droite.

{\centering%
\includesvg[scale=0.075]{\taskGraphicsFolder/graphics/2021-UY-11-solution.svg}\par}


\subsection*{It’s Informatics}

En informatique, il faut souvent résoudre des problèmes qui sont spécifiés par une série de \emph{contraintes} logiques. La tâche consiste à trouver une solution qui respecte toutes les contraintes. Des problème plus complexes que celui-ci peuvent être résolus en utilisant des \emph{opérateurs logiques} pour combiner les contraintes. La conjonction (${\wedge}$ ou encore opérateur ET) donne par exemple le résultat “vrai” pour l’expression A ${\wedge}$ B lorsque les deux contraintes A et B sont vraies. Dans cet exercice, ce serait donc: “l’arbre de gauche est plus petit” ${\wedge}$ “l’arbre de droite est plus grand”. On retrouve ce principe fondamental dans presque tous les domaines de l’informatique, par exemple dans beaucoup d’algorithmes de tri comme le \emph{tri à bulles} lors duquel la satisfaction aux contraintes de plusieurs objets d’une liste est évaluée avant de les déplacer, si nécessaire, à une autre position. Ce procédé est répété jusqu’à ce que la liste soit complètement triée.

{\raggedright

\subsection*{Keywords and Websites}

\begin{itemize}
  \item Pensée algorithmique (\emph{algorithmic thinking})
  \item Opérateur logique: \href{https://fr.wikipedia.org/wiki/Connecteur_logique}{\BrochureUrlText{https://fr.wikipedia.org/wiki/Connecteur\_logique}}
  \item Tri à bulles: \href{https://fr.wikipedia.org/wiki/Tri_\%C3\%A0_bulles}{\BrochureUrlText{https://fr.wikipedia.org/wiki/Tri\_à\_bulles}}
  \item Tri: \href{https://sorting.at/}{\BrochureUrlText{https://sorting.at/}}
  \item Problème de satisfaction de contraintes: \href{https://fr.wikipedia.org/wiki/Probl\%C3\%A8me_de_satisfaction_de_contraintes}{\BrochureUrlText{https://fr.wikipedia.org/wiki/Problème\_de\_satisfaction\_de\_contraintes}}
\end{itemize}


}
\end{document}
